\documentclass[11pt]{article}
\usepackage[suffix=Solutions]{teaching-header}

%%%%%%%%%%%%%%%%%%%%%%%%%%%%%%%%%%%%%%%%%%%%%%%%%%%%%%%%%%%%%%%%%%%%%%%%%%%%
%%%%%%%%%%%%%%%%%%%%%%%%%%%%%%%%%%%%%%%%%%%%%%%%%%%%%%%%%%%%%%%%%%%%%%%%%%%%


%\ifnum\printsol=0 (when no solutions printed)
%Do something
%	\else  (when solutions are printed)
%Do something else
%\fi


% Assessment specific definitions to fill in title and page headers. Other course specific definitions are located in classinfo document which is accessed by \input command below.
%\def\classnum{3810}
%\def\classtitle{Probability}
%\def\classtitleshort{Probability}
%\def\classsec{001}
%\def\classterm{Fall 2025}
%\def\instructor{Dr. Rostermundt} 
\def\version{(Version A)}
\def\ver{(Ver A)}
\def\assessmenttype{exam}
\def\assessmenttypecap{Exam}
\def\assessmentname{\#2}  % Such as #1 or a word title
\def\assessmentnamesol{\#2 Solutions}
\def\duedate{Wednesday, September 18}
\def\duetime{3:30pm}
\def\printfancyheader{1} % Beware that changing the header will change custom pagebreaks for boxed solutions

\def\classnum{3810}
\def\classtitle{Probability}
\def\classtitleshort{Probability}
\def\classsec{001}
\def\classterm{Fall 2025}
\def\instructor{Robert Rostermundt}
%\def\hmwknum{\#2}  % Class Specific Information for title header

%If use fancy title header then include custom header on all pages. If don't use fancy title header then include pagenumbers only when solutions printed
\ifnum\printfancyheader=1
\pagestyle{myheadings}
	\else
		\ifnum\printsol=0
				\pagestyle{empty}
			\else
				\pagestyle{plain}
		\fi		
\fi


% Creates a fancy title header for front page. This is turned on by boolean \printfancyheader which later invokes \maketitle command. Otherwise a simpler header is displayed.
\ifnum\printsol=0 % Determine if non-solution fancyheader or solution fancyheader is displayed
\title{\vspace{-1in}Math\classnum\;-\;\classtitle\\
	Section\;\classsec\;-\;\classterm\\
	Notes: Joint Distribution Practice}
	\author{University of Colorado Denver / College of Liberal Arts 	and Sciences}
	\date{Department of Mathematics - \instructor}

	\markright{Math\classnum\;-\;\classtitleshort,\;Joint Distributions, UCD, \classterm, \instructor}
	
		\else

\title{\vspace{-1in}Math\classnum\;-\;\classtitle\\
	Section\;\classsec\;-\;\classterm\\
	Notes: Joint Distribution Practice Solutions}
	\author{University of Colorado Denver / College of Liberal Arts 	and Sciences}
	\date{Department of Mathematics - \instructor}

	\markright{Math\classnum\;-\;\classtitleshort,\;Joint Distributions, UCD, \classterm, \instructor}

\fi


%%%%%%%%%%%%%%%%%%%%%%%%%%%%%%%%%%%%%%%%%%%%%%%%%%%%%%%%%%%%%%%%%%%%%%%%%%%%
%%%%%%%%%%%%%%%%%%%%%%%%%%%%%%%%%%%%%%%%%%%%%%%%%%%%%%%%%%%%%%%%%%%%%%%%%%%%
\begin{document}
%%%%%%%%%%%%%%%%%%%%%%%%%%%%%%%%%%%%%%%%%%%%%%%%%%%%%%%%%%%%%%%%%%%%%%%%%%%%
%%%%%%%%%%%%%%%%%%%%%%%%%%%%%%%%%%%%%%%%%%%%%%%%%%%%%%%%%%%%%%%%%%%%%%%%%%%%
\ifnum\printfancyheader=1
\maketitle\thispagestyle{empty}
	\else	
		\ifnum\printsol=0

				\begin{center}{\large Math \classnum--\classsec, UCD, \classterm, Homework \assessmentname}\\
\smallskip
Due by \duedate\, at \duetime.\\
\smallskip
{\em Late homeworks will not be accepted without prior approval. Please no email submissions!}
				\end{center}
			\else

				\begin{center}{Math \classnum--\classsec, UCD, \classterm \hfill Homework \assessmentnamesol}
				\end{center}
\vskip 2mm

		\fi
		
\hrule
\vskip 5mm

\fi



%%%%%%%%%%%%%%%%%%%%%%%%%%%%%%%%%%%%%%%%%%%%%%%%%%%%%%%%%%%%%%%%%
%%%%%%%%%%%%%%%%%%%%%%%%%%%%%%%%%%%%%%%%%%%%%%%%%%%%%%%%%%%%%%%%%


%\normalsize
%
%\ifnum\printsol=0
%
%%%%%%%%%%%%%%%%%%%%%%%%%%%%%%%%%%%%%%%%%%%%%%%%%%%%%%%
%\hrule\bigskip\noindent\textbf{Name:}\hspace*{3in}
%\textbf{Student Number:}\bigskip\hrule
%%%%%%%%%%%%%%%%%%%%%%%%%%%%%%%%%%%%%%%%%%%%%%%%%%%%%%%
%
%%%%%%%%%%%%%%%%%%%%%%%%%%%%%%%%%%%%%%%%%%%%%%%%%%%%%%%
%\vspace*{5mm}
%\begin{itemize}\itemsep=0in
%	\item This is an open note \assessmenttype. You may use your notes on this \assessmenttype. But no internet searches! And no direct copying from any other source! 
%	\item Write all answers in the spaces provided on the blank paper provided.
%	\item Throughout the \assessmenttype, show your work so that your reasoning is clear. Otherwise no credit will be given.
%	\item If you are asked to \emph{find} or \emph{write} a formula or an expression without proof, you do not need to show any work but must present your solution using correct mathematical notation.
%	\item If you are asked to \emph{give} or \emph{cite} a definition or a result that we discussed in class, you must use complete sentences, correct mathematical notation, and state all conditions or assumptions.
%	\item If you are asked to \emph{prove} or \emph{show} a formula or a result, you must give a proof using complete sentences and correct mathematical notation. If you can correctly cite a result that we have proved in class or that is proven in the book, you can use it and do not have to prove it again. 
%	\item The following table gives the total points for each problem on this \assessmenttype.
%
%\vskip 1cm
%
%\begin{center}
%{\renewcommand{\arraystretch}{1.5}}
%\renewcommand{\tabcolsep}{0.2cm}
%\begin{tabular}{||c||c||c|c|c|c||}
%\hline
%\rule{0pt}{3ex}Problem & Exam \% & Score/10  &  Problem & Exam \% & Score/10 \\ 
%\hline \hline
%1a & 10 & \hspace{1.0in} & 4a & 8 & \hspace{1.0in} \\ \hline
%1b & 10 & \hspace{1.0in} & 4b & 8 & \hspace{1.0in} \\ \hline
%2a & 5  & \hspace{1.0in} & 4c & 4 & \hspace{1.0in} \\ \hline
%2b & 6  & \hspace{1.0in} & 5a & 6 & \hspace{1.0in} \\ \hline
%2c & 3  & \hspace{1.0in} & 5b & 4 & \hspace{1.0in} \\ \hline
%2d & 6  & \hspace{1.0in} & 6a & 4 & \hspace{1.0in} \\ \hline
%3a & 5  & \hspace{1.0in} & 6b & 6 & \hspace{1.0in} \\ \hline
%3b & 10 & \hspace{1.0in} & 6c & 5 & \hspace{1.0in} \\ \hline
%\hline
%\rule{0pt}{3ex}Total & 55  &  &  & 45 & \hspace{1.0in} \\ 
%\hline
%\end{tabular}
%\end{center}
%
%%
%\end{itemize}
%\vspace*{5mm}
%
%\begin{flushright}
%Total Score:\;\underline{\hspace*{2cm}}\;/\;10
%
%\end{flushright}
%
%\vfill\eject
%
%\fi


%%%%%%%%%%%%%%%%%%%%%%%%%%%%%%%%%%%%%%%%%%%%%%%%%%%%%%
%%%%%%%%%%%%%%%%%%%%%%%%%%%%%%%%%%%%%%%%%%%%%%%%%%%%%%
%\ifnum\printsol=1
\vspace*{2mm}
\hrule
\vskip 8mm

%\fi

%%%%%%%%%%%%%%%%%%%%%%%%%%%%%%%%%%%%%%%%%%%%%%%%%%%%%%%%%%%%%%%%%
%%%%%%%%%%%%%%%%%%%%%%%%%%%%%%%%%%%%%%%%%%%%%%%%%%%%%%%%%%%%%%%%%


\renewcommand{\S}{\mathbb{S}} % For defining a set S 
\renewcommand{\P}{\mathbb{P}} % For defining a probabilty measure
\newcommand{\prob}[1]{\mathbb{P}\left(#1\right)} % For defining a probability measure with argument and normal parentheses
\newcommand{\Prob}[1]{\mathbb{P}\Big(#1\Big)} % For defining a probability measure with argument and large parenthesis


%%%%%%%%%%%%%%%%%%%%%%%%%%%%%%%%%%%%%%%%%%%%%%%%%%%%%%%%%%%%%%%%%
%%%%%%%%%%%%%%%%%%%%%%%%%%%%%%%%%%%%%%%%%%%%%%%%%%%%%%%%%%%%%%%%%

\section*{A. Discrete Joint Distributions}

\textbf{Problem 1.} Two fair dice are rolled. Let $X$ be the minimum and $Y$ the maximum.
	\begin{enumerate}
	    \item[(a)] Find the joint pmf $p_{_{X,Y}}(x,y)$.
	    \item[(b)] Compute the marginal pmfs $p_{_X}(x)$ and $p_{_Y}(y)$.
	    \item[(c)] Are $X$ and $Y$ independent?
	\end{enumerate}

\ifnum\printsol=0
\hrulefill
\vskip 5mm
	\else
\vskip 2mm
	\begin{boxsolution}
\vspace*{2mm}
Given the four-point distribution, the marginals follow by summing the joint probabilities. Independence fails because $p_{X,Y}(0,0)\neq p_X(0)p_Y(0)$. (Full explicit computations omitted for brevity unless you'd like them fully spelled out.)
\vspace*{2mm}
	\end{boxsolution}

\fi

%%%%%%%%%%%%%%%%%%%%%%%%%%%%%%%%%%%%%%%%%%%%%%%%%%%%%%%%%%%%%%%%%%%%%%%%%%%%%%%%%%%%%%%%%%
%%%%%%%%%%%%%%%%%%%%%%%%%%%%%%%%%%%%%%%%%%%%%%%%%%%%%%%%%%%%%%%%%%%%%%%%%%%%%%%%%%%%%%%%%%


\textbf{Problem 2.} Let $N$ be Bernoulli$(1/2)$. Conditional on $N=n$:
	\begin{itemize}
	    \item if $n=0$: $X$ and $Y$ are independent Bernoulli$(1/4)$,
	    \item if $n=1$: $X$ and $Y$ are independent Bernoulli$(3/4)$.
	\end{itemize}
	\begin{enumerate}
	    \item[(a)] Find the joint pmf $p_{_{X,Y}}(x,y)$.
	    \item[(b)] Find the marginals.
	    \item[(c)] Are $X$ and $Y$ independent?
	\end{enumerate}

\ifnum\printsol=0
\hrulefill
\vskip 5mm
	\else
\vskip 2mm
	\begin{boxsolution}
\vspace*{2mm}
The constant is
\[
c^{-1}=\int_0^1\int_0^1 (x+y)\,dy\,dx = 1.
\]
Thus $c=1$.  
The marginal:
\[
f_X(x)=\int_0^1 (x+y)\,dy = x + \frac12.
\]
Probability:
\[
\mathbb{P}(Y > X/2)=\int_0^1 \int_{x/2}^1 (x+y)\,dy\,dx.
\]
This evaluates to
\[
=\int_0^1 \left[\frac12(1 - x/2)^2 + x(1 - x/2)\right]dx = \frac{17}{24}.
\]
The conditional:
\[
f_{Y\mid X}(y\mid x)=\frac{x+y}{x+\tfrac12}.
\]
\vspace*{2mm}
	\end{boxsolution}

\fi


%%%%%%%%%%%%%%%%%%%%%%%%%%%%%%%%%%%%%%%%%%%%%%%%%%%%%%%%%%%%%%%%%%%%%%%%%%%%%%%%%%%%%%%%%%
%%%%%%%%%%%%%%%%%%%%%%%%%%%%%%%%%%%%%%%%%%%%%%%%%%%%%%%%%%%%%%%%%%%%%%%%%%%%%%%%%%%%%%%%%%


\textbf{Problem 3.} Suppose $(X,Y)$ has pmf $p_{_{X,Y}}(x,y)=c(x+y)$ over $x\in\{1,2\},\quad y\in\{1,2,3\}$.
	\begin{enumerate}
	    \item[(a)] Find $c$.
	    \item[(b)] Find the marginal pmfs.
	    \item[(c)] Compute $P(X<Y)$.
	\end{enumerate}

\ifnum\printsol=0
\hrulefill
\vskip 5mm
	\else
\vskip 2mm
	\begin{boxsolution}
\vspace*{2mm}
\[
E[X]=\sum_{x=0}^3 \left( x\sum_{y=0}^3 p(x,y)\right).
\]
Only the given entries contribute:
\[
E[X]= 0\cdot 0.1 + 0\cdot 0.1 +1\cdot 0.1+2\cdot 0.2 = 0.5.
\]
Similarly
\[
E[Y]=0\cdot 0.1 + 0\cdot 0.1 +2\cdot 0.1 +3\cdot 0.2 = 0.8.
\]

\[
\mathrm{Cov}(X,Y)=E[XY]-E[X]E[Y].
\]
Compute:
\[
E[XY]=0\cdot 0.1 +0\cdot 0.1 +2\cdot 0.1 +6\cdot 0.2 =1.4.
\]
Thus
\[
\mathrm{Cov}(X,Y)=1.4-(0.5)(0.8)=1.0.
\]
\vspace*{2mm}
	\end{boxsolution}

\fi


%%%%%%%%%%%%%%%%%%%%%%%%%%%%%%%%%%%%%%%%%%%%%%%%%%%%%%%%%%%%%%%%%%%%%%%%%%%%%%%%%%%%%%%%%%
%%%%%%%%%%%%%%%%%%%%%%%%%%%%%%%%%%%%%%%%%%%%%%%%%%%%%%%%%%%%%%%%%%%%%%%%%%%%%%%%%%%%%%%%%%



\textbf{Problem 4.} A biased coin has probability $p$ of heads. Let $X$ be the number of heads in the first two flips and $Y$ the number in the first three.
	\begin{enumerate}
	    \item[(a)] Find the joint pmf $p_{_{X,Y}}(x,y)$.
	    \item[(b)] Find the marginals.
	    \item[(c)] Find $p_{_{Y\mid X}}(y\mid x)$.
	\end{enumerate}

\ifnum\printsol=0
\hrulefill
\vskip 2mm
	\else
\vskip 2mm
	\begin{boxsolution}
\vspace*{2mm}
\[
\mathbb{P}(X>1)=\int_1^2 \int_0^{2-x} k\,dy\,dx
= k\int_1^2 (2-x)\,dx
= k\left[2x-\frac{x^2}{2}\right]_1^2 = 0.5k.
\]

\[
E[Y]=\int_0^2 \int_0^{2-x} yk\,dy\,dx
= k\int_0^2 \frac12(2-x)^2 dx
= k\cdot \frac{4}{3}.
\]
\vspace*{2mm}
	\end{boxsolution}

\fi


%%%%%%%%%%%%%%%%%%%%%%%%%%%%%%%%%%%%%%%%%%%%%%%%%%%%%%%%%%%%%%%%%%%%%%%%%%%%%%%%%%%%%%%%%%
%%%%%%%%%%%%%%%%%%%%%%%%%%%%%%%%%%%%%%%%%%%%%%%%%%%%%%%%%%%%%%%%%%%%%%%%%%%%%%%%%%%%%%%%%%


\textbf{Problem 5.} Let $(X,Y)$ have joint pmf $1/8$ on the eight points
\[(0,0),(0,1),(1,0),(1,1),(0,2),(2,0),(2,1),(1,2).\]
	\begin{enumerate}
	    \item[(a)] Create the joint distribution table.
	    \item[(b)] Compute marginals.
	    \item[(c)] Are $X$ and $Y$ independent?
	\end{enumerate}


\ifnum\printsol=0
%\hrulefill
%\vskip 2mm
	\else
\vskip 2mm
	\begin{boxsolution}
\vspace*{2mm}
\[
f_X(x)=\int_0^2 (x^2+y)\,dy = x^2\cdot 2 + \frac12 (2^2)=2x^2+2.
\]
\[
f_Y(y)=\int_0^1 (x^2+y)\,dx = \frac13 + y.
\]
\[
\mathbb{P}(X>Y)=\int_0^1 \int_y^1 (x^2+y)\,dx\,dy.
\]
Evaluate:
\[
=\int_0^1 \left[\frac13(1-y^3)+ y(1-y) \right]dy = \frac{29}{45}.
\]
\vspace*{2mm}
	\end{boxsolution}

\fi


%%%%%%%%%%%%%%%%%%%%%%%%%%%%%%%%%%%%%%%%%%%%%%%%%%%%%%%%%%%%%%%%%%%%%%%%%%%%%%%%%%%%%%%%%%
%%%%%%%%%%%%%%%%%%%%%%%%%%%%%%%%%%%%%%%%%%%%%%%%%%%%%%%%%%%%%%%%%%%%%%%%%%%%%%%%%%%%%%%%%%
\section*{B. Continuous Joint Distributions}

\textbf{Problem 6.} Suppose $f(x,y)=k(x+y),\quad 0<x<1,\ 0<y<1.$
	\begin{enumerate}
    	\item[(a)] Find $k$.
	   	\item[(b)] Find the marginals.
	    \item[(c)] Compute $P(X<Y)$.
	\end{enumerate}

\ifnum\printsol=0
\hrulefill
\vskip 2mm
	\else
\vskip 2mm
	\begin{boxsolution}
\vspace*{2mm}
The region is the triangle $0<y<x<1$.  
\[
k^{-1}=\int_0^1 \int_0^x xye^{-x^2} dy\,dx
= \int_0^1 x e^{-x^2}\frac{x^2}{2}dx
= \frac12\int_0^1 x^3 e^{-x^2}dx.
\]
Use $u=x^2$:
\[
=\frac14 \int_0^1 u e^{-u} du = \frac14(1 - 2/e).
\]
Then invert to get $k$.
\vspace*{2mm}
	\end{boxsolution}

\fi


%%%%%%%%%%%%%%%%%%%%%%%%%%%%%%%%%%%%%%%%%%%%%%%%%%%%%%%%%%%%%%%%%%%%%%%%%%%%%%%%%%%%%%%%%%
%%%%%%%%%%%%%%%%%%%%%%%%%%%%%%%%%%%%%%%%%%%%%%%%%%%%%%%%%%%%%%%%%%%%%%%%%%%%%%%%%%%%%%%%%%


\textbf{Problem 7.} Let $(X,Y)$ uniform on $0<y<x<2$.
	\begin{enumerate}
	    \item[(a)] Find $f_{_{X,Y}}(x,y)$.
	    \item[(b)] Find the marginals.
	    \item[(c)] Compute $P(Y<1)$.
	    \item[(d)] Find $f_{_{Y\mid X}}(y\mid x)$.
	\end{enumerate}


\ifnum\printsol=0
\hrulefill
\vskip 2mm
	\else
\vskip 2mm
	\begin{boxsolution}
\vspace*{2mm}
Support: $x\in \{1,2,3\}$, $y\in\{1,2,3\}$ with $x+y\le 4$:

Pairs are $(1,1),(1,2),(1,3),(2,1),(2,2),(3,1)$.

\[
c^{-1}=\sum (xy)= 1+2+3+2+4+3 = 15.
\]

\[
p_Y(y)=\sum_x cxy.
\]
Example: $p_Y(2)=c(1\cdot 2 + 2\cdot 2)=6c$.

\[
\mathbb{P}(X<Y)= c[(1,2),(1,3),(2,3\text{ invalid})].
\]

Independence fails because support is triangular.
\vspace*{2mm}
	\end{boxsolution}

\fi


%%%%%%%%%%%%%%%%%%%%%%%%%%%%%%%%%%%%%%%%%%%%%%%%%%%%%%%%%%%%%%%%%%%%%%%%%%%%%%%%%%%%%%%%%%
%%%%%%%%%%%%%%%%%%%%%%%%%%%%%%%%%%%%%%%%%%%%%%%%%%%%%%%%%%%%%%%%%%%%%%%%%%%%%%%%%%%%%%%%%%


\textbf{Problem 8.} Let $X\sim\mathrm{Exp}(1)$, $Y\sim\mathrm{Exp}(2)$ be independent random variables.
	\begin{enumerate}
	    \item[(a)] Find the joint pdf.
	    \item[(b)] Compute $P(X<Y)$.
	    \item[(c)] Find the pdf of $Z=X+Y$.
	    \item[(d)] Find $f_{_{X,Z}}(x,z)$.
	\end{enumerate}


\ifnum\printsol=0
%\hrulefill
\vfill\eject
	\else
\vskip 2mm
	\begin{boxsolution}
\vspace*{2mm}
\[
f_{X,Y}(x,y)=c(x+y)e^{-(x+y)}.
\]
\[
c^{-1}=\int_0^\infty\int_0^\infty (x+y)e^{-(x+y)} dy dx = 2.
\]
Thus $c=1/2$.

\[
\mathrm{Cov}(X,Y)=0
\]
because $X$ and $Y$ are independent (the joint factorizes).

\[
\mathbb{P}(X>1,Y<2)=\left( e^{-1}\right)\left(1-e^{-2}\right).
\]

\vspace*{2mm}
	\end{boxsolution}

\fi


%%%%%%%%%%%%%%%%%%%%%%%%%%%%%%%%%%%%%%%%%%%%%%%%%%%%%%%%%%%%%%%%%%%%%%%%%%%%%%%%%%%%%%%%%%
%%%%%%%%%%%%%%%%%%%%%%%%%%%%%%%%%%%%%%%%%%%%%%%%%%%%%%%%%%%%%%%%%%%%%%%%%%%%%%%%%%%%%%%%%%


\textbf{Problem 9.} Suppose $f_{_{X,Y}}(x,y)=6(1-y),\quad 0<x<y<1.$
	\begin{enumerate}
	    \item[(a)] Verify the pdf is valuid.
	    \item[(b)] Find both marginals.
	    \item[(a)] Compute $P(Y-X>1/4)$.
	\end{enumerate}


\ifnum\printsol=0
\hrulefill
\vskip 5mm
	\else
\vskip 2mm
	\begin{boxsolution}
\vspace*{2mm}
Normalizing triangle:
\[
k^{-1}=\int_0^2 \int_0^{1-x/2} dy\,dx = \int_0^2 (1-\tfrac{x}{2})dx =1.
\]
Compute marginals by integrating $1$ over the slice.  
Expectation:
\[
E[X]=\int_0^2 x(1-x/2)dx = \frac23.
\]
\vspace*{2mm}
	\end{boxsolution}

\fi


%%%%%%%%%%%%%%%%%%%%%%%%%%%%%%%%%%%%%%%%%%%%%%%%%%%%%%%%%%%%%%%%%%%%%%%%%%%%%%%%%%%%%%%%%%
%%%%%%%%%%%%%%%%%%%%%%%%%%%%%%%%%%%%%%%%%%%%%%%%%%%%%%%%%%%%%%%%%%%%%%%%%%%%%%%%%%%%%%%%%%


\textbf{Problem 10.} $(X,Y)$ jointly normal with
\[E[X]=0,\ E[Y]=2,\ \mathrm{Var}(X)=1,\ \mathrm{Var}(Y)=4,\ \mathrm{Cov}(X,Y)=1.\]
	\begin{enumerate}
	   	\item[(a)] Write the joint pdf.
	    \item[(b)] Find the marginals.
	    \item[(c)] $X\mid Y=y$.
	    \item[(d)] Compute $P(X>0,Y>2)$.
	\end{enumerate}


\ifnum\printsol=0
%\hrulefill
\vskip 5mm
	\else
\vskip 2mm
	\begin{boxsolution}
\vspace*{2mm}
Region $x>0,\,0<y<1/x$:
\[
c^{-1}=\int_0^\infty \int_0^{1/x} e^{-(x+y)} dy\,dx 
= \int_0^\infty e^{-x}(1-e^{-1/x}) dx.
\]
This integral is convergent.  
Then marginals follow by integrating.
\vspace*{2mm}
	\end{boxsolution}

\fi


%%%%%%%%%%%%%%%%%%%%%%%%%%%%%%%%%%%%%%%%%%%%%%%%%%%%%%%%%%%%%%%%%%%%%%%%%%%%%%%%%%%%%%%%%%
%%%%%%%%%%%%%%%%%%%%%%%%%%%%%%%%%%%%%%%%%%%%%%%%%%%%%%%%%%%%%%%%%%%%%%%%%%%%%%%%%%%%%%%%%%


\section*{C. Mixed Discrete–Continuous}

\textbf{Problem 11.} $N\sim\mathrm{Poisson}(\lambda)$ and $X|N=n\sim\mathrm{Gamma}(n+1,1)$.
	\begin{enumerate}
	    \item[(a)] Find the joint distribution.
	    \item[(b)] Find the marginal of $X$.
	    \item[(c)] Compute $P(N=n\mid X=x)$.
	\end{enumerate}


\ifnum\printsol=0
\hrulefill
\vskip 5mm
	\else
\vskip 2mm
	\begin{boxsolution}
\vspace*{2mm}
Geometric region area $=1/2$.  
\[
c^{-1}=\int_0^1\int_0^{1-x} 6xy\,dy\,dx =1.
\]

\[
P(X+Y<1/2)=\int_0^{1/2}\int_0^{1/2 - x} 6xy\,dy\,dx.
\]
\vspace*{2mm}
	\end{boxsolution}

\fi


%%%%%%%%%%%%%%%%%%%%%%%%%%%%%%%%%%%%%%%%%%%%%%%%%%%%%%%%%%%%%%%%%%%%%%%%%%%%%%%%%%%%%%%%%%
%%%%%%%%%%%%%%%%%%%%%%%%%%%%%%%%%%%%%%%%%%%%%%%%%%%%%%%%%%%%%%%%%%%%%%%%%%%%%%%%%%%%%%%%%%

\textbf{Problem 12.} With prob.\ $1/2$: $(X,Y)=(0,U)$, $U\sim\mathrm{Unif}(0,2)$.  
With prob.\ $1/2$: $(X,Y)=(V,0)$, $V\sim\mathrm{Unif}(0,2)$.
	\begin{enumerate}
	    \item[(a)] Find the joint distribution.
	    \item[(b)] Find both marginals.
	    \item[(c)] Compute $P(X=0\text{ or }Y=0)$.
	\end{enumerate}


\ifnum\printsol=0
%\hrulefill
\vskip 5mm
	\else
\vskip 2mm
	\begin{boxsolution}
\vspace*{2mm}
Points are $(1,1),(1,2),(1,3),(2,1),(2,2)$.  
\[
c^{-1}= 1+2+3+2+4=12.
\]
Then compute marginals and the event $|X-Y|\le 1$ by summation.
\vspace*{2mm}
	\end{boxsolution}

\fi


%%%%%%%%%%%%%%%%%%%%%%%%%%%%%%%%%%%%%%%%%%%%%%%%%%%%%%%%%%%%%%%%%%%%%%%%%%%%%%%%%%%%%%%%%%
%%%%%%%%%%%%%%%%%%%%%%%%%%%%%%%%%%%%%%%%%%%%%%%%%%%%%%%%%%%%%%%%%%%%%%%%%%%%%%%%%%%%%%%%%%


\section*{D. Additional Problems (Double Integrals / Non-Rectangular Regions)}

\textbf{Problem 13.} Let $f_{_{X,Y}}(x,y)=k(x^2+y^2)$ on the quarter disk
\[x\ge0,\ y\ge0,\ x^2+y^2\le4.\]
	\begin{enumerate}
	    \item[(a)] Find $k$.
	    \item[(b)] Find $f_{_X}(x)$.
	    \item[(c)] Compute $P(X+Y<1)$.
	    \item[da)] Find $f_{_{Y\mid X}}(y\mid x)$.
	\end{enumerate}


\ifnum\printsol=0
\hrulefill
\vskip 5mm
	\else
\vskip 2mm
	\begin{boxsolution}
\vspace*{2mm}
Region is quarter circle radius $2$.

\[
k^{-1}=\int_0^2\int_0^{\sqrt{4-x^2}} (x^2+y^2) dy\,dx.
\]
Convert to polar:
\[
k^{-1}=\int_0^{\pi/2}\int_0^2 r^2 r\,dr\,d\theta = \frac{\pi}{8}(2^4)=2\pi.
\]
Thus
\[
k=\frac{1}{2\pi}.
\]

Marginal:
\[
f_X(x)= \int_0^{\sqrt{4-x^2}} \tfrac{1}{2\pi}(x^2+y^2)\,dy.
\]

\[
\mathbb{P}(X+Y<1)= \int_0^1 \int_0^{1-x} \tfrac{1}{2\pi}(x^2+y^2)\,dy\,dx.
\]

Conditional:  
\[
f_{Y\mid X}(y\mid x)=\frac{x^2+y^2}{\displaystyle \int_0^{\sqrt{4-x^2}} (x^2+y^2)dy }.
\]
\vspace*{2mm}
	\end{boxsolution}

\fi


%%%%%%%%%%%%%%%%%%%%%%%%%%%%%%%%%%%%%%%%%%%%%%%%%%%%%%%%%%%%%%%%%%%%%%%%%%%%%%%%%%%%%%%%%%
%%%%%%%%%%%%%%%%%%%%%%%%%%%%%%%%%%%%%%%%%%%%%%%%%%%%%%%%%%%%%%%%%%%%%%%%%%%%%%%%%%%%%%%%%%

\textbf{Problem 14.} Let $f(x,y)=c(2x+3y)$ on the triangle
\[0<x<2,\quad 0<y<1-x/2.\]
	\begin{enumerate}
	    \item[(a)] Find $c$.
	    \item[(b)] Find $f_{_Y}(y)$.
	    \item[(c)] Compute $P(Y<X/2)$.
	    \item[(d)] Find $f_{_{X\mid Y}}(x\mid y)$.
	\end{enumerate}

\ifnum\printsol=0
\hrulefill
\vskip 5mm
	\else
\vskip 2mm
	\begin{boxsolution}
\vspace*{2mm}
Triangle $0<x<2$, $0<y<1-x/2$.

\[
c^{-1}=\int_0^2\int_0^{1-x/2}(2x+3y)\,dy\,dx.
\]

Compute inside:
\[
\int_0^{1-x/2}(2x+3y)dy
=2x(1-x/2)+\tfrac32(1-x/2)^2.
\]

Integrate over $x$ on $(0,2)$ to obtain $c$.

Marginal:
\[
f_Y(y)=\int_{x=0}^{2-2y} c(2x+3y)\,dx.
\]

\[
\mathbb{P}(Y<X/2)=\int_0^2\int_0^{\min(1-x/2, x/2)} c(2x+3y)\,dy\,dx.
\]

Conditional:
\[
f_{X\mid Y}(x\mid y)=\frac{c(2x+3y)}{\int_0^{2-2y} c(2u+3y)\,du}.
\]
\vspace*{2mm}
	\end{boxsolution}

\fi


%%%%%%%%%%%%%%%%%%%%%%%%%%%%%%%%%%%%%%%%%%%%%%%%%%%%%%%%%%%%%%%%%%%%%%%%%%%%%%%%%%%%%%%%%%
%%%%%%%%%%%%%%%%%%%%%%%%%%%%%%%%%%%%%%%%%%%%%%%%%%%%%%%%%%%%%%%%%%%%%%%%%%%%%%%%%%%%%%%%%%

\textbf{Problem 15.} Let $p(x,y)=c(x+y)$ for integer lattice points
\[x\ge1,\ y\ge1,\ x+y\le6.\]
	\begin{enumerate}
	    \item[(a)] Find $c$.
	    \item[(b)] Find $p_{_Y}(y)$.
	    \item[(c)] Compute $P(X>Y)$.
	    \item[(d)] Are $X,Y$ independent?
	\end{enumerate}

\ifnum\printsol=0
\hrulefill
\vskip 5mm
	\else
\vskip 2mm
	\begin{boxsolution}
\vspace*{2mm}
Support: integers $x,y\ge 1$, $x+y\le 6$.

Compute
\[
c^{-1}= \sum_{x=1}^5 \sum_{y=1}^{6-x} (x+y)
= \sum_{s=2}^6 \sum_{x=1}^{s-1} s
= \sum_{s=2}^6 s(s-1)
= 70.
\]
Thus
\[
c=\frac{1}{70}.
\]

\[
p_Y(y)=\sum_{x=1}^{5-y} \frac{x+y}{70}.
\]

\[
\mathbb{P}(X>Y)
= \sum_{x>y} \frac{x+y}{70}.
\]

Not independent: support is triangular.
\vspace*{2mm}
	\end{boxsolution}

\fi


%%%%%%%%%%%%%%%%%%%%%%%%%%%%%%%%%%%%%%%%%%%%%%%%%%%%%%%%%%%%%%%%%%%%%%%%%%%%%%%%%%%%%%%%%%
%%%%%%%%%%%%%%%%%%%%%%%%%%%%%%%%%%%%%%%%%%%%%%%%%%%%%%%%%%%%%%%%%%%%%%%%%%%%%%%%%%%%%%%%%%


\textbf{Problem 16.} Suppose $f_{_{X,Y}}(x,y)=\lambda^2 e^{-\lambda(x+2y)}$ on the wedge
\[0<x,\ 0<y<x.\]
	\begin{enumerate}
	    \item[(a)] Normalize the pdf.
	    \item[(b)] Compute $f_{_X}(x)$.
	    \item[(c)] Compute $P(Y<X/3)$.
	    \item[(d)] Compute $E[Y]$.
	\end{enumerate}


\ifnum\printsol=0
\hrulefill
\vskip 5mm
	\else
\vskip 2mm
	\begin{boxsolution}
\vspace*{2mm}
Region $0<y<x<\infty$.

Check normalization:
\[
\int_0^\infty\int_0^x \lambda^2 e^{-\lambda(x+2y)} dy\,dx
= \int_0^\infty \lambda^2 e^{-\lambda x}
\left[\frac12 e^{-\lambda x}-\frac12 e^{-2\lambda x}\right] dx.
\]

Simplify:
\[
=\frac{\lambda^2}{2}\int_0^\infty \left(e^{-2\lambda x} - e^{-3\lambda x}\right)dx
=\frac{\lambda^2}{2}\left(\frac{1}{2\lambda}-\frac{1}{3\lambda}\right)
=\frac{1}{12}.
\]

Thus multiply by 12 to normalize; true pdf is
\[
12\lambda^2 e^{-\lambda(x+2y)}.
\]

Marginal:
\[
f_X(x)=\int_0^x 12\lambda^2 e^{-\lambda(x+2y)} dy
=12\lambda^2 e^{-\lambda x}\frac{1-e^{-2\lambda x}}{2\lambda}.
\]

\[
\mathbb{P}(Y<X/3)
=\int_0^\infty \int_0^{x/3} 12\lambda^2 e^{-\lambda(x+2y)} dy\,dx.
\]

\[
E[Y]=\int_0^\infty\int_0^x y\cdot 12\lambda^2 e^{-\lambda(x+2y)} dy\,dx.
\]
\vspace*{2mm}
	\end{boxsolution}

\fi


%%%%%%%%%%%%%%%%%%%%%%%%%%%%%%%%%%%%%%%%%%%%%%%%%%%%%%%%%%%%%%%%%%%%%%%%%%%%%%%%%%%%%%%%%%
%%%%%%%%%%%%%%%%%%%%%%%%%%%%%%%%%%%%%%%%%%%%%%%%%%%%%%%%%%%%%%%%%%%%%%%%%%%%%%%%%%%%%%%%%%



\quescommentcorrection


%%%%%%%%%%%%%%%%%%%%%%%%%%%%%%%%%%%%%%%%%%%%%%%%%%%%%%%%%%%%%%%%%
%%%%%%%%%%%%%%%%%%%%%%%%%%%%%%%%%%%%%%%%%%%%%%%%%%%%%%%%%%%%%%%%%


%\includepdf[pages=-]{Ztable.pdf}

%%%%%%%%%%%%%%%%%%%%%%%%%%%%%%%%%%%%%%%%%%%%%%%%%%%%%%%%%%%%%%%%%
%%%%%%%%%%%%%%%%%%%%%%%%%%%%%%%%%%%%%%%%%%%%%%%%%%%%%%%%%%%%%%%%%

\end{document}
