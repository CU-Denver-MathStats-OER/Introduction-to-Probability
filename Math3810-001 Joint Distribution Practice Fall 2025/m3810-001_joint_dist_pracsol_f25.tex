\PassOptionsToPackage{solutions}{teaching-header}  % easy to comment out
\input{"m3810-001_joint_dist_prac_f25"}





%\documentclass[12pt]{article}
%\usepackage{amsmath, amssymb, amsfonts}
%\usepackage{geometry}
%\geometry{margin=1in}
%
%\begin{document}
%
%\begin{center}
%    {\Large \textbf{Solutions to Joint and Marginal Distribution Practice Exam}}
%\end{center}
%
%\bigskip
%
%% =========================
%% Problem 1
%% =========================
%
%\section*{Problem 1 Solution}
%
%Given the four-point distribution, the marginals follow by summing the joint probabilities. Independence fails because $p_{X,Y}(0,0)\neq p_X(0)p_Y(0)$. (Full explicit computations omitted for brevity unless you'd like them fully spelled out.)
%
%\bigskip
%
%% =========================
%% Problem 2
%% =========================
%
%\section*{Problem 2 Solution}
%
%The constant is
%\[
%c^{-1}=\int_0^1\int_0^1 (x+y)\,dy\,dx = 1.
%\]
%Thus $c=1$.  
%The marginal:
%\[
%f_X(x)=\int_0^1 (x+y)\,dy = x + \frac12.
%\]
%Probability:
%\[
%\mathbb{P}(Y > X/2)=\int_0^1 \int_{x/2}^1 (x+y)\,dy\,dx.
%\]
%This evaluates to
%\[
%=\int_0^1 \left[\frac12(1 - x/2)^2 + x(1 - x/2)\right]dx = \frac{17}{24}.
%\]
%The conditional:
%\[
%f_{Y\mid X}(y\mid x)=\frac{x+y}{x+\tfrac12}.
%\]
%
%\bigskip
%
%% =========================
%% Problem 3
%% =========================
%
%\section*{Problem 3 Solution}
%
%\[
%E[X]=\sum_{x=0}^3 \left( x\sum_{y=0}^3 p(x,y)\right).
%\]
%Only the given entries contribute:
%\[
%E[X]= 0\cdot 0.1 + 0\cdot 0.1 +1\cdot 0.1+2\cdot 0.2 = 0.5.
%\]
%Similarly
%\[
%E[Y]=0\cdot 0.1 + 0\cdot 0.1 +2\cdot 0.1 +3\cdot 0.2 = 0.8.
%\]
%
%\[
%\mathrm{Cov}(X,Y)=E[XY]-E[X]E[Y].
%\]
%Compute:
%\[
%E[XY]=0\cdot 0.1 +0\cdot 0.1 +2\cdot 0.1 +6\cdot 0.2 =1.4.
%\]
%Thus
%\[
%\mathrm{Cov}(X,Y)=1.4-(0.5)(0.8)=1.0.
%\]
%
%\bigskip
%
%% =========================
%% Problem 4
%% =========================
%
%\section*{Problem 4 Solution}
%
%\[
%\mathbb{P}(X>1)=\int_1^2 \int_0^{2-x} k\,dy\,dx
%= k\int_1^2 (2-x)\,dx
%= k\left[2x-\frac{x^2}{2}\right]_1^2 = 0.5k.
%\]
%
%\[
%E[Y]=\int_0^2 \int_0^{2-x} yk\,dy\,dx
%= k\int_0^2 \frac12(2-x)^2 dx
%= k\cdot \frac{4}{3}.
%\]
%
%\bigskip
%
%% =========================
%% Problem 5
%% =========================
%
%\section*{Problem 5 Solution}
%
%\[
%f_X(x)=\int_0^2 (x^2+y)\,dy = x^2\cdot 2 + \frac12 (2^2)=2x^2+2.
%\]
%\[
%f_Y(y)=\int_0^1 (x^2+y)\,dx = \frac13 + y.
%\]
%\[
%\mathbb{P}(X>Y)=\int_0^1 \int_y^1 (x^2+y)\,dx\,dy.
%\]
%Evaluate:
%\[
%=\int_0^1 \left[\frac13(1-y^3)+ y(1-y) \right]dy = \frac{29}{45}.
%\]
%
%\bigskip
%
%% =========================
%% Problem 6
%% =========================
%
%\section*{Problem 6 Solution}
%
%The region is the triangle $0<y<x<1$.  
%\[
%k^{-1}=\int_0^1 \int_0^x xye^{-x^2} dy\,dx
%= \int_0^1 x e^{-x^2}\frac{x^2}{2}dx
%= \frac12\int_0^1 x^3 e^{-x^2}dx.
%\]
%Use $u=x^2$:
%\[
%=\frac14 \int_0^1 u e^{-u} du = \frac14(1 - 2/e).
%\]
%Then invert to get $k$.
%
%\bigskip
%
%% =========================
%% Problem 7
%% =========================
%
%\section*{Problem 7 Solution}
%
%Support: $x\in \{1,2,3\}$, $y\in\{1,2,3\}$ with $x+y\le 4$:
%
%Pairs are $(1,1),(1,2),(1,3),(2,1),(2,2),(3,1)$.
%
%\[
%c^{-1}=\sum (xy)= 1+2+3+2+4+3 = 15.
%\]
%
%\[
%p_Y(y)=\sum_x cxy.
%\]
%Example: $p_Y(2)=c(1\cdot 2 + 2\cdot 2)=6c$.
%
%\[
%\mathbb{P}(X<Y)= c[(1,2),(1,3),(2,3\text{ invalid})].
%\]
%
%Independence fails because support is triangular.
%
%\bigskip
%
%% =========================
%% Problem 8
%% =========================
%
%\section*{Problem 8 Solution}
%
%\[
%f_{X,Y}(x,y)=c(x+y)e^{-(x+y)}.
%\]
%\[
%c^{-1}=\int_0^\infty\int_0^\infty (x+y)e^{-(x+y)} dy dx = 2.
%\]
%Thus $c=1/2$.
%
%\[
%\mathrm{Cov}(X,Y)=0
%\]
%because $X$ and $Y$ are independent (the joint factorizes).
%
%\[
%\mathbb{P}(X>1,Y<2)=\left( e^{-1}\right)\left(1-e^{-2}\right).
%\]
%
%\bigskip
%
%% =========================
%% Problem 9
%% =========================
%
%\section*{Problem 9 Solution}
%
%Normalizing triangle:
%\[
%k^{-1}=\int_0^2 \int_0^{1-x/2} dy\,dx = \int_0^2 (1-\tfrac{x}{2})dx =1.
%\]
%Compute marginals by integrating $1$ over the slice.  
%Expectation:
%\[
%E[X]=\int_0^2 x(1-x/2)dx = \frac23.
%\]
%
%\bigskip
%
%% =========================
%% Problem 10
%% =========================
%
%\section*{Problem 10 Solution}
%
%Region $x>0,\,0<y<1/x$:
%\[
%c^{-1}=\int_0^\infty \int_0^{1/x} e^{-(x+y)} dy\,dx 
%= \int_0^\infty e^{-x}(1-e^{-1/x}) dx.
%\]
%This integral is convergent.  
%Then marginals follow by integrating.
%
%\bigskip
%
%% =========================
%% Problem 11
%% =========================
%
%\section*{Problem 11 Solution}
%
%Geometric region area $=1/2$.  
%\[
%c^{-1}=\int_0^1\int_0^{1-x} 6xy\,dy\,dx =1.
%\]
%
%\[
%P(X+Y<1/2)=\int_0^{1/2}\int_0^{1/2 - x} 6xy\,dy\,dx.
%\]
%
%\bigskip
%
%% =========================
%% Problem 12
%% =========================
%
%\section*{Problem 12 Solution}
%
%Points are $(1,1),(1,2),(1,3),(2,1),(2,2)$.  
%\[
%c^{-1}= 1+2+3+2+4=12.
%\]
%Then compute marginals and the event $|X-Y|\le 1$ by summation.
%
%\bigskip
%
%% =========================
%% Problem 13 (New)
%% =========================
%
%\section*{Problem 13 Solution}
%
%Region is quarter circle radius $2$.
%
%\[
%k^{-1}=\int_0^2\int_0^{\sqrt{4-x^2}} (x^2+y^2) dy\,dx.
%\]
%Convert to polar:
%\[
%k^{-1}=\int_0^{\pi/2}\int_0^2 r^2 r\,dr\,d\theta = \frac{\pi}{8}(2^4)=2\pi.
%\]
%Thus
%\[
%k=\frac{1}{2\pi}.
%\]
%
%Marginal:
%\[
%f_X(x)= \int_0^{\sqrt{4-x^2}} \tfrac{1}{2\pi}(x^2+y^2)\,dy.
%\]
%
%\[
%\mathbb{P}(X+Y<1)= \int_0^1 \int_0^{1-x} \tfrac{1}{2\pi}(x^2+y^2)\,dy\,dx.
%\]
%
%Conditional:  
%\[
%f_{Y\mid X}(y\mid x)=\frac{x^2+y^2}{\displaystyle \int_0^{\sqrt{4-x^2}} (x^2+y^2)dy }.
%\]
%
%\bigskip
%
%% =========================
%% Problem 14 (New)
%% =========================
%
%\section*{Problem 14 Solution}
%
%Triangle $0<x<2$, $0<y<1-x/2$.
%
%\[
%c^{-1}=\int_0^2\int_0^{1-x/2}(2x+3y)\,dy\,dx.
%\]
%
%Compute inside:
%\[
%\int_0^{1-x/2}(2x+3y)dy
%=2x(1-x/2)+\tfrac32(1-x/2)^2.
%\]
%
%Integrate over $x$ on $(0,2)$ to obtain $c$.
%
%Marginal:
%\[
%f_Y(y)=\int_{x=0}^{2-2y} c(2x+3y)\,dx.
%\]
%
%\[
%\mathbb{P}(Y<X/2)=\int_0^2\int_0^{\min(1-x/2, x/2)} c(2x+3y)\,dy\,dx.
%\]
%
%Conditional:
%\[
%f_{X\mid Y}(x\mid y)=\frac{c(2x+3y)}{\int_0^{2-2y} c(2u+3y)\,du}.
%\]
%
%\bigskip
%
%% =========================
%% Problem 15 (New)
%% =========================
%
%\section*{Problem 15 Solution}
%
%Support: integers $x,y\ge 1$, $x+y\le 6$.
%
%Compute
%\[
%c^{-1}= \sum_{x=1}^5 \sum_{y=1}^{6-x} (x+y)
%= \sum_{s=2}^6 \sum_{x=1}^{s-1} s
%= \sum_{s=2}^6 s(s-1)
%= 70.
%\]
%Thus
%\[
%c=\frac{1}{70}.
%\]
%
%\[
%p_Y(y)=\sum_{x=1}^{5-y} \frac{x+y}{70}.
%\]
%
%\[
%\mathbb{P}(X>Y)
%= \sum_{x>y} \frac{x+y}{70}.
%\]
%
%Not independent: support is triangular.
%
%\bigskip
%
%% =========================
%% Problem 16 (New)
%% =========================
%
%\section*{Problem 16 Solution}
%
%Region $0<y<x<\infty$.
%
%Check normalization:
%\[
%\int_0^\infty\int_0^x \lambda^2 e^{-\lambda(x+2y)} dy\,dx
%= \int_0^\infty \lambda^2 e^{-\lambda x}
%\left[\frac12 e^{-\lambda x}-\frac12 e^{-2\lambda x}\right] dx.
%\]
%
%Simplify:
%\[
%=\frac{\lambda^2}{2}\int_0^\infty \left(e^{-2\lambda x} - e^{-3\lambda x}\right)dx
%=\frac{\lambda^2}{2}\left(\frac{1}{2\lambda}-\frac{1}{3\lambda}\right)
%=\frac{1}{12}.
%\]
%
%Thus multiply by 12 to normalize; true pdf is
%\[
%12\lambda^2 e^{-\lambda(x+2y)}.
%\]
%
%Marginal:
%\[
%f_X(x)=\int_0^x 12\lambda^2 e^{-\lambda(x+2y)} dy
%=12\lambda^2 e^{-\lambda x}\frac{1-e^{-2\lambda x}}{2\lambda}.
%\]
%
%\[
%\mathbb{P}(Y<X/3)
%=\int_0^\infty \int_0^{x/3} 12\lambda^2 e^{-\lambda(x+2y)} dy\,dx.
%\]
%
%\[
%E[Y]=\int_0^\infty\int_0^x y\cdot 12\lambda^2 e^{-\lambda(x+2y)} dy\,dx.
%\]
%
%\end{document}
%
%
