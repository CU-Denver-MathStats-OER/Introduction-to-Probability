%%%%%%%%%%%%%%%%%%%%%%%%%%%%%%%%%%%%%%%%%%%%%%%%%%%%%%
\documentclass[11pt]{article}
%%%%%%%%%%%%%%%%%%%%%%%%%%%%%%%%%%%%%%%%%%%%%%%%%%%%%%

\usepackage{amsmath}
\usepackage{amsthm}
\usepackage{amssymb}
\usepackage{latexsym}
\usepackage{graphicx}
\usepackage{color}
\usepackage{verbatim}
\usepackage{float}
\usepackage{multicol}
\usepackage{xcolor}
\usepackage{listings}
\usepackage{tikz}
\usetikzlibrary{arrows.meta, positioning, calc}
\usetikzlibrary{decorations.pathmorphing}
\usepackage{tcolorbox}
\tcbuselibrary{breakable}
\usepackage{cancel}


\newtcolorbox{solutionbox}{
  breakable,
  colback=blue!5!white,
  colframe=blue!50!black,
  title=Solution,
  sharp corners,
  boxrule=0.8pt
}

\newtcolorbox{hintbox}{
  breakable,
  colback=gray!10!white,
  colframe=gray!50!black,
  title=Hint,
  sharp corners,
  boxrule=0.5pt
}

% Unnumbered theorem
\newtheorem*{thm*}{Theorem}

\lstdefinelanguage{R}{
      keywords={if,else,while,for,in,next,break,function,TRUE,FALSE,NULL,Inf,NA,NaN,switch,repeat,return,require,library},
      keywordstyle=\color{blue}\bfseries,
      identifierstyle=\color{black},
      comment=[l]{\#},
      commentstyle=\color{gray}\ttfamily,
      string=[b]{"},
      stringstyle=\color{red}\ttfamily,
      morecomment=[l]{//},
      morestring=[b]{'},
      sensitive=true,
      morekeywords={print,summary,plot,lm,glm,data,frame,read.csv,write.csv,factor,levels,names,colnames,rownames,
      head,tail,str,dim,length,class,typeof,mode,is.na,is.null,is.finite,is.infinite,is.nan,as.numeric,as.character,
      as.factor,as.Date,as.POSIXct,as.matrix,as.data.frame,rbind,cbind,merge,subset,aggregate,tapply,apply,lapply,sapply,
      mapply,vapply,replicate,seq,rep,c,list,matrix,array,data.frame,table,hist,boxplot,barplot,pie,curve,lines,points,text,
      abline,legend,par,mtext,title,xlab,ylab,xlim,ylim,main,sub,col,pch,cex,lty,lwd,type,bg,fg,args,options,warnings,errors,
      message,stop,warning,error,try,tryCatch,withCallingHandlers,on.exit,debug,browser,trace,recover,options,getOption,setOption},
    }


\setlength{\textheight}{9in}
\setlength{\textwidth}{6in}
\addtolength{\topmargin}{-2cm}
\addtolength{\oddsidemargin}{-1cm}
\parindent=0in


\def\classnum{3810}
\def\classtitle{Probability}
\def\classtitleshort{Probability}
\def\classsec{001}
\def\classterm{Fall 2025}
\def\instructor{Robert Rostermundt}
%\def\hmwknum{\#2}


%%%%%%%%%%%%%%%%%%%%%%%%%%%%%%%%%%%%%%%%%%%%%%%%%%%%%%%%%
%%%%%%%%%%%%%%%%%%%%%%%%%  Colors  %%%%%%%%%%%%%%%%%%%%%%
%%%%%%%%%%%%%%%%%%%%%%%%%%%%%%%%%%%%%%%%%%%%%%%%%%%%%%%%%

\definecolor{Green}{rgb}{0,.5,0}
%use for definitions
\definecolor{Red}{rgb}{.8,.2,0}
%use for emphasis
\definecolor{Yellow}{rgb}{.6,.6,.1}
%use for part titles
\definecolor{Cyan}{rgb}{.2,.6,.7}
%use for comments
\definecolor{Purple}{rgb}{.4,0,1}
%use for examples
\definecolor{deepred}{rgb}{.53,.29,.24}
%use for important points
\definecolor{Black}{rgb}{0,0,0}
%use for washout
\definecolor{Grey}{rgb}{.45,.45,.45}
% use for theorems
\newcommand{\tred}[1]{\textcolor{Red}{#1}}
\newcommand{\tgreen}[1]{\textcolor{Green}{#1}}
\newcommand{\tcyan}[1]{\textcolor{Cyan}{#1}}
\newcommand{\tyellow}[1]{\textcolor{Yellow}{#1}}
\newcommand{\tpurple}[1]{\textcolor{Purple}{#1}}
\newcommand{\tblack}[1]{\textcolor{Black}{#1}}
\newcommand{\tgrey}[1]{\textcolor{Grey}{#1}}
\newcommand{\tdeepred}[1]{\textcolor{deepred}{#1}}
\newcommand{\ttt}[1]{\texttt{#1}}

%%%%%%%%%%%%%%%%%%%%%%%%%%%%%%%%%%%%%%%%%%%%%%%%%%%%%%%%%
%%%%%%%%%%%%%%%%%%%%%%%%%  Theorem Environments  %%%%%%%%
%%%%%%%%%%%%%%%%%%%%%%%%%%%%%%%%%%%%%%%%%%%%%%%%%%%%%%%%%

\theoremstyle{plain}
\newtheorem{thm}{Theorem}
\newtheorem{axiom}{Axiom}
\newtheorem{cor}{Corollary}
\newtheorem{lemma}{Lemma}
\newtheorem{prop}{Proposition}
\newtheorem{ques}{Question}
\theoremstyle{definition}
\newtheorem{defn}{Definition}
\theoremstyle{remark}
\newtheorem{remark}{Remark}
\theoremstyle{definition}
\newtheorem{ex}{Example}
\numberwithin{equation}{section}
\newtheorem{prob}{Problem}
\numberwithin{equation}{section}


%%%%%%%%%%%%%%%%%%%%%%%%%%%%%%%%%%%%%%%%%%%%%%%%%%%%%%%%%
%%%%%%%%%%%%%%%%%%%%%%%%%  Math    %%%%%%%%%%%%%%%%%%%%%%
%%%%%%%%%%%%%%%%%%%%%%%%%%%%%%%%%%%%%%%%%%%%%%%%%%%%%%%%%


\newcommand{\abs}[1]{\left\lvert{#1}\right\rvert}
\newcommand{\card}[1]{\lvert{#1}\rvert}
\newcommand{\union}{\cup}
\newcommand{\Union}{\bigcup}
\newcommand{\inter}{\cap}
\newcommand{\Inter}{\bigcap}
%\newcommand{\hint}[1]{\medskip\newline\emph{Hint: #1}}
%\newcommand{\note}[1]{\medskip\newline\emph{Note: #1}}
\newcommand{\points}[1]{[#1 points]}
\newcommand{\totalpoints}[1]{[#1 points total]}
\newcommand{\ds}{\displaystyle}
\newcommand{\ben}{\begin{enumerate}}
\newcommand{\een}{\end{enumerate}}
\newcommand{\bi}{\begin{itemize}}
\newcommand{\ei}{\end{itemize}}
\newcommand{\beq}{\begin{eqnarray*}}
\newcommand{\eeq}{\end{eqnarray*}}
\newcommand{\bieq}{\begin{IEEEeqnarray}{rCl}}
\newcommand{\bieqx}{\begin{IEEEeqnarray}{+rCl+x*}}
\newcommand{\eieq}{\end{IEEEeqnarray}}
\newcommand{\nn}{\nonumber}
%\renewcommand{\i}{\item}
\newcommand{\bpm}{\begin{pmatrix}}
\newcommand{\epm}{\end{pmatrix}}
\newcommand{\sol}{\indent{\bf\emph{Solution:}}}
\newcommand{\ssol}{\indent{\\[2mm]\bf\emph{Solution:}}\;}
\newcommand{\hint}{\indent{\bf\emph{Hint}:}\;}
\newcommand{\note}{\indent{\bf\emph{Note}:}\;}
\newcommand{\vsk}{\vskip 2mm}
%%%%%%%%%%%%%%%%%%%%%%%%% Calculus %%%%%%%%%%%%%%%%%%%%%%%%%%%%
\newcommand{\dd}[2]{\ds\frac{d}{d{#1}}\left[{#2}\right]}
\newcommand{\der}[2]{\ds\frac{d{#1}}{d{#2}}}
\newcommand{\lmt}[3]{\ds\lim_{{#1}\to{#2}}{#3}}
\renewcommand{\iint}[2]{\ds\int{#1}\,d{#2}}
\newcommand{\dint}[4]{\ds\int^{#4}_{#3}{#1}\,d{#2}}
\renewcommand{\Delta}{\triangle}
%%%%%%%%%%%%%%%%%%%%%%%%% Number Sets %%%%%%%%%%%%%%%%%%%%%%%%%%
\newcommand{\N}{\mathbb{N}}
\newcommand{\Z}{\mathbb{Z}}
\newcommand{\Q}{\mathbb{Q}}
\newcommand{\R}{\mathbb{R}}
\newcommand{\C}{\mathbb{C}}
\newcommand{\F}{\mathcal{F}}
\renewcommand{\P}{\mathbb{P}}
\newcommand{\E}{\mathcal{E}}
\renewcommand{\o}{\omega}
\renewcommand{\O}{\Omega}
%%%%%%%%%%%%%%%%%%%%%%%%% Vectors %%%%%%%%%%%%%%%%%%%%%%%%%%%%%
\newcommand{\x}{\bar{x}}
\renewcommand{\v}{\bar{v}}
\newcommand{\y}{\bar{y}}
\newcommand{\z}{\bar{z}}
\newcommand{\w}{\bar{w}}
\renewcommand{\u}{\bar{u}}
\renewcommand{\b}{\bar{b}}
\newcommand{\e}{\bar{e}}
\renewcommand{\a}{\vec{a}}
\renewcommand{\r}{\vec{r}}
\newcommand{\vv}{\vec{v}}
\newcommand{\vecPQ}[2]{\overrightarrow{#1}{#2}}
\newcommand{\vecV}[1]{\overrightarrow{#1}}
\newcommand{\la}{\langle}
\newcommand{\ra}{\rangle}
%%%%%%%%%%%%%%%%%%%%%%%%%%% Vector Spaces %%%%%%%%%%%%%%%%%%%%
\newcommand{\rn}{\ensuremath{\mathbb{R}^n}}
\renewcommand{\rm}{\ensuremath{\mathbb{R}^m}}
\newcommand{\re}{\mathbb{R}}
\newcommand{\Pn}{\mathbb{P}_n}
\newcommand{\B}{\mathcal{B}}
%%%%%%%%%%%%%%%%%%%%%%%%%%% Graphics %%%%%%%%%%%%%%%%%%%%%%%%
\newcommand{\cg}[2]{\begin{center}
\includegraphics[scale={#1}]{{#2}}
\end{center}}
\makeatletter
\def\imod#1{\allowbreak\mkern10mu({\operator@font mod}\,\,#1)}
\makeatother

%%%%%%%%%%%%%%%%%%%%%%%%%%%%%%%%%%%%%%%%%%%%%%%%%%%%%%%%%%%%%%%%%%%%%%%%%%%%%%%%%%%%%%%%%%%%%%
%%%%%%%%%%%%%%%%%%%%%%%%%%%%%% Defined Fonts %%%%%%%%%%%%%%%%%%%%%%%%%%%%%%%%%%%%%%%%%%%%%%%%%
%%%%%%%%%%%%%%%%%%%%%%%%%%%%%%%%%%%%%%%%%%%%%%%%%%%%%%%%%%%%%%%%%%%%%%%%%%%%%%%%%%%%%%%%%%%%%%

\font\minihelv=phvr at 6pt
\font\helv=phvr at 10pt
\font\medhelv=phvr at 16pt
\font\bighelv=phvr at 20pt
\font\hugehelv=phvr at 36pt
\font\mybigfont=phvr at 16pt
\font\mymediumfont=phvr at 14pt
\font\mediumhelv=phvr at 14pt
\font\mybfit=ptmbi at 12pt


%%%%%%%%%%%%%%%%%%%%%%%%%%%%%%%%%%%%%%%%%%%%%%%%%%%%%%%%%%%%%%%%%%%%%%%%%%%%%%%%%%%%%%%%%%%%%%%
%%%%%%%%%%%%%%%%%%%%%%%%%%%%%% Other Commands %%%%%%%%%%%%%%%%%%%%%%%%%%%%%%%%%%%%%%%%%%%%%%%%%
%%%%%%%%%%%%%%%%%%%%%%%%%%%%%%%%%%%%%%%%%%%%%%%%%%%%%%%%%%%%%%%%%%%%%%%%%%%%%%%%%%%%%%%%%%%%%%%
%\setlength\fboxrule{.5pt}
%\newcommand{\latexpicborder}[3]{
%\setlength\fboxsep{30pt}
%\begin{figure}[hb]
%\begin{center}
%\fbox{
%\input{#1}
%}
%\caption{#2}
%\label{#3}
%\end{center}
%\end{figure}
%\setlength\fboxsep{0pt}
%}
%
%\newcommand{\latexpic}[2]{
%\begin{figure}[hb]
%\begin{center}
%\input{#1}
%\vspace*{8mm}
%\caption{#2}
%\end{center}
%\end{figure}
%}

%\begin{minipage}[b]{0.6\linewidth}
%......
%\end{minipage}
%\hspace{0.5cm}
%\begin{minipage}[t]{0.4\linewidth}
%\centering
%\includegraphics[scale=.5]{m1401_ex3_g4.eps}
%\end{minipage}
%\end{figure}


%%%%%%%%%%%%%%%%%%%%%%%%%%%%%%%%%%%%%%%%%%%%%%%%%%%%%%%%%%%%%%%%%%%%%%%%%%%%%%%%%%%%%%%%%%%%%%
%%%%%%%%%%%%%%%%%%%%%%%%%%% IEEEeqnarray Notes %%%%%%%%%%%%%%%%%%%%%%%%%%%%%%%%%%%%%%%%%%%%%%%
%%%%%%%%%%%%%%%%%%%%%%%%%%%%%%%%%%%%%%%%%%%%%%%%%%%%%%%%%%%%%%%%%%%%%%%%%%%%%%%%%%%%%%%%%%%%%%


%Any number of columns can be specified with IEEEeqnarray: {c} will give only one
%column with all entries centered, or {rCll} would add a fourth, left-justified
%column to use for comments. Moreover, beside l, c, r, L, C, R for math mode
%entries there are also s, t, u for left, centered, and right text mode entries.
%Additional space can be added with . and / and ? in increasing order.
%
%
%\begin{proof}
%This is a proof that ends
%with an equation array:
%\begin{IEEEeqnarray*}{+rCl+x*}
%a & = & b + c \\
%& = & d + e. & \qedhere
%\end{IEEEeqnarray*}
%\end{proof}
%Note that the + in {+rCl+x*} denotes stretchable spaces, one on the left
%of the equations (which, if not specified, will be done automatically by
%IEEEeqnarray!) and one on the right of the equations. But now on the right,
%after the stretching column, we add an empty column x. This column will be
%only needed on the last line when we will put the \qedhere command there.
%Finally, we specify a *. This is a null-space that prevents IEEEeqnarray to
%add another unwanted +-space!


% The following environments enable custom numbering of theorems so that the numbers agree % with the numbering in the textbook being used. 
%
%  Usage examples:
%\begin{customthm}{2.2}\label{eight}
%Every theorem must be numbered by hand.
%\end{customthm}
%
%Here is a reference to theorem~\ref{eight}.
%
%\begin{customthm}{2.3}[Parenthetical comment]\label{nine}
%Statement
%\end{customthm}
%
%Here is a reference to theorem~\ref{nine}


\newtheorem{innercustomthm}{Theorem}
\newenvironment{customthm}[1]
  {\renewcommand\theinnercustomthm{#1}\innercustomthm}
  {\endinnercustomthm}
  
  \newtheorem{innercustomprop}{Proposition}
\newenvironment{customprop}[1]
  {\renewcommand\theinnercustomprop{#1}\innercustomprop}
  {\endinnercustomprop}
  
    \newtheorem{innercustomlem}{Lemma}
\newenvironment{customlem}[1]
  {\renewcommand\theinnercustomlem{#1}\innercustomlem}
  {\endinnercustomlem}
  
    \newtheorem{innercustomconj}{Conjecture}
\newenvironment{customconj}[1]
  {\renewcommand\theinnercustomconj{#1}\innercustomconj}
  {\endinnercustomconj}
  
    \newtheorem{innercustomclaim}{Claim}
\newenvironment{customclaim}[1]
  {\renewcommand\theinnercustomclaim{#1}\innercustomclaim}
  {\endinnercustomclaim}
  
    \newtheorem{innercustomcor}{Corollary}
\newenvironment{customcor}[1]
  {\renewcommand\theinnercustomcor{#1}\innercustomcor}
  {\endinnercustomcor}
  
    \newtheorem{innercustomdef}{Definition}
\newenvironment{customdef}[1]
  {\renewcommand\theinnercustomdef{#1}\innercustomdef}
  {\endinnercustomdef}
  
    \newtheorem{innercustomex}{Example}
\newenvironment{customex}[1]
  {\renewcommand\theinnercustomex{#1}\innercustomex}
  {\endinnercustomex}
  
    \newtheorem{innercustomass}{Assumption}
\newenvironment{customass}[1]
  {\renewcommand\theinnercustomass{#1}\innercustomass}
  {\endinnercustomass}
  
      \newtheorem{innercustomax}{Axiom}
\newenvironment{customax}[1]
  {\renewcommand\theinnercustomax{#1}\innercustomax}
  {\endinnercustomax}
  

\vfuzz2pt % Don't report over-full v-boxes if over-edge is small
\hfuzz2pt % Don't report over-full h-boxes if over-edge is small

\renewcommand{\ni}{\noindent}


%%%%%%%%%%%%%%%%%%%%%%%%%%%%%%%%%%%%%%%%%%%%%%%%%%%%%%
%%%%%%%%%%%%%%%%%%%%%%%%%%%%%%%%%%%%%%%%%%%%%%%%%%%%%%

\pagestyle{myheadings}

%%%%%%%%%%%%%%%%%%%%%%%%%%%%%%%%%%%%%%%%%%%%%%%%%%%%%%

%%%%%%%%%%%%%%%%%%%%%%%%%%%%%%%%%%%%%%%%%%%%%%%%%%%%%%
%%%%%%%%%%%%%%%%%%%%%%%%%   Document Body   %%%%%%%%%%
%%%%%%%%%%%%%%%%%%%%%%%%%%%%%%%%%%%%%%%%%%%%%%%%%%%%%%

%\def\classnum{3810}
%\def\classtitle{Probability}
%\def\classtitleshort{Probability}
%\def\classsec{001}
%\def\classterm{Fall 2025}
%\def\instructor{Robert Rostermundt}
\def\printsol{0}


	\title{\vspace{-1in}Math\classnum\;-\;\classtitle\\
	Section\;\classsec\;-\;\classterm\\
	Notes: Expectation and Densities\\
	Through Conditioning}
	\author{University of Colorado Denver / College of Liberal Arts 	and Sciences}
	\date{Department of Mathematics - \instructor}

	\markright{Math\classnum\;-\;\classtitleshort, UCD, \classterm, \instructor}



%%%%%%%%%%%%%%%%%%%%%%%%%%%%%%%%%%%%%%%%%%%%%%%%%%%%%%
\begin{document}\maketitle\thispagestyle{empty}
%%%%%%%%%%%%%%%%%%%%%%%%%%%%%%%%%%%%%%%%%%%%%%%%%%%%%%


%%%%%%%%%%%%%%%%%%%%%%%%%%%%%%%%%%%%%%%%%%%%%%%%%%%%%%%%%%%%%%%%%%%%%%%%%%%%%%%%%%%%%%%%%%%%%%%%%%%%%%
\vspace*{2mm}
\hrule
\vskip 8mm


%%%%%%%%%%%%%%%%%%%%%%%%%%%%%%%%%%%%%%%%%%%%%%%%%%%%%%%%%%%%%%%%%%%%%%%%%
%%%%%%%%%%%%%%%%%%%%%%%%%%%%%%%%%%%%%%%%%%%%%%%%%%%%%%%%%%%%%%%%%%%%%%%%%


%%%%%%%%%%%%%%%%%%%%%%%%%%%%%%%%%%%%%%%%%%%%%%%%%%%%%%%%%%%%%%%%%%%%%%%%%%%%%%%%%%%%%%%%%%%%%%%%%%%%%%
\section*{The Problem:}
%%%%%%%%%%%%%%%%%%%%%%%%%%%%%%%%%%%%%%%%%%%%%%%%%%%%%%%%%%%%%%%%%%%%%%%%%%%%%%%%%%%%%%%%%%%%%%%%%%%%%%


\noindent The train leave the station every fifteen minutes starting on the hour. We arrive at the train station at a random time between 7:10am and 7:30am. What is our expected waiting time for the train?
\vskip 2mm
Let $W$ equal the waiting time until our train leaves the station and let $T$ be the time we arrive at the train station. Then $Im(W)=[0,15)$ with, as of yet, an unknown distribution. If we let $T=0$ correspond to 7:10am we see that $T\sim Uni(0,20)$. And so we are looking for $E[W]$. Our definition of expected value is
\[E[W]=\ds\int^{\infty}_{-\infty}w\cdot f_{_W}(w)\,dw.\]
However, at this point we do not know the distribution of $W$ and do not know it's density function $f_{_W}$.


%%%%%%%%%%%%%%%%%%%%%%%%%%%%%%%%%%%%%%%%%%%%%%%%%%%%%%%%%%%%%%%%%%%%%%%%%%%%%%%%%%%%%%%%%%%%%%%%%%%%%%
\section*{Theoretical Tools:}
%%%%%%%%%%%%%%%%%%%%%%%%%%%%%%%%%%%%%%%%%%%%%%%%%%%%%%%%%%%%%%%%%%%%%%%%%%%%%%%%%%%%%%%%%%%%%%%%%%%%%%

\noindent We will compute this expectation in two ways by conditioning on the following events:

	\begin{itemize}
		\item[] $A:=\big\{0<T\le 5\big\}$ which is the event that we arrive between 7:10am and 7:15am.
		\item[] $B:=\big\{5<T< 20\big\}$ which is the event that we arrive between 7:15am and 7:30am.
	\end{itemize} 

Then since $A$ and $B$ partition the sample space we can compute the expected value as
\[E[W]=\P(A)\cdot E\big[\,W\,|\,A\,\big]+\P(B)\cdot E\big[\,W\,|\,B\,\big].\]
Alternatively, we can describe the density for $W$ using conditional densities as
\[f_{_W}(w)=\P(A)\cdot f_{_{W|A}}(w)+\P(B)\cdot f_{_{W|B}}(w).\]
From here we can use integration to directly compute the expectation.

%%%%%%%%%%%%%%%%%%%%%%%%%%%%%%%%%%%%%%%%%%%%%%%%%%%%%%%%%%%%%%%%%%%%%%%%%%%%%%%%%%%%%%%%%%%%%%%%%%%%%%
\section*{Calculations:}
%%%%%%%%%%%%%%%%%%%%%%%%%%%%%%%%%%%%%%%%%%%%%%%%%%%%%%%%%%%%%%%%%%%%%%%%%%%%%%%%%%%%%%%%%%%%%%%%%%%%%%

\noindent We first make the easy calculations 
\beq
\P(A)&=&\P(0<T\le 5)=F_{_{T}}(5)=\ds\frac{5-0}{20-0}=\ds\frac{1}{4}\\
\P(B)&=&1-\P(A)=\ds\frac{3}{4}\\
\eeq
\paragraph{Method \#1:}
We need to find $E\big[\,W\,|\,A\,\big]$ and $E\big[\,W\,|\,B\,\big]$. To do this we will find the relationship between $T$ and $W$. In particular,
\[W=\left\{\begin{array}{ccl}
5-T&:&0<T\le 5\\
20-T&:&5<T<20\\
\end{array}\right.\]
Then
\[E\big[\,W\,|\,A\,\big]=E\big[\,5-T\,\big|\,0<T\le 5\,\big]=5-E\big[\,T\,\big|\,0<T\le 5\,\big]=5-\ds\frac{5}{2}.\]
To see this, we have the conditional distribution $T\,\big|\,0<T\le 5\sim Uni(0,5)$ and so $E\big[\,T\,|\,0<T\le 5\,\big]=5/2$. That gives us
\[E\big[\,W\,|\,A\,\big]=\ds\frac{5}{2}.\]
Similarly,
\[E\big[\,W\,|\,B\,\big]=E\big[\,20-T\,\big|\,5<T<20\,\big]=20-E\big[\,T\,\big|\,5<T<20\,\big]=20-\ds\frac{25}{2}.\]
To see this, we have the conditional distribution $T\,\big|\,5<T<20\sim Uni(5,20)$ and so $E\big[\,T\,|\,5<T<20\,\big]=25/2$. That gives us
\[E\big[\,W\,|\,A\,\big]=\ds\frac{15}{2}.\]
Putting the pieces together we have
\[E[W]=\underbrace{\P(A)}_{\text{\textcolor{red}{1/4}}}\cdot\underbrace{E\big[\,W\,|\,A\,\big]}_{\text{\textcolor{red}{5/2}}}+\underbrace{\P(B)}_{\text{\textcolor{red}{3/4}}}\cdot\underbrace{E\big[\,W\,|\,B\,\big]}_{\text{\textcolor{red}{15/2}}}=\ds\frac{50}{8}\]
So the expected waiting time is 6.25 minutes.
\vskip 1cm
\paragraph{Method \#2:}
Alternatively we compute the density function as
\[f_{_W}(w)=\underbrace{\P(A)}_{\text{\textcolor{red}{1/4}}}\cdot f_{_{W|A}}(w)+\underbrace{\P(B)}_{\text{\textcolor{red}{3/4}}}\cdot f_{_{W|B}}(w)\]
and then integrate. Letting $w(t)=5-t$, with the inverse relationship $t(w)=5-w$, we have from our change of variables formula that
\[f_{_{W|A}}(w)=f_{_{T|A}}\big(\,t(w)\,\big)\cdot\left|\ds\frac{dt}{dw}\right|=f_{_{T|A}}(5-w)\cdot|-1|=\ds\frac{1}{5},\;\text{when}\;0<w\le 5.\]
Similarly, letting $w(t)=20-t$, with the inverse relationship $t(w)=20-w$, we again have from our change of variables formula that
\[f_{_{W|B}}(w)=f_{_{T|B}}\big(\,t(w)\,\big)\cdot\left|\ds\frac{dt}{dw}\right|=f_{_{T|B}}(20-w)\cdot|-1|=\ds\frac{1}{15},\;\text{when}\;0<w\le 15.\]
Putting these pieces together, and seeing that $f_{_{W|A}}$ only has a non-zero contribution to $f_{_W}(w)$ when $0\le w\le 5$, we have
\[f_{_W}(w)=\underbrace{\P(A)}_{\text{\textcolor{red}{1/4}}}\cdot\underbrace{f_{_{W|A}}(w)}_{\stackrel{\text{\textcolor{red}{1/5 when}}}{\text{\textcolor{red}{$0\le w\le 5$}}}}+\underbrace{\P(B)}_{\text{\textcolor{red}{3/4}}}\cdot \underbrace{f_{_{W|B}}(w)}_{\stackrel{\text{\textcolor{red}{1/15 when}}}{\text{\textcolor{red}{$0\le w\le 15$}}}}\]
\vskip 5mm
%\vfill\eject

Finally we have the density function $f_{_W}$.
\vskip 5mm
\begin{multicols}{2}
\[f_{_W}(w)=\left\{\begin{array}{ccl}
\ds\frac{1}{10}&:&0\le w\le 5\\
\\
\ds\frac{1}{20}&:&5<w<15\\
\\
0&:&\text{Otherwise}.
\end{array}\right.\]
\columnbreak
\begin{center}
\includegraphics[scale=0.6]{density_of_w.jpg}
\end{center}
\end{multicols}
\vskip 5mm
A quick check verifies that
\vskip 1mm
\[\ds\int^{\infty}_{-\infty}f_{_W}(w)\,dw=1\]
\vskip 1mm
and $f_{_W}$ is a valid density function. We can now compute
\[E[W]=\ds\int^{\infty}_{-\infty}w\cdot f_{_W}(w)\,dw=\ds\int^{w=5}_{w=0}\ds\frac{w}{10}\,dw+\ds\int^{w=15}_{w=5}\ds\frac{w}{20}\,dw=\ds\frac{250}{40}=6.25\]
as we saw in our previous calculations.
\vskip 5mm

%%%%%%%%%%%%%%%%%%%%%%%%%%%%%%%%%%%%%%%%%%%%%%%%%%%%%%%%%%%%%%%%%%%%%%%%%%%%%%%%%%%%%%%%%%%%%%%%%%%%%%
\section*{Simulations:} 
%%%%%%%%%%%%%%%%%%%%%%%%%%%%%%%%%%%%%%%%%%%%%%%%%%%%%%%%%%%%%%%%%%%%%%%%%%%%%%%%%%%%%%%%%%%%%%%%%%%%%%
I have simulated this process in R. Below are graphics from four different simulations with number trials $n=50, 100, 1000, 100000$. The R code is provided after the graphics.
\vskip 1cm

\includegraphics[scale=0.48]{w-density_simulation_50.jpeg}\quad
\includegraphics[scale=0.48]{w-density_simulation_100.jpeg}\\
\vskip 5mm
Notice how the empirical mean differs noticeably from the theoretical mean with relatively small sample sizes $n=50$ and $n=100$. 
\vskip 5mm
\includegraphics[scale=0.48]{w-density_simulation_1000.jpeg}\quad
\includegraphics[scale=0.48]{w-density_simulation_100000.jpeg}\\
\vskip 5mm
Notice that the empirical mean differs very little from the theoretical mean with larger sample sizes $n=1000$ and $n=100000$. Also, the empirical density fits the theoretical density quite nicely for ``large" sample sizes.
\vskip 5mm

Next are a couple of simulations of the distribution function of $W$ with sample sizes $n=1000$ and $n=10,000$.
\vskip 5mm
\includegraphics[scale=0.45]{w-distribution_simulation_1000.jpeg}\quad
\includegraphics[scale=0.45]{w-distribution_simulation_10000.jpeg}\\
\vskip 5mm

\paragraph{The R Code:} Here is the R code used for the above simulations.
\vspace*{2mm}
\small 
\begin{lstlisting}[language=R]

# Code to Simulate the Density Function and Expectation of the Waiting Time

set.seed(123)
n <- 100000  # number of simulations

# Step 1: Simulate random arrival times (in minutes after 7:10)
T <- runif(n, 0, 20)


# Step 2: Compute waiting times
W <- ifelse(T < 5, 5 - T, 20 - T)


# Step 3: Define theoretical PDF of W
f_W <- function(w) {
  ifelse(w < 0, 0,
         ifelse(w < 5, 1/10,
                ifelse(w <= 15, 1/20, 0)))
}


# Step 4: Compute theoretical quantities
E_theory <- (1/10)*integrate(function(w) w, 0, 5)$value +
  (1/20)*integrate(function(w) w, 5, 15)$value
E_W2 <- (1/10)*integrate(function(w) w^2, 0, 5)$value +
  (1/20)*integrate(function(w) w^2, 5, 15)$value
Var_theory <- E_W2 - E_theory^2


# Step 5: Empirical stats
E_sim <- mean(W)
Var_sim <- var(W)

# Step 6: Plot simulated vs. theoretical density
hist(W, breaks = 50, freq = FALSE, col = "lightblue",
     main = sprintf("Density of Waiting Time W (n = %d)", n),
     xlab = "w (minutes)", ylab = "f_W(w)", xlim = c(0, 15))
curve(f_W(x), from = 0, to = 15, add = TRUE, col = "red", lwd = 2, lty = 2)

# Add mean lines
abline(v = E_sim, col = "blue", lwd = 2, lty = 3)
abline(v = E_theory, col = "darkgreen", lwd = 2, lty = 2)

# Add legend
legend("topright",
       legend = c("Empirical density (simulation)",
                  "Theoretical PDF",
                  sprintf("Empirical mean = %.2f", E_sim),
                  sprintf("Theoretical mean = %.2f", E_theory)),
       col = c("lightblue", "red", "blue", "darkgreen"),
       lwd = c(10, 2, 2, 2), lty = c(1, 2, 3, 2), bty = "n")

# Step 7: Print summary results
cat("=== Simulation Summary ===\n")
cat(sprintf("Number of trials: %d\n", n))
cat(sprintf("Empirical mean E[W] = %.4f\n", E_sim))
cat(sprintf("Theoretical mean E[W] = %.4f\n", E_theory))
cat(sprintf("Empirical variance Var(W) = %.4f\n", Var_sim))
cat(sprintf("Theoretical variance Var(W) = %.4f\n", Var_theory))



##############################################################################
##############################################################################



#Code to Simulate the Distribution Function of the Waiting Time

set.seed(123)
n <- 10000  # number of simulations

# Step 1: Simulate random arrival times
T <- runif(n, 0, 20)

# Step 2: Compute waiting times
W <- ifelse(T < 5, 5 - T, 20 - T)

# Step 3: Theoretical CDF of W
F_W <- function(w) {
  ifelse(w < 0, 0,
         ifelse(w < 5, w/10,
                ifelse(w <= 15, (w + 5)/20, 1)))
}


# Step 4: Plot empirical and theoretical CDFs
plot(
  ecdf(W),
  col = "blue", lwd = 2,
  pch = ".", cex = 0.5,  # smaller dots
  main = sprintf("Distribution Function of Waiting Time W (n = %d)", n),
  xlab = "w (minutes)", ylab = "F_W(w)", xlim = c(0, 15)
)
curve(F_W(x), from = 0, to = 15, add = TRUE, col = "red", lwd = 2, lty = 2)

legend(
  "bottomright",
  legend = c("Empirical CDF (simulation)", "Theoretical CDF"),
  col = c("blue", "red"),
  lwd = 2, lty = c(1, 2)
)

\end{lstlisting}

%# Step 7: Print summary results
%cat("=== Simulation Summary ===\n")
%cat(sprintf("Number of trials: %d\n", n))
%cat(sprintf("Empirical mean E[W] ≈ %.4f\n", E_sim))
%cat(sprintf("Theoretical mean E[W] = %.4f\n", E_theory))
%cat(sprintf("Empirical variance Var(W) ≈ %.4f\n", Var_sim))
%cat(sprintf("Theoretical variance Var(W) = %.4f\n", Var_theory))




\vskip 1cm
\hrule
\vskip 5mm
\begin{center}
\bf Please let me know if you have any questions, comments, or corrections!
\end{center}

%%%%%%%%%%%%%%%%%%%%%%%%%%%%%%%%%%%%%%%%%%%%%%%%%%%%%%
\end{document}
%%%%%%%%%%%%%%%%%%%%%%%%%%%%%%%%%%%%%%%%%%%%%%%%%%%%%%
