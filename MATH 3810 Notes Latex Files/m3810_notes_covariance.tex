%%%%%%%%%%%%%%%%%%%%%%%%%%%%%%%%%%%%%%%%%%%%%%%%%%%%%%
\documentclass[11pt]{article}
%%%%%%%%%%%%%%%%%%%%%%%%%%%%%%%%%%%%%%%%%%%%%%%%%%%%%%

\usepackage{amsmath}
\usepackage{amsthm}
\usepackage{amssymb}
\usepackage{latexsym}
\usepackage{graphicx}
\usepackage{color}
\usepackage{verbatim}
\usepackage{float}
\usepackage{multicol}
\usepackage{xcolor}
\usepackage{listings}
\usepackage{tikz}
\usetikzlibrary{arrows.meta, positioning, calc}
\usetikzlibrary{decorations.pathmorphing}
\usepackage{tcolorbox}
\tcbuselibrary{breakable}
\usepackage{cancel}


\newtcolorbox{solutionbox}{
  breakable,
  colback=blue!5!white,
  colframe=blue!50!black,
  title=Solution,
  sharp corners,
  boxrule=0.8pt
}

\newtcolorbox{hintbox}{
  breakable,
  colback=gray!10!white,
  colframe=gray!50!black,
  title=Hint,
  sharp corners,
  boxrule=0.5pt
}

% Unnumbered theorem
\newtheorem*{thm*}{Theorem}

\lstdefinelanguage{R}{
      keywords={if,else,while,for,in,next,break,function,TRUE,FALSE,NULL,Inf,NA,NaN,switch,repeat,return,require,library},
      keywordstyle=\color{blue}\bfseries,
      identifierstyle=\color{black},
      comment=[l]{\#},
      commentstyle=\color{gray}\ttfamily,
      string=[b]{"},
      stringstyle=\color{red}\ttfamily,
      morecomment=[l]{//},
      morestring=[b]{'},
      sensitive=true,
      morekeywords={print,summary,plot,lm,glm,data,frame,read.csv,write.csv,factor,levels,names,colnames,rownames,
      head,tail,str,dim,length,class,typeof,mode,is.na,is.null,is.finite,is.infinite,is.nan,as.numeric,as.character,
      as.factor,as.Date,as.POSIXct,as.matrix,as.data.frame,rbind,cbind,merge,subset,aggregate,tapply,apply,lapply,sapply,
      mapply,vapply,replicate,seq,rep,c,list,matrix,array,data.frame,table,hist,boxplot,barplot,pie,curve,lines,points,text,
      abline,legend,par,mtext,title,xlab,ylab,xlim,ylim,main,sub,col,pch,cex,lty,lwd,type,bg,fg,args,options,warnings,errors,
      message,stop,warning,error,try,tryCatch,withCallingHandlers,on.exit,debug,browser,trace,recover,options,getOption,setOption},
    }


\setlength{\textheight}{9in}
\setlength{\textwidth}{6in}
\addtolength{\topmargin}{-2cm}
\addtolength{\oddsidemargin}{-1cm}
\parindent=0in


\def\classnum{3810}
\def\classtitle{Probability}
\def\classtitleshort{Probability}
\def\classsec{001}
\def\classterm{Fall 2025}
\def\instructor{Robert Rostermundt}
%\def\hmwknum{\#2}


%%%%%%%%%%%%%%%%%%%%%%%%%%%%%%%%%%%%%%%%%%%%%%%%%%%%%%%%%
%%%%%%%%%%%%%%%%%%%%%%%%%  Colors  %%%%%%%%%%%%%%%%%%%%%%
%%%%%%%%%%%%%%%%%%%%%%%%%%%%%%%%%%%%%%%%%%%%%%%%%%%%%%%%%

\definecolor{Green}{rgb}{0,.5,0}
%use for definitions
\definecolor{Red}{rgb}{.8,.2,0}
%use for emphasis
\definecolor{Yellow}{rgb}{.6,.6,.1}
%use for part titles
\definecolor{Cyan}{rgb}{.2,.6,.7}
%use for comments
\definecolor{Purple}{rgb}{.4,0,1}
%use for examples
\definecolor{deepred}{rgb}{.53,.29,.24}
%use for important points
\definecolor{Black}{rgb}{0,0,0}
%use for washout
\definecolor{Grey}{rgb}{.45,.45,.45}
% use for theorems
\newcommand{\tred}[1]{\textcolor{Red}{#1}}
\newcommand{\tgreen}[1]{\textcolor{Green}{#1}}
\newcommand{\tcyan}[1]{\textcolor{Cyan}{#1}}
\newcommand{\tyellow}[1]{\textcolor{Yellow}{#1}}
\newcommand{\tpurple}[1]{\textcolor{Purple}{#1}}
\newcommand{\tblack}[1]{\textcolor{Black}{#1}}
\newcommand{\tgrey}[1]{\textcolor{Grey}{#1}}
\newcommand{\tdeepred}[1]{\textcolor{deepred}{#1}}
\newcommand{\ttt}[1]{\texttt{#1}}

%%%%%%%%%%%%%%%%%%%%%%%%%%%%%%%%%%%%%%%%%%%%%%%%%%%%%%%%%
%%%%%%%%%%%%%%%%%%%%%%%%%  Theorem Environments  %%%%%%%%
%%%%%%%%%%%%%%%%%%%%%%%%%%%%%%%%%%%%%%%%%%%%%%%%%%%%%%%%%

\theoremstyle{plain}
\newtheorem{thm}{Theorem}
\newtheorem{axiom}{Axiom}
\newtheorem{cor}{Corollary}
\newtheorem{lemma}{Lemma}
\newtheorem{prop}{Proposition}
\newtheorem{ques}{Question}
\theoremstyle{definition}
\newtheorem{defn}{Definition}
\theoremstyle{remark}
\newtheorem{remark}{Remark}
\theoremstyle{definition}
\newtheorem{ex}{Example}
\numberwithin{equation}{section}
\newtheorem{prob}{Problem}
\numberwithin{equation}{section}


%%%%%%%%%%%%%%%%%%%%%%%%%%%%%%%%%%%%%%%%%%%%%%%%%%%%%%%%%
%%%%%%%%%%%%%%%%%%%%%%%%%  Math    %%%%%%%%%%%%%%%%%%%%%%
%%%%%%%%%%%%%%%%%%%%%%%%%%%%%%%%%%%%%%%%%%%%%%%%%%%%%%%%%


\newcommand{\abs}[1]{\left\lvert{#1}\right\rvert}
\newcommand{\card}[1]{\lvert{#1}\rvert}
\newcommand{\union}{\cup}
\newcommand{\Union}{\bigcup}
\newcommand{\inter}{\cap}
\newcommand{\Inter}{\bigcap}
%\newcommand{\hint}[1]{\medskip\newline\emph{Hint: #1}}
%\newcommand{\note}[1]{\medskip\newline\emph{Note: #1}}
\newcommand{\points}[1]{[#1 points]}
\newcommand{\totalpoints}[1]{[#1 points total]}
\newcommand{\ds}{\displaystyle}
\newcommand{\ben}{\begin{enumerate}}
\newcommand{\een}{\end{enumerate}}
\newcommand{\bi}{\begin{itemize}}
\newcommand{\ei}{\end{itemize}}
\newcommand{\beq}{\begin{eqnarray*}}
\newcommand{\eeq}{\end{eqnarray*}}
\newcommand{\bieq}{\begin{IEEEeqnarray}{rCl}}
\newcommand{\bieqx}{\begin{IEEEeqnarray}{+rCl+x*}}
\newcommand{\eieq}{\end{IEEEeqnarray}}
\newcommand{\nn}{\nonumber}
%\renewcommand{\i}{\item}
\newcommand{\bpm}{\begin{pmatrix}}
\newcommand{\epm}{\end{pmatrix}}
\newcommand{\sol}{\indent{\bf\emph{Solution:}}}
\newcommand{\ssol}{\indent{\\[2mm]\bf\emph{Solution:}}\;}
\newcommand{\hint}{\indent{\bf\emph{Hint}:}\;}
\newcommand{\note}{\indent{\bf\emph{Note}:}\;}
\newcommand{\vsk}{\vskip 2mm}
%%%%%%%%%%%%%%%%%%%%%%%%% Calculus %%%%%%%%%%%%%%%%%%%%%%%%%%%%
\newcommand{\dd}[2]{\ds\frac{d}{d{#1}}\left[{#2}\right]}
\newcommand{\der}[2]{\ds\frac{d{#1}}{d{#2}}}
\newcommand{\lmt}[3]{\ds\lim_{{#1}\to{#2}}{#3}}
\renewcommand{\iint}[2]{\ds\int{#1}\,d{#2}}
\newcommand{\dint}[4]{\ds\int^{#4}_{#3}{#1}\,d{#2}}
\renewcommand{\Delta}{\triangle}
%%%%%%%%%%%%%%%%%%%%%%%%% Number Sets %%%%%%%%%%%%%%%%%%%%%%%%%%
\newcommand{\N}{\mathbb{N}}
\newcommand{\Z}{\mathbb{Z}}
\newcommand{\Q}{\mathbb{Q}}
\newcommand{\R}{\mathbb{R}}
\newcommand{\C}{\mathbb{C}}
\newcommand{\F}{\mathcal{F}}
\renewcommand{\P}{\mathbb{P}}
\newcommand{\E}{\mathcal{E}}
\renewcommand{\o}{\omega}
\renewcommand{\O}{\Omega}
%%%%%%%%%%%%%%%%%%%%%%%%% Vectors %%%%%%%%%%%%%%%%%%%%%%%%%%%%%
\newcommand{\x}{\bar{x}}
\renewcommand{\v}{\bar{v}}
\newcommand{\y}{\bar{y}}
\newcommand{\z}{\bar{z}}
\newcommand{\w}{\bar{w}}
\renewcommand{\u}{\bar{u}}
\renewcommand{\b}{\bar{b}}
\newcommand{\e}{\bar{e}}
\renewcommand{\a}{\vec{a}}
\renewcommand{\r}{\vec{r}}
\newcommand{\vv}{\vec{v}}
\newcommand{\vecPQ}[2]{\overrightarrow{#1}{#2}}
\newcommand{\vecV}[1]{\overrightarrow{#1}}
\newcommand{\la}{\langle}
\newcommand{\ra}{\rangle}
%%%%%%%%%%%%%%%%%%%%%%%%%%% Vector Spaces %%%%%%%%%%%%%%%%%%%%
\newcommand{\rn}{\ensuremath{\mathbb{R}^n}}
\renewcommand{\rm}{\ensuremath{\mathbb{R}^m}}
\newcommand{\re}{\mathbb{R}}
\newcommand{\Pn}{\mathbb{P}_n}
\newcommand{\B}{\mathcal{B}}
%%%%%%%%%%%%%%%%%%%%%%%%%%% Graphics %%%%%%%%%%%%%%%%%%%%%%%%
\newcommand{\cg}[2]{\begin{center}
\includegraphics[scale={#1}]{{#2}}
\end{center}}
\makeatletter
\def\imod#1{\allowbreak\mkern10mu({\operator@font mod}\,\,#1)}
\makeatother

%%%%%%%%%%%%%%%%%%%%%%%%%%%%%%%%%%%%%%%%%%%%%%%%%%%%%%%%%%%%%%%%%%%%%%%%%%%%%%%%%%%%%%%%%%%%%%
%%%%%%%%%%%%%%%%%%%%%%%%%%%%%% Defined Fonts %%%%%%%%%%%%%%%%%%%%%%%%%%%%%%%%%%%%%%%%%%%%%%%%%
%%%%%%%%%%%%%%%%%%%%%%%%%%%%%%%%%%%%%%%%%%%%%%%%%%%%%%%%%%%%%%%%%%%%%%%%%%%%%%%%%%%%%%%%%%%%%%

\font\minihelv=phvr at 6pt
\font\helv=phvr at 10pt
\font\medhelv=phvr at 16pt
\font\bighelv=phvr at 20pt
\font\hugehelv=phvr at 36pt
\font\mybigfont=phvr at 16pt
\font\mymediumfont=phvr at 14pt
\font\mediumhelv=phvr at 14pt
\font\mybfit=ptmbi at 12pt


%%%%%%%%%%%%%%%%%%%%%%%%%%%%%%%%%%%%%%%%%%%%%%%%%%%%%%%%%%%%%%%%%%%%%%%%%%%%%%%%%%%%%%%%%%%%%%%
%%%%%%%%%%%%%%%%%%%%%%%%%%%%%% Other Commands %%%%%%%%%%%%%%%%%%%%%%%%%%%%%%%%%%%%%%%%%%%%%%%%%
%%%%%%%%%%%%%%%%%%%%%%%%%%%%%%%%%%%%%%%%%%%%%%%%%%%%%%%%%%%%%%%%%%%%%%%%%%%%%%%%%%%%%%%%%%%%%%%
%\setlength\fboxrule{.5pt}
%\newcommand{\latexpicborder}[3]{
%\setlength\fboxsep{30pt}
%\begin{figure}[hb]
%\begin{center}
%\fbox{
%\input{#1}
%}
%\caption{#2}
%\label{#3}
%\end{center}
%\end{figure}
%\setlength\fboxsep{0pt}
%}
%
%\newcommand{\latexpic}[2]{
%\begin{figure}[hb]
%\begin{center}
%\input{#1}
%\vspace*{8mm}
%\caption{#2}
%\end{center}
%\end{figure}
%}

%\begin{minipage}[b]{0.6\linewidth}
%......
%\end{minipage}
%\hspace{0.5cm}
%\begin{minipage}[t]{0.4\linewidth}
%\centering
%\includegraphics[scale=.5]{m1401_ex3_g4.eps}
%\end{minipage}
%\end{figure}


%%%%%%%%%%%%%%%%%%%%%%%%%%%%%%%%%%%%%%%%%%%%%%%%%%%%%%%%%%%%%%%%%%%%%%%%%%%%%%%%%%%%%%%%%%%%%%
%%%%%%%%%%%%%%%%%%%%%%%%%%% IEEEeqnarray Notes %%%%%%%%%%%%%%%%%%%%%%%%%%%%%%%%%%%%%%%%%%%%%%%
%%%%%%%%%%%%%%%%%%%%%%%%%%%%%%%%%%%%%%%%%%%%%%%%%%%%%%%%%%%%%%%%%%%%%%%%%%%%%%%%%%%%%%%%%%%%%%


%Any number of columns can be specified with IEEEeqnarray: {c} will give only one
%column with all entries centered, or {rCll} would add a fourth, left-justified
%column to use for comments. Moreover, beside l, c, r, L, C, R for math mode
%entries there are also s, t, u for left, centered, and right text mode entries.
%Additional space can be added with . and / and ? in increasing order.
%
%
%\begin{proof}
%This is a proof that ends
%with an equation array:
%\begin{IEEEeqnarray*}{+rCl+x*}
%a & = & b + c \\
%& = & d + e. & \qedhere
%\end{IEEEeqnarray*}
%\end{proof}
%Note that the + in {+rCl+x*} denotes stretchable spaces, one on the left
%of the equations (which, if not specified, will be done automatically by
%IEEEeqnarray!) and one on the right of the equations. But now on the right,
%after the stretching column, we add an empty column x. This column will be
%only needed on the last line when we will put the \qedhere command there.
%Finally, we specify a *. This is a null-space that prevents IEEEeqnarray to
%add another unwanted +-space!


% The following environments enable custom numbering of theorems so that the numbers agree % with the numbering in the textbook being used. 
%
%  Usage examples:
%\begin{customthm}{2.2}\label{eight}
%Every theorem must be numbered by hand.
%\end{customthm}
%
%Here is a reference to theorem~\ref{eight}.
%
%\begin{customthm}{2.3}[Parenthetical comment]\label{nine}
%Statement
%\end{customthm}
%
%Here is a reference to theorem~\ref{nine}


\newtheorem{innercustomthm}{Theorem}
\newenvironment{customthm}[1]
  {\renewcommand\theinnercustomthm{#1}\innercustomthm}
  {\endinnercustomthm}
  
  \newtheorem{innercustomprop}{Proposition}
\newenvironment{customprop}[1]
  {\renewcommand\theinnercustomprop{#1}\innercustomprop}
  {\endinnercustomprop}
  
    \newtheorem{innercustomlem}{Lemma}
\newenvironment{customlem}[1]
  {\renewcommand\theinnercustomlem{#1}\innercustomlem}
  {\endinnercustomlem}
  
    \newtheorem{innercustomconj}{Conjecture}
\newenvironment{customconj}[1]
  {\renewcommand\theinnercustomconj{#1}\innercustomconj}
  {\endinnercustomconj}
  
    \newtheorem{innercustomclaim}{Claim}
\newenvironment{customclaim}[1]
  {\renewcommand\theinnercustomclaim{#1}\innercustomclaim}
  {\endinnercustomclaim}
  
    \newtheorem{innercustomcor}{Corollary}
\newenvironment{customcor}[1]
  {\renewcommand\theinnercustomcor{#1}\innercustomcor}
  {\endinnercustomcor}
  
    \newtheorem{innercustomdef}{Definition}
\newenvironment{customdef}[1]
  {\renewcommand\theinnercustomdef{#1}\innercustomdef}
  {\endinnercustomdef}
  
    \newtheorem{innercustomex}{Example}
\newenvironment{customex}[1]
  {\renewcommand\theinnercustomex{#1}\innercustomex}
  {\endinnercustomex}
  
    \newtheorem{innercustomass}{Assumption}
\newenvironment{customass}[1]
  {\renewcommand\theinnercustomass{#1}\innercustomass}
  {\endinnercustomass}
  
      \newtheorem{innercustomax}{Axiom}
\newenvironment{customax}[1]
  {\renewcommand\theinnercustomax{#1}\innercustomax}
  {\endinnercustomax}
  

\vfuzz2pt % Don't report over-full v-boxes if over-edge is small
\hfuzz2pt % Don't report over-full h-boxes if over-edge is small

\renewcommand{\ni}{\noindent}


%%%%%%%%%%%%%%%%%%%%%%%%%%%%%%%%%%%%%%%%%%%%%%%%%%%%%%
%%%%%%%%%%%%%%%%%%%%%%%%%%%%%%%%%%%%%%%%%%%%%%%%%%%%%%

\pagestyle{myheadings}

%%%%%%%%%%%%%%%%%%%%%%%%%%%%%%%%%%%%%%%%%%%%%%%%%%%%%%

%%%%%%%%%%%%%%%%%%%%%%%%%%%%%%%%%%%%%%%%%%%%%%%%%%%%%%
%%%%%%%%%%%%%%%%%%%%%%%%%   Document Body   %%%%%%%%%%
%%%%%%%%%%%%%%%%%%%%%%%%%%%%%%%%%%%%%%%%%%%%%%%%%%%%%%

%\def\classnum{3810}
%\def\classtitle{Probability}
%\def\classtitleshort{Probability}
%\def\classsec{001}
%\def\classterm{Fall 2025}
%\def\instructor{Robert Rostermundt}
\def\printsol{0}


	\title{\vspace{-1in}Math\classnum\;-\;\classtitle\\
	Section\;\classsec\;-\;\classterm\\
	Notes: Covariance}
	\author{University of Colorado Denver / College of Liberal Arts 	and Sciences}
	\date{Department of Mathematics - \instructor}

	\markright{Math\classnum\;-\;\classtitleshort, UCD, \classterm, \instructor}



%%%%%%%%%%%%%%%%%%%%%%%%%%%%%%%%%%%%%%%%%%%%%%%%%%%%%%
\begin{document}\maketitle\thispagestyle{empty}
%%%%%%%%%%%%%%%%%%%%%%%%%%%%%%%%%%%%%%%%%%%%%%%%%%%%%%



%%%%%%%%%%%%%%%%%%%%%%%%%%%%%%%%%%%%%%%%%%%%%%%%%%%%%%%%%%%%%%%%%%%%%%%%%%%%%%%%%%%%%%%%%%%%%%%%%%%%%%
\vspace*{2mm}
\hrule
\vskip 8mm


%%%%%%%%%%%%%%%%%%%%%%%%%%%%%%%%%%%%%%%%%%%%%%%%%%%%%%%%%%%%%%%%%%%%%%%%%
%%%%%%%%%%%%%%%%%%%%%%%%%%%%%%%%%%%%%%%%%%%%%%%%%%%%%%%%%%%%%%%%%%%%%%%%%

%%%%%%%%%%%%%%%%%%%%%%%%%%%%%%%%%%%%%%%%%%%%%%%%%%%%%%%%%%%%%%%%%%%%%%%%%%%%%%%%%%%%%%%%%%%%%%%%%%%%%%
\section*{Introduction to Covariance:}
%%%%%%%%%%%%%%%%%%%%%%%%%%%%%%%%%%%%%%%%%%%%%%%%%%%%%%%%%%%%%%%%%%%%%%%%%%%%%%%%%%%%%%%%%%%%%%%%%%%%%%

Covariance is a numerical measure describing how two random variables vary together. It generalizes the concept of variance:
	\begin{itemize}
		\item Variance measures how one variable varies around its mean.
		\item Covariance measures how two variables jointly vary around their means. 
	\end{itemize}

Here is the formal definition.
\vskip 5mm	
	\begin{defn}
Let $X$ and $Y$ be random variables with finite expected values,
$\mu_X = E[X]$ and $\mu_Y = E[Y]$. The \textbf{covariance} of $X$ and $Y$ is defined to be
\[Cov(X,Y) = E\left[(X - \mu_X)(Y - \mu_Y)\right].\]
	\end{defn}
\vskip 2mm	
Interpretation:
	\begin{itemize}
		\item $Cov(X,Y)>0$: when $X$ is above its mean, $Y$ tends to be above its mean (positive relationship).
		\item $Cov(X,Y)<0$: when $X$ is above its mean, $Y$ tends to be below its mean (negative relationship).
		\item Cov(X,Y)=0: no linear relationship.	
	\end{itemize}
%If one variable tends to be above its mean when the other is above its mean, 
%the covariance will be positive; if one is above its mean when the other is below, 
%the covariance will be negative.

\subsection*{Alternative Formulas}

By expanding the definition, we obtain the equivalent expression
\[\mathrm{Cov}(X,Y) = E[XY] - E[X]E[Y].\]
This is very easy to verify. 
	\begin{proof}
Assume $\mu_X=E[X]$ and $\mu_Y=E[Y]$ exist. Then
\[\mathrm{Cov}(X,Y)=E\big[(X-\mu_X)(Y-\mu_Y)\big]
=E\big[XY - X\mu_Y - \mu_X Y + \mu_X\mu_Y\big].\]
By linearity of expectation and the fact that constants factor out,
\[\mathrm{Cov}(X,Y)=E[XY]-\mu_YE[X]-\mu_XE[Y]+\mu_X\mu_Y
=E[XY]-\mu_X\mu_Y.\]
Therefore, $\mathrm{Cov}(X,Y)=E[XY]-E[X]E[Y]$ as desired.\\
	\end{proof}
\vskip 5mm
The term $E[X]E[Y]$ represents what the expectation of the product would be 
if $X$ and $Y$ were unrelated; covariance measures the deviation from this baseline. To help 
see this consider random variables $X$ and $Y$. The expectation of the product $XY$ is  
\[E[XY] = \sum_x \sum_y xy \, P(X=x,Y=y)\]
in the discrete case, and
\[E[XY] = \int_{-\infty}^\infty \int_{-\infty}^\infty xy \, f_{X,Y}(x,y)\,dx\,dy\]
in the continuous case. This expectation is taken with respect to the \emph{joint distribution} 
of $(X,Y)$. If large values of $X$ tend to occur with large values of $Y$, 
then the product $XY$ is typically large, making $E[XY]$ large. If large 
values of one tend to occur with small values of the other, then $E[XY]$ 
tends to be smaller.

%%%%%%%%%%%%%%%%%%%%%%%%%%%%%%%%%%%%%%%%%

\subsection*{Relationship to Variance}

Variance is a special case of covariance:
\[
\mathrm{Var}(X) = \mathrm{Cov}(X,X).
\]
Using the identity above,
\[
\mathrm{Var}(X) = E[X^2] - (E[X])^2.
\]

\subsection*{Covariance and Independence}

If $X$ and $Y$ are independent, then we can prove that
\[E[XY] = E[X]E[Y].\]
It follows, that then
\[\mathrm{Cov}(X,Y) = 0.\]
However, the converse is not true in general; zero covariance does not imply 
independence except in special cases such as jointly normal random variables.

\subsection*{Variance of a Sum}

Covariance plays a central role in computing the variance of a sum.  
Using the definition of variance,
\[
\mathrm{Var}(X+Y) 
= E\!\left[(X+Y - E[X+Y])^2\right].
\]

Expanding the square gives
\[
(X+Y - \mu_X - \mu_Y)^2
= (X-\mu_X)^2 + (Y-\mu_Y)^2 + 2(X-\mu_X)(Y-\mu_Y).
\]

Taking expectations,
\[
\mathrm{Var}(X+Y)
= \mathrm{Var}(X) + \mathrm{Var}(Y) + 2\mathrm{Cov}(X,Y).
\]

If $X$ and $Y$ are independent, then $\mathrm{Cov}(X,Y)=0$ and the formula 
reduces to
\[
\mathrm{Var}(X+Y) = \mathrm{Var}(X) + \mathrm{Var}(Y).
\]
We can extend this to the following result. If $X_1,X_2,\dots,X_n$ are independent random variables then
\[Var\big[X_1+X_2+\cdots+X_n\big]=Var[X_1]+Var[X_2]+\cdots+Var[X_n].\]
This result is key in an easy derivation of the variance of a binomial random variable and also in understanding the normalization of a random variable:
\[X\;\longrightarrow \ds\frac{X-\mu_{_X}}{\sigma_{_X}/\sqrt{n}}.\]
\vskip 5mm
One of the weaknesses of covariance is that the value is highly dependent on units. That is, if we change units of measurement we will change the value of the covariance. We would like to have  a measurement that is independent of units.
\vskip 2mm
\section*{Correlation Coefficient:}

The correlation coefficient provides a scale-free measure of the linear
relationship between two random variables. It's value is independent of units and is defined by
\[
\rho_{X,Y} = \frac{\mathrm{Cov}(X,Y)}
{\sqrt{\mathrm{Var}(X)}\sqrt{\mathrm{Var}(Y)}},
\]
and satisfies $-1 \le \rho_{X,Y} \le 1$. 
	\begin{itemize}
		\item A value of $\rho=1$ indicates a perfect
positive linear relationship
		\item A value of $\rho=-1$ indicates a perfect
negative linear relationship
		\item A value of $\rho=0$ indicates no linear relationship.
	\end{itemize}

Correlation is a normalized version of covariance and is therefore 
dimensionless. Independence implies $\rho_{X,Y}=0$, although the converse is 
generally false.

\subsection*{Concrete Numerical Example}

Consider the dataset consisting of the four pairs
\[(1,2),\quad (2,3),\quad (3,6),\quad (4,8).\]

	\begin{figure}[h!]
		\begin{center}
			\includegraphics[scale=0.5]{correlation_plot1.jpg}
			\caption{Scatter Plot of Example Data Set}
			\label{figcorrplot1}
		\end{center}			
	\end{figure}

\subsubsection*{Step 1: Sample Means}
\[\bar{x} = \frac{1+2+3+4}{4} = 2.5, \qquad\bar{y} = \frac{2+3+6+8}{4} = 4.75.\]

\subsubsection*{Step 2: Sample Covariance}

\note For a sample $\{(x_i,y_i)\}_{i=1}^n$, the \textbf{sample covariance} is
\[s_{XY} = \frac{1}{n-1} \sum_{i=1}^n (x_i - \bar{x})(y_i - \bar{y}),\]
where $\bar{x}$ and $\bar{y}$ are the sample means.
\vskip 5mm

\[s_{XY} = \frac{1}{n-1}\sum_{i=1}^4 (x_i - \bar{x})(y_i - \bar{y}).\]

We compute each product:
\[\begin{array}{c|c|c|c|c}
x_i & y_i & x_i - \bar{x} & y_i - \bar{y} & (x_i-\bar{x})(y_i-\bar{y}) \\
\hline
1 & 2 & -1.5 & -2.75 & 4.125 \\
2 & 3 & -0.5 & -1.75 & 0.875 \\
3 & 6 & \phantom{-}0.5 & \phantom{-}1.25 & 0.625 \\
4 & 8 & \phantom{-}1.5 & \phantom{-}3.25 & 4.875 \\
\end{array}\]

Summing the products:
\[4.125 + 0.875 + 0.625 + 4.875 = 10.5,\]
so
\[s_{XY} = \frac{10.5}{3} = 3.5.\]

\subsubsection*{Step 3: Sample Variances}

For $X$:
\[s_X^2=\frac{(-1.5)^2+(-0.5)^2+(0.5)^2+(1.5)^2}{3}=\frac{5}{3} \approx 1.6667.\]

For $Y$:
\[s_Y^2 = \frac{(-2.75)^2 + (-1.75)^2 + (1.25)^2 + (3.25)^2}{3}=\frac{25.75}{3} \approx 8.5833.\]

\subsubsection*{Step 4: Correlation Coefficient}

\[r=\frac{s_{XY}}{\sqrt{s_X^2}\sqrt{s_Y^2}}=\frac{3.5}{\sqrt{1.6667}\sqrt{8.5833}}.\]

Evaluating the square roots:
\[\sqrt{1.6667} \approx 1.290,\qquad\sqrt{8.5833} \approx 2.930.\]

Thus
\[r \approx \frac{3.5}{1.290 \cdot 2.930}=\frac{3.5}{3.7797}\approx 0.926.\]

This value indicates a \textbf{strong positive linear relationship} 
between $X$ and $Y$ which is visible in the scatter plot in Figure \ref{figcorrplot1} shown above.

\subsection*{Numerical Example: Uncorrelated but Dependent}

It's possible to have correlation equal to zero and be dependent. Here is a graphic example. 

	\begin{figure}[h!]
		\begin{center}
			\includegraphics[scale=0.5]{correlation_plot2.jpg}
			\caption{Scatter Plot of Example Data Set}
			\label{figcorrplot1}
		\end{center}			
	\end{figure}

This second plot clearly shows a curved pattern (a parabola), meaning strong dependence, even though the correlation is zero.
\vskip 5mm

Here is another concrete example using the five observations
\[(-2,4),\;(-1,1),\;(0,0),\;(1,1),\;(2,4),\]
so $X=\{-2,-1,0,1,2\}$ and $Y=X^2=\{4,1,0,1,4\}$.
\vskip 2mm
\subsubsection*{Step 1: Sample Means}
\[\bar X = \frac{-2-1+0+1+2}{5}=0,\qquad
\bar Y = \frac{4+1+0+1+4}{5}=2.\]

Deviations and products:
\[\begin{array}{c|c|c|c|c}
x_i & y_i & x_i-\bar X & y_i-\bar Y & (x_i-\bar X)(y_i-\bar Y)\\\hline
-2 & 4 & -2 & 2 & -4\\
-1 & 1 & -1 & -1 & 1\\
0  & 0 & 0  & -2 & 0\\
1  & 1 & 1  & -1 & -1\\
2  & 4 & 2  & 2  & 4
\end{array}\]
The sum of the products is $-4+1+0-1+4=0$.

\subsubsection*{Step 2: Sample Covariance}
\[s_{XY}=\frac{1}{n-1}\sum_{i=1}^n (x_i-\bar X)(y_i-\bar Y)=\frac{0}{4}=0.\]

\subsubsection*{Step 3: Sample Variances}
\[s_X^2=\frac{4+1+0+1+4}{4}=\frac{10}{4}=2.5,\qquad
s_Y^2=\frac{4+1+4+1+4}{4}=\frac{14}{4}=3.5.\]

\subsubsection*{Step 4: Sample Correlation}
\[r=\frac{s_{XY}}{\sqrt{s_X^2}\sqrt{s_Y^2}}=0.\]

Thus $r=0$ despite $Y$ depending on $X$ via $Y=X^2$. The zero correlation occurs because the
positive and negative contributions to the covariance cancel due to symmetry. Important point is
that zero covariance/correlation does not imply independence.


\vskip 1cm
\hrule
\vskip 5mm
\begin{center}
\bf Please let me know if you have any questions, comments, or corrections!
\end{center}

%%%%%%%%%%%%%%%%%%%%%%%%%%%%%%%%%%%%%%%%%%%%%%%%%%%%%%
\end{document}
%%%%%%%%%%%%%%%%%%%%%%%%%%%%%%%%%%%%%%%%%%%%%%%%%%%%%%

%%%%%%%%%%%%%%%%%%%%%%%%%%%%%%%%%%%%%%%%%%%%%%%%%%%%%%%%%%%%%%
%%%%%%%%%%%%%%%%%%%%%%%%%%%%%%%%%%%%%%%%%%%%%%%%%%%%%%%%%%%%%%

