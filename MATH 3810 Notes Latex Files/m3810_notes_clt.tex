%%%%%%%%%%%%%%%%%%%%%%%%%%%%%%%%%%%%%%%%%%%%%%%%%%%%%%
\documentclass[11pt]{article}
%%%%%%%%%%%%%%%%%%%%%%%%%%%%%%%%%%%%%%%%%%%%%%%%%%%%%%

\usepackage{amsmath}
\usepackage{amsthm}
\usepackage{amssymb}
\usepackage{latexsym}
\usepackage{graphicx}
\usepackage{color}
\usepackage{verbatim}
\usepackage{float}
\usepackage{multicol}
\usepackage{xcolor}
\usepackage{listings}
\usepackage{tikz}
\usetikzlibrary{arrows.meta, positioning, calc}
\usetikzlibrary{decorations.pathmorphing}
\usepackage{tcolorbox}
\tcbuselibrary{breakable}
\usepackage{cancel}


\newtcolorbox{solutionbox}{
  breakable,
  colback=blue!5!white,
  colframe=blue!50!black,
  title=Solution,
  sharp corners,
  boxrule=0.8pt
}

\newtcolorbox{hintbox}{
  breakable,
  colback=gray!10!white,
  colframe=gray!50!black,
  title=Hint,
  sharp corners,
  boxrule=0.5pt
}

% Unnumbered theorem
\newtheorem*{thm*}{Theorem}

\lstdefinelanguage{R}{
      keywords={if,else,while,for,in,next,break,function,TRUE,FALSE,NULL,Inf,NA,NaN,switch,repeat,return,require,library},
      keywordstyle=\color{blue}\bfseries,
      identifierstyle=\color{black},
      comment=[l]{\#},
      commentstyle=\color{gray}\ttfamily,
      string=[b]{"},
      stringstyle=\color{red}\ttfamily,
      morecomment=[l]{//},
      morestring=[b]{'},
      sensitive=true,
      morekeywords={print,summary,plot,lm,glm,data,frame,read.csv,write.csv,factor,levels,names,colnames,rownames,
      head,tail,str,dim,length,class,typeof,mode,is.na,is.null,is.finite,is.infinite,is.nan,as.numeric,as.character,
      as.factor,as.Date,as.POSIXct,as.matrix,as.data.frame,rbind,cbind,merge,subset,aggregate,tapply,apply,lapply,sapply,
      mapply,vapply,replicate,seq,rep,c,list,matrix,array,data.frame,table,hist,boxplot,barplot,pie,curve,lines,points,text,
      abline,legend,par,mtext,title,xlab,ylab,xlim,ylim,main,sub,col,pch,cex,lty,lwd,type,bg,fg,args,options,warnings,errors,
      message,stop,warning,error,try,tryCatch,withCallingHandlers,on.exit,debug,browser,trace,recover,options,getOption,setOption},
    }


\setlength{\textheight}{9in}
\setlength{\textwidth}{6in}
\addtolength{\topmargin}{-2cm}
\addtolength{\oddsidemargin}{-1cm}
\parindent=0in


\def\classnum{3810}
\def\classtitle{Probability}
\def\classtitleshort{Probability}
\def\classsec{001}
\def\classterm{Fall 2025}
\def\instructor{Robert Rostermundt}
%\def\hmwknum{\#2}


%%%%%%%%%%%%%%%%%%%%%%%%%%%%%%%%%%%%%%%%%%%%%%%%%%%%%%%%%
%%%%%%%%%%%%%%%%%%%%%%%%%  Colors  %%%%%%%%%%%%%%%%%%%%%%
%%%%%%%%%%%%%%%%%%%%%%%%%%%%%%%%%%%%%%%%%%%%%%%%%%%%%%%%%

\definecolor{Green}{rgb}{0,.5,0}
%use for definitions
\definecolor{Red}{rgb}{.8,.2,0}
%use for emphasis
\definecolor{Yellow}{rgb}{.6,.6,.1}
%use for part titles
\definecolor{Cyan}{rgb}{.2,.6,.7}
%use for comments
\definecolor{Purple}{rgb}{.4,0,1}
%use for examples
\definecolor{deepred}{rgb}{.53,.29,.24}
%use for important points
\definecolor{Black}{rgb}{0,0,0}
%use for washout
\definecolor{Grey}{rgb}{.45,.45,.45}
% use for theorems
\newcommand{\tred}[1]{\textcolor{Red}{#1}}
\newcommand{\tgreen}[1]{\textcolor{Green}{#1}}
\newcommand{\tcyan}[1]{\textcolor{Cyan}{#1}}
\newcommand{\tyellow}[1]{\textcolor{Yellow}{#1}}
\newcommand{\tpurple}[1]{\textcolor{Purple}{#1}}
\newcommand{\tblack}[1]{\textcolor{Black}{#1}}
\newcommand{\tgrey}[1]{\textcolor{Grey}{#1}}
\newcommand{\tdeepred}[1]{\textcolor{deepred}{#1}}
\newcommand{\ttt}[1]{\texttt{#1}}

%%%%%%%%%%%%%%%%%%%%%%%%%%%%%%%%%%%%%%%%%%%%%%%%%%%%%%%%%
%%%%%%%%%%%%%%%%%%%%%%%%%  Theorem Environments  %%%%%%%%
%%%%%%%%%%%%%%%%%%%%%%%%%%%%%%%%%%%%%%%%%%%%%%%%%%%%%%%%%

\theoremstyle{plain}
\newtheorem{thm}{Theorem}
\newtheorem{axiom}{Axiom}
\newtheorem{cor}{Corollary}
\newtheorem{lemma}{Lemma}
\newtheorem{prop}{Proposition}
\newtheorem{ques}{Question}
\theoremstyle{definition}
\newtheorem{defn}{Definition}
\theoremstyle{remark}
\newtheorem{remark}{Remark}
\theoremstyle{definition}
\newtheorem{ex}{Example}
\numberwithin{equation}{section}
\newtheorem{prob}{Problem}
\numberwithin{equation}{section}


%%%%%%%%%%%%%%%%%%%%%%%%%%%%%%%%%%%%%%%%%%%%%%%%%%%%%%%%%
%%%%%%%%%%%%%%%%%%%%%%%%%  Math    %%%%%%%%%%%%%%%%%%%%%%
%%%%%%%%%%%%%%%%%%%%%%%%%%%%%%%%%%%%%%%%%%%%%%%%%%%%%%%%%


\newcommand{\abs}[1]{\left\lvert{#1}\right\rvert}
\newcommand{\card}[1]{\lvert{#1}\rvert}
\newcommand{\union}{\cup}
\newcommand{\Union}{\bigcup}
\newcommand{\inter}{\cap}
\newcommand{\Inter}{\bigcap}
%\newcommand{\hint}[1]{\medskip\newline\emph{Hint: #1}}
%\newcommand{\note}[1]{\medskip\newline\emph{Note: #1}}
\newcommand{\points}[1]{[#1 points]}
\newcommand{\totalpoints}[1]{[#1 points total]}
\newcommand{\ds}{\displaystyle}
\newcommand{\ben}{\begin{enumerate}}
\newcommand{\een}{\end{enumerate}}
\newcommand{\bi}{\begin{itemize}}
\newcommand{\ei}{\end{itemize}}
\newcommand{\beq}{\begin{eqnarray*}}
\newcommand{\eeq}{\end{eqnarray*}}
\newcommand{\bieq}{\begin{IEEEeqnarray}{rCl}}
\newcommand{\bieqx}{\begin{IEEEeqnarray}{+rCl+x*}}
\newcommand{\eieq}{\end{IEEEeqnarray}}
\newcommand{\nn}{\nonumber}
%\renewcommand{\i}{\item}
\newcommand{\bpm}{\begin{pmatrix}}
\newcommand{\epm}{\end{pmatrix}}
\newcommand{\sol}{\indent{\bf\emph{Solution:}}}
\newcommand{\ssol}{\indent{\\[2mm]\bf\emph{Solution:}}\;}
\newcommand{\hint}{\indent{\bf\emph{Hint}:}\;}
\newcommand{\note}{\indent{\bf\emph{Note}:}\;}
\newcommand{\vsk}{\vskip 2mm}
%%%%%%%%%%%%%%%%%%%%%%%%% Calculus %%%%%%%%%%%%%%%%%%%%%%%%%%%%
\newcommand{\dd}[2]{\ds\frac{d}{d{#1}}\left[{#2}\right]}
\newcommand{\der}[2]{\ds\frac{d{#1}}{d{#2}}}
\newcommand{\lmt}[3]{\ds\lim_{{#1}\to{#2}}{#3}}
\renewcommand{\iint}[2]{\ds\int{#1}\,d{#2}}
\newcommand{\dint}[4]{\ds\int^{#4}_{#3}{#1}\,d{#2}}
\renewcommand{\Delta}{\triangle}
%%%%%%%%%%%%%%%%%%%%%%%%% Number Sets %%%%%%%%%%%%%%%%%%%%%%%%%%
\newcommand{\N}{\mathbb{N}}
\newcommand{\Z}{\mathbb{Z}}
\newcommand{\Q}{\mathbb{Q}}
\newcommand{\R}{\mathbb{R}}
\newcommand{\C}{\mathbb{C}}
\newcommand{\F}{\mathcal{F}}
\renewcommand{\P}{\mathbb{P}}
\newcommand{\E}{\mathcal{E}}
\renewcommand{\o}{\omega}
\renewcommand{\O}{\Omega}
%%%%%%%%%%%%%%%%%%%%%%%%% Vectors %%%%%%%%%%%%%%%%%%%%%%%%%%%%%
\newcommand{\x}{\bar{x}}
\renewcommand{\v}{\bar{v}}
\newcommand{\y}{\bar{y}}
\newcommand{\z}{\bar{z}}
\newcommand{\w}{\bar{w}}
\renewcommand{\u}{\bar{u}}
\renewcommand{\b}{\bar{b}}
\newcommand{\e}{\bar{e}}
\renewcommand{\a}{\vec{a}}
\renewcommand{\r}{\vec{r}}
\newcommand{\vv}{\vec{v}}
\newcommand{\vecPQ}[2]{\overrightarrow{#1}{#2}}
\newcommand{\vecV}[1]{\overrightarrow{#1}}
\newcommand{\la}{\langle}
\newcommand{\ra}{\rangle}
%%%%%%%%%%%%%%%%%%%%%%%%%%% Vector Spaces %%%%%%%%%%%%%%%%%%%%
\newcommand{\rn}{\ensuremath{\mathbb{R}^n}}
\renewcommand{\rm}{\ensuremath{\mathbb{R}^m}}
\newcommand{\re}{\mathbb{R}}
\newcommand{\Pn}{\mathbb{P}_n}
\newcommand{\B}{\mathcal{B}}
%%%%%%%%%%%%%%%%%%%%%%%%%%% Graphics %%%%%%%%%%%%%%%%%%%%%%%%
\newcommand{\cg}[2]{\begin{center}
\includegraphics[scale={#1}]{{#2}}
\end{center}}
\makeatletter
\def\imod#1{\allowbreak\mkern10mu({\operator@font mod}\,\,#1)}
\makeatother

%%%%%%%%%%%%%%%%%%%%%%%%%%%%%%%%%%%%%%%%%%%%%%%%%%%%%%%%%%%%%%%%%%%%%%%%%%%%%%%%%%%%%%%%%%%%%%
%%%%%%%%%%%%%%%%%%%%%%%%%%%%%% Defined Fonts %%%%%%%%%%%%%%%%%%%%%%%%%%%%%%%%%%%%%%%%%%%%%%%%%
%%%%%%%%%%%%%%%%%%%%%%%%%%%%%%%%%%%%%%%%%%%%%%%%%%%%%%%%%%%%%%%%%%%%%%%%%%%%%%%%%%%%%%%%%%%%%%

\font\minihelv=phvr at 6pt
\font\helv=phvr at 10pt
\font\medhelv=phvr at 16pt
\font\bighelv=phvr at 20pt
\font\hugehelv=phvr at 36pt
\font\mybigfont=phvr at 16pt
\font\mymediumfont=phvr at 14pt
\font\mediumhelv=phvr at 14pt
\font\mybfit=ptmbi at 12pt


%%%%%%%%%%%%%%%%%%%%%%%%%%%%%%%%%%%%%%%%%%%%%%%%%%%%%%%%%%%%%%%%%%%%%%%%%%%%%%%%%%%%%%%%%%%%%%%
%%%%%%%%%%%%%%%%%%%%%%%%%%%%%% Other Commands %%%%%%%%%%%%%%%%%%%%%%%%%%%%%%%%%%%%%%%%%%%%%%%%%
%%%%%%%%%%%%%%%%%%%%%%%%%%%%%%%%%%%%%%%%%%%%%%%%%%%%%%%%%%%%%%%%%%%%%%%%%%%%%%%%%%%%%%%%%%%%%%%
%\setlength\fboxrule{.5pt}
%\newcommand{\latexpicborder}[3]{
%\setlength\fboxsep{30pt}
%\begin{figure}[hb]
%\begin{center}
%\fbox{
%\input{#1}
%}
%\caption{#2}
%\label{#3}
%\end{center}
%\end{figure}
%\setlength\fboxsep{0pt}
%}
%
%\newcommand{\latexpic}[2]{
%\begin{figure}[hb]
%\begin{center}
%\input{#1}
%\vspace*{8mm}
%\caption{#2}
%\end{center}
%\end{figure}
%}

%\begin{minipage}[b]{0.6\linewidth}
%......
%\end{minipage}
%\hspace{0.5cm}
%\begin{minipage}[t]{0.4\linewidth}
%\centering
%\includegraphics[scale=.5]{m1401_ex3_g4.eps}
%\end{minipage}
%\end{figure}


%%%%%%%%%%%%%%%%%%%%%%%%%%%%%%%%%%%%%%%%%%%%%%%%%%%%%%%%%%%%%%%%%%%%%%%%%%%%%%%%%%%%%%%%%%%%%%
%%%%%%%%%%%%%%%%%%%%%%%%%%% IEEEeqnarray Notes %%%%%%%%%%%%%%%%%%%%%%%%%%%%%%%%%%%%%%%%%%%%%%%
%%%%%%%%%%%%%%%%%%%%%%%%%%%%%%%%%%%%%%%%%%%%%%%%%%%%%%%%%%%%%%%%%%%%%%%%%%%%%%%%%%%%%%%%%%%%%%


%Any number of columns can be specified with IEEEeqnarray: {c} will give only one
%column with all entries centered, or {rCll} would add a fourth, left-justified
%column to use for comments. Moreover, beside l, c, r, L, C, R for math mode
%entries there are also s, t, u for left, centered, and right text mode entries.
%Additional space can be added with . and / and ? in increasing order.
%
%
%\begin{proof}
%This is a proof that ends
%with an equation array:
%\begin{IEEEeqnarray*}{+rCl+x*}
%a & = & b + c \\
%& = & d + e. & \qedhere
%\end{IEEEeqnarray*}
%\end{proof}
%Note that the + in {+rCl+x*} denotes stretchable spaces, one on the left
%of the equations (which, if not specified, will be done automatically by
%IEEEeqnarray!) and one on the right of the equations. But now on the right,
%after the stretching column, we add an empty column x. This column will be
%only needed on the last line when we will put the \qedhere command there.
%Finally, we specify a *. This is a null-space that prevents IEEEeqnarray to
%add another unwanted +-space!


% The following environments enable custom numbering of theorems so that the numbers agree % with the numbering in the textbook being used. 
%
%  Usage examples:
%\begin{customthm}{2.2}\label{eight}
%Every theorem must be numbered by hand.
%\end{customthm}
%
%Here is a reference to theorem~\ref{eight}.
%
%\begin{customthm}{2.3}[Parenthetical comment]\label{nine}
%Statement
%\end{customthm}
%
%Here is a reference to theorem~\ref{nine}


\newtheorem{innercustomthm}{Theorem}
\newenvironment{customthm}[1]
  {\renewcommand\theinnercustomthm{#1}\innercustomthm}
  {\endinnercustomthm}
  
  \newtheorem{innercustomprop}{Proposition}
\newenvironment{customprop}[1]
  {\renewcommand\theinnercustomprop{#1}\innercustomprop}
  {\endinnercustomprop}
  
    \newtheorem{innercustomlem}{Lemma}
\newenvironment{customlem}[1]
  {\renewcommand\theinnercustomlem{#1}\innercustomlem}
  {\endinnercustomlem}
  
    \newtheorem{innercustomconj}{Conjecture}
\newenvironment{customconj}[1]
  {\renewcommand\theinnercustomconj{#1}\innercustomconj}
  {\endinnercustomconj}
  
    \newtheorem{innercustomclaim}{Claim}
\newenvironment{customclaim}[1]
  {\renewcommand\theinnercustomclaim{#1}\innercustomclaim}
  {\endinnercustomclaim}
  
    \newtheorem{innercustomcor}{Corollary}
\newenvironment{customcor}[1]
  {\renewcommand\theinnercustomcor{#1}\innercustomcor}
  {\endinnercustomcor}
  
    \newtheorem{innercustomdef}{Definition}
\newenvironment{customdef}[1]
  {\renewcommand\theinnercustomdef{#1}\innercustomdef}
  {\endinnercustomdef}
  
    \newtheorem{innercustomex}{Example}
\newenvironment{customex}[1]
  {\renewcommand\theinnercustomex{#1}\innercustomex}
  {\endinnercustomex}
  
    \newtheorem{innercustomass}{Assumption}
\newenvironment{customass}[1]
  {\renewcommand\theinnercustomass{#1}\innercustomass}
  {\endinnercustomass}
  
      \newtheorem{innercustomax}{Axiom}
\newenvironment{customax}[1]
  {\renewcommand\theinnercustomax{#1}\innercustomax}
  {\endinnercustomax}
  

\vfuzz2pt % Don't report over-full v-boxes if over-edge is small
\hfuzz2pt % Don't report over-full h-boxes if over-edge is small

\renewcommand{\ni}{\noindent}


%%%%%%%%%%%%%%%%%%%%%%%%%%%%%%%%%%%%%%%%%%%%%%%%%%%%%%
%%%%%%%%%%%%%%%%%%%%%%%%%%%%%%%%%%%%%%%%%%%%%%%%%%%%%%

\pagestyle{myheadings}

%%%%%%%%%%%%%%%%%%%%%%%%%%%%%%%%%%%%%%%%%%%%%%%%%%%%%%

%%%%%%%%%%%%%%%%%%%%%%%%%%%%%%%%%%%%%%%%%%%%%%%%%%%%%%
%%%%%%%%%%%%%%%%%%%%%%%%%   Document Body   %%%%%%%%%%
%%%%%%%%%%%%%%%%%%%%%%%%%%%%%%%%%%%%%%%%%%%%%%%%%%%%%%

%\def\classnum{3810}
%\def\classtitle{Probability}
%\def\classtitleshort{Probability}
%\def\classsec{001}
%\def\classterm{Fall 2025}
%\def\instructor{Robert Rostermundt}
\def\printsol{0}


	\title{\vspace{-1in}Math\classnum\;-\;\classtitle\\
	Section\;\classsec\;-\;\classterm\\
	Notes: Central Limit Theorem}
	\author{University of Colorado Denver / College of Liberal Arts 	and Sciences}
	\date{Department of Mathematics - \instructor}

	\markright{Math\classnum\;-\;\classtitleshort, UCD, \classterm, \instructor}



%%%%%%%%%%%%%%%%%%%%%%%%%%%%%%%%%%%%%%%%%%%%%%%%%%%%%%
\begin{document}\maketitle\thispagestyle{empty}
%%%%%%%%%%%%%%%%%%%%%%%%%%%%%%%%%%%%%%%%%%%%%%%%%%%%%%



%%%%%%%%%%%%%%%%%%%%%%%%%%%%%%%%%%%%%%%%%%%%%%%%%%%%%%%%%%%%%%%%%%%%%%%%%%%%%%%%%%%%%%%%%%%%%%%%%%%%%%
\vspace*{2mm}
\hrule
\vskip 8mm


%%%%%%%%%%%%%%%%%%%%%%%%%%%%%%%%%%%%%%%%%%%%%%%%%%%%%%%%%%%%%%%%%%%%%%%%%
%%%%%%%%%%%%%%%%%%%%%%%%%%%%%%%%%%%%%%%%%%%%%%%%%%%%%%%%%%%%%%%%%%%%%%%%%


%%%%%%%%%%%%%%%%%%%%%%%%%%%%%%%%%%%%%%%%%%%%%%%%%%%%%%%%%%%%%%%%%%%%%%%%%%%%%%%%%%%%%%%%%%%%%%%%%%%%%%
\section*{The Problem:}
%%%%%%%%%%%%%%%%%%%%%%%%%%%%%%%%%%%%%%%%%%%%%%%%%%%%%%%%%%%%%%%%%%%%%%%%%%%%%%%%%%%%%%%%%%%%%%%%%%%%%%


\noindent In probability, and especially statistics, we are interested in the distribution of the sample mean. That is, if we choose a random sample $X_1, X_2, X_3,\dots, X_n$ from a population described by a random variable $X$ with defined expectation $E[X]=\mu$ and variance $Var[X]=\sigma^2$, we would like to understand the distribution of the random variable
\[\overline{X}_n=\ds\frac{X_1+X_2+X_3+\cdots+X_n}{n}.\]
As an example, consider the pollster problem. Suppose the true proportion of the population that supports candidate A is $p$. In order to estimate the true proportion, a polling company asks $n$ randomly selected voters whether they’ll vote for candidate A (say, ``Yes" or ``No"). Let
\[X_i=\left\{\begin{array}{ccl}
1&:&\text{The }i^{th}\text{ person surveyed answers yes},\\
0&:&\text{The }i^{th}\text{ person surveyed answers no}.\\
\end{array}
\right.\]
Then $X_i\sim Ber(p)$ and the sample proportion is the random variable
\[\hat{p}_{_n}=\ds\frac{X_1+X_2+\cdots+X_n}{n}.\]


We already know that $E[\hat{p}_{_n}]=p$ and $Var(\hat{p}_{_n})=\sqrt{p(1-p)}/n$. We also know from the Weak law of Large Numbers that $\hat{p}_{_n}$ converges in probability to $p$. That is, we know $\hat{p}_{_n}\stackrel{p}{\longrightarrow}p$. But that doesn't tell us about the actual distribution of $\hat{p}_{_n}$. Fortunately, the Central Limit Theorem will help with this.


%%%%%%%%%%%%%%%%%%%%%%%%%%%%%%%%%%%%%%%%%%%%%%%%%%%%%%%%%%%%%%%%%%%%%%%%%%%%%%%%%%%%%%%%%%%%%%%%%%%%%%
\section*{Theoretical Tools:}
%%%%%%%%%%%%%%%%%%%%%%%%%%%%%%%%%%%%%%%%%%%%%%%%%%%%%%%%%%%%%%%%%%%%%%%%%%%%%%%%%%%%%%%%%%%%%%%%%%%%%%

\noindent As a reminder, the Weak Law of Large Numbers states that $\overline{X}_n$ converges in probability to $\mu=E[X]$. What is convergence in probability? It is a notion of a sequence of random variables $\{X_n\}$ becoming ``close to" another ransom variable $X$. Here is the formal definition.

\begin{defn}
We say that the sequence of random variables $\{X_n\}$ {\bf\emph{converges in probability}} to $X$ if, for all $\epsilon>0$, we have
\[\ds\lim_{n\to\infty}\P\left(\big|X_n-X\big|>\epsilon\right)=0.\]
We denote this as $X_n\stackrel{p}{\longrightarrow}X$.
\end{defn}
\vskip 5mm
For the Weak law of Large Numbers the sequence $\{\overline{X}_n\}$ is converging to a constant random variable $\mu$.
\vskip 5mm
\begin{thm}[WLLN]
Let $X$ be a random variable with finite expected value; i.e., $\mu=E[X]<\infty$, and $X_1,X_2,\dots,X_n$ be an independent random sample from the population with distribution $X$. If we define the sample mean $\overline{X}_n=\ds\frac{X_1+\cdots+X_n}{n}$, we have $\overline{X}_n\stackrel{p}{\longrightarrow}\mu$; i.e., the sample mean converges in probability to the expected value $\mu$.
\end{thm}

\ni The following graphics (Figure 1 and Figure 2) should demonstrate the idea. We see running averages for 50 different simulations and an $\epsilon$ tolerance about the value $\mu=0$. As the sample sizes increase the probability of being outside of the shown $\epsilon$ band should decreases, and so we would expect to see a lower proportion of our sample means outside of the $\epsilon$ band about $\mu=0$.\\

\vskip 5mm

\begin{figure}[h!]
\begin{center}
\includegraphics[trim=0cm 2cm 0cm 2cm, clip=true, scale=0.22]{wlln_plot_1.jpeg}
\caption{WLLN - Sample Sizes of $n\le 1000$}
\end{center}
\end{figure}

\begin{figure}[h!]
\begin{center}
\includegraphics[trim=0cm 2cm 0cm 2cm, clip=true, scale=0.22]{wlln_plot_2.jpeg}\\
\caption{WLLN - Sample Sizes of $n\le 2000$}
\end{center}
\end{figure}

\vskip 2mm
But there are many other ways to talk about convergence of random variables. One of them is convergence in distribution.
\vskip 2mm

\begin{defn}
We say the sequence of random variables $\{X_n\}$ {\bf\emph{converges in distribution}} to $X$, denoted $X_n\stackrel{D}{\longrightarrow}X$, if 
\[\ds\lim_{n\to\infty}F_{_{X_n}}=F_X\;\text{for all}\;x\;\text{where}\;F_X\;\text{is continuous}.\]
\end{defn}
\vskip 5mm
Now for the main theorem.
\vskip 5mm
\begin{thm}[Lindeberg-Levy CLT]
Let $X_1,X_2,X_3,\dots$ be a sequence of i.i.d. random variables with finite $E[X_i]=\mu_{_X}$ and finite $\text{Var}[X_i]=\sigma^{2}_{_X}$, and let $\overline{X}_n=(X_1+\cdots+X_n)/n$. Then the sequence of random variables $\sqrt{n}\big(\overline{X}_n-\mu_{_X}\big)$ converges in distribution to a normal distribution $N(0,\sigma^{2}_{_X})$;
\[\sqrt{n}\left(\ds\frac{1}{n}\ds\sum^{n}_{i=1}X_i-\mu_{_X}\right)\stackrel{D}{\longrightarrow}N(0,\sigma^{2}_{_X}).\]
\end{thm}
\vskip 2mm
Alternately, if $\sigma^{2}_{_X}>0$, we could state that a standardized sample mean converges in distribution to a standard normal. That is,
\[\ds\frac{\overline{X}_n-\mu_{_X}}{\sigma_{X}/\sqrt{n}}\stackrel{D}{\longrightarrow}N(0,1).\]
That means, for any $z\in\R$ we have
\[\lim_{n\to\infty}\P\Big(\ds\frac{\big(\overline{X}_n-\mu_{_X}}{\sigma_x/\sqrt{n}}\le z\Big)=\Phi(z)\]
where $\Phi$ is the distribution function for the standard normal distribution.
\vskip 5mm
Here are some simulations of the density of the standardized $\overline{X}_n=X_1+X_2+\cdots+X_n$ with an exponential random variable $X\sim exp(2)$ for $n=3, 10, 100, 1000$. 
\vskip 5mm
\includegraphics[scale=0.42]{m3810_expdensity_clt_3.jpeg}\quad
\includegraphics[scale=0.42]{m3810_expdensity_clt_10.jpeg}
\vskip 5mm
\includegraphics[scale=0.42]{m3810_expdensity_clt_100.jpeg}\quad
\includegraphics[scale=0.42]{m3810_expdensity_clt_1000.jpeg}
\vskip 5mm
The statement about the Central Limit Theorem concerns the distribution function of the standardized sample mean. Here are some simulations of the distribution function of the standardized sample mean $\overline{X}_n=X_1+X_2+\cdots+X_n$ with an exponential random variable $X\sim exp(2)$ for $n=3, 10, 100, 1000$. 
\vskip 5mm
\hspace*{8mm}\includegraphics[scale=0.35]{m3810_expdist_clt_3.jpeg}\qquad
\includegraphics[scale=0.35]{m3810_expdist_clt_10.jpeg}
\vskip 5mm
\hspace*{8mm}\includegraphics[scale=0.35]{m3810_expdist_clt_100.jpeg}\qquad
\includegraphics[scale=0.35]{m3810_expdist_clt_1000.jpeg}

\vfill\eject

\ni The usefulness of the theorem is that the distribution of $\sqrt{n}\big(\overline{X}_n-\mu_{_X}\big)$ approaches normality regardless of the shape of the distribution of the individual $X_i$'s. Moreover, for large enough $n$ the random variable $S_n=X_1+\cdots+X_n$ will be approximately normal. To see this, while letting $Z\sim N(0,1)$, we have for large values of $n$
\[\ds\frac{\overline{X}_n-\mu_{_X}}{\sigma_{X}/\sqrt{n}}\approx Z\quad\Longrightarrow\quad S_n\approx\sigma\sqrt{n}Z+n\mu\] and $\sigma\sqrt{n}Z+n\mu\sim N(n\mu,\sigma^2n)$.
\vskip 5mm
This justifies the common assumption of normality when the randomness of a process can be attributed to a sum of random components. Let's consider a couple of examples, one continuous example and one discrete example. We start with a random variable $X$ with a probability density function that is a piecewise polynomial. The mean of this distribution is 0 and its standard deviation is 1.\\

\begin{figure}[h!]
\begin{center}
\includegraphics[scale=2]{clt_1.jpg}
\end{center}
\caption{Graph of Probability Density Function of $X$}
\end{figure}

\vskip 5mm

\ni Then assuming $X_1,X_2,X_3,X_4$ are i.i.d. each having the same distribution as $X$, using convolution we compute the density functions for $Y_2=X_1+X_2$, $Y_3=X_1+X_2+X_3$ and $Y_4=X_1+X_2+X_3+X_4$ (and then scale each so that they have variance one). You can see the corresponding graphs below.

\begin{figure}[h!]
\begin{center}
\includegraphics[scale=2]{clt_2.jpg}
\end{center}
\caption{Graph of Probability Density Function of $Y_2=X_1+X_2$}
\end{figure}

\begin{figure}[h!]
\begin{center}
\includegraphics[scale=2]{clt_3.jpg}
\end{center}
\caption{Graph of Probability Density Function of $Y_3=X_1+X_2+X_3$}
\end{figure}

\begin{figure}[h!]
\begin{center}
\includegraphics[scale=2]{clt_4.jpg}
\end{center}
\caption{Graph of Probability Density Function of $Y_4=X_1+X_2+X_3+X_4$}
\end{figure}

\vfill\eject

\ni This density appears qualitatively very similar to a normal density after only summing four copies of the random variable $X$.\\

\ni We next consider a discrete example. Suppose that $X$ is a discrete uniform random variable with $Im(X)=\{1,2,3\}$. Then we have
\[p_{_X}(x)=\left\{\begin{array}{rcl}
1/3&:&x=1,2,3\\
0&:&\text{Otherwise}
\end{array}\right.\]
Then let $Y=X_1+X_2$ and $W=X_1+X_2+X_3$ where $X_1,X_2,X_3$ are i.i.d having the same distribution as $X$. Then we have
\[p_{_Y}(y)=\left\{\begin{array}{rcl}
1/9&:&y=2,6\\
2/9&:&y=3,4\\
3/9&:&y=5\\
0&:&\text{Otherwise}
\end{array}\right.\]
and
\[p_{_W}(w)=\left\{\begin{array}{rcl}
1/27&:&w=3,9\\
3/27&:&w=4,8\\
6/27&:&w=5,7\\
7/27&:&w=6\\
0&:&\text{Otherwise}
\end{array}\right.\]
The degree of the resemblance to the bell-shaped curve can be quantified as follows. Consider
\[\P(W\le 7)=\P(X_1+X_2+X_3\le 7)=\ds\frac{1}{27}+\ds\frac{3}{27}+\ds\frac{6}{27}+\ds\frac{7}{27}+\ds\frac{6}{27}=\ds\frac{23}{27}\approx 0.85185.\]
Then since $E[W]=6$ and $\text{Var}[W]=2$, and using a continuity correction, we compute
\[0.85185\approx\P(W\le 7.5)=\P\left(\ds\frac{W-6}{\sqrt{2}}\le\ds\frac{7.5-6}{\sqrt{2}}\right)=\P\left(\ds\frac{W-6}{\sqrt{2}}\le 1.0606602\right).\]
Then if $Z\sim N(0,1)$ we have.
\[\P(Z\le 1.0606602)=\Phi(1.0606602)\approx 0.85558.\]
This example shows how the distribution for the standardized $\big(W-E[W]\Big)/\sigma_{_W}$ has probabilities very close to those of the standard normal distribution.
 


%%%%%%%%%%%%%%%%%%%%%%%%%%%%%%%%%%%%%%%%%%%%%%%%%%%%%%%%%%%%%%%%%%%%%%%%%%%%%%%%%%%%%%%%%%%%%%%%%%%%%%
\section*{Calculations:}
%%%%%%%%%%%%%%%%%%%%%%%%%%%%%%%%%%%%%%%%%%%%%%%%%%%%%%%%%%%%%%%%%%%%%%%%%%%%%%%%%%%%%%%%%%%%%%%%%%%%%%

\noindent We now return to the pollster problem. Suppose we want $\P\big(\big|\hat{p}_{_n}-p\big|<E\big)=0.05$ for some tolerance $E$. By the Central Limit Theorem we have
\[\ds\frac{\hat{p}_{_n}-p}{\sqrt{p(1-p)/n}}\sim N(0,1)\]
Therefore will look at the tails of the normal distribution and we have
\[E=z_{_{0.975}}\sqrt{\ds\frac{p(1-p)}{n}}\quad\Longrightarrow\quad n=\ds\frac{z^2_{_{0.975}}p(1-p)}{E^2}.\]
	\begin{figure}[h!]
		\begin{center}
			\includegraphics[scale=0.5]{normal_dist_intervals.png}
		\end{center}
	\caption{Z-scores for Normal Distribution}
	\end{figure}
	
Since $p$ is unknown we can choose $p=0.5$ for the most conservative estimate and use $z_{0.975}=1.96$. Then for $E=0.03$ we have a conservative estimate
\[n=\ds\frac{(1.96)^2\cdot 0.25}{(0.03)^2}\approx 1067.111.\]
So a sample size of $n=1068$ would give us $\P\big(\big|\hat{p}_{_n}-p\big|<0.03\big)\le0.05$.

%%%%%%%%%%%%%%%%%%%%%%%%%%%%%%%%%%%%%%%%%%%%%%%%%%%%%%%%%%%%%%%%%%%%%%%%%%%%%%%%%%%%%%%%%%%%%%%%%%%%%%
\section*{The R Code:} 
%%%%%%%%%%%%%%%%%%%%%%%%%%%%%%%%%%%%%%%%%%%%%%%%%%%%%%%%%%%%%%%%%%%%%%%%%%%%%%%%%%%%%%%%%%%%%%%%%%%%%%
 Here is the R code used for the above simulations.
\vspace*{2mm}
\small 
\begin{lstlisting}[language=R]

# ==========================================
# Weak Law of Large Numbers Demonstration
# ==========================================

set.seed(123)   # for reproducibility

k <- 1000       # number of observations per sample
j <- 50         # number of independent sequences
e <- 0.04       # epsilon threshold

# Generate matrix of random variables
x <- matrix(2 * rbinom(k * j, size = 1, prob = 0.5) - 1, ncol = j)

# Compute running averages for each column
y <- apply(x, 2, function(z) cumsum(z) / seq_along(z))

# Plot running averages
matplot(y, type = "l",
        lty = 1, col = rainbow(j),
        ylim = c(-0.4, 0.4),
        main = "Weak Law of Large Numbers Simulation",
        xlab = "Number of trials (n)",
        ylab = "Running average of Xn")

# Add horizontal lines for epsilon and 0
abline(h = c(-e, e), lty = 2, lwd = 2, col = "red")
abline(h = 0, lty = 1, lwd = 2, col = "black")

# Add legend
legend("topright",
       legend = c(expression(paste("|X| <", epsilon)), "Mean (0)"),
       col = c("red", "black"),
       lty = c(2, 1),
       bty = "n")



# ==========================================
# Central Limit Theorem Simulation (Exponential Distribution)
# ==========================================


# --- User-defined parameters ---
lambda <- 2       # rate parameter of exponential
n <- 1000           # sample size per mean
m <- 1000         # number of replications
hist_color <- "orchid"  # histogram color


# --- CLT Simulation ---
# set.seed(123)  # for reproducibility


mu <- 1 / lambda
sigma2 <- 1 / (lambda^2)

# Generate m sample means of size n
data <- replicate(m, (mean(rexp(n, lambda)) - mu) / sqrt(sigma2 / n))

# --- Plotting ---
hist(data,
     prob = TRUE,
     breaks = "FD",
     col = hist_color,
     border = "white",
     main = bquote("CLT Simulation: Standardized Mean of Exponential
     (" * .(lambda) * ") with n = " * .(n)),
     xlab = "Standardized sample mean (Z)",
     ylab = "Density",
     cex.main = 1.1,
     xlim = c(-4, 4))

# Overlay theoretical normal curve
curve(dnorm(x), from = -4, to = 4, col = "darkblue", lwd = 2, add = TRUE)

# Add grid and legend
grid(col = "gray80", lty = 1)
legend("topright",
       legend = c("Simulated standardized means", "Standard Normal PDF"),
       fill = c(hist_color, NA),
       border = c(NA, NA),
       col = c(NA, "darkblue"),
       lwd = c(NA, 2),
       bty = "n")



# =======================================================
# CLT Simulation: Distribution Function for Exponential
# ======================================================


# --- User-defined parameters ---
lambda <- 2    # rate parameter
n <- 1000        # sample size per mean
m <- 1000      # number of repetitions
line_color <- "orchid4"

# --- Simulation ---
set.seed(123)
mu <- 1 / lambda
sigma2 <- 1 / (lambda^2)
X <- replicate(m, (mean(rexp(n, lambda)) - mu) / sqrt(sigma2 / n))

# --- Plot Empirical and Theoretical CDFs ---
plot(ecdf(X),
     col = line_color,
     main = bquote("CLT Simulation: ECDF of Standardized Mean of Exponential
     (" * .(lambda) * ") with n = " * .(n)),
     xlab = "Standardized sample mean (Z)",
     ylab = "Cumulative Probability",
     cex.main = 1.1,
     lwd = 2)

# Overlay theoretical standard normal CDF
curve(pnorm(x), from = -4, to = 4, col = "darkorange", lwd = 2, add = TRUE)

# Add grid and legend
grid(col = "gray80", lty = 1)
legend("bottomright",
       legend = c("Empirical CDF (Simulated)", "Standard Normal CDF"),
       col = c(line_color, "darkorange"),
       lwd = 2,
       bty = "n")

\end{lstlisting}



\vskip 1cm
\hrule
\vskip 5mm
\begin{center}
\bf Please let me know if you have any questions, comments, or corrections!
\end{center}

%%%%%%%%%%%%%%%%%%%%%%%%%%%%%%%%%%%%%%%%%%%%%%%%%%%%%%
\end{document}
%%%%%%%%%%%%%%%%%%%%%%%%%%%%%%%%%%%%%%%%%%%%%%%%%%%%%%