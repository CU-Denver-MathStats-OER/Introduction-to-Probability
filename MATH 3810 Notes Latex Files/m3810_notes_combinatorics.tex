%%%%%%%%%%%%%%%%%%%%%%%%%%%%%%%%%%%%%%%%%%%%%%%%%%%%%%
\documentclass[11pt]{article}
%%%%%%%%%%%%%%%%%%%%%%%%%%%%%%%%%%%%%%%%%%%%%%%%%%%%%%

\usepackage{amsmath}
\usepackage{amsthm}
\usepackage{amssymb}
\usepackage{latexsym}
\usepackage{graphicx}
\usepackage{color}
\usepackage{verbatim}
\usepackage{float}
\usepackage{multicol}
\usepackage{xcolor}
\usepackage{listings}
\usepackage{tikz}
\usetikzlibrary{arrows.meta, positioning, calc}
\usetikzlibrary{decorations.pathmorphing}
\usepackage{tcolorbox}
\tcbuselibrary{breakable}
\usepackage{cancel}


\newtcolorbox{solutionbox}{
  breakable,
  colback=blue!5!white,
  colframe=blue!50!black,
  title=Solution,
  sharp corners,
  boxrule=0.8pt
}

\newtcolorbox{hintbox}{
  breakable,
  colback=gray!10!white,
  colframe=gray!50!black,
  title=Hint,
  sharp corners,
  boxrule=0.5pt
}

% Unnumbered theorem
\newtheorem*{thm*}{Theorem}

\lstdefinelanguage{R}{
      keywords={if,else,while,for,in,next,break,function,TRUE,FALSE,NULL,Inf,NA,NaN,switch,repeat,return,require,library},
      keywordstyle=\color{blue}\bfseries,
      identifierstyle=\color{black},
      comment=[l]{\#},
      commentstyle=\color{gray}\ttfamily,
      string=[b]{"},
      stringstyle=\color{red}\ttfamily,
      morecomment=[l]{//},
      morestring=[b]{'},
      sensitive=true,
      morekeywords={print,summary,plot,lm,glm,data,frame,read.csv,write.csv,factor,levels,names,colnames,rownames,
      head,tail,str,dim,length,class,typeof,mode,is.na,is.null,is.finite,is.infinite,is.nan,as.numeric,as.character,
      as.factor,as.Date,as.POSIXct,as.matrix,as.data.frame,rbind,cbind,merge,subset,aggregate,tapply,apply,lapply,sapply,
      mapply,vapply,replicate,seq,rep,c,list,matrix,array,data.frame,table,hist,boxplot,barplot,pie,curve,lines,points,text,
      abline,legend,par,mtext,title,xlab,ylab,xlim,ylim,main,sub,col,pch,cex,lty,lwd,type,bg,fg,args,options,warnings,errors,
      message,stop,warning,error,try,tryCatch,withCallingHandlers,on.exit,debug,browser,trace,recover,options,getOption,setOption},
    }


\setlength{\textheight}{9in}
\setlength{\textwidth}{6in}
\addtolength{\topmargin}{-2cm}
\addtolength{\oddsidemargin}{-1cm}
\parindent=0in


\def\classnum{3810}
\def\classtitle{Probability}
\def\classtitleshort{Probability}
\def\classsec{001}
\def\classterm{Fall 2025}
\def\instructor{Robert Rostermundt}
%\def\hmwknum{\#2}


%%%%%%%%%%%%%%%%%%%%%%%%%%%%%%%%%%%%%%%%%%%%%%%%%%%%%%%%%
%%%%%%%%%%%%%%%%%%%%%%%%%  Colors  %%%%%%%%%%%%%%%%%%%%%%
%%%%%%%%%%%%%%%%%%%%%%%%%%%%%%%%%%%%%%%%%%%%%%%%%%%%%%%%%

\definecolor{Green}{rgb}{0,.5,0}
%use for definitions
\definecolor{Red}{rgb}{.8,.2,0}
%use for emphasis
\definecolor{Yellow}{rgb}{.6,.6,.1}
%use for part titles
\definecolor{Cyan}{rgb}{.2,.6,.7}
%use for comments
\definecolor{Purple}{rgb}{.4,0,1}
%use for examples
\definecolor{deepred}{rgb}{.53,.29,.24}
%use for important points
\definecolor{Black}{rgb}{0,0,0}
%use for washout
\definecolor{Grey}{rgb}{.45,.45,.45}
% use for theorems
\newcommand{\tred}[1]{\textcolor{Red}{#1}}
\newcommand{\tgreen}[1]{\textcolor{Green}{#1}}
\newcommand{\tcyan}[1]{\textcolor{Cyan}{#1}}
\newcommand{\tyellow}[1]{\textcolor{Yellow}{#1}}
\newcommand{\tpurple}[1]{\textcolor{Purple}{#1}}
\newcommand{\tblack}[1]{\textcolor{Black}{#1}}
\newcommand{\tgrey}[1]{\textcolor{Grey}{#1}}
\newcommand{\tdeepred}[1]{\textcolor{deepred}{#1}}
\newcommand{\ttt}[1]{\texttt{#1}}

%%%%%%%%%%%%%%%%%%%%%%%%%%%%%%%%%%%%%%%%%%%%%%%%%%%%%%%%%
%%%%%%%%%%%%%%%%%%%%%%%%%  Theorem Environments  %%%%%%%%
%%%%%%%%%%%%%%%%%%%%%%%%%%%%%%%%%%%%%%%%%%%%%%%%%%%%%%%%%

\theoremstyle{plain}
\newtheorem{thm}{Theorem}
\newtheorem{axiom}{Axiom}
\newtheorem{cor}{Corollary}
\newtheorem{lemma}{Lemma}
\newtheorem{prop}{Proposition}
\newtheorem{ques}{Question}
\theoremstyle{definition}
\newtheorem{defn}{Definition}
\theoremstyle{remark}
\newtheorem{remark}{Remark}
\theoremstyle{definition}
\newtheorem{ex}{Example}
\numberwithin{equation}{section}
\newtheorem{prob}{Problem}
\numberwithin{equation}{section}


%%%%%%%%%%%%%%%%%%%%%%%%%%%%%%%%%%%%%%%%%%%%%%%%%%%%%%%%%
%%%%%%%%%%%%%%%%%%%%%%%%%  Math    %%%%%%%%%%%%%%%%%%%%%%
%%%%%%%%%%%%%%%%%%%%%%%%%%%%%%%%%%%%%%%%%%%%%%%%%%%%%%%%%


\newcommand{\abs}[1]{\left\lvert{#1}\right\rvert}
\newcommand{\card}[1]{\lvert{#1}\rvert}
\newcommand{\union}{\cup}
\newcommand{\Union}{\bigcup}
\newcommand{\inter}{\cap}
\newcommand{\Inter}{\bigcap}
%\newcommand{\hint}[1]{\medskip\newline\emph{Hint: #1}}
%\newcommand{\note}[1]{\medskip\newline\emph{Note: #1}}
\newcommand{\points}[1]{[#1 points]}
\newcommand{\totalpoints}[1]{[#1 points total]}
\newcommand{\ds}{\displaystyle}
\newcommand{\ben}{\begin{enumerate}}
\newcommand{\een}{\end{enumerate}}
\newcommand{\bi}{\begin{itemize}}
\newcommand{\ei}{\end{itemize}}
\newcommand{\beq}{\begin{eqnarray*}}
\newcommand{\eeq}{\end{eqnarray*}}
\newcommand{\bieq}{\begin{IEEEeqnarray}{rCl}}
\newcommand{\bieqx}{\begin{IEEEeqnarray}{+rCl+x*}}
\newcommand{\eieq}{\end{IEEEeqnarray}}
\newcommand{\nn}{\nonumber}
%\renewcommand{\i}{\item}
\newcommand{\bpm}{\begin{pmatrix}}
\newcommand{\epm}{\end{pmatrix}}
\newcommand{\sol}{\indent{\bf\emph{Solution:}}}
\newcommand{\ssol}{\indent{\\[2mm]\bf\emph{Solution:}}\;}
\newcommand{\hint}{\indent{\bf\emph{Hint}:}\;}
\newcommand{\note}{\indent{\bf\emph{Note}:}\;}
\newcommand{\vsk}{\vskip 2mm}
%%%%%%%%%%%%%%%%%%%%%%%%% Calculus %%%%%%%%%%%%%%%%%%%%%%%%%%%%
\newcommand{\dd}[2]{\ds\frac{d}{d{#1}}\left[{#2}\right]}
\newcommand{\der}[2]{\ds\frac{d{#1}}{d{#2}}}
\newcommand{\lmt}[3]{\ds\lim_{{#1}\to{#2}}{#3}}
\renewcommand{\iint}[2]{\ds\int{#1}\,d{#2}}
\newcommand{\dint}[4]{\ds\int^{#4}_{#3}{#1}\,d{#2}}
\renewcommand{\Delta}{\triangle}
%%%%%%%%%%%%%%%%%%%%%%%%% Number Sets %%%%%%%%%%%%%%%%%%%%%%%%%%
\newcommand{\N}{\mathbb{N}}
\newcommand{\Z}{\mathbb{Z}}
\newcommand{\Q}{\mathbb{Q}}
\newcommand{\R}{\mathbb{R}}
\newcommand{\C}{\mathbb{C}}
\newcommand{\F}{\mathcal{F}}
\renewcommand{\P}{\mathbb{P}}
\newcommand{\E}{\mathcal{E}}
\renewcommand{\o}{\omega}
\renewcommand{\O}{\Omega}
%%%%%%%%%%%%%%%%%%%%%%%%% Vectors %%%%%%%%%%%%%%%%%%%%%%%%%%%%%
\newcommand{\x}{\bar{x}}
\renewcommand{\v}{\bar{v}}
\newcommand{\y}{\bar{y}}
\newcommand{\z}{\bar{z}}
\newcommand{\w}{\bar{w}}
\renewcommand{\u}{\bar{u}}
\renewcommand{\b}{\bar{b}}
\newcommand{\e}{\bar{e}}
\renewcommand{\a}{\vec{a}}
\renewcommand{\r}{\vec{r}}
\newcommand{\vv}{\vec{v}}
\newcommand{\vecPQ}[2]{\overrightarrow{#1}{#2}}
\newcommand{\vecV}[1]{\overrightarrow{#1}}
\newcommand{\la}{\langle}
\newcommand{\ra}{\rangle}
%%%%%%%%%%%%%%%%%%%%%%%%%%% Vector Spaces %%%%%%%%%%%%%%%%%%%%
\newcommand{\rn}{\ensuremath{\mathbb{R}^n}}
\renewcommand{\rm}{\ensuremath{\mathbb{R}^m}}
\newcommand{\re}{\mathbb{R}}
\newcommand{\Pn}{\mathbb{P}_n}
\newcommand{\B}{\mathcal{B}}
%%%%%%%%%%%%%%%%%%%%%%%%%%% Graphics %%%%%%%%%%%%%%%%%%%%%%%%
\newcommand{\cg}[2]{\begin{center}
\includegraphics[scale={#1}]{{#2}}
\end{center}}
\makeatletter
\def\imod#1{\allowbreak\mkern10mu({\operator@font mod}\,\,#1)}
\makeatother

%%%%%%%%%%%%%%%%%%%%%%%%%%%%%%%%%%%%%%%%%%%%%%%%%%%%%%%%%%%%%%%%%%%%%%%%%%%%%%%%%%%%%%%%%%%%%%
%%%%%%%%%%%%%%%%%%%%%%%%%%%%%% Defined Fonts %%%%%%%%%%%%%%%%%%%%%%%%%%%%%%%%%%%%%%%%%%%%%%%%%
%%%%%%%%%%%%%%%%%%%%%%%%%%%%%%%%%%%%%%%%%%%%%%%%%%%%%%%%%%%%%%%%%%%%%%%%%%%%%%%%%%%%%%%%%%%%%%

\font\minihelv=phvr at 6pt
\font\helv=phvr at 10pt
\font\medhelv=phvr at 16pt
\font\bighelv=phvr at 20pt
\font\hugehelv=phvr at 36pt
\font\mybigfont=phvr at 16pt
\font\mymediumfont=phvr at 14pt
\font\mediumhelv=phvr at 14pt
\font\mybfit=ptmbi at 12pt


%%%%%%%%%%%%%%%%%%%%%%%%%%%%%%%%%%%%%%%%%%%%%%%%%%%%%%%%%%%%%%%%%%%%%%%%%%%%%%%%%%%%%%%%%%%%%%%
%%%%%%%%%%%%%%%%%%%%%%%%%%%%%% Other Commands %%%%%%%%%%%%%%%%%%%%%%%%%%%%%%%%%%%%%%%%%%%%%%%%%
%%%%%%%%%%%%%%%%%%%%%%%%%%%%%%%%%%%%%%%%%%%%%%%%%%%%%%%%%%%%%%%%%%%%%%%%%%%%%%%%%%%%%%%%%%%%%%%
%\setlength\fboxrule{.5pt}
%\newcommand{\latexpicborder}[3]{
%\setlength\fboxsep{30pt}
%\begin{figure}[hb]
%\begin{center}
%\fbox{
%\input{#1}
%}
%\caption{#2}
%\label{#3}
%\end{center}
%\end{figure}
%\setlength\fboxsep{0pt}
%}
%
%\newcommand{\latexpic}[2]{
%\begin{figure}[hb]
%\begin{center}
%\input{#1}
%\vspace*{8mm}
%\caption{#2}
%\end{center}
%\end{figure}
%}

%\begin{minipage}[b]{0.6\linewidth}
%......
%\end{minipage}
%\hspace{0.5cm}
%\begin{minipage}[t]{0.4\linewidth}
%\centering
%\includegraphics[scale=.5]{m1401_ex3_g4.eps}
%\end{minipage}
%\end{figure}


%%%%%%%%%%%%%%%%%%%%%%%%%%%%%%%%%%%%%%%%%%%%%%%%%%%%%%%%%%%%%%%%%%%%%%%%%%%%%%%%%%%%%%%%%%%%%%
%%%%%%%%%%%%%%%%%%%%%%%%%%% IEEEeqnarray Notes %%%%%%%%%%%%%%%%%%%%%%%%%%%%%%%%%%%%%%%%%%%%%%%
%%%%%%%%%%%%%%%%%%%%%%%%%%%%%%%%%%%%%%%%%%%%%%%%%%%%%%%%%%%%%%%%%%%%%%%%%%%%%%%%%%%%%%%%%%%%%%


%Any number of columns can be specified with IEEEeqnarray: {c} will give only one
%column with all entries centered, or {rCll} would add a fourth, left-justified
%column to use for comments. Moreover, beside l, c, r, L, C, R for math mode
%entries there are also s, t, u for left, centered, and right text mode entries.
%Additional space can be added with . and / and ? in increasing order.
%
%
%\begin{proof}
%This is a proof that ends
%with an equation array:
%\begin{IEEEeqnarray*}{+rCl+x*}
%a & = & b + c \\
%& = & d + e. & \qedhere
%\end{IEEEeqnarray*}
%\end{proof}
%Note that the + in {+rCl+x*} denotes stretchable spaces, one on the left
%of the equations (which, if not specified, will be done automatically by
%IEEEeqnarray!) and one on the right of the equations. But now on the right,
%after the stretching column, we add an empty column x. This column will be
%only needed on the last line when we will put the \qedhere command there.
%Finally, we specify a *. This is a null-space that prevents IEEEeqnarray to
%add another unwanted +-space!


% The following environments enable custom numbering of theorems so that the numbers agree % with the numbering in the textbook being used. 
%
%  Usage examples:
%\begin{customthm}{2.2}\label{eight}
%Every theorem must be numbered by hand.
%\end{customthm}
%
%Here is a reference to theorem~\ref{eight}.
%
%\begin{customthm}{2.3}[Parenthetical comment]\label{nine}
%Statement
%\end{customthm}
%
%Here is a reference to theorem~\ref{nine}


\newtheorem{innercustomthm}{Theorem}
\newenvironment{customthm}[1]
  {\renewcommand\theinnercustomthm{#1}\innercustomthm}
  {\endinnercustomthm}
  
  \newtheorem{innercustomprop}{Proposition}
\newenvironment{customprop}[1]
  {\renewcommand\theinnercustomprop{#1}\innercustomprop}
  {\endinnercustomprop}
  
    \newtheorem{innercustomlem}{Lemma}
\newenvironment{customlem}[1]
  {\renewcommand\theinnercustomlem{#1}\innercustomlem}
  {\endinnercustomlem}
  
    \newtheorem{innercustomconj}{Conjecture}
\newenvironment{customconj}[1]
  {\renewcommand\theinnercustomconj{#1}\innercustomconj}
  {\endinnercustomconj}
  
    \newtheorem{innercustomclaim}{Claim}
\newenvironment{customclaim}[1]
  {\renewcommand\theinnercustomclaim{#1}\innercustomclaim}
  {\endinnercustomclaim}
  
    \newtheorem{innercustomcor}{Corollary}
\newenvironment{customcor}[1]
  {\renewcommand\theinnercustomcor{#1}\innercustomcor}
  {\endinnercustomcor}
  
    \newtheorem{innercustomdef}{Definition}
\newenvironment{customdef}[1]
  {\renewcommand\theinnercustomdef{#1}\innercustomdef}
  {\endinnercustomdef}
  
    \newtheorem{innercustomex}{Example}
\newenvironment{customex}[1]
  {\renewcommand\theinnercustomex{#1}\innercustomex}
  {\endinnercustomex}
  
    \newtheorem{innercustomass}{Assumption}
\newenvironment{customass}[1]
  {\renewcommand\theinnercustomass{#1}\innercustomass}
  {\endinnercustomass}
  
      \newtheorem{innercustomax}{Axiom}
\newenvironment{customax}[1]
  {\renewcommand\theinnercustomax{#1}\innercustomax}
  {\endinnercustomax}
  

\vfuzz2pt % Don't report over-full v-boxes if over-edge is small
\hfuzz2pt % Don't report over-full h-boxes if over-edge is small

\renewcommand{\ni}{\noindent}


%%%%%%%%%%%%%%%%%%%%%%%%%%%%%%%%%%%%%%%%%%%%%%%%%%%%%%
%%%%%%%%%%%%%%%%%%%%%%%%%%%%%%%%%%%%%%%%%%%%%%%%%%%%%%

\pagestyle{myheadings}

%%%%%%%%%%%%%%%%%%%%%%%%%%%%%%%%%%%%%%%%%%%%%%%%%%%%%%

%%%%%%%%%%%%%%%%%%%%%%%%%%%%%%%%%%%%%%%%%%%%%%%%%%%%%%
%%%%%%%%%%%%%%%%%%%%%%%%%   Document Body   %%%%%%%%%%
%%%%%%%%%%%%%%%%%%%%%%%%%%%%%%%%%%%%%%%%%%%%%%%%%%%%%%

%\def\classnum{3810}
%\def\classtitle{Probability}
%\def\classtitleshort{Probability}
%\def\classsec{001}
%\def\classterm{Fall 2025}
%\def\instructor{Robert Rostermundt}
\def\printsol{0}


	\title{\vspace{-1in}Math\classnum\;-\;\classtitle\\
	Section\;\classsec\;-\;\classterm\\
	Notes: Combinatorics}
	\author{University of Colorado Denver / College of Liberal Arts 	and Sciences}
	\date{Department of Mathematics - \instructor}

	\markright{Math\classnum\;-\;\classtitleshort, UCD, \classterm, \instructor}



%%%%%%%%%%%%%%%%%%%%%%%%%%%%%%%%%%%%%%%%%%%%%%%%%%%%%%
\begin{document}\maketitle\thispagestyle{empty}
%%%%%%%%%%%%%%%%%%%%%%%%%%%%%%%%%%%%%%%%%%%%%%%%%%%%%%


%%%%%%%%%%%%%%%%%%%%%%%%%%%%%%%%%%%%%%%%%%%%%%%%%%%%%%%%%%%%%%%%%%%%%%%%%%%%%%%%%%%%%%%%%%%%%%%%%%%%%%
\vspace*{2mm}
\hrule
\vskip 8mm


%%%%%%%%%%%%%%%%%%%%%%%%%%%%%%%%%%%%%%%%%%%%%%%%%%%%%%%%%%%%%%%%%%%%%%%%%
%%%%%%%%%%%%%%%%%%%%%%%%%%%%%%%%%%%%%%%%%%%%%%%%%%%%%%%%%%%%%%%%%%%%%%%%%


%%%%%%%%%%%%%%%%%%%%%%%%%%%%%%%%%%%%%%%%%%%%%%%%%%%%%%%%%%%%%%%%%%%%%%%%%%%%%%%%%%%%%%%%%%%%%%%%%%%%%%
\section*{The Problem:}
%%%%%%%%%%%%%%%%%%%%%%%%%%%%%%%%%%%%%%%%%%%%%%%%%%%%%%%%%%%%%%%%%%%%%%%%%%%%%%%%%%%%%%%%%%%%%%%%%%%%%%

\noindent Suppose that $\big(\Omega,\mathcal F,\P\big)$ is a probability space where $|\Omega|<\infty$.
If we assume that all outcomes are equally likely, then $\P$ is a discrete uniform probability
measure and for all events $A\in\mathcal F$ we have
\[
\P(A)=\frac{|A|}{|\Omega|}.
\]
Thus, computing probabilities in finite uniform sample spaces reduces to counting the number of
outcomes in sets. The mathematical tools used to determine the size of finite sets fall under the
study of \emph{combinatorics}. In this section we introduce the basic counting principles needed
for applied probability.

%%%%%%%%%%%%%%%%%%%%%%%%%%%%%%%%%%%%%%%%%%%%%%%%%%%%%%%%%%%%%%%%%%%%%%%%%%%%%%%%%%%%%%%%%%%%%%%%%%%%%%
\section*{Theoretical Tools:}
%%%%%%%%%%%%%%%%%%%%%%%%%%%%%%%%%%%%%%%%%%%%%%%%%%%%%%%%%%%%%%%%%%%%%%%%%%%%%%%%%%%%%%%%%%%%%%%%%%%%%%

Below are the basic counting principles we will need for this course.
\vskip 5mm
\begin{thm}[Product Rule]
Suppose a task can be completed in $k$ stages. If stage $i$ can be completed in $n_i$ ways,
independently of the choices made in the other stages, then the total number of ways to
complete the task is
\[n_1 n_2 \cdots n_k.\]
\end{thm}
\vskip 2mm
\noindent\emph{Example.} A password consists of two letters followed by three digits.
There are $26$ choices for each letter and $10$ choices for each digit. By the product rule,
the total number of possible passwords is
\[26^2 \cdot 10^3.\]
\vskip 5mm

\begin{defn}[Permutation]
A \emph{permutation} of a set of $n$ distinct objects is an ordering of those objects.
\end{defn}
\vskip 2mm
\noindent\emph{Example.} If three students Alice, Bob, and Carol line up, then
$(\text{Alice},\text{Bob},\text{Carol})$ and $(\text{Bob},\text{Alice},\text{Carol})$
represent two different permutations of the same set of students.
\vskip 5mm

\begin{thm}[Number of Permutations]
The number of permutations of $n$ distinct objects is
\[n! = n(n-1)(n-2)\cdots 2\cdot 1.\]
\end{thm}
\vskip 2mm
\noindent\emph{Example.} The number of ways to arrange $5$ distinct books on a shelf is
\[5! = 120.\]
\vskip 5mm

\begin{thm}[$k$-Permutations]
The number of ordered selections of $k$ distinct objects chosen from a set of $n$ distinct
objects is
\[P(n,k)=\frac{n!}{(n-k)!}.\]
\end{thm}
\vskip 2mm
\noindent\emph{Example.} Suppose $10$ runners compete in a race. The number of possible
ways to award gold, silver, and bronze medals is
\[P(10,3)=\frac{10!}{7!}=10\cdot 9\cdot 8.\]
\vskip 5mm

\begin{defn}[Combination]
A \emph{combination} is a selection of objects where order does not matter.
\end{defn}
\vskip 2mm
\noindent\emph{Example.} Choosing students $\{\text{Alice},\text{Bob},\text{Carol}\}$ to form
a committee is the same selection regardless of the order in which the names are listed.
\vskip 5mm
\begin{thm}[Combination Formula]
The number of ways to choose $k$ objects from a set of $n$ distinct objects is
\[\binom{n}{k}=\frac{n!}{k!(n-k)!}.\]
\end{thm}
\vskip 5mm

\noindent\emph{Example.} The number of ways to choose a committee of $3$ people from $10$ is
\[
\binom{10}{3}=120.
\]

\begin{defn}[Sampling Methods]
When selecting a sample from a set of objects there are two basic sampling methods:
	\begin{itemize}
		\item \textbf{Sampling without replacement:} Once an object is selected, it cannot be selected again.
Counting typically involves permutations or combinations.
		\item \textbf{Sampling with replacement:} Objects may be selected multiple times. Counting typically
uses the product rule.
	\end{itemize}
\end{defn}

\noindent\emph{Example.} Drawing $5$ cards from a deck without replacement leads to combinations,
since order does not matter. In contrast, rolling a die $3$ times is sampling with replacement,
and the number of possible outcomes is $6^3$.

%%%%%%%%%%%%%%%%%%%%%%%%%%%%%%%%%%%%%%%%%%%%%%%%%%%%%%%%%%%%%%%%%%%%%%%%%%%%%%%%%%%%%%%%%%%%%%%%%%%%%%
\section*{Proofs:}
%%%%%%%%%%%%%%%%%%%%%%%%%%%%%%%%%%%%%%%%%%%%%%%%%%%%%%%%%%%%%%%%%%%%%%%%%%%%%%%%%%%%%%%%%%%%%%%%%%%%%%

\paragraph*{Proof of the Product Rule:}

Suppose a task consists of two stages. If stage 1 can be completed in $n_1$ ways and stage 2
can be completed in $n_2$ ways independently of stage 1, then for each of the $n_1$ choices in stage
1 there are $n_2$ choices in stage 2. Hence, there are $n_1 n_2$ total outcomes.
The general case follows by induction.

\paragraph*{Proof that the Number of Permutations is $n!$:}

\noindent To form a permutation of $n$ distinct objects, we choose
\begin{itemize}
\item one of $n$ objects for the first position,
\item one of the remaining $n-1$ objects for the second position,
\item $\vdots$
\item one object for the final position.
\end{itemize}
By the product rule, the total number of permutations is
\[
n(n-1)\cdots 1 = n!.
\]

\paragraph*{Proof of the Combination Formula:}

\noindent First count the number of ordered selections of $k$ distinct objects from $n$:
\[
P(n,k)=\frac{n!}{(n-k)!}.
\]
Each unordered selection of $k$ objects corresponds to exactly $k!$ orderings. Dividing by $k!$
yields
\[
\binom{n}{k}=\frac{P(n,k)}{k!}=\frac{n!}{k!(n-k)!}.
\]
\vskip 1cm

%%%%%%%%%%%%%%%%%%%%%%%%%%%%%%%%%%%%%%%%%%%%%%%%%%%%%%%%%%%%%%%%%%%%%%%%%%%%%%%%%%%%%%%%%%%%%%%%%%%%%%
\section*{Example Problems:}
%%%%%%%%%%%%%%%%%%%%%%%%%%%%%%%%%%%%%%%%%%%%%%%%%%%%%%%%%%%%%%%%%%%%%%%%%%%%%%%%%%%%%%%%%%%%%%%%%%%%%%

\subsection*{Basic Counting Examples}

	\begin{enumerate}
		\item \textbf{Passwords.} How many 4-letter passwords can be formed using the English alphabet if
repetition is allowed?

	\emph{Solution.} Each position has 26 choices. By the product rule, the number of passwords is
\[26^4.\]

		\item \textbf{Arrangements.} How many ways can 5 students line up in a row?

\emph{Solution.} This is the number of permutations of 5 objects:
\[5! = 120.\]

		\item \textbf{Committees.} How many ways can a committee of 3 people be chosen from a group of 10?

\emph{Solution.} Since order does not matter, the number of committees is
\[\binom{10}{3} = 120.\]

		\item \textbf{License Plates.} A license plate consists of two letters followed by three digits.
How many different license plates are possible if repetition is allowed?

		\item \textbf{Ordered Selections.} From a group of 8 people, how many ways are there to choose a
president, vice president, and treasurer?

		\item \textbf{Unordered Selections.} From the same group of 8 people, how many distinct committees
of 3 people can be formed?

		\item \textbf{Sampling With Replacement.} A die is rolled 4 times. How many possible sequences of
outcomes are there?

		\item \textbf{Sampling Without Replacement.} A bag contains 6 distinct balls. Three balls are
drawn one at a time without replacement. How many possible ordered outcomes are there?

		\item \textbf{Order Matters or Not?} For each of the following situations, determine whether order
matters and compute the number of possible outcomes:
			\begin{enumerate}
				\item Choosing 2 cards from a standard deck.
				\item Awarding gold and silver medals in a race with 10 runners.
				\item Selecting a 4-digit PIN where digits may repeat.
			\end{enumerate}

		\item \textbf{Conceptual Question.} Explain why selecting a committee of 3 people from a group of
10 is counted using combinations, while assigning 3 distinct jobs to people from the same group
is counted using permutations.
	
	\end{enumerate}


\subsection*{Classic Probability Examples}

	\begin{enumerate}
		\item \textbf{Coin Tosses.} Suppose 3 fair coins are tossed. What is the probability of obtaining
exactly 2 heads?

\emph{Solution.} The sample space has $2^3=8$ outcomes. The number of outcomes with exactly 2 heads
is $\binom{3}{2}=3$. Therefore,
\[\P(\text{exactly 2 heads})=\ds\frac{3}{8}.\]

		\item \textbf{General Coin Toss Model.} Suppose $n$ fair coins are tossed. What is the probability
of obtaining exactly $k$ heads?

\emph{Solution.} The sample space has $2^n$ outcomes. The number of outcomes with exactly $k$ heads
is $\binom{n}{k}$. Thus,
\[\P(\text{exactly $k$ heads})=\ds\frac{\binom{n}{k}}{2^n}.\]

		\item \textbf{Cards.} A standard deck has 52 cards. What is the probability that a 5-card hand
contains exactly 2 aces?

\emph{Solution.} The total number of 5-card hands is $\binom{52}{5}$. The number of hands with
exactly 2 aces is
\[\binom{4}{2}\binom{48}{3}.\]
Thus,
\[\P=\ds\frac{\binom{4}{2}\binom{48}{3}}{\binom{52}{5}}.\]

		\item \textbf{At Least One Ace.} What is the probability that a 5-card hand contains at least one
ace?

\emph{Solution.} Rather than counting directly, we use complements. The number of hands with no
aces is $\binom{48}{5}$. Hence,
\[\P(\text{at least one ace}) = 1 - \ds\frac{\binom{48}{5}}{\binom{52}{5}}.\]

		\item \textbf{Sampling Without Replacement (Urn Model).} An urn contains 5 red balls and 7 blue
balls. Two balls are drawn at random without replacement. What is the probability that both balls
are red?

\emph{Solution.} The total number of ways to choose 2 balls from 12 is $\binom{12}{2}$. The number
of ways to choose 2 red balls from the 5 red balls is $\binom{5}{2}$. Therefore,
\[\P(\text{both red})=\ds\frac{\binom{5}{2}}{\binom{12}{2}}.\]

		\item \textbf{Sampling With Replacement.} An urn contains 5 red balls and 7 blue balls. Two balls
are drawn independently with replacement. What is the probability that both balls are red?

\emph{Solution.} Each draw has probability $5/12$ of being red. Since the draws are independent,
\[\P(\text{both red})=\left(\ds\frac{5}{12}\right)^2.\]

		\item \textbf{Birthday Problem (Finite Version).} Suppose $n$ people are selected independently
and uniformly from $N$ possible birthdays. What is the probability that all birthdays are distinct?

\emph{Solution.} The total number of birthday assignments is $N^n$. The number of assignments with
all distinct birthdays is
\[P(N,n)=\frac{N!}{(N-n)!}.\]
Hence,
\[\P(\text{all distinct})=\ds\frac{P(N,n)}{N^n}.\]

		\item \textbf{At Least One Shared Birthday.} Under the same assumptions, what is the probability
that at least two people share a birthday?

\emph{Solution.} Using complements,
\[\P(\text{at least one match}) = 1 - \P(\text{all distinct})=1-\ds\frac{P(N,n)}{N^n}.\]

		\item \textbf{Hypergeometric Model.} A shipment contains 20 items, of which 4 are defective. If
5 items are selected at random without replacement, what is the probability that exactly 1 item
is defective?

\emph{Solution.} The total number of ways to select 5 items is $\binom{20}{5}$. The number of ways
to select exactly 1 defective item is
\[\binom{4}{1}\binom{16}{4}.\]
Thus,
\[\P(\text{exactly 1 defective})=\ds\frac{\binom{4}{1}\binom{16}{4}}{\binom{20}{5}}.\]

		\item \textbf{Practice: Binomial vs.\ Hypergeometric.} A box contains 10 light bulbs, 3 of which
are defective.
			\begin{enumerate}
				\item If 4 bulbs are selected \emph{with replacement}, find the probability that exactly 2 are
defective.
				\item If 4 bulbs are selected \emph{without replacement}, find the probability that exactly 2 are
defective.
			\end{enumerate}

\end{enumerate}

\vskip 1cm

%%%%%%%%%%%%%%%%%%%%%%%%%%%%%%%%%%%%%%%%%%%%%%%%%%%%%%%%%%%%%%%%%%%%%%%%%%%%%%%%%%%%%%%%%%%%%%%%%%%%%%
\section*{Supplement: Why ${n\choose k}$ Is Called a Binomial Coefficient}
%%%%%%%%%%%%%%%%%%%%%%%%%%%%%%%%%%%%%%%%%%%%%%%%%%%%%%%%%%%%%%%%%%%%%%%%%%%%%%%%%%%%%%%%%%%%%%%%%%%%%%

\noindent The quantity
\[\binom{n}{k}=\frac{n!}{k!(n-k)!}\]
is called a \emph{binomial coefficient} because it appears as the coefficient of the term
$x^k y^{n-k}$ in the expansion of a binomial expression of the form $(x+y)^n$.

\paragraph*{The Binomial Theorem:}

\begin{thm}[Binomial Theorem]
For any integer $n \ge 0$,
\small
\[(x+y)^n = \sum_{k=0}^n \binom{n}{k} x^k y^{\,n-k}=
\binom{n}{0}y^n+\binom{n}{1}xy^{n-1}+\binom{n}{2}x^2 y^{n-2}+\cdots+\binom{n}{n} x^n.\]
\end{thm}

\normalsize

\noindent Thus, the binomial coefficients $\binom{n}{k}$ describe how many times each monomial
$x^k y^{n-k}$ appears when the product $(x+y)(x+y)\cdots(x+y)$ is expanded.

\paragraph*{Combinatorial Interpretation:}

\noindent Consider the product
\[(x+y)^n = \underbrace{(x+y)(x+y)\cdots(x+y)}_{n \text{ factors}}.\]
To obtain a term of the form $x^k y^{n-k}$, we must choose $x$ from exactly $k$ of the $n$ factors
and $y$ from the remaining $n-k$ factors. The number of ways to choose which $k$ factors contribute
an $x$ is
\[\binom{n}{k}.\]
This explains both the algebraic and combinatorial meaning of the binomial coefficient.

\paragraph*{Illustrative Examples:}

	\begin{enumerate}
		\item \textbf{Expansion of a Binomial.} Expanding $(x+y)^3$ gives
\[(x+y)^3 = x^3 + 3x^2y + 3xy^2 + y^3.\]
The coefficients $1,3,3,1$ are precisely
\[\binom{3}{0}, \binom{3}{1}, \binom{3}{2}, \binom{3}{3}.\]

		\item \textbf{Numerical Example.} Using the binomial theorem,
\[(1+1)^n = \sum_{k=0}^n \binom{n}{k}.\]
Since $(1+1)^n = 2^n$, this shows that
\[\sum_{k=0}^n \binom{n}{k} = 2^n.\]
This identity has a natural combinatorial interpretation: the total number of subsets of an
$n$-element set is $2^n$.

			\item \textbf{Connection to Coin Tosses.} When $n$ fair coins are tossed, each outcome corresponds
to a term in the expansion of $(H+T)^n$. The coefficient $\binom{n}{k}$ counts the number of outcomes
with exactly $k$ heads and $n-k$ tails, explaining why binomial coefficients arise naturally in
probability problems.

	\end{enumerate}



%%%%%%%%%%%%%%%%%%%%%%%%%%%%%%%%%%%%%%%%%%%%%%%%%%%%%%%%%%%%%%%%%%%%%%%%%%%%%%%%%%%%%%%%%%%%%%%%%%%%%%
%\section*{The R Code:} 
%%%%%%%%%%%%%%%%%%%%%%%%%%%%%%%%%%%%%%%%%%%%%%%%%%%%%%%%%%%%%%%%%%%%%%%%%%%%%%%%%%%%%%%%%%%%%%%%%%%%%%
%Here is the R code used for the above simulations.
%\vspace*{2mm}
%\small 
%\begin{lstlisting}[language=R]
%
%
%
%\end{lstlisting}



\vskip 1cm
\hrule
\vskip 5mm
\begin{center}
\bf Please let me know if you have any questions, comments, or corrections!
\end{center}

%%%%%%%%%%%%%%%%%%%%%%%%%%%%%%%%%%%%%%%%%%%%%%%%%%%%%%
\end{document}
%%%%%%%%%%%%%%%%%%%%%%%%%%%%%%%%%%%%%%%%%%%%%%%%%%%%%%