%%%%%%%%%%%%%%%%%%%%%%%%%%%%%%%%%%%%%%%%%%%%%%%%%%%%%%
\documentclass[11pt]{article}
%%%%%%%%%%%%%%%%%%%%%%%%%%%%%%%%%%%%%%%%%%%%%%%%%%%%%%

\usepackage{amsmath}
\usepackage{amsthm}
\usepackage{amssymb}
\usepackage{latexsym}
\usepackage{graphicx}
\usepackage{color}
\usepackage{verbatim}
\usepackage{float}
\usepackage{multicol}
\usepackage{xcolor}
\usepackage{listings}
\usepackage{tikz}
\usetikzlibrary{arrows.meta, positioning, calc}
\usetikzlibrary{decorations.pathmorphing}
\usepackage{tcolorbox}
\tcbuselibrary{breakable}
\usepackage{cancel}


\newtcolorbox{solutionbox}{
  breakable,
  colback=blue!5!white,
  colframe=blue!50!black,
  title=Solution,
  sharp corners,
  boxrule=0.8pt
}

\newtcolorbox{hintbox}{
  breakable,
  colback=gray!10!white,
  colframe=gray!50!black,
  title=Hint,
  sharp corners,
  boxrule=0.5pt
}

% Unnumbered theorem
\newtheorem*{thm*}{Theorem}

\lstdefinelanguage{R}{
      keywords={if,else,while,for,in,next,break,function,TRUE,FALSE,NULL,Inf,NA,NaN,switch,repeat,return,require,library},
      keywordstyle=\color{blue}\bfseries,
      identifierstyle=\color{black},
      comment=[l]{\#},
      commentstyle=\color{gray}\ttfamily,
      string=[b]{"},
      stringstyle=\color{red}\ttfamily,
      morecomment=[l]{//},
      morestring=[b]{'},
      sensitive=true,
      morekeywords={print,summary,plot,lm,glm,data,frame,read.csv,write.csv,factor,levels,names,colnames,rownames,
      head,tail,str,dim,length,class,typeof,mode,is.na,is.null,is.finite,is.infinite,is.nan,as.numeric,as.character,
      as.factor,as.Date,as.POSIXct,as.matrix,as.data.frame,rbind,cbind,merge,subset,aggregate,tapply,apply,lapply,sapply,
      mapply,vapply,replicate,seq,rep,c,list,matrix,array,data.frame,table,hist,boxplot,barplot,pie,curve,lines,points,text,
      abline,legend,par,mtext,title,xlab,ylab,xlim,ylim,main,sub,col,pch,cex,lty,lwd,type,bg,fg,args,options,warnings,errors,
      message,stop,warning,error,try,tryCatch,withCallingHandlers,on.exit,debug,browser,trace,recover,options,getOption,setOption},
    }


\setlength{\textheight}{9in}
\setlength{\textwidth}{6in}
\addtolength{\topmargin}{-2cm}
\addtolength{\oddsidemargin}{-1cm}
\parindent=0in


\def\classnum{3810}
\def\classtitle{Probability}
\def\classtitleshort{Probability}
\def\classsec{001}
\def\classterm{Fall 2025}
\def\instructor{Robert Rostermundt}
%\def\hmwknum{\#2}


%%%%%%%%%%%%%%%%%%%%%%%%%%%%%%%%%%%%%%%%%%%%%%%%%%%%%%%%%
%%%%%%%%%%%%%%%%%%%%%%%%%  Colors  %%%%%%%%%%%%%%%%%%%%%%
%%%%%%%%%%%%%%%%%%%%%%%%%%%%%%%%%%%%%%%%%%%%%%%%%%%%%%%%%

\definecolor{Green}{rgb}{0,.5,0}
%use for definitions
\definecolor{Red}{rgb}{.8,.2,0}
%use for emphasis
\definecolor{Yellow}{rgb}{.6,.6,.1}
%use for part titles
\definecolor{Cyan}{rgb}{.2,.6,.7}
%use for comments
\definecolor{Purple}{rgb}{.4,0,1}
%use for examples
\definecolor{deepred}{rgb}{.53,.29,.24}
%use for important points
\definecolor{Black}{rgb}{0,0,0}
%use for washout
\definecolor{Grey}{rgb}{.45,.45,.45}
% use for theorems
\newcommand{\tred}[1]{\textcolor{Red}{#1}}
\newcommand{\tgreen}[1]{\textcolor{Green}{#1}}
\newcommand{\tcyan}[1]{\textcolor{Cyan}{#1}}
\newcommand{\tyellow}[1]{\textcolor{Yellow}{#1}}
\newcommand{\tpurple}[1]{\textcolor{Purple}{#1}}
\newcommand{\tblack}[1]{\textcolor{Black}{#1}}
\newcommand{\tgrey}[1]{\textcolor{Grey}{#1}}
\newcommand{\tdeepred}[1]{\textcolor{deepred}{#1}}
\newcommand{\ttt}[1]{\texttt{#1}}

%%%%%%%%%%%%%%%%%%%%%%%%%%%%%%%%%%%%%%%%%%%%%%%%%%%%%%%%%
%%%%%%%%%%%%%%%%%%%%%%%%%  Theorem Environments  %%%%%%%%
%%%%%%%%%%%%%%%%%%%%%%%%%%%%%%%%%%%%%%%%%%%%%%%%%%%%%%%%%

\theoremstyle{plain}
\newtheorem{thm}{Theorem}
\newtheorem{axiom}{Axiom}
\newtheorem{cor}{Corollary}
\newtheorem{lemma}{Lemma}
\newtheorem{prop}{Proposition}
\newtheorem{ques}{Question}
\theoremstyle{definition}
\newtheorem{defn}{Definition}
\theoremstyle{remark}
\newtheorem{remark}{Remark}
\theoremstyle{definition}
\newtheorem{ex}{Example}
\numberwithin{equation}{section}
\newtheorem{prob}{Problem}
\numberwithin{equation}{section}


%%%%%%%%%%%%%%%%%%%%%%%%%%%%%%%%%%%%%%%%%%%%%%%%%%%%%%%%%
%%%%%%%%%%%%%%%%%%%%%%%%%  Math    %%%%%%%%%%%%%%%%%%%%%%
%%%%%%%%%%%%%%%%%%%%%%%%%%%%%%%%%%%%%%%%%%%%%%%%%%%%%%%%%


\newcommand{\abs}[1]{\left\lvert{#1}\right\rvert}
\newcommand{\card}[1]{\lvert{#1}\rvert}
\newcommand{\union}{\cup}
\newcommand{\Union}{\bigcup}
\newcommand{\inter}{\cap}
\newcommand{\Inter}{\bigcap}
%\newcommand{\hint}[1]{\medskip\newline\emph{Hint: #1}}
%\newcommand{\note}[1]{\medskip\newline\emph{Note: #1}}
\newcommand{\points}[1]{[#1 points]}
\newcommand{\totalpoints}[1]{[#1 points total]}
\newcommand{\ds}{\displaystyle}
\newcommand{\ben}{\begin{enumerate}}
\newcommand{\een}{\end{enumerate}}
\newcommand{\bi}{\begin{itemize}}
\newcommand{\ei}{\end{itemize}}
\newcommand{\beq}{\begin{eqnarray*}}
\newcommand{\eeq}{\end{eqnarray*}}
\newcommand{\bieq}{\begin{IEEEeqnarray}{rCl}}
\newcommand{\bieqx}{\begin{IEEEeqnarray}{+rCl+x*}}
\newcommand{\eieq}{\end{IEEEeqnarray}}
\newcommand{\nn}{\nonumber}
%\renewcommand{\i}{\item}
\newcommand{\bpm}{\begin{pmatrix}}
\newcommand{\epm}{\end{pmatrix}}
\newcommand{\sol}{\indent{\bf\emph{Solution:}}}
\newcommand{\ssol}{\indent{\\[2mm]\bf\emph{Solution:}}\;}
\newcommand{\hint}{\indent{\bf\emph{Hint}:}\;}
\newcommand{\note}{\indent{\bf\emph{Note}:}\;}
\newcommand{\vsk}{\vskip 2mm}
%%%%%%%%%%%%%%%%%%%%%%%%% Calculus %%%%%%%%%%%%%%%%%%%%%%%%%%%%
\newcommand{\dd}[2]{\ds\frac{d}{d{#1}}\left[{#2}\right]}
\newcommand{\der}[2]{\ds\frac{d{#1}}{d{#2}}}
\newcommand{\lmt}[3]{\ds\lim_{{#1}\to{#2}}{#3}}
\renewcommand{\iint}[2]{\ds\int{#1}\,d{#2}}
\newcommand{\dint}[4]{\ds\int^{#4}_{#3}{#1}\,d{#2}}
\renewcommand{\Delta}{\triangle}
%%%%%%%%%%%%%%%%%%%%%%%%% Number Sets %%%%%%%%%%%%%%%%%%%%%%%%%%
\newcommand{\N}{\mathbb{N}}
\newcommand{\Z}{\mathbb{Z}}
\newcommand{\Q}{\mathbb{Q}}
\newcommand{\R}{\mathbb{R}}
\newcommand{\C}{\mathbb{C}}
\newcommand{\F}{\mathcal{F}}
\renewcommand{\P}{\mathbb{P}}
\newcommand{\E}{\mathcal{E}}
\renewcommand{\o}{\omega}
\renewcommand{\O}{\Omega}
%%%%%%%%%%%%%%%%%%%%%%%%% Vectors %%%%%%%%%%%%%%%%%%%%%%%%%%%%%
\newcommand{\x}{\bar{x}}
\renewcommand{\v}{\bar{v}}
\newcommand{\y}{\bar{y}}
\newcommand{\z}{\bar{z}}
\newcommand{\w}{\bar{w}}
\renewcommand{\u}{\bar{u}}
\renewcommand{\b}{\bar{b}}
\newcommand{\e}{\bar{e}}
\renewcommand{\a}{\vec{a}}
\renewcommand{\r}{\vec{r}}
\newcommand{\vv}{\vec{v}}
\newcommand{\vecPQ}[2]{\overrightarrow{#1}{#2}}
\newcommand{\vecV}[1]{\overrightarrow{#1}}
\newcommand{\la}{\langle}
\newcommand{\ra}{\rangle}
%%%%%%%%%%%%%%%%%%%%%%%%%%% Vector Spaces %%%%%%%%%%%%%%%%%%%%
\newcommand{\rn}{\ensuremath{\mathbb{R}^n}}
\renewcommand{\rm}{\ensuremath{\mathbb{R}^m}}
\newcommand{\re}{\mathbb{R}}
\newcommand{\Pn}{\mathbb{P}_n}
\newcommand{\B}{\mathcal{B}}
%%%%%%%%%%%%%%%%%%%%%%%%%%% Graphics %%%%%%%%%%%%%%%%%%%%%%%%
\newcommand{\cg}[2]{\begin{center}
\includegraphics[scale={#1}]{{#2}}
\end{center}}
\makeatletter
\def\imod#1{\allowbreak\mkern10mu({\operator@font mod}\,\,#1)}
\makeatother

%%%%%%%%%%%%%%%%%%%%%%%%%%%%%%%%%%%%%%%%%%%%%%%%%%%%%%%%%%%%%%%%%%%%%%%%%%%%%%%%%%%%%%%%%%%%%%
%%%%%%%%%%%%%%%%%%%%%%%%%%%%%% Defined Fonts %%%%%%%%%%%%%%%%%%%%%%%%%%%%%%%%%%%%%%%%%%%%%%%%%
%%%%%%%%%%%%%%%%%%%%%%%%%%%%%%%%%%%%%%%%%%%%%%%%%%%%%%%%%%%%%%%%%%%%%%%%%%%%%%%%%%%%%%%%%%%%%%

\font\minihelv=phvr at 6pt
\font\helv=phvr at 10pt
\font\medhelv=phvr at 16pt
\font\bighelv=phvr at 20pt
\font\hugehelv=phvr at 36pt
\font\mybigfont=phvr at 16pt
\font\mymediumfont=phvr at 14pt
\font\mediumhelv=phvr at 14pt
\font\mybfit=ptmbi at 12pt


%%%%%%%%%%%%%%%%%%%%%%%%%%%%%%%%%%%%%%%%%%%%%%%%%%%%%%%%%%%%%%%%%%%%%%%%%%%%%%%%%%%%%%%%%%%%%%%
%%%%%%%%%%%%%%%%%%%%%%%%%%%%%% Other Commands %%%%%%%%%%%%%%%%%%%%%%%%%%%%%%%%%%%%%%%%%%%%%%%%%
%%%%%%%%%%%%%%%%%%%%%%%%%%%%%%%%%%%%%%%%%%%%%%%%%%%%%%%%%%%%%%%%%%%%%%%%%%%%%%%%%%%%%%%%%%%%%%%
%\setlength\fboxrule{.5pt}
%\newcommand{\latexpicborder}[3]{
%\setlength\fboxsep{30pt}
%\begin{figure}[hb]
%\begin{center}
%\fbox{
%\input{#1}
%}
%\caption{#2}
%\label{#3}
%\end{center}
%\end{figure}
%\setlength\fboxsep{0pt}
%}
%
%\newcommand{\latexpic}[2]{
%\begin{figure}[hb]
%\begin{center}
%\input{#1}
%\vspace*{8mm}
%\caption{#2}
%\end{center}
%\end{figure}
%}

%\begin{minipage}[b]{0.6\linewidth}
%......
%\end{minipage}
%\hspace{0.5cm}
%\begin{minipage}[t]{0.4\linewidth}
%\centering
%\includegraphics[scale=.5]{m1401_ex3_g4.eps}
%\end{minipage}
%\end{figure}


%%%%%%%%%%%%%%%%%%%%%%%%%%%%%%%%%%%%%%%%%%%%%%%%%%%%%%%%%%%%%%%%%%%%%%%%%%%%%%%%%%%%%%%%%%%%%%
%%%%%%%%%%%%%%%%%%%%%%%%%%% IEEEeqnarray Notes %%%%%%%%%%%%%%%%%%%%%%%%%%%%%%%%%%%%%%%%%%%%%%%
%%%%%%%%%%%%%%%%%%%%%%%%%%%%%%%%%%%%%%%%%%%%%%%%%%%%%%%%%%%%%%%%%%%%%%%%%%%%%%%%%%%%%%%%%%%%%%


%Any number of columns can be specified with IEEEeqnarray: {c} will give only one
%column with all entries centered, or {rCll} would add a fourth, left-justified
%column to use for comments. Moreover, beside l, c, r, L, C, R for math mode
%entries there are also s, t, u for left, centered, and right text mode entries.
%Additional space can be added with . and / and ? in increasing order.
%
%
%\begin{proof}
%This is a proof that ends
%with an equation array:
%\begin{IEEEeqnarray*}{+rCl+x*}
%a & = & b + c \\
%& = & d + e. & \qedhere
%\end{IEEEeqnarray*}
%\end{proof}
%Note that the + in {+rCl+x*} denotes stretchable spaces, one on the left
%of the equations (which, if not specified, will be done automatically by
%IEEEeqnarray!) and one on the right of the equations. But now on the right,
%after the stretching column, we add an empty column x. This column will be
%only needed on the last line when we will put the \qedhere command there.
%Finally, we specify a *. This is a null-space that prevents IEEEeqnarray to
%add another unwanted +-space!


% The following environments enable custom numbering of theorems so that the numbers agree % with the numbering in the textbook being used. 
%
%  Usage examples:
%\begin{customthm}{2.2}\label{eight}
%Every theorem must be numbered by hand.
%\end{customthm}
%
%Here is a reference to theorem~\ref{eight}.
%
%\begin{customthm}{2.3}[Parenthetical comment]\label{nine}
%Statement
%\end{customthm}
%
%Here is a reference to theorem~\ref{nine}


\newtheorem{innercustomthm}{Theorem}
\newenvironment{customthm}[1]
  {\renewcommand\theinnercustomthm{#1}\innercustomthm}
  {\endinnercustomthm}
  
  \newtheorem{innercustomprop}{Proposition}
\newenvironment{customprop}[1]
  {\renewcommand\theinnercustomprop{#1}\innercustomprop}
  {\endinnercustomprop}
  
    \newtheorem{innercustomlem}{Lemma}
\newenvironment{customlem}[1]
  {\renewcommand\theinnercustomlem{#1}\innercustomlem}
  {\endinnercustomlem}
  
    \newtheorem{innercustomconj}{Conjecture}
\newenvironment{customconj}[1]
  {\renewcommand\theinnercustomconj{#1}\innercustomconj}
  {\endinnercustomconj}
  
    \newtheorem{innercustomclaim}{Claim}
\newenvironment{customclaim}[1]
  {\renewcommand\theinnercustomclaim{#1}\innercustomclaim}
  {\endinnercustomclaim}
  
    \newtheorem{innercustomcor}{Corollary}
\newenvironment{customcor}[1]
  {\renewcommand\theinnercustomcor{#1}\innercustomcor}
  {\endinnercustomcor}
  
    \newtheorem{innercustomdef}{Definition}
\newenvironment{customdef}[1]
  {\renewcommand\theinnercustomdef{#1}\innercustomdef}
  {\endinnercustomdef}
  
    \newtheorem{innercustomex}{Example}
\newenvironment{customex}[1]
  {\renewcommand\theinnercustomex{#1}\innercustomex}
  {\endinnercustomex}
  
    \newtheorem{innercustomass}{Assumption}
\newenvironment{customass}[1]
  {\renewcommand\theinnercustomass{#1}\innercustomass}
  {\endinnercustomass}
  
      \newtheorem{innercustomax}{Axiom}
\newenvironment{customax}[1]
  {\renewcommand\theinnercustomax{#1}\innercustomax}
  {\endinnercustomax}
  

\vfuzz2pt % Don't report over-full v-boxes if over-edge is small
\hfuzz2pt % Don't report over-full h-boxes if over-edge is small

\renewcommand{\ni}{\noindent}


%%%%%%%%%%%%%%%%%%%%%%%%%%%%%%%%%%%%%%%%%%%%%%%%%%%%%%
%%%%%%%%%%%%%%%%%%%%%%%%%%%%%%%%%%%%%%%%%%%%%%%%%%%%%%

\pagestyle{myheadings}

%%%%%%%%%%%%%%%%%%%%%%%%%%%%%%%%%%%%%%%%%%%%%%%%%%%%%%

%%%%%%%%%%%%%%%%%%%%%%%%%%%%%%%%%%%%%%%%%%%%%%%%%%%%%%
%%%%%%%%%%%%%%%%%%%%%%%%%   Document Body   %%%%%%%%%%
%%%%%%%%%%%%%%%%%%%%%%%%%%%%%%%%%%%%%%%%%%%%%%%%%%%%%%

%\def\classnum{3810}
%\def\classtitle{Probability}
%\def\classtitleshort{Probability}
%\def\classsec{001}
%\def\classterm{Fall 2025}
%\def\instructor{Robert Rostermundt}
\def\printsol{0}


	\title{\vspace{-1in}Math\classnum\;-\;\classtitle\\
	Section\;\classsec\;-\;\classterm\\
	Notes: Conditional Density and Expectation}
	\author{University of Colorado Denver / College of Liberal Arts 	and Sciences}
	\date{Department of Mathematics - \instructor}

	\markright{Math\classnum\;-\;\classtitleshort, UCD, \classterm, \instructor}



%%%%%%%%%%%%%%%%%%%%%%%%%%%%%%%%%%%%%%%%%%%%%%%%%%%%%%
\begin{document}\maketitle\thispagestyle{empty}
%%%%%%%%%%%%%%%%%%%%%%%%%%%%%%%%%%%%%%%%%%%%%%%%%%%%%%


%%%%%%%%%%%%%%%%%%%%%%%%%%%%%%%%%%%%%%%%%%%%%%%%%%%%%%%%%%%%%%%%%%%%%%%%%%%%%%%%%%%%%%%%%%%%%%%%%%%%%%
\vspace*{2mm}
\hrule
\vskip 8mm


%%%%%%%%%%%%%%%%%%%%%%%%%%%%%%%%%%%%%%%%%%%%%%%%%%%%%%%%%%%%%%%%%%%%%%%%%
%%%%%%%%%%%%%%%%%%%%%%%%%%%%%%%%%%%%%%%%%%%%%%%%%%%%%%%%%%%%%%%%%%%%%%%%%


%%%%%%%%%%%%%%%%%%%%%%%%%%%%%%%%%%%%%%%%%%%%%%%%%%%%%%%%%%%%%%%%%%%%%%%%%%%%%%%%%%%%%%%%%%%%%%%%%%%%%%
\section*{The Problem:}
%%%%%%%%%%%%%%%%%%%%%%%%%%%%%%%%%%%%%%%%%%%%%%%%%%%%%%%%%%%%%%%%%%%%%%%%%%%%%%%%%%%%%%%%%%%%%%%%%%%%%%


\noindent We break a stick of length $l$ twice at random. First at length $y$, and then the second break of length $x$. Find the expected length of the stick after the second break. We let $Y$ be the length after the first break, and $X$ be the length after the second break. Then $Y\sim Uni(0,l)$ and $X\sim Uni(0,Y)$ and we are searching for $E[X]$. Since we don't initially have a density function $f_{_X}$ we will use some theoretical tools to build this density function and then compute $E[X]=\ds\int^{\infty}_{-\infty}x\cdot f_{_X}(x)\,dx$.
	\begin{figure}[h!]
		\begin{center}
			\includegraphics[scale=0.4]{stick_break_simulation_gr1.jpeg}
		\end{center}
		\caption{Simulated Stick Breaking $(n=10)$}
	\end{figure}

%%%%%%%%%%%%%%%%%%%%%%%%%%%%%%%%%%%%%%%%%%%%%%%%%%%%%%%%%%%%%%%%%%%%%%%%%%%%%%%%%%%%%%%%%%%%%%%%%%%%%%
\section*{Theoretical Tools:}
%%%%%%%%%%%%%%%%%%%%%%%%%%%%%%%%%%%%%%%%%%%%%%%%%%%%%%%%%%%%%%%%%%%%%%%%%%%%%%%%%%%%%%%%%%%%%%%%%%%%%%

\noindent Here are the basic tools for this problem.

	\begin{itemize}
		\item[$\bullet$] {\bf\emph{Joint Density Function}}: Let $X,Y$ be continuous jointly distributed random variables. Then we can describe the joint density 
function as
\[f_{_{X,Y}}(x,y)=f_{_Y}(y)f_{_{X|Y}}(x|y)\]
where $f_{_Y}(y)$ is the marginal density of $Y$ and $f_{_{X|Y}}(x|y)$ is a conditional density function.

		\item[$\bullet$] {\bf\emph{Marginal Density Function}}: Given a joint density function $f_{_{X,Y}}$ we can compute the marginal density functions of $X$ and $Y$ as follows:
\beq
f_{_X}(x)&=&\ds\int^{\infty}_{-\infty}f_{_{X,Y}}(x,y)\,dy\\
f_{_Y}(y)&=&\ds\int^{\infty}_{-\infty}f_{_{X,Y}}(x,y)\,dx\\
\eeq

	\end{itemize}

%%%%%%%%%%%%%%%%%%%%%%%%%%%%%%%%%%%%%%%%%%%%%%%%%%%%%%%%%%%%%%%%%%%%%%%%%%%%%%%%%%%%%%%%%%%%%%%%%%%%%%
\section*{Calculations:}
%%%%%%%%%%%%%%%%%%%%%%%%%%%%%%%%%%%%%%%%%%%%%%%%%%%%%%%%%%%%%%%%%%%%%%%%%%%%%%%%%%%%%%%%%%%%%%%%%%%%%%

\noindent Since $Y\sim Uni(0,l)$ we have $f_{_Y}(y)=1\l$ when $0<y<l$. Then $f_{_{X|Y=y}}(x|y)=1/y$, when $0<x<y$. From this and the above formula for the joint density function we have
\[f_{_{X,Y}}(x,y)=\left\{\begin{array}{ccl}
\ds\frac{1}{ly}&:&0<y<l,\,0<x<y\\
0&:&\text{Otherwise}.
\end{array}
\right.\]
Here is the support for the joint distribution. The horizontal arrow is a setup aid for integration with respect to $y$.
	\begin{figure}[h!]
		\begin{center}
			\includegraphics[scale=0.5]{stick_break_region.jpg}
		\end{center}
		\caption{Support for the Joint Distribution}
	\end{figure}
	\vskip 1mm
Now we have that, for $0<x<l$,
\[f_{_X}(x)=\ds\int^{\infty}_{-\infty}f_{_{X,Y}}(x,y)\,dy=\ds\int^{y=l}_{y=x}\ds\frac{1}{ly}\,dy=\ds\frac{1}{l}\ln\left(\ds\frac{l}{x}\right).\]
Now we are ready to compute the expectation.
\[E[X]=\ds\int^{\infty}_{-\infty}x\cdot f_{_X}(x)\,dx=\ds\int^{x=l}_{x=0}\ds\frac{x}{l}\cdot\ln\left(\ds\frac{l}{x}\right)\,dx=\cdots=\ds\frac{l}{4}\]
where the integral is calculated using integration by parts. So we have, as intuition would suggest, that
\[E[X]=\ds\frac{l}{4}.\]

%%%%%%%%%%%%%%%%%%%%%%%%%%%%%%%%%%%%%%%%%%%%%%%%%%%%%%%%%%%%%%%%%%%%%%%%%%%%%%%%%%%%%%%%%%%%%%%%%%%%%%
\section*{Simulations:}
%%%%%%%%%%%%%%%%%%%%%%%%%%%%%%%%%%%%%%%%%%%%%%%%%%%%%%%%%%%%%%%%%%%%%%%%%%%%%%%%%%%%%%%%%%%%%%%%%%%%%%
I have simulated this process in R with a $l=1$ unit length stick. So the theoretical expectation is $E[X]=1/4$. Below are graphics from the simulation showing the number updates $n=50, 500, 5000, 10000$. The R code is provided the end of the document.
\vskip 1cm

\includegraphics[scale=0.48]{stick_break_simulation_50.jpeg}\quad
\includegraphics[scale=0.48]{stick_break_simulation_500.jpeg}\\
\vskip 5mm
\includegraphics[scale=0.48]{stick_break_simulation_5000.jpeg}\quad
\includegraphics[scale=0.48]{stick_break_simulation_10000.jpeg}\\
\vskip 5mm



\vfill\eject

%%%%%%%%%%%%%%%%%%%%%%%%%%%%%%%%%%%%%%%%%%%%%%%%%%%%%%%%%%%%%%%%%%%%%%%%%%%%%%%%%%%%%%%%%%%%%%%%%%%%%%
\section*{The R Code:}
%%%%%%%%%%%%%%%%%%%%%%%%%%%%%%%%%%%%%%%%%%%%%%%%%%%%%%%%%%%%%%%%%%%%%%%%%%%%%%%%%%%%%%%%%%%%%%%%%%%%%%
Here is the R code used for the above simulations.
\vspace*{2mm}
\small 
\begin{lstlisting}[language=R]

set.seed(123)

n <- 10000   # number of experiments
l <- 1     # stick length

# Step 1: first break (uniform on 0..l)
y <- runif(n, 0, l)

# Step 2: second break (uniform on 0..y)
x <- runif(n, 0, y)

# Three resulting pieces
piece1 <- x
piece2 <- y - x
piece3 <- l - y

cat("Sample means:\n")
cat("E[piece1] ~", mean(piece1), "  theoretical l/4 =", l/4, "\n")
cat("E[piece2] ~", mean(piece2), "  theoretical l/4 =", l/4, "\n")
cat("E[piece3] ~", mean(piece3), "  theoretical l/2 =", l/2, "\n")

# Plot histograms with n in the title
par(mfrow=c(1,3))
hist(piece1, breaks=60, col="skyblue",
     main=paste0("piece1 = x, n=", n), xlim=c(0,l))
hist(piece2, breaks=60, col="lightgreen",
     main=paste0("piece2 = y - x, n=", n), xlim=c(0,l))
hist(piece3, breaks=60, col="salmon",
     main=paste0("piece3 = l - y, n=", n), xlim=c(0,l))
par(mfrow=c(1,1))

# Theoretical density of X (first break on first piece)
xx <- seq(1e-6, l-1e-6, length.out=400)
fX_theory <- (1/l) * log(l/xx)

hist(piece1, breaks=80, probability=TRUE, col="lightgray",
     main=paste0("Final Length X with Theoretical Density f(x)  (n=", n,")"),
     xlab="x", xlim=c(0,l))
lines(xx, fX_theory, col="blue", lwd=2)

# Add empirical mean of X as vertical line
abline(v = mean(piece1), col = "red", lwd = 2, lty = 2)
text(mean(piece1), max(hist(piece1, plot=FALSE)$density)*0.9,
     labels = paste0("Mean = ", round(mean(piece1), 3)), col = "red", pos = 4)
\end{lstlisting}



\vskip 1cm
\hrule
\vskip 5mm
\begin{center}
\bf Please let me know if you have any questions, comments, or corrections!
\end{center}

%%%%%%%%%%%%%%%%%%%%%%%%%%%%%%%%%%%%%%%%%%%%%%%%%%%%%%
\end{document}
%%%%%%%%%%%%%%%%%%%%%%%%%%%%%%%%%%%%%%%%%%%%%%%%%%%%%%