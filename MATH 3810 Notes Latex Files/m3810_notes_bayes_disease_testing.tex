%%%%%%%%%%%%%%%%%%%%%%%%%%%%%%%%%%%%%%%%%%%%%%%%%%%%%%
\documentclass[11pt]{article}
%%%%%%%%%%%%%%%%%%%%%%%%%%%%%%%%%%%%%%%%%%%%%%%%%%%%%%

\usepackage{amsmath}
\usepackage{amsthm}
\usepackage{amssymb}
\usepackage{latexsym}
\usepackage{graphicx}
\usepackage{color}
\usepackage{verbatim}
\usepackage{float}
\usepackage{multicol}
\usepackage{xcolor}
\usepackage{listings}
\usepackage{tikz}
\usetikzlibrary{arrows.meta, positioning, calc}
\usetikzlibrary{decorations.pathmorphing}
\usepackage{tcolorbox}
\tcbuselibrary{breakable}
\usepackage{cancel}


\newtcolorbox{solutionbox}{
  breakable,
  colback=blue!5!white,
  colframe=blue!50!black,
  title=Solution,
  sharp corners,
  boxrule=0.8pt
}

\newtcolorbox{hintbox}{
  breakable,
  colback=gray!10!white,
  colframe=gray!50!black,
  title=Hint,
  sharp corners,
  boxrule=0.5pt
}

% Unnumbered theorem
\newtheorem*{thm*}{Theorem}

\lstdefinelanguage{R}{
      keywords={if,else,while,for,in,next,break,function,TRUE,FALSE,NULL,Inf,NA,NaN,switch,repeat,return,require,library},
      keywordstyle=\color{blue}\bfseries,
      identifierstyle=\color{black},
      comment=[l]{\#},
      commentstyle=\color{gray}\ttfamily,
      string=[b]{"},
      stringstyle=\color{red}\ttfamily,
      morecomment=[l]{//},
      morestring=[b]{'},
      sensitive=true,
      morekeywords={print,summary,plot,lm,glm,data,frame,read.csv,write.csv,factor,levels,names,colnames,rownames,
      head,tail,str,dim,length,class,typeof,mode,is.na,is.null,is.finite,is.infinite,is.nan,as.numeric,as.character,
      as.factor,as.Date,as.POSIXct,as.matrix,as.data.frame,rbind,cbind,merge,subset,aggregate,tapply,apply,lapply,sapply,
      mapply,vapply,replicate,seq,rep,c,list,matrix,array,data.frame,table,hist,boxplot,barplot,pie,curve,lines,points,text,
      abline,legend,par,mtext,title,xlab,ylab,xlim,ylim,main,sub,col,pch,cex,lty,lwd,type,bg,fg,args,options,warnings,errors,
      message,stop,warning,error,try,tryCatch,withCallingHandlers,on.exit,debug,browser,trace,recover,options,getOption,setOption},
    }


\setlength{\textheight}{9in}
\setlength{\textwidth}{6in}
\addtolength{\topmargin}{-2cm}
\addtolength{\oddsidemargin}{-1cm}
\parindent=0in


\def\classnum{3810}
\def\classtitle{Probability}
\def\classtitleshort{Probability}
\def\classsec{001}
\def\classterm{Fall 2025}
\def\instructor{Robert Rostermundt}
%\def\hmwknum{\#2}


%%%%%%%%%%%%%%%%%%%%%%%%%%%%%%%%%%%%%%%%%%%%%%%%%%%%%%%%%
%%%%%%%%%%%%%%%%%%%%%%%%%  Colors  %%%%%%%%%%%%%%%%%%%%%%
%%%%%%%%%%%%%%%%%%%%%%%%%%%%%%%%%%%%%%%%%%%%%%%%%%%%%%%%%

\definecolor{Green}{rgb}{0,.5,0}
%use for definitions
\definecolor{Red}{rgb}{.8,.2,0}
%use for emphasis
\definecolor{Yellow}{rgb}{.6,.6,.1}
%use for part titles
\definecolor{Cyan}{rgb}{.2,.6,.7}
%use for comments
\definecolor{Purple}{rgb}{.4,0,1}
%use for examples
\definecolor{deepred}{rgb}{.53,.29,.24}
%use for important points
\definecolor{Black}{rgb}{0,0,0}
%use for washout
\definecolor{Grey}{rgb}{.45,.45,.45}
% use for theorems
\newcommand{\tred}[1]{\textcolor{Red}{#1}}
\newcommand{\tgreen}[1]{\textcolor{Green}{#1}}
\newcommand{\tcyan}[1]{\textcolor{Cyan}{#1}}
\newcommand{\tyellow}[1]{\textcolor{Yellow}{#1}}
\newcommand{\tpurple}[1]{\textcolor{Purple}{#1}}
\newcommand{\tblack}[1]{\textcolor{Black}{#1}}
\newcommand{\tgrey}[1]{\textcolor{Grey}{#1}}
\newcommand{\tdeepred}[1]{\textcolor{deepred}{#1}}
\newcommand{\ttt}[1]{\texttt{#1}}

%%%%%%%%%%%%%%%%%%%%%%%%%%%%%%%%%%%%%%%%%%%%%%%%%%%%%%%%%
%%%%%%%%%%%%%%%%%%%%%%%%%  Theorem Environments  %%%%%%%%
%%%%%%%%%%%%%%%%%%%%%%%%%%%%%%%%%%%%%%%%%%%%%%%%%%%%%%%%%

\theoremstyle{plain}
\newtheorem{thm}{Theorem}
\newtheorem{axiom}{Axiom}
\newtheorem{cor}{Corollary}
\newtheorem{lemma}{Lemma}
\newtheorem{prop}{Proposition}
\newtheorem{ques}{Question}
\theoremstyle{definition}
\newtheorem{defn}{Definition}
\theoremstyle{remark}
\newtheorem{remark}{Remark}
\theoremstyle{definition}
\newtheorem{ex}{Example}
\numberwithin{equation}{section}
\newtheorem{prob}{Problem}
\numberwithin{equation}{section}


%%%%%%%%%%%%%%%%%%%%%%%%%%%%%%%%%%%%%%%%%%%%%%%%%%%%%%%%%
%%%%%%%%%%%%%%%%%%%%%%%%%  Math    %%%%%%%%%%%%%%%%%%%%%%
%%%%%%%%%%%%%%%%%%%%%%%%%%%%%%%%%%%%%%%%%%%%%%%%%%%%%%%%%


\newcommand{\abs}[1]{\left\lvert{#1}\right\rvert}
\newcommand{\card}[1]{\lvert{#1}\rvert}
\newcommand{\union}{\cup}
\newcommand{\Union}{\bigcup}
\newcommand{\inter}{\cap}
\newcommand{\Inter}{\bigcap}
%\newcommand{\hint}[1]{\medskip\newline\emph{Hint: #1}}
%\newcommand{\note}[1]{\medskip\newline\emph{Note: #1}}
\newcommand{\points}[1]{[#1 points]}
\newcommand{\totalpoints}[1]{[#1 points total]}
\newcommand{\ds}{\displaystyle}
\newcommand{\ben}{\begin{enumerate}}
\newcommand{\een}{\end{enumerate}}
\newcommand{\bi}{\begin{itemize}}
\newcommand{\ei}{\end{itemize}}
\newcommand{\beq}{\begin{eqnarray*}}
\newcommand{\eeq}{\end{eqnarray*}}
\newcommand{\bieq}{\begin{IEEEeqnarray}{rCl}}
\newcommand{\bieqx}{\begin{IEEEeqnarray}{+rCl+x*}}
\newcommand{\eieq}{\end{IEEEeqnarray}}
\newcommand{\nn}{\nonumber}
%\renewcommand{\i}{\item}
\newcommand{\bpm}{\begin{pmatrix}}
\newcommand{\epm}{\end{pmatrix}}
\newcommand{\sol}{\indent{\bf\emph{Solution:}}}
\newcommand{\ssol}{\indent{\\[2mm]\bf\emph{Solution:}}\;}
\newcommand{\hint}{\indent{\bf\emph{Hint}:}\;}
\newcommand{\note}{\indent{\bf\emph{Note}:}\;}
\newcommand{\vsk}{\vskip 2mm}
%%%%%%%%%%%%%%%%%%%%%%%%% Calculus %%%%%%%%%%%%%%%%%%%%%%%%%%%%
\newcommand{\dd}[2]{\ds\frac{d}{d{#1}}\left[{#2}\right]}
\newcommand{\der}[2]{\ds\frac{d{#1}}{d{#2}}}
\newcommand{\lmt}[3]{\ds\lim_{{#1}\to{#2}}{#3}}
\renewcommand{\iint}[2]{\ds\int{#1}\,d{#2}}
\newcommand{\dint}[4]{\ds\int^{#4}_{#3}{#1}\,d{#2}}
\renewcommand{\Delta}{\triangle}
%%%%%%%%%%%%%%%%%%%%%%%%% Number Sets %%%%%%%%%%%%%%%%%%%%%%%%%%
\newcommand{\N}{\mathbb{N}}
\newcommand{\Z}{\mathbb{Z}}
\newcommand{\Q}{\mathbb{Q}}
\newcommand{\R}{\mathbb{R}}
\newcommand{\C}{\mathbb{C}}
\newcommand{\F}{\mathcal{F}}
\renewcommand{\P}{\mathbb{P}}
\newcommand{\E}{\mathcal{E}}
\renewcommand{\o}{\omega}
\renewcommand{\O}{\Omega}
%%%%%%%%%%%%%%%%%%%%%%%%% Vectors %%%%%%%%%%%%%%%%%%%%%%%%%%%%%
\newcommand{\x}{\bar{x}}
\renewcommand{\v}{\bar{v}}
\newcommand{\y}{\bar{y}}
\newcommand{\z}{\bar{z}}
\newcommand{\w}{\bar{w}}
\renewcommand{\u}{\bar{u}}
\renewcommand{\b}{\bar{b}}
\newcommand{\e}{\bar{e}}
\renewcommand{\a}{\vec{a}}
\renewcommand{\r}{\vec{r}}
\newcommand{\vv}{\vec{v}}
\newcommand{\vecPQ}[2]{\overrightarrow{#1}{#2}}
\newcommand{\vecV}[1]{\overrightarrow{#1}}
\newcommand{\la}{\langle}
\newcommand{\ra}{\rangle}
%%%%%%%%%%%%%%%%%%%%%%%%%%% Vector Spaces %%%%%%%%%%%%%%%%%%%%
\newcommand{\rn}{\ensuremath{\mathbb{R}^n}}
\renewcommand{\rm}{\ensuremath{\mathbb{R}^m}}
\newcommand{\re}{\mathbb{R}}
\newcommand{\Pn}{\mathbb{P}_n}
\newcommand{\B}{\mathcal{B}}
%%%%%%%%%%%%%%%%%%%%%%%%%%% Graphics %%%%%%%%%%%%%%%%%%%%%%%%
\newcommand{\cg}[2]{\begin{center}
\includegraphics[scale={#1}]{{#2}}
\end{center}}
\makeatletter
\def\imod#1{\allowbreak\mkern10mu({\operator@font mod}\,\,#1)}
\makeatother

%%%%%%%%%%%%%%%%%%%%%%%%%%%%%%%%%%%%%%%%%%%%%%%%%%%%%%%%%%%%%%%%%%%%%%%%%%%%%%%%%%%%%%%%%%%%%%
%%%%%%%%%%%%%%%%%%%%%%%%%%%%%% Defined Fonts %%%%%%%%%%%%%%%%%%%%%%%%%%%%%%%%%%%%%%%%%%%%%%%%%
%%%%%%%%%%%%%%%%%%%%%%%%%%%%%%%%%%%%%%%%%%%%%%%%%%%%%%%%%%%%%%%%%%%%%%%%%%%%%%%%%%%%%%%%%%%%%%

\font\minihelv=phvr at 6pt
\font\helv=phvr at 10pt
\font\medhelv=phvr at 16pt
\font\bighelv=phvr at 20pt
\font\hugehelv=phvr at 36pt
\font\mybigfont=phvr at 16pt
\font\mymediumfont=phvr at 14pt
\font\mediumhelv=phvr at 14pt
\font\mybfit=ptmbi at 12pt


%%%%%%%%%%%%%%%%%%%%%%%%%%%%%%%%%%%%%%%%%%%%%%%%%%%%%%%%%%%%%%%%%%%%%%%%%%%%%%%%%%%%%%%%%%%%%%%
%%%%%%%%%%%%%%%%%%%%%%%%%%%%%% Other Commands %%%%%%%%%%%%%%%%%%%%%%%%%%%%%%%%%%%%%%%%%%%%%%%%%
%%%%%%%%%%%%%%%%%%%%%%%%%%%%%%%%%%%%%%%%%%%%%%%%%%%%%%%%%%%%%%%%%%%%%%%%%%%%%%%%%%%%%%%%%%%%%%%
%\setlength\fboxrule{.5pt}
%\newcommand{\latexpicborder}[3]{
%\setlength\fboxsep{30pt}
%\begin{figure}[hb]
%\begin{center}
%\fbox{
%\input{#1}
%}
%\caption{#2}
%\label{#3}
%\end{center}
%\end{figure}
%\setlength\fboxsep{0pt}
%}
%
%\newcommand{\latexpic}[2]{
%\begin{figure}[hb]
%\begin{center}
%\input{#1}
%\vspace*{8mm}
%\caption{#2}
%\end{center}
%\end{figure}
%}

%\begin{minipage}[b]{0.6\linewidth}
%......
%\end{minipage}
%\hspace{0.5cm}
%\begin{minipage}[t]{0.4\linewidth}
%\centering
%\includegraphics[scale=.5]{m1401_ex3_g4.eps}
%\end{minipage}
%\end{figure}


%%%%%%%%%%%%%%%%%%%%%%%%%%%%%%%%%%%%%%%%%%%%%%%%%%%%%%%%%%%%%%%%%%%%%%%%%%%%%%%%%%%%%%%%%%%%%%
%%%%%%%%%%%%%%%%%%%%%%%%%%% IEEEeqnarray Notes %%%%%%%%%%%%%%%%%%%%%%%%%%%%%%%%%%%%%%%%%%%%%%%
%%%%%%%%%%%%%%%%%%%%%%%%%%%%%%%%%%%%%%%%%%%%%%%%%%%%%%%%%%%%%%%%%%%%%%%%%%%%%%%%%%%%%%%%%%%%%%


%Any number of columns can be specified with IEEEeqnarray: {c} will give only one
%column with all entries centered, or {rCll} would add a fourth, left-justified
%column to use for comments. Moreover, beside l, c, r, L, C, R for math mode
%entries there are also s, t, u for left, centered, and right text mode entries.
%Additional space can be added with . and / and ? in increasing order.
%
%
%\begin{proof}
%This is a proof that ends
%with an equation array:
%\begin{IEEEeqnarray*}{+rCl+x*}
%a & = & b + c \\
%& = & d + e. & \qedhere
%\end{IEEEeqnarray*}
%\end{proof}
%Note that the + in {+rCl+x*} denotes stretchable spaces, one on the left
%of the equations (which, if not specified, will be done automatically by
%IEEEeqnarray!) and one on the right of the equations. But now on the right,
%after the stretching column, we add an empty column x. This column will be
%only needed on the last line when we will put the \qedhere command there.
%Finally, we specify a *. This is a null-space that prevents IEEEeqnarray to
%add another unwanted +-space!


% The following environments enable custom numbering of theorems so that the numbers agree % with the numbering in the textbook being used. 
%
%  Usage examples:
%\begin{customthm}{2.2}\label{eight}
%Every theorem must be numbered by hand.
%\end{customthm}
%
%Here is a reference to theorem~\ref{eight}.
%
%\begin{customthm}{2.3}[Parenthetical comment]\label{nine}
%Statement
%\end{customthm}
%
%Here is a reference to theorem~\ref{nine}


\newtheorem{innercustomthm}{Theorem}
\newenvironment{customthm}[1]
  {\renewcommand\theinnercustomthm{#1}\innercustomthm}
  {\endinnercustomthm}
  
  \newtheorem{innercustomprop}{Proposition}
\newenvironment{customprop}[1]
  {\renewcommand\theinnercustomprop{#1}\innercustomprop}
  {\endinnercustomprop}
  
    \newtheorem{innercustomlem}{Lemma}
\newenvironment{customlem}[1]
  {\renewcommand\theinnercustomlem{#1}\innercustomlem}
  {\endinnercustomlem}
  
    \newtheorem{innercustomconj}{Conjecture}
\newenvironment{customconj}[1]
  {\renewcommand\theinnercustomconj{#1}\innercustomconj}
  {\endinnercustomconj}
  
    \newtheorem{innercustomclaim}{Claim}
\newenvironment{customclaim}[1]
  {\renewcommand\theinnercustomclaim{#1}\innercustomclaim}
  {\endinnercustomclaim}
  
    \newtheorem{innercustomcor}{Corollary}
\newenvironment{customcor}[1]
  {\renewcommand\theinnercustomcor{#1}\innercustomcor}
  {\endinnercustomcor}
  
    \newtheorem{innercustomdef}{Definition}
\newenvironment{customdef}[1]
  {\renewcommand\theinnercustomdef{#1}\innercustomdef}
  {\endinnercustomdef}
  
    \newtheorem{innercustomex}{Example}
\newenvironment{customex}[1]
  {\renewcommand\theinnercustomex{#1}\innercustomex}
  {\endinnercustomex}
  
    \newtheorem{innercustomass}{Assumption}
\newenvironment{customass}[1]
  {\renewcommand\theinnercustomass{#1}\innercustomass}
  {\endinnercustomass}
  
      \newtheorem{innercustomax}{Axiom}
\newenvironment{customax}[1]
  {\renewcommand\theinnercustomax{#1}\innercustomax}
  {\endinnercustomax}
  

\vfuzz2pt % Don't report over-full v-boxes if over-edge is small
\hfuzz2pt % Don't report over-full h-boxes if over-edge is small

\renewcommand{\ni}{\noindent}


%%%%%%%%%%%%%%%%%%%%%%%%%%%%%%%%%%%%%%%%%%%%%%%%%%%%%%
%%%%%%%%%%%%%%%%%%%%%%%%%%%%%%%%%%%%%%%%%%%%%%%%%%%%%%

\pagestyle{myheadings}

%%%%%%%%%%%%%%%%%%%%%%%%%%%%%%%%%%%%%%%%%%%%%%%%%%%%%%

%%%%%%%%%%%%%%%%%%%%%%%%%%%%%%%%%%%%%%%%%%%%%%%%%%%%%%
%%%%%%%%%%%%%%%%%%%%%%%%%   Document Body   %%%%%%%%%%
%%%%%%%%%%%%%%%%%%%%%%%%%%%%%%%%%%%%%%%%%%%%%%%%%%%%%%

%\def\classnum{3810}
%\def\classtitle{Probability}
%\def\classtitleshort{Probability}
%\def\classsec{001}
%\def\classterm{Fall 2025}
%\def\instructor{Robert Rostermundt}
\def\printsol{0}


	\title{\vspace{-1in}Math\classnum\;-\;\classtitle\\
	Section\;\classsec\;-\;\classterm\\
	Notes: Bayes' Rule and Disease Testing}
	\author{University of Colorado Denver / College of Liberal Arts 	and Sciences}
	\date{Department of Mathematics - \instructor}

	\markright{Math\classnum\;-\;\classtitleshort, UCD, \classterm, \instructor}



%%%%%%%%%%%%%%%%%%%%%%%%%%%%%%%%%%%%%%%%%%%%%%%%%%%%%%
\begin{document}\maketitle\thispagestyle{empty}
%%%%%%%%%%%%%%%%%%%%%%%%%%%%%%%%%%%%%%%%%%%%%%%%%%%%%%



%%%%%%%%%%%%%%%%%%%%%%%%%%%%%%%%%%%%%%%%%%%%%%%%%%%%%%%%%%%%%%%%%%%%%%%%%%%%%%%%%%%%%%%%%%%%%%%%%%%%%%
\vspace*{2mm}
\hrule
\vskip 8mm


%%%%%%%%%%%%%%%%%%%%%%%%%%%%%%%%%%%%%%%%%%%%%%%%%%%%%%%%%%%%%%%%%%%%%%%%%
%%%%%%%%%%%%%%%%%%%%%%%%%%%%%%%%%%%%%%%%%%%%%%%%%%%%%%%%%%%%%%%%%%%%%%%%%


%%%%%%%%%%%%%%%%%%%%%%%%%%%%%%%%%%%%%%%%%%%%%%%%%%%%%%%%%%%%%%%%%%%%%%%%%%%%%%%%%%%%%%%%%%%%%%%%%%%%%%
\section*{The Problem:}
%%%%%%%%%%%%%%%%%%%%%%%%%%%%%%%%%%%%%%%%%%%%%%%%%%%%%%%%%%%%%%%%%%%%%%%%%%%%%%%%%%%%%%%%%%%%%%%%%%%%%%

\noindent We are tested for a particular disease. We know that 5\% of the population has the disease and we also have empirical data about the accuracy of the test. The test has sensitivity 95\% -- the ability of the test to correctly identify those who have the disease). Also, the test specificity is 90\% -- the ability of the test to correctly identify those who do not have the disease. We want to know the probability of having the disease given a positive test result.

%%%%%%%%%%%%%%%%%%%%%%%%%%%%%%%%%%%%%%%%%%%%%%%%%%%%%%%%%%%%%%%%%%%%%%%%%%%%%%%%%%%%%%%%%%%%%%%%%%%%%%
\section*{Simulations:}
%%%%%%%%%%%%%%%%%%%%%%%%%%%%%%%%%%%%%%%%%%%%%%%%%%%%%%%%%%%%%%%%%%%%%%%%%%%%%%%%%%%%%%%%%%%%%%%%%%%%%%
I have simulated this process in R. Below are graphics from four different simulations with running proportions (up to $n=100,000$) of those having the disease from patients testing positive. In each simulation the proportion seems to be converging to a common proportion $p=1/3$. The R code is provided the end of the document.

\begin{center}
\includegraphics[scale=0.31]{m3810_bayes_medical_simulation1.jpeg}\qquad\qquad
\includegraphics[scale=0.31]{m3810_bayes_medical_simulation2.jpeg}\\
\vskip 5mm
\includegraphics[scale=0.31]{m3810_bayes_medical_simulation3.jpeg}\qquad\qquad
\includegraphics[scale=0.31]{m3810_bayes_medical_simulation4.jpeg}\\
\end{center}

\vfill\eject

The following graphic shows the results of 50 different simulations. We can see that in all simulations the proportion of those having the disease given a positive test result are converging to the proportion $p=1/3$.
\vskip 1mm
\begin{figure}[h!]
\begin{center}
\includegraphics[scale=0.48]{bayes_medical_simulation_multiple.jpeg}
\caption{50 Simulations of True Positive Test Results}
\end{center}
\end{figure}
\vskip 5mm
Given these simulations and our frequency approach to probability theory we might suspect that the true probability is about $p=1/3$.

%%%%%%%%%%%%%%%%%%%%%%%%%%%%%%%%%%%%%%%%%%%%%%%%%%%%%%%%%%%%%%%%%%%%%%%%%%%%%%%%%%%%%%%%%%%%%%%%%%%%%%
\section*{Theoretical Tools:}
%%%%%%%%%%%%%%%%%%%%%%%%%%%%%%%%%%%%%%%%%%%%%%%%%%%%%%%%%%%%%%%%%%%%%%%%%%%%%%%%%%%%%%%%%%%%%%%%%%%%%%

\noindent 

	\begin{itemize}
		\item {\bf\emph{Conditional Probability:}} Let $(\Omega,\mathcal{F},\P)$ be a probability space and $A,B\in\mathcal{F}$ where $\P(B)\not=0$. Then we define the conditional probability of $A$ given $B$ as
\[\P\big(A\,|\,B\big)=\ds\frac{\P\big(A\cap B\big)}{\P(B)}.\]
We note that from this we can write 
\[\P\big(A\cap B\big)=\P\big(B\big)\cdot\P\big(A\,|\,B\big)\quad\text{and}\quad\P\big(A\cap B\big)=\P\big(A\big)\cdot\P\big(B\,|\,A\big).\]
		\item {\bf\emph{Total Probability:}} Let $(\Omega,\mathcal{F},\P)$ be a probability space where $B\in\mathcal{F}$ is a non-zero probability event. Moreover suppose $A_1,A_2,\dots,A_n$ forms a partition of the sample space $\Omega$ where $\P(A_k)\not=0$ when $1\le k\le n$. Then we can write
\[\P\big(B\big)=\P\big(A_1\big)\cdot\P\big(B\,|\,A_1\big)+\P\big(A_2\big)\cdot\P\big(B\,|\,A_2\big)+\cdots+\P\big(A_n\big)\cdot\P\big(B\,|\,A_n\big).\]
This is one of the main tools in a ``divide and conquer" strategy in evaluating probability.

\vfill\eject

		\item {\bf\emph{Baye's Rule:}} Let $(\Omega,\mathcal{F},\P)$ be a probability space where $B\in\mathcal{F}$ is a non-zero probability event. Moreover suppose $A_1,A_2,\dots,A_n$ forms a partition of the sample space $\Omega$ where $\P(A_k)\not=0$ when $1\le k\le n$. Then we can write
\[\P\big(A_k\,|\,B\big)=\ds\frac{\P\big(A_k\big)\P\big(B\,|\,A_k\big)}{\ds\sum^{n}_{k=1}\P\big(A_k\big)\cdot\P\big(B\,|\,A_k\big)}\]		
Bayes' Rule is simply a combination of conditional probability and total probability, but it is important when trying to make probabalistic inference. In particular, we can think of $A_1,A_2,\dots,A_n$ as $n$ possible states of the world and event $B$ as evidence obtained from experiment. Then $\P(A_k\,|\,B)$ can be interpreted as follows: what is the probability that we are in the state $A_k$ given the evidence outcome in event $B$?		
	\end{itemize} 


%%%%%%%%%%%%%%%%%%%%%%%%%%%%%%%%%%%%%%%%%%%%%%%%%%%%%%%%%%%%%%%%%%%%%%%%%%%%%%%%%%%%%%%%%%%%%%%%%%%%%%
\section*{Calculations:}
%%%%%%%%%%%%%%%%%%%%%%%%%%%%%%%%%%%%%%%%%%%%%%%%%%%%%%%%%%%%%%%%%%%%%%%%%%%%%%%%%%%%%%%%%%%%%%%%%%%%%%

\noindent Let $D$ be the event that we have the disease and $E$ be the event that we test positive for the disease. Then, since $D$ and $D^C$ partition the sample space, we have
\beq
\P\big(D\,|\,E\big)&=&\ds\frac{\P\big(D\big)\cdot\P\big(E\,|\,D\big)}{\P\big(D\big)\cdot\P\big(E\,|\,D\big)+\P\big(D^C\big)\cdot\P\big(E\,|\,D^C\big)}\\
\\
&=&\ds\frac{(0.05)\cdot(0.95)}{(0.05)\cdot(0.95)+(0.95)\cdot(0.10)}\\
\\
&=&0.\overline{333}
\eeq 
That is, $\P\big(D\,|\,E\big)=1/3$ as we suspected from the simulations. So just testing positive with a very accurate test does not mean you have a high probability of actually having the disease. With a little thought, the reason for this surprising result as the low probability of having the disease.

\vskip 1cm


%%%%%%%%%%%%%%%%%%%%%%%%%%%%%%%%%%%%%%%%%%%%%%%%%%%%%%%%%%%%%%%%%%%%%%%%%%%%%%%%%%%%%%%%%%%%%%%%%%%%%%
\section*{The R Code:}
%%%%%%%%%%%%%%%%%%%%%%%%%%%%%%%%%%%%%%%%%%%%%%%%%%%%%%%%%%%%%%%%%%%%%%%%%%%%%%%%%%%%%%%%%%%%%%%%%%%%%%

Here is the R code used for the above simulations.
\vspace*{2mm}
\small 
\begin{lstlisting}[language=R]

########################################################################
# This simulates the probability of having a particular disease
# given a positive test result.

# install.packages("ggrepel") if necessary
library(ggplot2)
library(ggrepel)  # for table labels on plot

# --- Parameters ---
p_disease <- 0.05
sens <- 0.95
spec <- 0.90

# True Bayesian probability
p_true <- p_disease * sens / (p_disease * sens + (1 - p_disease) * (1 - spec))

# --- Simulation function ---
simulate_medical <- function(n, p_disease, sens, spec) {
  disease <- rbinom(n, 1, p_disease)   # 1 = has disease, 0 = does not
  test <- integer(n)
  
  # Conditional test results
  test[disease == 1] <- rbinom(sum(disease == 1), 1, sens)
  test[disease == 0] <- rbinom(sum(disease == 0), 1, 1 - spec)
  
  # Probability of disease given positive test
  p_given_positive <- mean(disease[test == 1])
  return(p_given_positive)
}

# --- Simulate multiple sample sizes ---
sample_sizes <- c(100, 500, 1000, 5000, 10000, 50000, 1e5)
results <- sapply(sample_sizes, simulate_medical,
                  p_disease = p_disease,
                  sens = sens,
                  spec = spec)

# --- Create table ---
df <- data.frame(
  n = sample_sizes,
  p_given_positive = results
)


# --- Plot ---
ggplot(df, aes(x = n, y = p_given_positive)) +
  geom_line(size = 1.2, color = "blue") +
  geom_point(size = 2) +
  scale_x_log10() +
  geom_hline(yintercept = p_true, linetype = "dotted", color = "red", size = 1) +
  geom_text(aes(label = round(p_given_positive,3)), vjust = -0.5, size = 3.5) +
  labs(
  title = paste0("Single Simulation: Probability of Disease Given Positive Test\n",
                      "Parameters: Prob(disease) = ", p_disease,
                      ", Sensitivity = ", sens,
                      ", Specificity = ", spec),
       subtitle = paste0("Red dotted line = true probability (Bayes): ", 
       round(p_true,3)),
       x = "Number of Simulated People (log scale)",
       y = "Estimated Probability") +
  theme_minimal()


print(df)  # Table output




################################################################################
################################################################################
# Run Multiple Simulations
################################################################################
################################################################################

library(ggplot2)
library(dplyr)

# --- Parameters ---
p_disease <- 0.05
sens <- 0.95
spec <- 0.90

# True Bayesian probability
p_true <- p_disease * sens / (p_disease * sens + (1 - p_disease) * (1 - spec))

# --- Simulation function ---
simulate_medical <- function(n, p_disease, sens, spec) {
  disease <- rbinom(n, 1, p_disease)
  test <- integer(n)
  test[disease == 1] <- rbinom(sum(disease == 1), 1, sens)
  test[disease == 0] <- rbinom(sum(disease == 0), 1, 1 - spec)
  mean(disease[test == 1])
}

# --- Multiple simulations per sample size ---
sample_sizes <- c(100, 500, 1000, 5000, 10000)
n_sims <- 50

all_results <- lapply(sample_sizes, function(n) {
  replicate(n_sims, simulate_medical(n, p_disease, sens, spec))
})

# --- Convert to long-format data frame ---
df_long <- data.frame(
  n = rep(sample_sizes, each = n_sims),
  p_given_positive = unlist(all_results)
)

# --- Summarize mean +/- SD ---
df_summary <- df_long %>%
  group_by(n) %>%
  summarize(
    mean_p = mean(p_given_positive),
    sd_p = sd(p_given_positive)
  ) %>%
  ungroup() %>%
  mutate(label = paste0(round(mean_p,3), " +/- ", round(sd_p,3)))


# --- Plot ---
ggplot(df_long, aes(x = n, y = p_given_positive)) +
  geom_jitter(width = 0.1, height = 0, alpha = 0.3, color = "blue") + 
  geom_line(data = df_summary, aes(x = n, y = mean_p), color = "darkblue", 
  size = 1.2) +  # mean line
  geom_errorbar(data = df_summary, 
                aes(x = n, ymin = mean_p - sd_p, ymax = mean_p + sd_p, y = mean_p),
                width = 0.05, color = "darkblue") +  # +/-1 SD
  geom_text(data = df_summary, aes(x = n, y = mean_p + 0.06, label = label), 
            size = 3.5, color = "black") +  # table overlay above points
  scale_x_log10() +
  geom_hline(yintercept = p_true, linetype = "dotted", color = "red", size = 1) +
  labs(
    title = paste0("Multiple Simulations: Probability Disease With Positive Test\n",
                   "Parameters: Prob(disease) = ", p_disease,
                   ", Sensitivity = ", sens,
                   ", Specificity = ", spec),
    subtitle = paste0(
      "Red dotted line = true probability (Bayes): ", round(p_true,3), "\n",
      "Blue line = mean of ", n_sims, " simulations; error bars = +/-1 SD"
    ),
    x = "Number of Simulated People (log scale)",
    y = "Estimated Probability"
  ) +
  theme_minimal()


# --- Print summary table ---
print(df_summary)

\end{lstlisting}

\vskip 1cm
\hrule
\vskip 5mm
\begin{center}
\bf Please let me know if you have any questions, comments, or corrections!
\end{center}

%%%%%%%%%%%%%%%%%%%%%%%%%%%%%%%%%%%%%%%%%%%%%%%%%%%%%%
\end{document}
%%%%%%%%%%%%%%%%%%%%%%%%%%%%%%%%%%%%%%%%%%%%%%%%%%%%%%
