\documentclass[11pt]{article}
\usepackage[suffix=Solutions]{teaching-header}

%%%%%%%%%%%%%%%%%%%%%%%%%%%%%%%%%%%%%%%%%%%%%%%%%%%%%%%%%%%%%%%%%%%%%%%%%%%%
%%%%%%%%%%%%%%%%%%%%%%%%%%%%%%%%%%%%%%%%%%%%%%%%%%%%%%%%%%%%%%%%%%%%%%%%%%%%


%\ifnum\printsol=0 (when no solutions printed)
%Do something
%	\else  (when solutions are printed)
%Do something else
%\fi


% Assessment specific definitions to fill in title and page headers. Other course specific definitions are located in classinfo document which is accessed by \input command below. 
\def\assessmenttype{????}
\def\assessmenttypecap{????}
\def\assessmentnameshort{????}
\def\assessmentname{\#???}  % Such as #1 or a word title
\def\assessmentnamesol{\#???? Solutions}
\def\duedate{Tuesday, November 5}
\def\duetime{9:30am}
\def\printfancyheader{1} % Beware that changing the header will change custom pagebreaks for boxed solutions

\def\classnum{3810}
\def\classtitle{Probability}
\def\classtitleshort{Probability}
\def\classsec{001}
\def\classterm{Fall 2025}
\def\instructor{Robert Rostermundt}
%\def\hmwknum{\#2}  % Class Specific Information for title header

%If use fancy title header then include custom header on all pages. If don't use fancy title header then include pagenumbers only when solutions printed
\ifnum\printfancyheader=1
\pagestyle{myheadings}
	\else
		\ifnum\printsol=0
				\pagestyle{empty}
			\else
				\pagestyle{plain}
		\fi		
\fi


% Creates a fancy title header for front page. This is turned on by boolean \printfancyheader which later invokes \maketitle command. Otherwise a simpler header is displayed.
\ifnum\printsol=0 % Determine if non-solution fancyheader or solution fancyheader is displayed
\title{\vspace{-1in}Math\classnum\;-\;\classtitle\\
	Section\;\classsec\;-\;\classterm\\
	\assessmenttypecap\;\assessmentname}
	\author{University of Colorado Denver / College of Liberal Arts 	and Sciences}
	\date{Department of Mathematics - Dr. \instructor}

	\markright{Math\classnum\;-\;\classtitleshort,\;\assessmenttypecap\;\assessmentname, UCD, \classterm, Dr. \instructor}
	
		\else

\title{\vspace{-1in}Math\classnum\;-\;\classtitle\\
	Section\;\classsec\;-\;\classterm\\
	\assessmenttypecap\;\assessmentnamesol}
	\author{University of Colorado Denver / College of Liberal Arts 	and Sciences}
	\date{Department of Mathematics - Dr. \instructor}

	\markright{Math\classnum\;-\;\classtitleshort,\;\assessmenttypecap\;\assessmentname, UCD, \classterm, Dr. \instructor}

\fi


%%%%%%%%%%%%%%%%%%%%%%%%%%%%%%%%%%%%%%%%%%%%%%%%%%%%%%%%%%%%%%%%%%%%%%%%%%%%
%%%%%%%%%%%%%%%%%%%%%%%%%%%%%%%%%%%%%%%%%%%%%%%%%%%%%%%%%%%%%%%%%%%%%%%%%%%%
\begin{document}
%%%%%%%%%%%%%%%%%%%%%%%%%%%%%%%%%%%%%%%%%%%%%%%%%%%%%%%%%%%%%%%%%%%%%%%%%%%%
%%%%%%%%%%%%%%%%%%%%%%%%%%%%%%%%%%%%%%%%%%%%%%%%%%%%%%%%%%%%%%%%%%%%%%%%%%%%
\ifnum\printfancyheader=1
\maketitle\thispagestyle{empty}
	\else	
		\ifnum\printsol=0

				\begin{center}{\large Math \classnum--\classsec, UCD, \classterm, Homework \assessmentname}\\
\smallskip
Due by \duedate\, at \duetime.\\
\smallskip
{\em Late homeworks will not be accepted without prior approval. Please no email submissions!}
				\end{center}
			\else

				\begin{center}{Math \classnum--\classsec, UCD, \classterm \hfill Homework \assessmentnamesol}
				\end{center}
\vskip 2mm

		\fi
		
\hrule
\vskip 5mm

\fi



%%%%%%%%%%%%%%%%%%%%%%%%%%%%%%%%%%%%%%%%%%%%%%%%%%%%%%%%%%%%%%%%%
%%%%%%%%%%%%%%%%%%%%%%%%%%%%%%%%%%%%%%%%%%%%%%%%%%%%%%%%%%%%%%%%%


\normalsize

\ifnum\printsol=0

%%%%%%%%%%%%%%%%%%%%%%%%%%%%%%%%%%%%%%%%%%%%%%%%%%%%%%
\hrule\bigskip\noindent\textbf{Name:}\hspace*{3in}
\textbf{Student Number:}\bigskip\hrule
%%%%%%%%%%%%%%%%%%%%%%%%%%%%%%%%%%%%%%%%%%%%%%%%%%%%%%

%%%%%%%%%%%%%%%%%%%%%%%%%%%%%%%%%%%%%%%%%%%%%%%%%%%%%%
\vspace*{5mm}
\begin{itemize}\itemsep=0in
	\item This is an open note \assessmenttype. You may use your notes on this \assessmenttype but may not use your book. All electronics must be turned off during the exam.
	\item Write all answers in the spaces provided on the \assessmenttype.
	\item Throughout the \assessmenttype, show your work so that your reasoning is clear. Otherwise no credit will be given.
	\item If you are asked to \emph{find} or \emph{write} a formula or an expression without proof, you do not need to show any work but must present your solution using correct mathematical notation.
	\item If you are asked to \emph{give} or \emph{cite} a definition or a result that we discussed in class, you must use complete sentences, correct mathematical notation, and state all conditions or assumptions.
	\item If you are asked to \emph{prove} or \emph{show} a formula or a result, you must give a proof using complete sentences and correct mathematical notation. If you can correctly cite a result that we have proved in class or that is proven in the book, you can use it and do not have to prove it again. 
	\item The following table gives the total points for each problem on this \assessmenttype.

\begin{center}
{\renewcommand{\arraystretch}{1.5}
\renewcommand{\tabcolsep}{0.2cm}
\begin{tabular}{||c||c||c|c|c|c||}
\hline
\rule{0pt}{3ex}Problem & Exam \% & Score  &  Problem & Exam \% & Score \\ 
\hline \hline
1  & ?? & \hspace{1.0in}  &   &    & \hspace{1.0in} \\ \hline
2  & ?? & \hspace{1.0in}  &   &    & \hspace{1.0in} \\ \hline
3  & ?? & \hspace{1.0in}  &   &    & \hspace{1.0in} \\ \hline
4  & ?? & \hspace{1.0in}  &   &    & \hspace{1.0in} \\ \hline
5  & ?? & \hspace{1.0in}  &   &    & \hspace{1.0in} \\ \hline
6  & ?? & \hspace{1.0in}  &   &    & \hspace{1.0in} \\ \hline
\hline
\rule{0pt}{3ex}Total &  100 &  &    &  & \hspace{1.0in} \\ \hline
\end{tabular}}
\end{center}

%
\end{itemize}
\vspace*{5mm}

\begin{flushright}
Total Score:\;\underline{\hspace*{2cm}}\;/\;10

\end{flushright}

\vfill\eject

\fi


%%%%%%%%%%%%%%%%%%%%%%%%%%%%%%%%%%%%%%%%%%%%%%%%%%%%%%
%%%%%%%%%%%%%%%%%%%%%%%%%%%%%%%%%%%%%%%%%%%%%%%%%%%%%%
\ifnum\printsol=1
\vspace*{2mm}
\hrule
\vskip 8mm

\fi

%%%%%%%%%%%%%%%%%%%%%%%%%%%%%%%%%%%%%%%%%%%%%%%%%%%%%%%%%%%%%%%%%
%%%%%%%%%%%%%%%%%%%%%%%%%%%%%%%%%%%%%%%%%%%%%%%%%%%%%%%%%%%%%%%%%


\renewcommand{\S}{\mathbb{S}} % For defining a set S 
\renewcommand{\P}{\mathbb{P}} % For defining a probabilty measure
\newcommand{\prob}[1]{\mathbb{P}\left(#1\right)} % For defining a probability measure with argument and normal parentheses
\newcommand{\Prob}[1]{\mathbb{P}\Big(#1\Big)} % For defining a probability measure with argument and large parenthesis


%%%%%%%%%%%%%%%%%%%%%%%%%%%%%%%%%%%%%%%%%%%%%%%%%%%%%%%%%%%%%%%%%
%%%%%%%%%%%%%%%%%%%%%%%%%%%%%%%%%%%%%%%%%%%%%%%%%%%%%%%%%%%%%%%%%


\begin{enumerate}

	\item\exampercent{???} Some of the following conditions are true for a probability measure $\P$. Mark the box for each condition that is true (and leave all other boxes blank). If a statement is false provide a counterexample or correct it in the space provided.
\ifnum\printsol=0

	\begin{enumerate}

		\item[] $\square$\qquad 
		\vskip 0.55in

		\item[] $\square$\qquad 
		\vskip 0.55in

		\item[] $\square$\qquad 
		\vskip 0.55in

		\item[] $\square$\qquad 
		\vskip 0.55in

		\item[] $\square$\qquad 
		\vskip 0.55in

		\item[] $\square$\qquad 
		\vskip 0.55in

		\item[] $\square$\qquad 
		\vskip 0.55in

		\item[] $\square$\qquad 
		\vskip 0.55in

		\item[] $\square$\qquad 
		\vskip 0.55in

		\item[] $\square$\qquad 

\vfill\eject

	\end{enumerate}

\fi
\ifnum\printsol=1
\ssol See the following checked and unchecked boxes and corresponding discussions.

	\begin{enumerate}

		\item[] $\boxtimes$\qquad 
		\vskip 2mm


		\item[] $\square$\qquad 
		\vskip 2mm
	\begin{discussion}
		
	\end{discussion}
		\vskip 2mm


		\vskip 1cm

	\end{enumerate}

\fi



%%%%%%%%%%%%%%%%%%%%%%%%%%%%%%%%%%%%%%%%%%%%%%%%%%%%%%%%%%%%%%%%%
%%%%%%%%%%%%%%%%%%%%%%%%%%%%%%%%%%%%%%%%%%%%%%%%%%%%%%%%%%%%%%%%%

	\item\exampercent{??} 

\boxsol{
	\vskip 2mm
	\begin{boxsolution}
			
	\end{boxsolution}
	
	\vskip 5mm

}

\ifnum\printsol=0
\vfill\eject

\fi


%%%%%%%%%%%%%%%%%%%%%%%%%%%%%%%%%%%%%%%%%%%%%%%%%%%%%%%%%%%%%%%%%
%%%%%%%%%%%%%%%%%%%%%%%%%%%%%%%%%%%%%%%%%%%%%%%%%%%%%%%%%%%%%%%%%



	\item\exampercent{??} 

\boxsol{
	\vskip 2mm
	\begin{boxsolution}
			
	\end{boxsolution}
	
	\vskip 5mm

}

\ifnum\printsol=0
\vfill\eject

\fi


%%%%%%%%%%%%%%%%%%%%%%%%%%%%%%%%%%%%%%%%%%%%%%%%%%%%%%%%%%%%%%%%%
%%%%%%%%%%%%%%%%%%%%%%%%%%%%%%%%%%%%%%%%%%%%%%%%%%%%%%%%%%%%%%%%%

	\item\exampercent{??} 

\boxsol{
	\vskip 2mm
	\begin{boxsolution}
			
	\end{boxsolution}
	
	\vskip 5mm

}

\ifnum\printsol=0
\vfill\eject

\fi


%%%%%%%%%%%%%%%%%%%%%%%%%%%%%%%%%%%%%%%%%%%%%%%%%%%%%%%%%%%%%%%%%
%%%%%%%%%%%%%%%%%%%%%%%%%%%%%%%%%%%%%%%%%%%%%%%%%%%%%%%%%%%%%%%%%

	\item\exampercent{??} 

\boxsol{
	\vskip 2mm
	\begin{boxsolution}
			
	\end{boxsolution}
	
	\vskip 5mm

}

\ifnum\printsol=0
\vfill\eject

\fi


%%%%%%%%%%%%%%%%%%%%%%%%%%%%%%%%%%%%%%%%%%%%%%%%%%%%%%%%%%%%%%%%%
%%%%%%%%%%%%%%%%%%%%%%%%%%%%%%%%%%%%%%%%%%%%%%%%%%%%%%%%%%%%%%%%%


\end{enumerate}


%%%%%%%%%%%%%%%%%%%%%%%%%%%%%%%%%%%%%%%%%%%%%%%%%%%%%%%%%%%%%%%%%
%%%%%%%%%%%%%%%%%%%%%%%%%%%%%%%%%%%%%%%%%%%%%%%%%%%%%%%%%%%%%%%%%


\quescommentcorrection


\end{document}

 
 