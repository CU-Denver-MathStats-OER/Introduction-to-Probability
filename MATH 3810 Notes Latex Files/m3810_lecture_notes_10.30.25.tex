%%%%%%%%%%%%%%%%%%%%%%%%%%%%%%%%%%%%%%%%%%%%%%%%%%%%%%
\documentclass[11pt]{article}
%%%%%%%%%%%%%%%%%%%%%%%%%%%%%%%%%%%%%%%%%%%%%%%%%%%%%%

\usepackage{amsmath}
\usepackage{amsthm}
\usepackage{amssymb}
\usepackage{latexsym}
\usepackage{graphicx}
\usepackage{color}
\usepackage{verbatim}
\usepackage{float}
\usepackage{multicol}
\usepackage{xcolor}
\usepackage{listings}
\usepackage{tikz}
\usetikzlibrary{arrows.meta, positioning, calc}
\usetikzlibrary{decorations.pathmorphing}
\usepackage{tcolorbox}
\tcbuselibrary{breakable}
\usepackage{cancel}


\newtcolorbox{solutionbox}{
  breakable,
  colback=blue!5!white,
  colframe=blue!50!black,
  title=Solution,
  sharp corners,
  boxrule=0.8pt
}

\newtcolorbox{hintbox}{
  breakable,
  colback=gray!10!white,
  colframe=gray!50!black,
  title=Hint,
  sharp corners,
  boxrule=0.5pt
}

% Unnumbered theorem
\newtheorem*{thm*}{Theorem}

\lstdefinelanguage{R}{
      keywords={if,else,while,for,in,next,break,function,TRUE,FALSE,NULL,Inf,NA,NaN,switch,repeat,return,require,library},
      keywordstyle=\color{blue}\bfseries,
      identifierstyle=\color{black},
      comment=[l]{\#},
      commentstyle=\color{gray}\ttfamily,
      string=[b]{"},
      stringstyle=\color{red}\ttfamily,
      morecomment=[l]{//},
      morestring=[b]{'},
      sensitive=true,
      morekeywords={print,summary,plot,lm,glm,data,frame,read.csv,write.csv,factor,levels,names,colnames,rownames,
      head,tail,str,dim,length,class,typeof,mode,is.na,is.null,is.finite,is.infinite,is.nan,as.numeric,as.character,
      as.factor,as.Date,as.POSIXct,as.matrix,as.data.frame,rbind,cbind,merge,subset,aggregate,tapply,apply,lapply,sapply,
      mapply,vapply,replicate,seq,rep,c,list,matrix,array,data.frame,table,hist,boxplot,barplot,pie,curve,lines,points,text,
      abline,legend,par,mtext,title,xlab,ylab,xlim,ylim,main,sub,col,pch,cex,lty,lwd,type,bg,fg,args,options,warnings,errors,
      message,stop,warning,error,try,tryCatch,withCallingHandlers,on.exit,debug,browser,trace,recover,options,getOption,setOption},
    }


\setlength{\textheight}{9in}
\setlength{\textwidth}{6in}
\addtolength{\topmargin}{-2cm}
\addtolength{\oddsidemargin}{-1cm}
\parindent=0in


\def\classnum{3810}
\def\classtitle{Probability}
\def\classtitleshort{Probability}
\def\classsec{001}
\def\classterm{Fall 2025}
\def\instructor{Robert Rostermundt}
%\def\hmwknum{\#2}


%%%%%%%%%%%%%%%%%%%%%%%%%%%%%%%%%%%%%%%%%%%%%%%%%%%%%%%%%
%%%%%%%%%%%%%%%%%%%%%%%%%  Colors  %%%%%%%%%%%%%%%%%%%%%%
%%%%%%%%%%%%%%%%%%%%%%%%%%%%%%%%%%%%%%%%%%%%%%%%%%%%%%%%%

\definecolor{Green}{rgb}{0,.5,0}
%use for definitions
\definecolor{Red}{rgb}{.8,.2,0}
%use for emphasis
\definecolor{Yellow}{rgb}{.6,.6,.1}
%use for part titles
\definecolor{Cyan}{rgb}{.2,.6,.7}
%use for comments
\definecolor{Purple}{rgb}{.4,0,1}
%use for examples
\definecolor{deepred}{rgb}{.53,.29,.24}
%use for important points
\definecolor{Black}{rgb}{0,0,0}
%use for washout
\definecolor{Grey}{rgb}{.45,.45,.45}
% use for theorems
\newcommand{\tred}[1]{\textcolor{Red}{#1}}
\newcommand{\tgreen}[1]{\textcolor{Green}{#1}}
\newcommand{\tcyan}[1]{\textcolor{Cyan}{#1}}
\newcommand{\tyellow}[1]{\textcolor{Yellow}{#1}}
\newcommand{\tpurple}[1]{\textcolor{Purple}{#1}}
\newcommand{\tblack}[1]{\textcolor{Black}{#1}}
\newcommand{\tgrey}[1]{\textcolor{Grey}{#1}}
\newcommand{\tdeepred}[1]{\textcolor{deepred}{#1}}
\newcommand{\ttt}[1]{\texttt{#1}}

%%%%%%%%%%%%%%%%%%%%%%%%%%%%%%%%%%%%%%%%%%%%%%%%%%%%%%%%%
%%%%%%%%%%%%%%%%%%%%%%%%%  Theorem Environments  %%%%%%%%
%%%%%%%%%%%%%%%%%%%%%%%%%%%%%%%%%%%%%%%%%%%%%%%%%%%%%%%%%

\theoremstyle{plain}
\newtheorem{thm}{Theorem}
\newtheorem{axiom}{Axiom}
\newtheorem{cor}{Corollary}
\newtheorem{lemma}{Lemma}
\newtheorem{prop}{Proposition}
\newtheorem{ques}{Question}
\theoremstyle{definition}
\newtheorem{defn}{Definition}
\theoremstyle{remark}
\newtheorem{remark}{Remark}
\theoremstyle{definition}
\newtheorem{ex}{Example}
\numberwithin{equation}{section}
\newtheorem{prob}{Problem}
\numberwithin{equation}{section}


%%%%%%%%%%%%%%%%%%%%%%%%%%%%%%%%%%%%%%%%%%%%%%%%%%%%%%%%%
%%%%%%%%%%%%%%%%%%%%%%%%%  Math    %%%%%%%%%%%%%%%%%%%%%%
%%%%%%%%%%%%%%%%%%%%%%%%%%%%%%%%%%%%%%%%%%%%%%%%%%%%%%%%%


\newcommand{\abs}[1]{\left\lvert{#1}\right\rvert}
\newcommand{\card}[1]{\lvert{#1}\rvert}
\newcommand{\union}{\cup}
\newcommand{\Union}{\bigcup}
\newcommand{\inter}{\cap}
\newcommand{\Inter}{\bigcap}
%\newcommand{\hint}[1]{\medskip\newline\emph{Hint: #1}}
%\newcommand{\note}[1]{\medskip\newline\emph{Note: #1}}
\newcommand{\points}[1]{[#1 points]}
\newcommand{\totalpoints}[1]{[#1 points total]}
\newcommand{\ds}{\displaystyle}
\newcommand{\ben}{\begin{enumerate}}
\newcommand{\een}{\end{enumerate}}
\newcommand{\bi}{\begin{itemize}}
\newcommand{\ei}{\end{itemize}}
\newcommand{\beq}{\begin{eqnarray*}}
\newcommand{\eeq}{\end{eqnarray*}}
\newcommand{\bieq}{\begin{IEEEeqnarray}{rCl}}
\newcommand{\bieqx}{\begin{IEEEeqnarray}{+rCl+x*}}
\newcommand{\eieq}{\end{IEEEeqnarray}}
\newcommand{\nn}{\nonumber}
%\renewcommand{\i}{\item}
\newcommand{\bpm}{\begin{pmatrix}}
\newcommand{\epm}{\end{pmatrix}}
\newcommand{\sol}{\indent{\bf\emph{Solution:}}}
\newcommand{\ssol}{\indent{\\[2mm]\bf\emph{Solution:}}\;}
\newcommand{\hint}{\indent{\bf\emph{Hint}:}\;}
\newcommand{\note}{\indent{\bf\emph{Note}:}\;}
\newcommand{\vsk}{\vskip 2mm}
%%%%%%%%%%%%%%%%%%%%%%%%% Calculus %%%%%%%%%%%%%%%%%%%%%%%%%%%%
\newcommand{\dd}[2]{\ds\frac{d}{d{#1}}\left[{#2}\right]}
\newcommand{\der}[2]{\ds\frac{d{#1}}{d{#2}}}
\newcommand{\lmt}[3]{\ds\lim_{{#1}\to{#2}}{#3}}
\renewcommand{\iint}[2]{\ds\int{#1}\,d{#2}}
\newcommand{\dint}[4]{\ds\int^{#4}_{#3}{#1}\,d{#2}}
\renewcommand{\Delta}{\triangle}
%%%%%%%%%%%%%%%%%%%%%%%%% Number Sets %%%%%%%%%%%%%%%%%%%%%%%%%%
\newcommand{\N}{\mathbb{N}}
\newcommand{\Z}{\mathbb{Z}}
\newcommand{\Q}{\mathbb{Q}}
\newcommand{\R}{\mathbb{R}}
\newcommand{\C}{\mathbb{C}}
\newcommand{\F}{\mathcal{F}}
\renewcommand{\P}{\mathbb{P}}
\newcommand{\E}{\mathcal{E}}
\renewcommand{\o}{\omega}
\renewcommand{\O}{\Omega}
%%%%%%%%%%%%%%%%%%%%%%%%% Vectors %%%%%%%%%%%%%%%%%%%%%%%%%%%%%
\newcommand{\x}{\bar{x}}
\renewcommand{\v}{\bar{v}}
\newcommand{\y}{\bar{y}}
\newcommand{\z}{\bar{z}}
\newcommand{\w}{\bar{w}}
\renewcommand{\u}{\bar{u}}
\renewcommand{\b}{\bar{b}}
\newcommand{\e}{\bar{e}}
\renewcommand{\a}{\vec{a}}
\renewcommand{\r}{\vec{r}}
\newcommand{\vv}{\vec{v}}
\newcommand{\vecPQ}[2]{\overrightarrow{#1}{#2}}
\newcommand{\vecV}[1]{\overrightarrow{#1}}
\newcommand{\la}{\langle}
\newcommand{\ra}{\rangle}
%%%%%%%%%%%%%%%%%%%%%%%%%%% Vector Spaces %%%%%%%%%%%%%%%%%%%%
\newcommand{\rn}{\ensuremath{\mathbb{R}^n}}
\renewcommand{\rm}{\ensuremath{\mathbb{R}^m}}
\newcommand{\re}{\mathbb{R}}
\newcommand{\Pn}{\mathbb{P}_n}
\newcommand{\B}{\mathcal{B}}
%%%%%%%%%%%%%%%%%%%%%%%%%%% Graphics %%%%%%%%%%%%%%%%%%%%%%%%
\newcommand{\cg}[2]{\begin{center}
\includegraphics[scale={#1}]{{#2}}
\end{center}}
\makeatletter
\def\imod#1{\allowbreak\mkern10mu({\operator@font mod}\,\,#1)}
\makeatother

%%%%%%%%%%%%%%%%%%%%%%%%%%%%%%%%%%%%%%%%%%%%%%%%%%%%%%%%%%%%%%%%%%%%%%%%%%%%%%%%%%%%%%%%%%%%%%
%%%%%%%%%%%%%%%%%%%%%%%%%%%%%% Defined Fonts %%%%%%%%%%%%%%%%%%%%%%%%%%%%%%%%%%%%%%%%%%%%%%%%%
%%%%%%%%%%%%%%%%%%%%%%%%%%%%%%%%%%%%%%%%%%%%%%%%%%%%%%%%%%%%%%%%%%%%%%%%%%%%%%%%%%%%%%%%%%%%%%

\font\minihelv=phvr at 6pt
\font\helv=phvr at 10pt
\font\medhelv=phvr at 16pt
\font\bighelv=phvr at 20pt
\font\hugehelv=phvr at 36pt
\font\mybigfont=phvr at 16pt
\font\mymediumfont=phvr at 14pt
\font\mediumhelv=phvr at 14pt
\font\mybfit=ptmbi at 12pt


%%%%%%%%%%%%%%%%%%%%%%%%%%%%%%%%%%%%%%%%%%%%%%%%%%%%%%%%%%%%%%%%%%%%%%%%%%%%%%%%%%%%%%%%%%%%%%%
%%%%%%%%%%%%%%%%%%%%%%%%%%%%%% Other Commands %%%%%%%%%%%%%%%%%%%%%%%%%%%%%%%%%%%%%%%%%%%%%%%%%
%%%%%%%%%%%%%%%%%%%%%%%%%%%%%%%%%%%%%%%%%%%%%%%%%%%%%%%%%%%%%%%%%%%%%%%%%%%%%%%%%%%%%%%%%%%%%%%
%\setlength\fboxrule{.5pt}
%\newcommand{\latexpicborder}[3]{
%\setlength\fboxsep{30pt}
%\begin{figure}[hb]
%\begin{center}
%\fbox{
%\input{#1}
%}
%\caption{#2}
%\label{#3}
%\end{center}
%\end{figure}
%\setlength\fboxsep{0pt}
%}
%
%\newcommand{\latexpic}[2]{
%\begin{figure}[hb]
%\begin{center}
%\input{#1}
%\vspace*{8mm}
%\caption{#2}
%\end{center}
%\end{figure}
%}

%\begin{minipage}[b]{0.6\linewidth}
%......
%\end{minipage}
%\hspace{0.5cm}
%\begin{minipage}[t]{0.4\linewidth}
%\centering
%\includegraphics[scale=.5]{m1401_ex3_g4.eps}
%\end{minipage}
%\end{figure}


%%%%%%%%%%%%%%%%%%%%%%%%%%%%%%%%%%%%%%%%%%%%%%%%%%%%%%%%%%%%%%%%%%%%%%%%%%%%%%%%%%%%%%%%%%%%%%
%%%%%%%%%%%%%%%%%%%%%%%%%%% IEEEeqnarray Notes %%%%%%%%%%%%%%%%%%%%%%%%%%%%%%%%%%%%%%%%%%%%%%%
%%%%%%%%%%%%%%%%%%%%%%%%%%%%%%%%%%%%%%%%%%%%%%%%%%%%%%%%%%%%%%%%%%%%%%%%%%%%%%%%%%%%%%%%%%%%%%


%Any number of columns can be specified with IEEEeqnarray: {c} will give only one
%column with all entries centered, or {rCll} would add a fourth, left-justified
%column to use for comments. Moreover, beside l, c, r, L, C, R for math mode
%entries there are also s, t, u for left, centered, and right text mode entries.
%Additional space can be added with . and / and ? in increasing order.
%
%
%\begin{proof}
%This is a proof that ends
%with an equation array:
%\begin{IEEEeqnarray*}{+rCl+x*}
%a & = & b + c \\
%& = & d + e. & \qedhere
%\end{IEEEeqnarray*}
%\end{proof}
%Note that the + in {+rCl+x*} denotes stretchable spaces, one on the left
%of the equations (which, if not specified, will be done automatically by
%IEEEeqnarray!) and one on the right of the equations. But now on the right,
%after the stretching column, we add an empty column x. This column will be
%only needed on the last line when we will put the \qedhere command there.
%Finally, we specify a *. This is a null-space that prevents IEEEeqnarray to
%add another unwanted +-space!


% The following environments enable custom numbering of theorems so that the numbers agree % with the numbering in the textbook being used. 
%
%  Usage examples:
%\begin{customthm}{2.2}\label{eight}
%Every theorem must be numbered by hand.
%\end{customthm}
%
%Here is a reference to theorem~\ref{eight}.
%
%\begin{customthm}{2.3}[Parenthetical comment]\label{nine}
%Statement
%\end{customthm}
%
%Here is a reference to theorem~\ref{nine}


\newtheorem{innercustomthm}{Theorem}
\newenvironment{customthm}[1]
  {\renewcommand\theinnercustomthm{#1}\innercustomthm}
  {\endinnercustomthm}
  
  \newtheorem{innercustomprop}{Proposition}
\newenvironment{customprop}[1]
  {\renewcommand\theinnercustomprop{#1}\innercustomprop}
  {\endinnercustomprop}
  
    \newtheorem{innercustomlem}{Lemma}
\newenvironment{customlem}[1]
  {\renewcommand\theinnercustomlem{#1}\innercustomlem}
  {\endinnercustomlem}
  
    \newtheorem{innercustomconj}{Conjecture}
\newenvironment{customconj}[1]
  {\renewcommand\theinnercustomconj{#1}\innercustomconj}
  {\endinnercustomconj}
  
    \newtheorem{innercustomclaim}{Claim}
\newenvironment{customclaim}[1]
  {\renewcommand\theinnercustomclaim{#1}\innercustomclaim}
  {\endinnercustomclaim}
  
    \newtheorem{innercustomcor}{Corollary}
\newenvironment{customcor}[1]
  {\renewcommand\theinnercustomcor{#1}\innercustomcor}
  {\endinnercustomcor}
  
    \newtheorem{innercustomdef}{Definition}
\newenvironment{customdef}[1]
  {\renewcommand\theinnercustomdef{#1}\innercustomdef}
  {\endinnercustomdef}
  
    \newtheorem{innercustomex}{Example}
\newenvironment{customex}[1]
  {\renewcommand\theinnercustomex{#1}\innercustomex}
  {\endinnercustomex}
  
    \newtheorem{innercustomass}{Assumption}
\newenvironment{customass}[1]
  {\renewcommand\theinnercustomass{#1}\innercustomass}
  {\endinnercustomass}
  
      \newtheorem{innercustomax}{Axiom}
\newenvironment{customax}[1]
  {\renewcommand\theinnercustomax{#1}\innercustomax}
  {\endinnercustomax}
  

\vfuzz2pt % Don't report over-full v-boxes if over-edge is small
\hfuzz2pt % Don't report over-full h-boxes if over-edge is small

\renewcommand{\ni}{\noindent}


%%%%%%%%%%%%%%%%%%%%%%%%%%%%%%%%%%%%%%%%%%%%%%%%%%%%%%
%%%%%%%%%%%%%%%%%%%%%%%%%%%%%%%%%%%%%%%%%%%%%%%%%%%%%%

\pagestyle{myheadings}

%%%%%%%%%%%%%%%%%%%%%%%%%%%%%%%%%%%%%%%%%%%%%%%%%%%%%%

%%%%%%%%%%%%%%%%%%%%%%%%%%%%%%%%%%%%%%%%%%%%%%%%%%%%%%
%%%%%%%%%%%%%%%%%%%%%%%%%   Document Body   %%%%%%%%%%
%%%%%%%%%%%%%%%%%%%%%%%%%%%%%%%%%%%%%%%%%%%%%%%%%%%%%%

%\def\classnum{3810}
%\def\classtitle{Probability}
%\def\classtitleshort{Probability}
%\def\classsec{001}
%\def\classterm{Fall 2025}
%\def\instructor{Robert Rostermundt}
\def\printsol{0}


	\title{\vspace{-1in}Math\classnum\;-\;\classtitle\\
	Section\;\classsec\;-\;\classterm\\
	Notes: More Expectation Through Conditioning}
	\author{University of Colorado Denver / College of Liberal Arts 	and Sciences}
	\date{Department of Mathematics - \instructor}

	\markright{Math\classnum\;-\;\classtitleshort, UCD, \classterm, \instructor}



%%%%%%%%%%%%%%%%%%%%%%%%%%%%%%%%%%%%%%%%%%%%%%%%%%%%%%
\begin{document}\maketitle\thispagestyle{empty}
%%%%%%%%%%%%%%%%%%%%%%%%%%%%%%%%%%%%%%%%%%%%%%%%%%%%%%



%%%%%%%%%%%%%%%%%%%%%%%%%%%%%%%%%%%%%%%%%%%%%%%%%%%%%%%%%%%%%%%%%%%%%%%%%%%%%%%%%%%%%%%%%%%%%%%%%%%%%%
\vspace*{2mm}
\hrule
\vskip 8mm

%%%%%%%%%%%%%%%%%%%%%%%%%%%%%%%%%%%%%%%%%%%%%%%%%%%%%%%%%%%%%%%%%%%%%%%%%
%%%%%%%%%%%%%%%%%%%%%%%%%%%%%%%%%%%%%%%%%%%%%%%%%%%%%%%%%%%%%%%%%%%%%%%%%

%%%%%%%%%%%%%%%%%%%%%%%%%%%%%%%%%%%%%%%%%%%%%%%%%%%%%%%%%%%%%%%%%%%%%%%%%%%%%%%%%%%%%%%%%%%%%%%%%%%%%%
\section*{The Problem:}
%%%%%%%%%%%%%%%%%%%%%%%%%%%%%%%%%%%%%%%%%%%%%%%%%%%%%%%%%%%%%%%%%%%%%%%%%%%%%%%%%%%%%%%%%%%%%%%%%%%%%%


\noindent A professor has accidentally scheduled two different appointments at the same time. One student show up on time and the other student shows up 5 minutes late. Assume that both appointment times are independent and each is modeled by an exponential distribution with average time 30 minutes. If $T$ is the elapsed time from the first student arrival until the end of the second appointment, find the expectation $E[T]$.
\vskip 2mm
Let $X_1$ be the time of the first appointment and $X_2$ be the time of the second appointment. Then we have $X_1\sim exp\big(\lambda=1/30\big)$ and $X_2\sim exp\big(\lambda=1/30\big)$ and $X_1\perp X_2$. We want to find
\[E[T]=\ds\int^{\infty}_{-\infty}t\cdot f_{_T}(t)\,dt.\]
However, at this point we do not know the distribution of $T$ and do not know it's density function $f_{_T}$. Before we make any theoretical calculations lets simulate the process and see if we can gain any insight.

%%%%%%%%%%%%%%%%%%%%%%%%%%%%%%%%%%%%%%%%%%%%%%%%%%%%%%%%%%%%%%%%%%%%%%%%%%%%%%%%%%%%%%%%%%%%%%%%%%%%%%
\section*{Simulations:}
%%%%%%%%%%%%%%%%%%%%%%%%%%%%%%%%%%%%%%%%%%%%%%%%%%%%%%%%%%%%%%%%%%%%%%%%%%%%%%%%%%%%%%%%%%%%%%%%%%%%%%
I have simulated this process in R. Below are graphics from four different simulations with number trials $n=50, 100, 1000, 100000$. The R code is provided the end of the document.
\vskip 1cm

\includegraphics[scale=0.48]{exp_waiting_simulation_50.jpeg}\quad
\includegraphics[scale=0.48]{exp_waiting_simulation_100.jpeg}\\
\vskip 5mm
\includegraphics[scale=0.48]{exp_waiting_simulation_1000.jpeg}\quad
\includegraphics[scale=0.48]{exp_waiting_simulation_100000.jpeg}\\
\vskip 5mm
\vskip 5mm
Each simulation is labeled with the true theoretical mean, but without this theoretical information these simulations might give us a sense for the desired expected value $E[T]$. 
\vskip 5mm
Next are a couple of comparisons of our simulations of the density function $f_{_T}$ with the density function of the sum $X_1+X_2$. Here we can see how the late arrival is affecting the expected elapsed time.
\vskip 8mm
\includegraphics[scale=0.48]{exp_waiting_compare_gr2.jpeg}\quad
\includegraphics[scale=0.48]{exp_waiting_compare_gr3.jpeg}\\

\vfill\eject


%%%%%%%%%%%%%%%%%%%%%%%%%%%%%%%%%%%%%%%%%%%%%%%%%%%%%%%%%%%%%%%%%%%%%%%%%%%%%%%%%%%%%%%%%%%%%%%%%%%%%%
\section*{Theoretical Tools:}
%%%%%%%%%%%%%%%%%%%%%%%%%%%%%%%%%%%%%%%%%%%%%%%%%%%%%%%%%%%%%%%%%%%%%%%%%%%%%%%%%%%%%%%%%%%%%%%%%%%%%%

\noindent We will compute this expectation in two ways by conditioning on the following events:

	\begin{itemize}
		\item[] $A:=\big\{0<X_1<5\big\}$ which is the event that the first appointment lasts less than 5 minutes.
		\item[] $B:=\big\{5\le X_1\big\}$ which is the event that the first appointment lasts at least 5 minutes.
	\end{itemize} 

Then since $A$ and $B$ partition the sample space we can compute the expected value as
\[E[T]=\P(A)\cdot E\big[\,T\,|\,A\,\big]+\P(B)\cdot E\big[\,T\,|\,B\,\big].\]


%%%%%%%%%%%%%%%%%%%%%%%%%%%%%%%%%%%%%%%%%%%%%%%%%%%%%%%%%%%%%%%%%%%%%%%%%%%%%%%%%%%%%%%%%%%%%%%%%%%%%%
\section*{Calculations:}
%%%%%%%%%%%%%%%%%%%%%%%%%%%%%%%%%%%%%%%%%%%%%%%%%%%%%%%%%%%%%%%%%%%%%%%%%%%%%%%%%%%%%%%%%%%%%%%%%%%%%%

\noindent We first make the easy calculations 
\beq
\P(A)&=&\P(0<X_1\le 5)=F_{_{X_1}}(5)=1-e^{-5/30}\\
\P(B)&=&1-\P(A)=e^{-5/30}\\
\eeq
We need to find $E\big[\,T\,|\,A\,\big]$ and $E\big[\,T\,|\,B\,\big]$. So 
\[E\big[\,T\,|\,A\,\big]=E\big[\,5+X_2\,\big|\,0<X_1<5\,\big]=5+E\big[\,X_2\,\big|\,0<X_1<5\,\big]=5+E[X_2]=35.\]
Similarly,
\[E\big[\,T\,|\,B\,\big]=E\big[\,X_1+X_2\,\big|\,5\le X_1\,\big]=E\big[\,X_1\,\big|\,5\ge X_1\,\big]+E\big[\,X_2\,\big|\,5\ge X_1\,\big]=5+E[X_1]+E[X_2]=65.\]
To see this, recall that the exponential random variable $X_1$ is memoryless. The following plot should help to see that the conditional distribution $X_1\,|\,X_1>5$ is identical to the distribution of $X_1$ but with all mass shifted to the right 5 units.
\begin{center}
\includegraphics[scale=0.3]{conditional_density_plot.jpg}
\end{center}
\vskip 8mm
Putting the pieces together we have
\[E[T]=\underbrace{\P(A)}_{\text{\textcolor{red}{$1-e^{-5/30}$}}}\cdot\underbrace{E\big[\,T\,|\,A\,\big]}_{\text{\textcolor{red}{35}}}+\underbrace{\P(B)}_{\text{\textcolor{red}{$e^{-5/30}$}}}\cdot\underbrace{E\big[\,W\,|\,B\,\big]}_{\text{\textcolor{red}{65}}}\]
So the expected waiting time is 60.394 minutes.
\vskip 1cm


\paragraph{The R Code:} Here is the R code used for the above simulations.
\vspace*{2mm}
\small 
\begin{lstlisting}[language=R]

# Simulated density and expectation of total elapsed time T

set.seed(123)
# parameters
n <- 100000          # number of simulated trials
lambda <- 1/30       # Exp rate, mean = 30 minutes



# simulate appointment durations
x1 <- rexp(n, rate = lambda)
x2 <- rexp(n, rate = lambda)

# second student arrives 5 minutes late
lateness <- 5

# compute relevant times for each trial
start1 <- rep(0, n)
end1   <- x1
arrive2 <- rep(lateness, n)
start2  <- pmax(end1, arrive2)
end2    <- start2 + x2
T <- end2   # total time from first arrival to second departure

# empirical mean
mean_T <- mean(T)

# print summary
cat("Estimated E[T] =", mean_T, "minutes\n")

# per-trial comparison (first 10)
results <- data.frame(
  trial = 1:10,
  X1 = round(x1[1:10], 2),
  X2 = round(x2[1:10], 2),
  end1 = round(end1[1:10], 2),
  start2 = round(start2[1:10], 2),
  end2 = round(end2[1:10], 2),
  Total = round(T[1:10], 2)
)
print(results)



# summary statistics
cat("\nMean(X1) =", mean(x1), "  Mean(X2) =", mean(x2), "\n")
cat("Mean(max(X1,5)) =", mean(pmax(x1,5)), "\n")
cat("Mean(Total)     =", mean_T, "\n")

# plot with n and empirical mean in legend
hist(T, breaks = 100, freq = FALSE,  col = "lightblue",
     main = paste("Simulated Distribution of Total Time (n =", n, ")"),
     xlab = "Minutes from first arrival to second departure")

abline(v = mean_T, col = "red", lwd = 2)
abline(v = 60.394, col = "blue", lwd = 2, lty = 2)

legend("topright",
       legend = c(
         paste0("Empirical mean = ", round(mean_T, 3)),
         "Theoretical mean = 60.394"
       ),
       col = c("red", "blue"),
       lwd = 2,
       lty = c(1, 2))

#########################################################################
#########################################################################
# Same simulation except comparison to Density of X1+X2 without Late Arrival


set.seed(123)

# parameters
n <- 200000
lambda <- 1/30     # Exp(mean=30)
lateness <- 5      # second student 5 minutes late

# simulate appointment durations
x1 <- rexp(n, rate = lambda)
x2 <- rexp(n, rate = lambda)

# total time from first arrival (0) to second departure
T <- pmax(x1, lateness) + x2

# compute means
emp_mean <- mean(T)
theory_mean <- 2 / lambda   # mean of X1 + X2

# histogram of simulated totals
hist(T, breaks = 100, freq = FALSE, col = "lightblue",
     main = paste("Total Time with Lateness =", lateness, "min  (n =", n, ")"),
     xlab = "Total time (minutes)", border = "white")

# overlay theoretical Gamma(2, rate=lambda) density
tseq <- seq(0, max(T), length.out = 1000)
lines(tseq, dgamma(tseq, shape = 2, rate = lambda), col = "blue", lwd = 2)

# overlay empirical density curve
lines(density(T), col = "red", lwd = 2, lty = 2)

# add vertical lines for means
abline(v = emp_mean, col = "red", lwd = 2, lty = 2)
abline(v = theory_mean, col = "blue", lwd = 2, lty = 3)

# legend
legend("topright",
       legend = c(
         paste0("Empirical density (mean = ", round(emp_mean, 3), ")"),
         paste0("Theoretical X1+X2 mean = ", theory_mean)
       ),
       col = c("red", "blue"),
       lwd = 2,
       lty = c(2, 3),
       bg = "white")
\end{lstlisting}



\vskip 1cm
\hrule
\vskip 5mm
\begin{center}
\bf Please let me know if you have any questions, comments, or corrections!
\end{center}

%%%%%%%%%%%%%%%%%%%%%%%%%%%%%%%%%%%%%%%%%%%%%%%%%%%%%%
\end{document}
%%%%%%%%%%%%%%%%%%%%%%%%%%%%%%%%%%%%%%%%%%%%%%%%%%%%%%
