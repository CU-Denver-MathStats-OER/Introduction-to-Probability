%%%%%%%%%%%%%%%%%%%%%%%%%%%%%%%%%%%%%%%%%%%%%%%%%%%%%%
\documentclass[11pt]{article}
%%%%%%%%%%%%%%%%%%%%%%%%%%%%%%%%%%%%%%%%%%%%%%%%%%%%%%

\usepackage{amsmath}
\usepackage{amsthm}
\usepackage{amssymb}
\usepackage{latexsym}
\usepackage{graphicx}
\usepackage{color}
\usepackage{verbatim}
\usepackage{float}
\usepackage{multicol}
\usepackage{xcolor}
\usepackage{listings}
\usepackage{tikz}
\usetikzlibrary{arrows.meta, positioning, calc}
\usetikzlibrary{decorations.pathmorphing}
\usepackage{tcolorbox}
\tcbuselibrary{breakable}

\newtcolorbox{solutionbox}{
  breakable,
  colback=blue!5!white,
  colframe=blue!50!black,
  title=Solution,
  sharp corners,
  boxrule=0.8pt
}

\newtcolorbox{hintbox}{
  breakable,
  colback=gray!10!white,
  colframe=gray!50!black,
  title=Hint,
  sharp corners,
  boxrule=0.5pt
}

% Unnumbered theorem
\newtheorem*{thm*}{Theorem}


\lstdefinelanguage{R}{
      keywords={if,else,while,for,in,next,break,function,TRUE,FALSE,NULL,Inf,NA,NaN,switch,repeat,return,require,library},
      keywordstyle=\color{blue}\bfseries,
      identifierstyle=\color{black},
      comment=[l]{\#},
      commentstyle=\color{gray}\ttfamily,
      string=[b]{"},
      stringstyle=\color{red}\ttfamily,
      morecomment=[l]{//},
      morestring=[b]{'},
      sensitive=true,
      morekeywords={print,summary,plot,lm,glm,data,frame,read.csv,write.csv,factor,levels,names,colnames,rownames,
      head,tail,str,dim,length,class,typeof,mode,is.na,is.null,is.finite,is.infinite,is.nan,as.numeric,as.character,
      as.factor,as.Date,as.POSIXct,as.matrix,as.data.frame,rbind,cbind,merge,subset,aggregate,tapply,apply,lapply,sapply,
      mapply,vapply,replicate,seq,rep,c,list,matrix,array,data.frame,table,hist,boxplot,barplot,pie,curve,lines,points,text,
      abline,legend,par,mtext,title,xlab,ylab,xlim,ylim,main,sub,col,pch,cex,lty,lwd,type,bg,fg,args,options,warnings,errors,
      message,stop,warning,error,try,tryCatch,withCallingHandlers,on.exit,debug,browser,trace,recover,options,getOption,setOption},
    }


\setlength{\textheight}{9in}
\setlength{\textwidth}{6in}
\addtolength{\topmargin}{-2cm}
\addtolength{\oddsidemargin}{-1cm}
\parindent=0in


\def\classnum{3810}
\def\classtitle{Probability}
\def\classtitleshort{Probability}
\def\classsec{001}
\def\classterm{Fall 2025}
\def\instructor{Robert Rostermundt}
%\def\hmwknum{\#2}


%%%%%%%%%%%%%%%%%%%%%%%%%%%%%%%%%%%%%%%%%%%%%%%%%%%%%%%%%
%%%%%%%%%%%%%%%%%%%%%%%%%  Colors  %%%%%%%%%%%%%%%%%%%%%%
%%%%%%%%%%%%%%%%%%%%%%%%%%%%%%%%%%%%%%%%%%%%%%%%%%%%%%%%%

\definecolor{Green}{rgb}{0,.5,0}
%use for definitions
\definecolor{Red}{rgb}{.8,.2,0}
%use for emphasis
\definecolor{Yellow}{rgb}{.6,.6,.1}
%use for part titles
\definecolor{Cyan}{rgb}{.2,.6,.7}
%use for comments
\definecolor{Purple}{rgb}{.4,0,1}
%use for examples
\definecolor{deepred}{rgb}{.53,.29,.24}
%use for important points
\definecolor{Black}{rgb}{0,0,0}
%use for washout
\definecolor{Grey}{rgb}{.45,.45,.45}
% use for theorems
\newcommand{\tred}[1]{\textcolor{Red}{#1}}
\newcommand{\tgreen}[1]{\textcolor{Green}{#1}}
\newcommand{\tcyan}[1]{\textcolor{Cyan}{#1}}
\newcommand{\tyellow}[1]{\textcolor{Yellow}{#1}}
\newcommand{\tpurple}[1]{\textcolor{Purple}{#1}}
\newcommand{\tblack}[1]{\textcolor{Black}{#1}}
\newcommand{\tgrey}[1]{\textcolor{Grey}{#1}}
\newcommand{\tdeepred}[1]{\textcolor{deepred}{#1}}
\newcommand{\ttt}[1]{\texttt{#1}}

%%%%%%%%%%%%%%%%%%%%%%%%%%%%%%%%%%%%%%%%%%%%%%%%%%%%%%%%%
%%%%%%%%%%%%%%%%%%%%%%%%%  Theorem Environments  %%%%%%%%
%%%%%%%%%%%%%%%%%%%%%%%%%%%%%%%%%%%%%%%%%%%%%%%%%%%%%%%%%

\theoremstyle{plain}
\newtheorem{thm}{Theorem}
\newtheorem{axiom}{Axiom}
\newtheorem{cor}{Corollary}
\newtheorem{lemma}{Lemma}
\newtheorem{prop}{Proposition}
\newtheorem{ques}{Question}
\theoremstyle{definition}
\newtheorem{defn}{Definition}
\theoremstyle{remark}
\newtheorem{remark}{Remark}
\theoremstyle{definition}
\newtheorem{ex}{Example}
\numberwithin{equation}{section}
\newtheorem{prob}{Problem}
\numberwithin{equation}{section}


%%%%%%%%%%%%%%%%%%%%%%%%%%%%%%%%%%%%%%%%%%%%%%%%%%%%%%%%%
%%%%%%%%%%%%%%%%%%%%%%%%%  Math    %%%%%%%%%%%%%%%%%%%%%%
%%%%%%%%%%%%%%%%%%%%%%%%%%%%%%%%%%%%%%%%%%%%%%%%%%%%%%%%%


\newcommand{\abs}[1]{\left\lvert{#1}\right\rvert}
\newcommand{\card}[1]{\lvert{#1}\rvert}
\newcommand{\union}{\cup}
\newcommand{\Union}{\bigcup}
\newcommand{\inter}{\cap}
\newcommand{\Inter}{\bigcap}
%\newcommand{\hint}[1]{\medskip\newline\emph{Hint: #1}}
%\newcommand{\note}[1]{\medskip\newline\emph{Note: #1}}
\newcommand{\points}[1]{[#1 points]}
\newcommand{\totalpoints}[1]{[#1 points total]}
\newcommand{\ds}{\displaystyle}
\newcommand{\ben}{\begin{enumerate}}
\newcommand{\een}{\end{enumerate}}
\newcommand{\bi}{\begin{itemize}}
\newcommand{\ei}{\end{itemize}}
\newcommand{\beq}{\begin{eqnarray*}}
\newcommand{\eeq}{\end{eqnarray*}}
\newcommand{\bieq}{\begin{IEEEeqnarray}{rCl}}
\newcommand{\bieqx}{\begin{IEEEeqnarray}{+rCl+x*}}
\newcommand{\eieq}{\end{IEEEeqnarray}}
\newcommand{\nn}{\nonumber}
%\renewcommand{\i}{\item}
\newcommand{\bpm}{\begin{pmatrix}}
\newcommand{\epm}{\end{pmatrix}}
\newcommand{\sol}{\indent{\bf\emph{Solution:}}}
\newcommand{\ssol}{\indent{\\[2mm]\bf\emph{Solution:}}\;}
\newcommand{\hint}{\indent{\bf\emph{Hint}:}\;}
\newcommand{\note}{\indent{\bf\emph{Note}:}\;}
\newcommand{\vsk}{\vskip 2mm}
%%%%%%%%%%%%%%%%%%%%%%%%% Calculus %%%%%%%%%%%%%%%%%%%%%%%%%%%%
\newcommand{\dd}[2]{\ds\frac{d}{d{#1}}\left[{#2}\right]}
\newcommand{\der}[2]{\ds\frac{d{#1}}{d{#2}}}
\newcommand{\lmt}[3]{\ds\lim_{{#1}\to{#2}}{#3}}
\renewcommand{\iint}[2]{\ds\int{#1}\,d{#2}}
\newcommand{\dint}[4]{\ds\int^{#4}_{#3}{#1}\,d{#2}}
\renewcommand{\Delta}{\triangle}
%%%%%%%%%%%%%%%%%%%%%%%%% Number Sets %%%%%%%%%%%%%%%%%%%%%%%%%%
\newcommand{\N}{\mathbb{N}}
\newcommand{\Z}{\mathbb{Z}}
\newcommand{\Q}{\mathbb{Q}}
\newcommand{\R}{\mathbb{R}}
\newcommand{\C}{\mathbb{C}}
\newcommand{\F}{\mathcal{F}}
\renewcommand{\P}{\mathbb{P}}
\newcommand{\E}{\mathcal{E}}
\renewcommand{\o}{\omega}
\renewcommand{\O}{\Omega}
%%%%%%%%%%%%%%%%%%%%%%%%% Vectors %%%%%%%%%%%%%%%%%%%%%%%%%%%%%
\newcommand{\x}{\bar{x}}
\renewcommand{\v}{\bar{v}}
\newcommand{\y}{\bar{y}}
\newcommand{\z}{\bar{z}}
\newcommand{\w}{\bar{w}}
\renewcommand{\u}{\bar{u}}
\renewcommand{\b}{\bar{b}}
\newcommand{\e}{\bar{e}}
\renewcommand{\a}{\vec{a}}
\renewcommand{\r}{\vec{r}}
\newcommand{\vv}{\vec{v}}
\newcommand{\vecPQ}[2]{\overrightarrow{#1}{#2}}
\newcommand{\vecV}[1]{\overrightarrow{#1}}
\newcommand{\la}{\langle}
\newcommand{\ra}{\rangle}
%%%%%%%%%%%%%%%%%%%%%%%%%%% Vector Spaces %%%%%%%%%%%%%%%%%%%%
\newcommand{\rn}{\ensuremath{\mathbb{R}^n}}
\renewcommand{\rm}{\ensuremath{\mathbb{R}^m}}
\newcommand{\re}{\mathbb{R}}
\newcommand{\Pn}{\mathbb{P}_n}
\newcommand{\B}{\mathcal{B}}
%%%%%%%%%%%%%%%%%%%%%%%%%%% Graphics %%%%%%%%%%%%%%%%%%%%%%%%
\newcommand{\cg}[2]{\begin{center}
\includegraphics[scale={#1}]{{#2}}
\end{center}}
\makeatletter
\def\imod#1{\allowbreak\mkern10mu({\operator@font mod}\,\,#1)}
\makeatother

%%%%%%%%%%%%%%%%%%%%%%%%%%%%%%%%%%%%%%%%%%%%%%%%%%%%%%%%%%%%%%%%%%%%%%%%%%%%%%%%%%%%%%%%%%%%%%
%%%%%%%%%%%%%%%%%%%%%%%%%%%%%% Defined Fonts %%%%%%%%%%%%%%%%%%%%%%%%%%%%%%%%%%%%%%%%%%%%%%%%%
%%%%%%%%%%%%%%%%%%%%%%%%%%%%%%%%%%%%%%%%%%%%%%%%%%%%%%%%%%%%%%%%%%%%%%%%%%%%%%%%%%%%%%%%%%%%%%

\font\minihelv=phvr at 6pt
\font\helv=phvr at 10pt
\font\medhelv=phvr at 16pt
\font\bighelv=phvr at 20pt
\font\hugehelv=phvr at 36pt
\font\mybigfont=phvr at 16pt
\font\mymediumfont=phvr at 14pt
\font\mediumhelv=phvr at 14pt
\font\mybfit=ptmbi at 12pt


%%%%%%%%%%%%%%%%%%%%%%%%%%%%%%%%%%%%%%%%%%%%%%%%%%%%%%%%%%%%%%%%%%%%%%%%%%%%%%%%%%%%%%%%%%%%%%%
%%%%%%%%%%%%%%%%%%%%%%%%%%%%%% Other Commands %%%%%%%%%%%%%%%%%%%%%%%%%%%%%%%%%%%%%%%%%%%%%%%%%
%%%%%%%%%%%%%%%%%%%%%%%%%%%%%%%%%%%%%%%%%%%%%%%%%%%%%%%%%%%%%%%%%%%%%%%%%%%%%%%%%%%%%%%%%%%%%%%
%\setlength\fboxrule{.5pt}
%\newcommand{\latexpicborder}[3]{
%\setlength\fboxsep{30pt}
%\begin{figure}[hb]
%\begin{center}
%\fbox{
%\input{#1}
%}
%\caption{#2}
%\label{#3}
%\end{center}
%\end{figure}
%\setlength\fboxsep{0pt}
%}
%
%\newcommand{\latexpic}[2]{
%\begin{figure}[hb]
%\begin{center}
%\input{#1}
%\vspace*{8mm}
%\caption{#2}
%\end{center}
%\end{figure}
%}

%\begin{minipage}[b]{0.6\linewidth}
%......
%\end{minipage}
%\hspace{0.5cm}
%\begin{minipage}[t]{0.4\linewidth}
%\centering
%\includegraphics[scale=.5]{m1401_ex3_g4.eps}
%\end{minipage}
%\end{figure}


%%%%%%%%%%%%%%%%%%%%%%%%%%%%%%%%%%%%%%%%%%%%%%%%%%%%%%%%%%%%%%%%%%%%%%%%%%%%%%%%%%%%%%%%%%%%%%
%%%%%%%%%%%%%%%%%%%%%%%%%%% IEEEeqnarray Notes %%%%%%%%%%%%%%%%%%%%%%%%%%%%%%%%%%%%%%%%%%%%%%%
%%%%%%%%%%%%%%%%%%%%%%%%%%%%%%%%%%%%%%%%%%%%%%%%%%%%%%%%%%%%%%%%%%%%%%%%%%%%%%%%%%%%%%%%%%%%%%


%Any number of columns can be specified with IEEEeqnarray: {c} will give only one
%column with all entries centered, or {rCll} would add a fourth, left-justified
%column to use for comments. Moreover, beside l, c, r, L, C, R for math mode
%entries there are also s, t, u for left, centered, and right text mode entries.
%Additional space can be added with . and / and ? in increasing order.
%
%
%\begin{proof}
%This is a proof that ends
%with an equation array:
%\begin{IEEEeqnarray*}{+rCl+x*}
%a & = & b + c \\
%& = & d + e. & \qedhere
%\end{IEEEeqnarray*}
%\end{proof}
%Note that the + in {+rCl+x*} denotes stretchable spaces, one on the left
%of the equations (which, if not specified, will be done automatically by
%IEEEeqnarray!) and one on the right of the equations. But now on the right,
%after the stretching column, we add an empty column x. This column will be
%only needed on the last line when we will put the \qedhere command there.
%Finally, we specify a *. This is a null-space that prevents IEEEeqnarray to
%add another unwanted +-space!


% The following environments enable custom numbering of theorems so that the numbers agree % with the numbering in the textbook being used. 
%
%  Usage examples:
%\begin{customthm}{2.2}\label{eight}
%Every theorem must be numbered by hand.
%\end{customthm}
%
%Here is a reference to theorem~\ref{eight}.
%
%\begin{customthm}{2.3}[Parenthetical comment]\label{nine}
%Statement
%\end{customthm}
%
%Here is a reference to theorem~\ref{nine}


\newtheorem{innercustomthm}{Theorem}
\newenvironment{customthm}[1]
  {\renewcommand\theinnercustomthm{#1}\innercustomthm}
  {\endinnercustomthm}
  
  \newtheorem{innercustomprop}{Proposition}
\newenvironment{customprop}[1]
  {\renewcommand\theinnercustomprop{#1}\innercustomprop}
  {\endinnercustomprop}
  
    \newtheorem{innercustomlem}{Lemma}
\newenvironment{customlem}[1]
  {\renewcommand\theinnercustomlem{#1}\innercustomlem}
  {\endinnercustomlem}
  
    \newtheorem{innercustomconj}{Conjecture}
\newenvironment{customconj}[1]
  {\renewcommand\theinnercustomconj{#1}\innercustomconj}
  {\endinnercustomconj}
  
    \newtheorem{innercustomclaim}{Claim}
\newenvironment{customclaim}[1]
  {\renewcommand\theinnercustomclaim{#1}\innercustomclaim}
  {\endinnercustomclaim}
  
    \newtheorem{innercustomcor}{Corollary}
\newenvironment{customcor}[1]
  {\renewcommand\theinnercustomcor{#1}\innercustomcor}
  {\endinnercustomcor}
  
    \newtheorem{innercustomdef}{Definition}
\newenvironment{customdef}[1]
  {\renewcommand\theinnercustomdef{#1}\innercustomdef}
  {\endinnercustomdef}
  
    \newtheorem{innercustomex}{Example}
\newenvironment{customex}[1]
  {\renewcommand\theinnercustomex{#1}\innercustomex}
  {\endinnercustomex}
  
    \newtheorem{innercustomass}{Assumption}
\newenvironment{customass}[1]
  {\renewcommand\theinnercustomass{#1}\innercustomass}
  {\endinnercustomass}
  
      \newtheorem{innercustomax}{Axiom}
\newenvironment{customax}[1]
  {\renewcommand\theinnercustomax{#1}\innercustomax}
  {\endinnercustomax}
  

\vfuzz2pt % Don't report over-full v-boxes if over-edge is small
\hfuzz2pt % Don't report over-full h-boxes if over-edge is small

\renewcommand{\ni}{\noindent}


%%%%%%%%%%%%%%%%%%%%%%%%%%%%%%%%%%%%%%%%%%%%%%%%%%%%%%
%%%%%%%%%%%%%%%%%%%%%%%%%%%%%%%%%%%%%%%%%%%%%%%%%%%%%%

\pagestyle{myheadings}

%%%%%%%%%%%%%%%%%%%%%%%%%%%%%%%%%%%%%%%%%%%%%%%%%%%%%%

%%%%%%%%%%%%%%%%%%%%%%%%%%%%%%%%%%%%%%%%%%%%%%%%%%%%%%
%%%%%%%%%%%%%%%%%%%%%%%%%   Document Body   %%%%%%%%%%
%%%%%%%%%%%%%%%%%%%%%%%%%%%%%%%%%%%%%%%%%%%%%%%%%%%%%%

%\def\classnum{3810}
%\def\classtitle{Probability}
%\def\classtitleshort{Probability}
%\def\classsec{001}
%\def\classterm{Fall 2025}
%\def\instructor{Robert Rostermundt}
\def\printsol{0}


	\title{\vspace{-1in}Math\classnum\;-\;\classtitle\\
	Section\;\classsec\;-\;\classterm\\
	Notes: Exponential Random Variables}
	\author{University of Colorado Denver / College of Liberal Arts 	and Sciences}
	\date{Department of Mathematics - \instructor}

	\markright{Math\classnum\;-\;\classtitleshort, UCD, \classterm, \instructor}



%%%%%%%%%%%%%%%%%%%%%%%%%%%%%%%%%%%%%%%%%%%%%%%%%%%%%%
\begin{document}\maketitle\thispagestyle{empty}
%%%%%%%%%%%%%%%%%%%%%%%%%%%%%%%%%%%%%%%%%%%%%%%%%%%%%%



%%%%%%%%%%%%%%%%%%%%%%%%%%%%%%%%%%%%%%%%%%%%%%%%%%%%%%%%%%%%%%%%%%%%%%%%%%%%%%%%%%%%%%%%%%%%%%%%%%%%%%
\vspace*{2mm}
\hrule
\vskip 8mm

%%%%%%%%%%%%%%%%%%%%%%%%%%%%%%%%%%%%%%%%%%%%%%%%%%%%%%%%%%%%%%%%%%%%%%%%%
%%%%%%%%%%%%%%%%%%%%%%%%%%%%%%%%%%%%%%%%%%%%%%%%%%%%%%%%%%%%%%%%%%%%%%%%%

%%%%%%%%%%%%%%%%%%%%%%%%%%%%%%%%%%%%%%%%%%%%%%%%%%%%%%%%%%%%%%%%%%%%%%%%%%%%%%%%%%%%%%%%%%%%%%%%%%%%%%
\section*{The Problem:}
%%%%%%%%%%%%%%%%%%%%%%%%%%%%%%%%%%%%%%%%%%%%%%%%%%%%%%%%%%%%%%%%%%%%%%%%%%%%%%%%%%%%%%%%%%%%%%%%%%%%%%

\noindent
Suppose events occur randomly over continuous time, such as customers arriving at a store,
radioactive particles decaying, or phone calls coming into a call center.
We are interested in the \emph{waiting time} until the first ``arrival" occurs.
\vskip 5mm
Let $X$ be the amount of time until the first arrival. We will assume events occur at a constant 
average rate $\lambda>0$ per unit time. Then $X$ is an \emph{exponential random variable} with 
parameter $\lambda$, denoted
\[X \sim \mathrm{Exp}(\lambda).\]
We have two goals:
	\begin{itemize}
		\item Find the density function $f_{_X}$ and the distribution function $F_{_X}$.
		\item Determine $\mathbb{E}[X]$ and $\mathrm{Var}(X)$.
	\end{itemize}

%The probability density function (pdf) of $X$ is
%\[
%f_X(x) = \lambda e^{-\lambda x}, \qquad x \ge 0.
%\]
%
%Our goal is to compute the expected waiting time
%\[
%\mathbb{E}[X] = \int_0^\infty x f_X(x)\,dx
%= \int_0^\infty x \lambda e^{-\lambda x}\,dx,
%\]
%and to understand the variance of $X$.

%%%%%%%%%%%%%%%%%%%%%%%%%%%%%%%%%%%%%%%%%%%%%%%%%%%%%%%%%%%%%%%%%%%%%%%%%%%%%%%%%%%%%%%%%%%%%%%%%%%%%%
\section*{Theoretical Tools:}
%%%%%%%%%%%%%%%%%%%%%%%%%%%%%%%%%%%%%%%%%%%%%%%%%%%%%%%%%%%%%%%%%%%%%%%%%%%%%%%%%%%%%%%%%%%%%%%%%%%%%%

\begin{itemize}

	\item \textbf{Poisson Process Assumptions:}  
A Poisson process with rate $\lambda>0$ satisfies:
		\begin{itemize}
			\item The number of arrivals in an interval of length $t$ has a
$\mathrm{Poisson}(\lambda t)$ distribution.
			\item Numbers of arrivals in disjoint time intervals are independent.
			\item Over a very short interval of length $\delta$,
\[\P(\text{one arrival}) = \lambda \delta + o(\delta),\qquad\P(\text{two or more arrivals}) = o(\delta).\]
		\end{itemize}

	\item \textbf{From Interval Probabilities to Densities:}  
If a continuous random variable $X$ satisfies
\[\P(x \le X \le x+\delta) = f(x)\,\delta + o(\delta),\]
then $f(x)$ is the probability density function of $X$.

	\item \textbf{Memoryless Property (Exponential Distribution):}  
If $X \sim \mathrm{Exp}(\lambda)$, then for all $s,t \ge 0$,
\[\P(X > s+t \mid X > t) = \P(X > s).\]

	\item \textbf{Law of Total Expectation:}  
If events $A$ and $A^c$ partition the sample space, then
\[\mathbb{E}[X]=\P(A)\,\mathbb{E}[X \mid A]+\P(A^c)\,\mathbb{E}[X \mid A^c].\]

	\item \textbf{Integration by Parts:}  
For differentiable functions $u=u(x)$ and $v=v(x)$,
\[\int u\,dv = uv - \int v\,du.\]

	\item \textbf{Useful Integrals:}  
For $\lambda>0$,
\[\int_0^\infty e^{-\lambda x}\,dx = \frac{1}{\lambda},\qquad\int_0^\infty x e^{-\lambda x}\,dx = \frac{1}{\lambda^2},
\qquad\int_0^\infty x^2 e^{-\lambda x}\,dx = \frac{2}{\lambda^3}.\]

\end{itemize}

\vskip 1cm

%%%%%%%%%%%%%%%%%%%%%%%%%%%%%%%%%%%%%%%%%%%%%%%%%%%%%%%%%%%%%%%%%%%%%%%%%%%%%%%%%%%%%%%%%%%%%%%%%%%%%%
\section*{Determining the Density $f_{_X}$:}
%%%%%%%%%%%%%%%%%%%%%%%%%%%%%%%%%%%%%%%%%%%%%%%%%%%%%%%%%%%%%%%%%%%%%%%%%%%%%%%%%%%%%%%%%%%%%%%%%%%%%%

We will start by assuming that we are waiting for the first arrival in a Poisson process.
\vskip 5mm

%%%%%%%%%%%%%%%%%%%%%%%%%%%%%%%%%%%%%%%%%%%%%%%%%%%%%%%%%%%%%%%%%%%%%%%%%%%%%%%%%%%%%%%%%%%%%%%%%%%%%%
\paragraph*{Step \#1: Waiting Times in a Poisson Process}
%%%%%%%%%%%%%%%%%%%%%%%%%%%%%%%%%%%%%%%%%%%%%%%%%%%%%%%%%%%%%%%%%%%%%%%%%%%%%%%%%%%%%%%%%%%%%%%%%%%%%%

\noindent Suppose events occur randomly over time according to a Poisson processwith rate $\lambda>0$.
A Poisson process with rate $\lambda>0$ satisfies:
		\begin{itemize}
			\item The number of arrivals in an interval of length $t$ has a
$\mathrm{Poisson}(\lambda t)$ distribution.
			\item Numbers of arrivals in disjoint time intervals are independent.
			\item Over a very short interval of length $\delta$, the probability of more than one event in a very short interval is negligible. Formally,
\[\P(\text{one arrival}) = \lambda \delta + o(\delta),\qquad\P(\text{two or more arrivals}) = o(\delta).\]
		\end{itemize}
\vskip 2mm

We are interested in the random variable
\[T_k = \text{time of the $k$th arrival},\]
that is, the amount of time we must wait until the $k$th event occurs.
Our first goal is to find the probability density function of $T_k$.

%%%%%%%%%%%%%%%%%%%%%%%%%%%%%%%%%%%%%%%%%%%%%%%%%%%%%%%%%%%%%%%%%%%%%%%%%%%%%%%%%%%%%%%%%%%%%%%%%%%%%%
\paragraph*{Step \#2: Deriving the Erlang Density}
%%%%%%%%%%%%%%%%%%%%%%%%%%%%%%%%%%%%%%%%%%%%%%%%%%%%%%%%%%%%%%%%%%%%%%%%%%%%%%%%%%%%%%%%%%%%%%%%%%%%%%

\noindent To find the density of $T_k$, we examine the probability that the $k$th
arrival occurs in a short interval $[x,x+\delta]$, where $\delta>0$ is small.

For this to happen, two things must occur:

\begin{enumerate}
\item Exactly $k-1$ arrivals occur in the time interval $[0,x]$.
\item Exactly one arrival occurs in the interval $(x,x+\delta]$.
\end{enumerate}

Because Poisson process increments are independent, the probability of both events occurring is the product of their probabilities. The number of arrivals in $[0,x]$ is $\mathrm{Poisson}(\lambda x)$, so
\[\P\Big(k-1\text{ arrivals in }(0,x]\,\Big)=\frac{(\lambda x)^{k-1}}{(k-1)!}e^{-\lambda x}.\]

\noindent The number of arrivals in $(x,x+\delta]$ is $\mathrm{Poisson}(\lambda\delta)$.
For small $\delta$,
\[\P\Big(1\text{ arrival in }(x,x+\delta]\,\Big)=\lambda\delta + o(\delta).\]

\vskip 2mm

\noindent
Multiplying these probabilities, we obtain
\[\P\big(x \le T_k \le x+\delta\big)=\frac{(\lambda x)^{k-1}}{(k-1)!}e^{-\lambda x}\cdot\lambda\delta+o(\delta).\]

Dividing by $\delta$ and letting $\delta\to 0$, we obtain the probability
density function of $T_k$:
\[f_{T_k}(x)=\lim_{\delta\to 0}\frac{\P(x \le T_k \le x+\delta)}{\delta}=
\frac{\lambda^k x^{k-1}}{(k-1)!}e^{-\lambda x},\qquad x \ge 0.\]

This distribution is called the \emph{Erlang distribution} with parameters
$k$ and $\lambda$.

%%%%%%%%%%%%%%%%%%%%%%%%%%%%%%%%%%%%%%%%%%%%%%%%%%%%%%%%%%%%%%%%%%%%%%%%%%%%%%%%%%%%%%%%%%%%%%%%%%%%%%
\paragraph*{Step \#3: From Erlang to Exponential}
%%%%%%%%%%%%%%%%%%%%%%%%%%%%%%%%%%%%%%%%%%%%%%%%%%%%%%%%%%%%%%%%%%%%%%%%%%%%%%%%%%%%%%%%%%%%%%%%%%%%%%

\noindent The exponential distribution arises as the special case $k=1$.
Setting $k=1$ in the Erlang density,
\[f_{T_1}(x)=\frac{\lambda^1 x^{0}}{0!}e^{-\lambda x}=\lambda e^{-\lambda x},\qquad x \ge 0.\]

Thus, the waiting time until the \emph{first} arrival in a Poisson process
with rate $\lambda$ has density
\[f_{_X}(x)=\lambda e^{-\lambda x},\qquad x \ge 0,\]
and we write
\[X \sim \mathrm{Exp}(\lambda).\]

\vskip 3mm

\noindent
This mirrors the discrete setting:
\begin{itemize}
\item Geometric distribution: number of Bernoulli trials until first success.
\item Exponential distribution: amount of time until first Poisson arrival.
\end{itemize}

Both distributions arise from the same idea of \emph{waiting for the first
``arrival"}, with geometric time measured in trials and exponential time measured
continuously.

\begin{tcolorbox}[colback=yellow!10!white,colframe=yellow!50!black,title=Remark,sharp corners,boxrule=0.8pt]
\textbf{Erlang vs.\ Gamma:}  

The Erlang distribution is a special case of the Gamma distribution with an \emph{integer shape parameter} \(k\). That is, if
\[
X \sim \text{Erlang}(k,\lambda),
\]
then
\[
f_X(x) = \frac{\lambda^k x^{k-1}}{(k-1)!} e^{-\lambda x}, \quad x \ge 0,
\]
which is equivalent to
\[
X \sim \text{Gamma}(\alpha=k, \lambda), \quad \Gamma(k)=(k-1)!.
\]

\noindent \textbf{Intuition:} Erlang is used in queueing/Poisson processes as the waiting time until the \(k\)th arrival, while Gamma allows non-integer shapes in more general statistical contexts.
\end{tcolorbox}


%%%%%%%%%%%%%%%%%%%%%%%%%%%%%%%%%%%%%%%%%%%%%%%%%%%%%%%%%%%%%%%%%%%%%%%%%%%%%%%%%%%%%%%%%%%%%%%%%%%%%%
\section*{Calculating $\mathbb{E}[X]$:}
%%%%%%%%%%%%%%%%%%%%%%%%%%%%%%%%%%%%%%%%%%%%%%%%%%%%%%%%%%%%%%%%%%%%%%%%%%%%%%%%%%%%%%%%%%%%%%%%%%%%%%

\noindent
Let $X \sim \mathrm{Exp}(\lambda)$. Then
\begin{align*}
\mathbb{E}[X]
&= \int_0^\infty x \lambda e^{-\lambda x}\,dx \\
&= \lambda \int_0^\infty x e^{-\lambda x}\,dx.
\end{align*}

Using integration by parts, let
\[u = x, \quad dv = e^{-\lambda x}dx\quad \Longrightarrow \quad
du = dx, \quad v = -\frac{1}{\lambda}e^{-\lambda x}.\]

Then
\begin{align*}
\int_0^\infty x e^{-\lambda x}\,dx
&= \left[-\frac{x}{\lambda}e^{-\lambda x}\right]_0^\infty
+ \frac{1}{\lambda}\int_0^\infty e^{-\lambda x}\,dx \\
&= 0 + \frac{1}{\lambda}\cdot\frac{1}{\lambda}
= \frac{1}{\lambda^2}.
\end{align*}

Therefore,
\[\mathbb{E}[X]=\lambda\cdot\frac{1}{\lambda^2}=\frac{1}{\lambda}.\]

\vskip 5mm

\noindent
\textbf{Intuition for } $\mathbb{E}[X] = 1/\lambda$:

\begin{itemize}
\item $\lambda$ represents the average number of events per unit time.
\item If events occur rapidly (large $\lambda$), we expect to wait less time.
\item If events are rare (small $\lambda$), the waiting time is longer.
\end{itemize}

For example:
\begin{itemize}
\item $\lambda=2$: on average, wait $1/2$ unit of time.
\item $\lambda=0.1$: on average, wait $10$ units of time.
\end{itemize}

\vskip 1cm

%%%%%%%%%%%%%%%%%%%%%%%%%%%%%%%%%%%%%%%%%%%%%%%%%%%%%%%%%%%%%%%%%%%%%%%%%%%%%%%%%%%%%%%%%%%%%%%%%%%%%%
\section*{Calculating and $\mathrm{Var}(X)$:}
%%%%%%%%%%%%%%%%%%%%%%%%%%%%%%%%%%%%%%%%%%%%%%%%%%%%%%%%%%%%%%%%%%%%%%%%%%%%%%%%%%%%%%%%%%%%%%%%%%%%%%

To compute the variance, we first find $\mathbb{E}[X^2]$.
\begin{align*}
\mathbb{E}[X^2]
&= \int^{\infty}_{x=0} x^2 \lambda e^{-\lambda x}\,dx \\
&= \lambda \int_0^\infty x^2 e^{-\lambda x}\,dx.
\end{align*}

Using integration by parts twice (or known gamma integrals),
\[\int^{\infty}_{x=0} x^2 e^{-\lambda x}\,dx = \frac{2}{\lambda^3}.\]

Thus,
\[\mathbb{E}[X^2]=\lambda\cdot\frac{2}{\lambda^3}= \frac{2}{\lambda^2}.\]

Now compute the variance:
\[\mathrm{Var}(X)=\mathbb{E}[X^2]-(\mathbb{E}[X])^2=\frac{2}{\lambda^2}-\frac{1}{\lambda^2}
=\frac{1}{\lambda^2}.\]

\vskip 5mm

\noindent
\textbf{Intuition for } $\mathrm{Var}(X)= 1/\lambda^2$:

\begin{itemize}
\item The variance scales with the square of the mean waiting time.
\item Faster event rates lead to more tightly clustered waiting times.
\item Slower rates produce larger variability in how long we must wait.
\end{itemize}

\vskip 5mm

%%%%%%%%%%%%%%%%%%%%%%%%%%%%%%%%%%%%%%%%%%%%%%%%%%%%%%%%%%%%%%%%%%%%%%%%%%%%%%%%%%%%%%%%%%%%%%%%%%%%%%
\section*{Using the Memoryless Property (Continuous-Time Argument):}
%%%%%%%%%%%%%%%%%%%%%%%%%%%%%%%%%%%%%%%%%%%%%%%%%%%%%%%%%%%%%%%%%%%%%%%%%%%%%%%%%%%%%%%%%%%%%%%%%%%%%%

Let $X \sim \mathrm{Exp}(\lambda)$ be the waiting time until the first arrival
in a Poisson process with rate $\lambda$. We derive $\mathbb{E}[X]$ using the memoryless property and a small-time
partition of the sample space.

\vskip 3mm

Fix $t>0$. The events $\{X \le t\}$ and $\{X>t\}$ form a partition of the
sample space. By the law of total expectation,
\beq\mathbb{E}[X]=\P(X \le t)\,\mathbb{E}[X \mid X \le t]+\P(X > t)\,\mathbb{E}[X \mid X > t].\eeq

Now condition on the event $\{X>t\}$.
If $X>t$, define the remaining waiting time
\[Y = X - t.\]
By the memoryless property of the exponential distribution,
\[Y \sim \mathrm{Exp}(\lambda),\quad \text{so} \quad\mathbb{E}[Y] = \mathbb{E}[X].\]
Therefore,
\[\mathbb{E}[X \mid X>t] = t + \mathbb{E}[X].\]

Substituting into the law of total expectation,
\begin{align*}
\mathbb{E}[X]
&= \P(X \le t)\,\mathbb{E}[X \mid X \le t]
+ \P(X > t)\big(t + \mathbb{E}[X]\big) \\[1mm]
&= \P(X \le t)\,\mathbb{E}[X \mid X \le t]
+ t\P(X>t)
+ \P(X>t)\mathbb{E}[X].
\end{align*}

Rearranging terms,
\[\P(X \le t)\,\mathbb{E}[X]=\P(X \le t)\,\mathbb{E}[X \mid X \le t]+t\P(X>t).\]

Dividing by $\P(X \le t)$ gives
\beq
\mathbb{E}[X]=\mathbb{E}[X \mid X \le t]+t\cdot\frac{\P(X>t)}{\P(X \le t)}.
\eeq

Now let $t\to 0^{+}$. For an exponential random variable,
\[\P(X \le t) = 1 - e^{-\lambda t}\sim\lambda t,\qquad\P(X>t)\to 1,\qquad
\mathbb{E}[X \mid X \le t] \to 0.\]

Taking the limit,
\[\mathbb{E}[X]=\lim_{t\to 0^{+}}\left(\mathbb{E}[X \mid X \le t]+t\frac{\P(X>t)}{\P(X \le t)}\right)=0+\cdot\frac{1}{\lambda}=\frac{1}{\lambda}.\]

%%%%%%%%%%%%%%%%%%%%%%%%%%%%%%%%%%%%%%%%%%%%%%%%%%%%%%%%%%%%%%%%%%%%%%%%%%%%%%%%%%%%%%%%%%%%%%%%%%%%%%
\begin{tcolorbox}[colback=gray!10!white,colframe=gray!60!black,
title={Why the Geometric Argument Fails in Continuous Time},
sharp corners, boxrule=0.5pt]
For a geometric random variable $X$, the event $\{X=1\}$ has positive
probability, which allows us to condition on the first trial:
\[\{X=1\}, \quad \{X>1\}.\]

In continuous time, however,
\[\P(X=0)=0.\]
There is no ``first instant’’ with positive probability, so conditioning on
$X=0$ is meaningless. Instead, we must condition on a \emph{small interval} of time:
\[\{X \le t\}, \quad \{X>t\},\]
and then let $t \downarrow 0$. This small-time partition plays the role of the first trial in the discrete
setting.
\end{tcolorbox}

%%%%%%%%%%%%%%%%%%%%%%%%%%%%%%%%%%%%%%%%%%%%%%%%%%%%%%%%%%%%%%%%%%%%%%%%%%%%%%%%%%%%%%%%%%%%%%%%%%%%%%
\paragraph*{Discrete vs.\ Continuous Waiting-Time Models}
%%%%%%%%%%%%%%%%%%%%%%%%%%%%%%%%%%%%%%%%%%%%%%%%%%%%%%%%%%%%%%%%%%%%%%%%%%%%%%%%%%%%%%%%%%%%%%%%%%%%%%

\begin{center}
\begin{tabular}{|c|c|}
\hline
\textbf{Geometric} & \textbf{Exponential} \\
\hline
Bernoulli trials & Poisson process \\
Discrete time & Continuous time \\
Success probability $p$ & Rate $\lambda$ \\
Wait for first success & Wait for first arrival \\
$\P(X=1)>0$ & $\P(X=0)=0$ \\
Condition on $\{X=1\}$ or $\{X>1\}$ & Condition on $\{X \le t\}$ or $\{X>t\}$ \\
Memoryless: $\P(X>s+t\mid X>t)=\P(X>s)$ & Same memoryless property \\
$\mathbb{E}[X]=1/p$ & $\mathbb{E}[X]=1/\lambda$ \\
\hline
\end{tabular}
\end{center}

\vskip 1cm
\hrule
\vskip 1cm

%%%%%%%%%%%%%%%%%%%%%%%%%%%%%%%%%%%%%%%%%%%%%%%%%%%%%%%%%%%%%%%%%%%%%%%%%%%%%%%%%%%%%%%%%%%%%%%%%%%%%%
\section*{Foundational Examples}
%%%%%%%%%%%%%%%%%%%%%%%%%%%%%%%%%%%%%%%%%%%%%%%%%%%%%%%%%%%%%%%%%%%%%%%%%%%%%%%%%%%%%%%%%%%%%%%%%%%%%%

The following examples illustrate the basic ideas behind exponential and
Erlang random variables.
Each example highlights a core concept that will be used repeatedly
throughout the course.

\vskip 5mm

%%%%%%%%%%%%%%%%%%%%%%%%%%%%%%%%%%%%%%%%%%%%%%%%%%%%%%%%%%%%%%%%%%%%%%%%%%%%%%%%%%%%%%%%%%%%%%%%%%%%%%
\subsection*{Example 1: Interpreting the Rate Parameter}
%%%%%%%%%%%%%%%%%%%%%%%%%%%%%%%%%%%%%%%%%%%%%%%%%%%%%%%%%%%%%%%%%%%%%%%%%%%%%%%%%%%%%%%%%%%%%%%%%%%%%%

\noindent
Suppose customers arrive at a help desk according to a Poisson process
with rate $\lambda=3$ per hour.

\begin{itemize}
\item[(a)] What is the expected time until the first customer arrives?
\item[(b)] What is the probability that no customer arrives in the first
$20$ minutes?
\end{itemize}

\begin{solutionbox}
\begin{itemize}
\item[(a)] The waiting time $X$ until the first arrival is exponential
with parameter $\lambda=3$.
Thus,
\[
\mathbb{E}[X]=\frac{1}{\lambda}=\frac{1}{3}\text{ hour }=20\text{ minutes}.
\]

\item[(b)] Twenty minutes is $1/3$ of an hour.
Therefore,
\[
\P(X>1/3)=e^{-3(1/3)}=e^{-1}.
\]
\end{itemize}
\end{solutionbox}

\vskip 5mm

%%%%%%%%%%%%%%%%%%%%%%%%%%%%%%%%%%%%%%%%%%%%%%%%%%%%%%%%%%%%%%%%%%%%%%%%%%%%%%%%%%%%%%%%%%%%%%%%%%%%%%
\subsection*{Example 2: Memorylessness in Words}
%%%%%%%%%%%%%%%%%%%%%%%%%%%%%%%%%%%%%%%%%%%%%%%%%%%%%%%%%%%%%%%%%%%%%%%%%%%%%%%%%%%%%%%%%%%%%%%%%%%%%%

\noindent
Let $X \sim \mathrm{Exp}(\lambda)$.
Suppose no arrival has occurred in the first $10$ minutes.

\begin{itemize}
\item[(a)] What is the distribution of the remaining waiting time?
\item[(b)] Does the past waiting time affect the future waiting time?
\end{itemize}

\begin{solutionbox}
\begin{itemize}
\item[(a)] By the memoryless property, the remaining waiting time has the
same exponential distribution as the original waiting time:
\[
X-10 \mid X>10 \sim \mathrm{Exp}(\lambda).
\]

\item[(b)] No. The exponential distribution has no memory.
The amount of time already waited provides no information about how much
longer we must wait.
\end{itemize}
\end{solutionbox}

\vskip 5mm

%%%%%%%%%%%%%%%%%%%%%%%%%%%%%%%%%%%%%%%%%%%%%%%%%%%%%%%%%%%%%%%%%%%%%%%%%%%%%%%%%%%%%%%%%%%%%%%%%%%%%%
\subsection*{Example 3: From Erlang to Exponential}
%%%%%%%%%%%%%%%%%%%%%%%%%%%%%%%%%%%%%%%%%%%%%%%%%%%%%%%%%%%%%%%%%%%%%%%%%%%%%%%%%%%%%%%%%%%%%%%%%%%%%%

\noindent
Let $T_k$ be the time of the $k$th arrival in a Poisson process with rate
$\lambda$.

\begin{itemize}
\item[(a)] Write the density function of $T_k$.
\item[(b)] What distribution do we obtain when $k=1$?
\end{itemize}

\begin{solutionbox}
\begin{itemize}
\item[(a)] The random variable $T_k$ has an Erlang distribution with density
\[
f_{T_k}(x)=\frac{\lambda^k x^{k-1}}{(k-1)!}e^{-\lambda x}, \qquad x\ge0.
\]

\item[(b)] When $k=1$, this reduces to
\[
f_{T_1}(x)=\lambda e^{-\lambda x}, \qquad x\ge0,
\]
which is the exponential distribution.
\end{itemize}
\end{solutionbox}

\vskip 5mm

%%%%%%%%%%%%%%%%%%%%%%%%%%%%%%%%%%%%%%%%%%%%%%%%%%%%%%%%%%%%%%%%%%%%%%%%%%%%%%%%%%%%%%%%%%%%%%%%%%%%%%
\subsection*{Example 4: Mean Waiting Time for Multiple Arrivals}
%%%%%%%%%%%%%%%%%%%%%%%%%%%%%%%%%%%%%%%%%%%%%%%%%%%%%%%%%%%%%%%%%%%%%%%%%%%%%%%%%%%%%%%%%%%%%%%%%%%%%%

\noindent
Let $T_3$ denote the time until the third arrival in a Poisson process
with rate $\lambda$.

\begin{itemize}
\item[(a)] Find $\mathbb{E}[T_3]$.
\item[(b)] Explain your answer intuitively.
\end{itemize}

\begin{solutionbox}
\begin{itemize}
\item[(a)] The Erlang distribution with parameter $k$ has mean
\[
\mathbb{E}[T_k]=\frac{k}{\lambda}.
\]
Thus,
\[
\mathbb{E}[T_3]=\frac{3}{\lambda}.
\]

\item[(b)] Waiting for three arrivals is equivalent to waiting for three
independent exponential waiting times and adding them together.
On average, each arrival takes $1/\lambda$ units of time.
\end{itemize}
\end{solutionbox}

\vskip 5mm

%%%%%%%%%%%%%%%%%%%%%%%%%%%%%%%%%%%%%%%%%%%%%%%%%%%%%%%%%%%%%%%%%%%%%%%%%%%%%%%%%%%%%%%%%%%%%%%%%%%%%%
\subsection*{Example 5: Discrete vs.\ Continuous Waiting}
%%%%%%%%%%%%%%%%%%%%%%%%%%%%%%%%%%%%%%%%%%%%%%%%%%%%%%%%%%%%%%%%%%%%%%%%%%%%%%%%%%%%%%%%%%%%%%%%%%%%%%

\noindent
Suppose we repeatedly flip a coin with probability of heads $p=0.25$,
and also observe arrivals in a Poisson process with rate $\lambda=0.25$.

\begin{itemize}
\item[(a)] What is the expected number of trials until the first head?
\item[(b)] What is the expected waiting time until the first arrival?
\item[(c)] Why are these models considered analogues?
\end{itemize}

\begin{solutionbox}
\begin{itemize}
\item[(a)] The waiting time is geometric with mean $1/p=4$ trials.
\item[(b)] The waiting time is exponential with mean $1/\lambda=4$ time units.
\item[(c)] Both models describe waiting until the first occurrence of an
event, differing only in whether time is measured discretely or continuously.
\end{itemize}
\end{solutionbox}


\vskip 1cm
\hrule
\vskip 1cm


%%%%%%%%%%%%%%%%%%%%%%%%%%%%%%%%%%%%%%%%%%%%%%%%%%%%%%%%%%%%%%%%%%%%%%%%%%%%%%%%%%%%%%%%%%%%%%%%%%%%%%
\section*{In-Class Practice Problems}
%%%%%%%%%%%%%%%%%%%%%%%%%%%%%%%%%%%%%%%%%%%%%%%%%%%%%%%%%%%%%%%%%%%%%%%%%%%%%%%%%%%%%%%%%%%%%%%%%%%%%%

The following problems are designed to be worked during class time.
Some are computational, while others focus on interpretation and structure.
Unless stated otherwise, assume all Poisson processes have rate $\lambda>0$.

\vskip 5mm

%%%%%%%%%%%%%%%%%%%%%%%%%%%%%%%%%%%%%%%%%%%%%%%%%%%%%%%%%%%%%%%%%%%%%%%%%%%%%%%%%%%%%%%%%%%%%%%%%%%%%%
\subsection*{Warm-Up: Interpreting the Exponential Distribution}
%%%%%%%%%%%%%%%%%%%%%%%%%%%%%%%%%%%%%%%%%%%%%%%%%%%%%%%%%%%%%%%%%%%%%%%%%%%%%%%%%%%%%%%%%%%%%%%%%%%%%%

\begin{enumerate}
\item Let $X \sim \mathrm{Exp}(\lambda)$.
Interpret the quantity $\P(X>t)$ in words.
What does the formula $\P(X>t)=e^{-\lambda t}$ tell you about long waiting times?

\item Suppose the average number of arrivals is $2$ per hour.
\begin{itemize}
\item What is the value of $\lambda$?
\item What is the expected waiting time until the first arrival?
\item Find $\P(X>30 \text{ minutes})$.
\end{itemize}

\item True or False (justify briefly):  
If $\mathbb{E}[X]=5$, then $\lambda=5$.
\end{enumerate}

\vskip 5mm

%%%%%%%%%%%%%%%%%%%%%%%%%%%%%%%%%%%%%%%%%%%%%%%%%%%%%%%%%%%%%%%%%%%%%%%%%%%%%%%%%%%%%%%%%%%%%%%%%%%%%%
\subsection*{Building the Erlang Distribution}
%%%%%%%%%%%%%%%%%%%%%%%%%%%%%%%%%%%%%%%%%%%%%%%%%%%%%%%%%%%%%%%%%%%%%%%%%%%%%%%%%%%%%%%%%%%%%%%%%%%%%%

\begin{enumerate}
\setcounter{enumi}{3}

\item Let $T_2$ be the time of the second arrival in a Poisson process.
Describe in words what must happen for $T_2$ to lie in the interval $[x,x+\delta]$.

\item Using the Poisson distribution, write an expression for
\[
\P(x \le T_2 \le x+\delta)
\]
in terms of $\lambda$, $x$, and $\delta$, ignoring terms of order $o(\delta)$.

\item Generalize the previous problem to $T_k$.
Which parts of the argument depend on $k$, and which do not?

\item Show that the Erlang density integrates to $1$:
\[
f_{T_k}(x) = \frac{\lambda^k x^{k-1}}{(k-1)!}e^{-\lambda x}, \quad x\ge0.
\]
(Hint: use repeated integration by parts or recall the gamma function.)
\end{enumerate}

\vskip 5mm

%%%%%%%%%%%%%%%%%%%%%%%%%%%%%%%%%%%%%%%%%%%%%%%%%%%%%%%%%%%%%%%%%%%%%%%%%%%%%%%%%%%%%%%%%%%%%%%%%%%%%%
\subsection*{Memoryless Property}
%%%%%%%%%%%%%%%%%%%%%%%%%%%%%%%%%%%%%%%%%%%%%%%%%%%%%%%%%%%%%%%%%%%%%%%%%%%%%%%%%%%%%%%%%%%%%%%%%%%%%%

\begin{enumerate}
\setcounter{enumi}{7}

\item Let $X \sim \mathrm{Exp}(\lambda)$.
Compute $\P(X>10 \mid X>5)$ and compare it to $\P(X>5)$.
What do you observe?

\item Suppose you have already waited $t$ minutes with no arrivals.
What is the distribution of the remaining waiting time?
Explain your reasoning without formulas.

\item Contrast this with the discrete case:
If $X \sim \mathrm{Geom}(p)$, what is the distribution of $X-1$ given $X>1$?
\end{enumerate}

\vskip 5mm

%%%%%%%%%%%%%%%%%%%%%%%%%%%%%%%%%%%%%%%%%%%%%%%%%%%%%%%%%%%%%%%%%%%%%%%%%%%%%%%%%%%%%%%%%%%%%%%%%%%%%%
\subsection*{Expectation via Small-Time Conditioning}
%%%%%%%%%%%%%%%%%%%%%%%%%%%%%%%%%%%%%%%%%%%%%%%%%%%%%%%%%%%%%%%%%%%%%%%%%%%%%%%%%%%%%%%%%%%%%%%%%%%%%%

\begin{enumerate}
\setcounter{enumi}{10}

\item Let $X \sim \mathrm{Exp}(\lambda)$ and fix $t>0$.
Write the law of total expectation using the partition
$\{X \le t\}$ and $\{X>t\}$.

\item Show that
\[
\mathbb{E}[X \mid X>t] = t + \mathbb{E}[X].
\]

\item Use the approximation $\P(X \le t) \approx \lambda t$ for small $t$
to explain why
\[
\mathbb{E}[X] \approx \frac{t}{\lambda t}
\quad \text{for small } t.
\]

\item Why does this argument fail if we try to condition on $\{X=0\}$
instead of $\{X \le t\}$?
\end{enumerate}

\vskip 5mm

%%%%%%%%%%%%%%%%%%%%%%%%%%%%%%%%%%%%%%%%%%%%%%%%%%%%%%%%%%%%%%%%%%%%%%%%%%%%%%%%%%%%%%%%%%%%%%%%%%%%%%
\subsection*{Comparing Discrete and Continuous Waiting Times}
%%%%%%%%%%%%%%%%%%%%%%%%%%%%%%%%%%%%%%%%%%%%%%%%%%%%%%%%%%%%%%%%%%%%%%%%%%%%%%%%%%%%%%%%%%%%%%%%%%%%%%

\begin{enumerate}
\setcounter{enumi}{14}

\item Fill in the blanks:

\begin{center}
\begin{tabular}{|c|c|}
\hline
\textbf{Geometric} & \textbf{Exponential} \\
\hline
Trials until success & \rule{4cm}{0.15mm} \\
Parameter $p$ & \rule{4cm}{0.15mm} \\
$\mathbb{E}[X]=\rule{1cm}{0.15mm}$ & $\mathbb{E}[X]=\rule{1cm}{0.15mm}$ \\
\hline
\end{tabular}
\end{center}

\item Give one reason (conceptual, not computational) why both distributions
are memoryless.

\item Suppose arrivals occur at rate $\lambda=0.2$.
Would you expect more or less variability in waiting times than if
$\lambda=2$?
Explain without computing a variance.
\end{enumerate}

\vskip 5mm

\noindent
\textbf{Reflection (Optional):}  
In what sense is the exponential distribution the “continuous analogue’’
of the geometric distribution?
Where does the analogy break down?


\vfill\eject

%%%%%%%%%%%%%%%%%%%%%%%%%%%%%%%%%%%%%%%%%%%%%%%%%%%%%%%%%%%%%%%%%%%%%%%%%%%%%%%%%%%%%%%%%%%%%%%%%%%%%%
\section*{Solutions to In-Class Practice Problems}
%%%%%%%%%%%%%%%%%%%%%%%%%%%%%%%%%%%%%%%%%%%%%%%%%%%%%%%%%%%%%%%%%%%%%%%%%%%%%%%%%%%%%%%%%%%%%%%%%%%%%%

%%%%%%%%%%%%%%%%%%%%%%%%%%%%%%%%%%%%%%%%%%%%%%%%%%%%%%%%%%%%%%%%%%%%%%%%%%%%%%%%%%%%%%%%%%%%%%%%%%%%%%
\subsection*{Warm-Up: Interpreting the Exponential Distribution}
%%%%%%%%%%%%%%%%%%%%%%%%%%%%%%%%%%%%%%%%%%%%%%%%%%%%%%%%%%%%%%%%%%%%%%%%%%%%%%%%%%%%%%%%%%%%%%%%%%%%%%

\begin{enumerate}
\item The quantity $\P(X>t)$ represents the probability that we must wait
\emph{longer than $t$ units of time} for the first arrival.
The formula $\P(X>t)=e^{-\lambda t}$ shows that long waiting times decay
exponentially and become increasingly unlikely.

\item If arrivals occur at an average rate of $2$ per hour, then $\lambda=2$.
\begin{itemize}
\item The expected waiting time is $\mathbb{E}[X]=1/\lambda=1/2$ hour.
\item Thirty minutes is $0.5$ hours, so
\[
\P(X>0.5)=e^{-2(0.5)}=e^{-1}.
\]
\end{itemize}

\item False.  
If $\mathbb{E}[X]=5$, then $1/\lambda=5$, so $\lambda=1/5$.
\end{enumerate}

\vskip 5mm

%%%%%%%%%%%%%%%%%%%%%%%%%%%%%%%%%%%%%%%%%%%%%%%%%%%%%%%%%%%%%%%%%%%%%%%%%%%%%%%%%%%%%%%%%%%%%%%%%%%%%%
\subsection*{Building the Erlang Distribution}
%%%%%%%%%%%%%%%%%%%%%%%%%%%%%%%%%%%%%%%%%%%%%%%%%%%%%%%%%%%%%%%%%%%%%%%%%%%%%%%%%%%%%%%%%%%%%%%%%%%%%%

\begin{enumerate}
\setcounter{enumi}{3}

\item For $T_2$ to lie in $[x,x+\delta]$:
\begin{itemize}
\item Exactly one arrival must occur in $[0,x]$.
\item Exactly one arrival must occur in $(x,x+\delta]$.
\end{itemize}

\item Using independence of increments,
\[
\P(x \le T_2 \le x+\delta)
=
\P(N(x)=1)\cdot\P(N(x+\delta)-N(x)=1)
=
(\lambda x)e^{-\lambda x}\cdot(\lambda\delta)+o(\delta).
\]

\item For $T_k$, exactly $k-1$ arrivals must occur in $[0,x]$, and exactly
one arrival must occur in $(x,x+\delta]$.
Only the first probability depends on $k$; the small-interval probability
does not.

\item We compute
\[
\int_0^\infty \frac{\lambda^k x^{k-1}}{(k-1)!}e^{-\lambda x}\,dx
=
\frac{\lambda^k}{(k-1)!}\cdot\frac{(k-1)!}{\lambda^k}
=1.
\]
Thus $f_{T_k}$ is a valid density.
\end{enumerate}

\vskip 5mm

%%%%%%%%%%%%%%%%%%%%%%%%%%%%%%%%%%%%%%%%%%%%%%%%%%%%%%%%%%%%%%%%%%%%%%%%%%%%%%%%%%%%%%%%%%%%%%%%%%%%%%
\subsection*{Memoryless Property}
%%%%%%%%%%%%%%%%%%%%%%%%%%%%%%%%%%%%%%%%%%%%%%%%%%%%%%%%%%%%%%%%%%%%%%%%%%%%%%%%%%%%%%%%%%%%%%%%%%%%%%

\begin{enumerate}
\setcounter{enumi}{7}

\item
\[
\P(X>10 \mid X>5)
=
\frac{\P(X>10)}{\P(X>5)}
=
\frac{e^{-10\lambda}}{e^{-5\lambda}}
=
e^{-5\lambda}
=
\P(X>5).
\]
This confirms the memoryless property.

\item The remaining waiting time has the same exponential distribution
as the original waiting time.
Past waiting provides no information about future waiting.

\item If $X\sim\mathrm{Geom}(p)$, then
\[
X-1 \mid X>1 \sim \mathrm{Geom}(p).
\]
The process restarts after each failure, just as the exponential clock
restarts after time has passed.
\end{enumerate}

\vskip 5mm

%%%%%%%%%%%%%%%%%%%%%%%%%%%%%%%%%%%%%%%%%%%%%%%%%%%%%%%%%%%%%%%%%%%%%%%%%%%%%%%%%%%%%%%%%%%%%%%%%%%%%%
\subsection*{Expectation via Small-Time Conditioning}
%%%%%%%%%%%%%%%%%%%%%%%%%%%%%%%%%%%%%%%%%%%%%%%%%%%%%%%%%%%%%%%%%%%%%%%%%%%%%%%%%%%%%%%%%%%%%%%%%%%%%%

\begin{enumerate}
\setcounter{enumi}{10}

\item The law of total expectation gives
\[
\mathbb{E}[X]
=
\P(X \le t)\,\mathbb{E}[X \mid X \le t]
+
\P(X>t)\,\mathbb{E}[X \mid X>t].
\]

\item If $X>t$, then the remaining waiting time $Y=X-t$ satisfies
$Y\sim\mathrm{Exp}(\lambda)$ by memorylessness.
Thus
\[
\mathbb{E}[X \mid X>t] = t + \mathbb{E}[X].
\]

\item For small $t$, we have $\P(X \le t)\approx \lambda t$ and
$\P(X>t)\approx 1$.
Since $\mathbb{E}[X \mid X \le t]\approx 0$,
\[
\mathbb{E}[X]
\approx
t\cdot\frac{1}{\lambda t}
=
\frac{1}{\lambda}.
\]

\item Conditioning on $\{X=0\}$ is invalid because $\P(X=0)=0$.
In continuous time, we must condition on intervals of positive length,
not single points.
\end{enumerate}

\vskip 5mm

%%%%%%%%%%%%%%%%%%%%%%%%%%%%%%%%%%%%%%%%%%%%%%%%%%%%%%%%%%%%%%%%%%%%%%%%%%%%%%%%%%%%%%%%%%%%%%%%%%%%%%
\subsection*{Comparing Discrete and Continuous Waiting Times}
%%%%%%%%%%%%%%%%%%%%%%%%%%%%%%%%%%%%%%%%%%%%%%%%%%%%%%%%%%%%%%%%%%%%%%%%%%%%%%%%%%%%%%%%%%%%%%%%%%%%%%

\begin{enumerate}
\setcounter{enumi}{14}

\item
\begin{center}
\begin{tabular}{|c|c|}
\hline
\textbf{Geometric} & \textbf{Exponential} \\
\hline
Trials until success & Time until first arrival \\
Parameter $p$ & Rate $\lambda$ \\
$\mathbb{E}[X]=1/p$ & $\mathbb{E}[X]=1/\lambda$ \\
\hline
\end{tabular}
\end{center}

\item Both distributions model waiting until the first occurrence of
an event, where each trial or instant has the same chance of success,
independent of the past.

\item With $\lambda=0.2$, arrivals are much slower, so waiting times are
longer and more variable.
With $\lambda=2$, arrivals are faster and waiting times are more tightly
clustered.
\end{enumerate}


%%%%%%%%%%%%%%%%%%%%%%%%%%%%%%%%%%%%%%%%%%%%%%%%%%%%%%%%%%%%%%%%%%%%%%%%%%%%%%%%%%%%%%%%%%%%%%%%%%%%%%%
%\section*{The R Code:} 
%%%%%%%%%%%%%%%%%%%%%%%%%%%%%%%%%%%%%%%%%%%%%%%%%%%%%%%%%%%%%%%%%%%%%%%%%%%%%%%%%%%%%%%%%%%%%%%%%%%%%%%
%
%Here is the R code used for the above simulations.
%\vspace*{2mm}
%\small 
%\begin{lstlisting}[language=R]
%
%\end{lstlisting}



\vskip 1cm
\hrule
\vskip 5mm
\begin{center}
\bf Please let me know if you have any questions, comments, or corrections!
\end{center}

%%%%%%%%%%%%%%%%%%%%%%%%%%%%%%%%%%%%%%%%%%%%%%%%%%%%%%
\end{document}
%%%%%%%%%%%%%%%%%%%%%%%%%%%%%%%%%%%%%%%%%%%%%%%%%%%%%%
