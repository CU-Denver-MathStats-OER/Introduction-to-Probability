\documentclass[11pt]{article}
\usepackage[suffix=Solutions]{teaching-header}

%%%%%%%%%%%%%%%%%%%%%%%%%%%%%%%%%%%%%%%%%%%%%%%%%%%%%%%%%%%%%%%%%%%%%%%%%%%%
%%%%%%%%%%%%%%%%%%%%%%%%%%%%%%%%%%%%%%%%%%%%%%%%%%%%%%%%%%%%%%%%%%%%%%%%%%%%


%\ifnum\printsol=0 (when no solutions printed)
%Do something
%	\else  (when solutions are printed)
%Do something else
%\fi


% Assessment specific definitions to fill in title and page headers. Other course specific definitions are located in classinfo document which is accessed by \input command below. 
%\def\classnum{3810}
%\def\classtitle{Probability}
%\def\classtitleshort{Probability}
%\def\classsec{001}
%\def\classterm{Fall 2025}
%\def\instructor{Robert Rostermundt}
\def\assessmenttype{homework}
\def\assessmenttypecap{Homework}
\def\assessmentname{\#4}  % Such as #1 or a word title
\def\assessmentnamesol{\#4 Solutions}
\def\duedate{Wednesday, September 18}
\def\duetime{3:30pm}
\def\printfancyheader{1} % Beware that changing the header will change custom pagebreaks for boxed solutions

\def\classnum{3810}
\def\classtitle{Probability}
\def\classtitleshort{Probability}
\def\classsec{001}
\def\classterm{Fall 2025}
\def\instructor{Robert Rostermundt}
%\def\hmwknum{\#2}  % Class Specific Information for title header

%If use fancy title header then include custom header on all pages. If don't use fancy title header then include pagenumbers only when solutions printed
\ifnum\printfancyheader=1
\pagestyle{myheadings}
	\else
		\ifnum\printsol=0
				\pagestyle{empty}
			\else
				\pagestyle{plain}
		\fi		
\fi


% Creates a fancy title header for front page. This is turned on by boolean \printfancyheader which later invokes \maketitle command. Otherwise a simpler header is displayed.
\ifnum\printsol=0 % Determine if non-solution fancyheader or solution fancyheader is displayed
\title{\vspace{-1in}Math\classnum\;-\;\classtitle\\
	Section\;\classsec\;-\;\classterm\\
	Practice Problems: WLLN and CLT}
	\author{University of Colorado Denver / College of Liberal Arts 	and Sciences}
	\date{Department of Mathematics - \instructor}

	\markright{Math\,\classnum\;-\;\classtitleshort,\;Practice Problems: WLLN and CLT, UCD, \classterm, \instructor}
	
		\else

\title{\vspace{-1in}Math\classnum\;-\;\classtitle\\
	Section\;\classsec\;-\;\classterm\\
	Practice Problems: WLLN and CLT (Solutions)}
	\author{University of Colorado Denver / College of Liberal Arts 	and Sciences}
	\date{Department of Mathematics - \instructor}

	\markright{Math\classnum\;-\;\classtitleshort,\;Practice Problems: WLLN and CLT, UCD, \classterm, \instructor}

\fi


%%%%%%%%%%%%%%%%%%%%%%%%%%%%%%%%%%%%%%%%%%%%%%%%%%%%%%%%%%%%%%%%%%%%%%%%%%%%
%%%%%%%%%%%%%%%%%%%%%%%%%%%%%%%%%%%%%%%%%%%%%%%%%%%%%%%%%%%%%%%%%%%%%%%%%%%%
\begin{document}
%%%%%%%%%%%%%%%%%%%%%%%%%%%%%%%%%%%%%%%%%%%%%%%%%%%%%%%%%%%%%%%%%%%%%%%%%%%%
%%%%%%%%%%%%%%%%%%%%%%%%%%%%%%%%%%%%%%%%%%%%%%%%%%%%%%%%%%%%%%%%%%%%%%%%%%%%
\ifnum\printfancyheader=1
\maketitle\thispagestyle{empty}
	\else	
		\ifnum\printsol=0

				\begin{center}{\large Math \classnum--\classsec, UCD, \classterm, Homework \assessmentname}\\
\smallskip
Due by \duedate\, at \duetime.\\
\smallskip
{\em Late homeworks will not be accepted without prior approval. Please no email submissions!}
				\end{center}
			\else

				\begin{center}{Math \classnum--\classsec, UCD, \classterm \hfill Homework \assessmentnamesol}
				\end{center}
\vskip 2mm

		\fi
		
\hrule
\vskip 5mm

\fi



%%%%%%%%%%%%%%%%%%%%%%%%%%%%%%%%%%%%%%%%%%%%%%%%%%%%%%%%%%%%%%%%%
%%%%%%%%%%%%%%%%%%%%%%%%%%%%%%%%%%%%%%%%%%%%%%%%%%%%%%%%%%%%%%%%%


%\normalsize
%
%\ifnum\printsol=0
%
%%%%%%%%%%%%%%%%%%%%%%%%%%%%%%%%%%%%%%%%%%%%%%%%%%%%%%%
%\hrule\bigskip\noindent\textbf{Name:}\hspace*{3in}
%\textbf{Student Number:}\bigskip\hrule
%%%%%%%%%%%%%%%%%%%%%%%%%%%%%%%%%%%%%%%%%%%%%%%%%%%%%%%
%
%%%%%%%%%%%%%%%%%%%%%%%%%%%%%%%%%%%%%%%%%%%%%%%%%%%%%%%
%\vspace*{5mm}
%\begin{itemize}\itemsep=0in
%	\item This is an open note \assessmenttype. You may use your notes and your book on this \assessmenttype.
%%	\item Write all answers in the spaces provided on the \assessmenttype.
%	\item Throughout the \assessmenttype, show your work so that your reasoning is clear. Otherwise no credit will be given.
%	\item If you are asked to \emph{find} or \emph{write} a formula or an expression without proof, you do not need to show any work but must present your solution using correct mathematical notation.
%	\item If you are asked to \emph{give} or \emph{cite} a definition or a result that we discussed in class, you must use complete sentences, correct mathematical notation, and state all conditions or assumptions.
%	\item If you are asked to \emph{prove} or \emph{show} a formula or a result, you must give a proof using complete sentences and correct mathematical notation. If you can correctly cite a result that we have proved in class or that is proven in the book, you can use it and do not have to prove it again. 
%	\item The following table gives the total points for each problem on this \assessmenttype.
%
%\begin{center}
%{\renewcommand{\arraystretch}{1.5}
%\renewcommand{\tabcolsep}{0.2cm}
%\begin{tabular}{||c||c||c||c||c||c||}
%\hline
%\rule{0pt}{3ex}Problem & Points & Score/10 & Problem & Points & Score/10 \\ \hline \hline
%  &  & \hspace{1.0in}  &  & 10 & \hspace{1.0in}\\ \hline
%  &  & \hspace{1.0in}  &  & 10 & \hspace{1.0in}\\ \hline
%  &  & \hspace{1.0in}  &  & 10 & \hspace{1.0in}\\ \hline
%  &  & \hspace{1.0in}  &  & 10 & \hspace{1.0in}\\ \hline
%  &  & \hspace{1.0in}  &  & 10 & \hspace{1.0in}\\ \hline
%  &  & \hspace{1.0in}  &  & 10 & \hspace{1.0in}\\ \hline
%\hline
%\rule{0pt}{3ex}Total &  & \hspace{1.0in} & &  & \\ \hline
%\end{tabular}}
%\end{center}
%
%\vspace*{5mm}
%
%\end{itemize}
%\vspace*{5mm}
%
%\begin{flushright}
%Total Score:\;\underline{\hspace*{2cm}}\;/\;10
%
%\end{flushright}
%
%\vfill\eject
%
%\fi


%%%%%%%%%%%%%%%%%%%%%%%%%%%%%%%%%%%%%%%%%%%%%%%%%%%%%%%%%%%%%%%%%
%%%%%%%%%%%%%%%%%%%%%%%%%%%%%%%%%%%%%%%%%%%%%%%%%%%%%%%%%%%%%%%%%


\renewcommand{\S}{\mathbb{S}} % For defining a set S 
\renewcommand{\P}{\mathbb{P}} % For defining a probabilty measure
\newcommand{\prob}[1]{\mathbb{P}\left(#1\right)} % For defining a probability measure with argument and normal parentheses
\newcommand{\Prob}[1]{\mathbb{P}\Big(#1\Big)} % For defining a probability measure with argument and large parenthesis


%%%%%%%%%%%%%%%%%%%%%%%%%%%%%%%%%%%%%%%%%%%%%%%%%%%%%%%%%%%%%%%%%
%%%%%%%%%%%%%%%%%%%%%%%%%%%%%%%%%%%%%%%%%%%%%%%%%%%%%%%%%%%%%%%%%
%\ifnum\printsol=1
\vspace*{2mm}
\hrule
\vskip 8mm

%\fi


%%%%%%%%%%%%%%%%%%%%%%%%%%%%%%%%%%%%%%%%%%%%%%%%%%%%%%%%%%%%%%%%%%%%
%%%%%%%%%%%%%%%%%%%%%%%%%%%%%%%%%%%%%%%%%%%%%%%%%%%%%%%%%%%%%%%%%%%%



\section*{Part A: Weak Law of Large Numbers}

\paragraph{1. Basic WLLN Verification.}
Let $X_1, X_2, \dots$ be i.i.d.\ with
\[\P(X_i = 2)=\tfrac12,\qquad \P(X_i=-1)=\tfrac12.\]
Define $S_n = X_1+\cdots+X_n$ and $\bar X_n = S_n/n$.
\begin{itemize}
    \item[(a)] Compute $E[X_1]$ and $\operatorname{Var}(X_1)$.
    \item[(b)] Use Chebyshev’s inequality to show that $\bar X_n \to E[X_1]$ in probability.
\end{itemize}

\paragraph{2. WLLN for Random Variables with Increasing Variance.}
Let $X_n$ be independent with
\[E[X_n]=1, \qquad \operatorname{Var}(X_n)=\frac{1}{n}.\]
Does the LLN hold for $\bar X_n = \frac1n\sum_{k=1}^n X_k$?  

\paragraph{3. Sample Proportion Convergence.}
Let $X_1,\dots,X_n,\dots$ be i.i.d.\ Bernoulli($p$).  
Use the WLLN to show that the sample proportion
\[\hat p_n = \frac{1}{n}\sum_{i=1}^n X_i\]
converges in probability to $p$.  
Then compute $n$ such that
\[\P\big(| \hat p_n - p| > 0.05\big) < 0.01.\]

\paragraph{4. WLLN When Moments Do Not Exist.}
Let $X_1,\dots,X_n,\dots$ be i.i.d.\ with density
\[f(x)=\frac{1}{x^2}, \qquad x\ge 1.\]
Does the Weak Law hold for $\bar X_n$? Explain.
\ifnum\printsol=0
\vfill\eject
\fi

\paragraph{5. WLLN for Non-Identically Distributed Variables.}
Suppose $X_k$ are independent with
\[E[X_k] = 0,\qquad \operatorname{Var}(X_k)=\frac{1}{k}.\]
Consider $Y_n = \frac{1}{n} \sum_{k=1}^n X_k$.  
Does $Y_n \to 0$ in probability?

\section*{Part B: Central Limit Theorem}

\paragraph{6. CLT for Bernoulli Variables.}
Let $X_i\sim \text{Bernoulli}(p)$. Derive the CLT approximation for
\[\P\left(\frac{\hat p_n - p}{\sqrt{p(1-p)/n}} \le z\right).\]
Approximate
\[\P(|\hat p_n - p|<0.02)\]
when $p=0.3$ and $n=400$.

\paragraph{7. CLT for Poisson Variables.}
Let $X_i\sim \mathrm{Poisson}(\lambda)$.  
Approximate
\[\P(S_{200} \le 230),\]
where $S_{200}=\sum_{i=1}^{200} X_i$, using the CLT with continuity correction.

\paragraph{8. CLT Approximation Error.}
If $X_i\sim N(\mu, \sigma^2)$, redo the CLT derivation and identify the limiting distribution of $S_n$. Explain why the CLT is unnecessary here.

\paragraph{9. CLT for Uniform Distribution.}
Let $X_i\sim \mathrm{Uniform}(0,1)$. Approximate
\[\P(S_{50} > 30),\]
where $S_{50}=X_1+\cdots+X_{50}$.

\paragraph{10. Lindeberg Condition Check.}
Let $X_k$ be independent with
\[\P(X_k = k) = \frac{1}{2k},\qquad\P(X_k = -k)=\frac{1}{2k},\qquad\P(X_k = 0)=1-\frac{1}{k}.\]
\begin{itemize}
    \item[(a)] Compute $E[X_k]$ and $\operatorname{Var}(X_k)$.
    \item[(b)] Let $S_n=\sum_{k=1}^n X_k$. Does the CLT apply? Check whether the Lindeberg condition holds.
\end{itemize}

\paragraph{11. CLT + Delta Method.}
Let $X_i\sim \text{Exponential}(1)$. Approximate the distribution of
\[\sqrt{n}\big(\log(\bar X_n)-\log(1)\big).\]

\paragraph{12. CLT for Non-Rectangular Domains.}
Let $(X_i,Y_i)$ be i.i.d. points uniformly distributed on the unit disk. Consider
\[R_n = \frac{1}{n}\sum_{i=1}^n \sqrt{X_i^2 + Y_i^2}.\]
\begin{itemize}
    \item[(a)] Compute $\mu = E[\sqrt{X_1^2 + Y_1^2}]$.
    \item[(b)] Using the CLT, approximate the distribution of $\sqrt{n}(R_n - \mu)$.
\end{itemize}


\ifnum\printsol=0
\end{document}

\fi

%%%%%%%%%%%%%%%%%%%%%%%%%%%%%%%%%%%%%%%%%%%%%%%%%%%%%%%%%%%%%%%%%%%%%%%%%%%%%%%%%%%%%%%%%%%%%%%%%%%%%%%
%%%%%%%%%%%%%%%%%%%%%%%%%%%%%%%%%%%%%%%%%%%%%%%%%%%%%%%%%%%%%%%%%%%%%%%%%%%%%%%%%%%%%%%%%%%%%%%%%%%%%%%
%%%%%%%%%%%%%%%%%%%%%%%%%%%%%%%%%%%%%%%%%%%%%%%%%%%%%%%%%%%%%%%%%%%%%%%%%%%%%%%%%%%%%%%%%%%%%%%%%%%%%%%
%%%%%%%%%%%%%%%%%%%%%%%%%%%%%%%%%%%%%%%%%%%%%%%%%%%%%%%%%%%%%%%%%%%%%%%%%%%%%%%%%%%%%%%%%%%%%%%%%%%%%%%

\newpage
\newpage
\section*{Solutions}

\paragraph{1. Basic WLLN Verification.}

\emph{(a)}  
The expectation is computed directly from the definition:
\[
E[X_1]
= \sum_x x \, \P(X_1=x)
= 2\cdot \tfrac12 + (-1)\cdot \tfrac12
= \tfrac12.
\]
Next compute the second moment:
\[
E[X_1^2]
= 2^2\cdot \tfrac12 + (-1)^2\cdot \tfrac12
= \tfrac12(4) + \tfrac12(1)
= \tfrac52.
\]
Hence
\[
\operatorname{Var}(X_1)
= E[X_1^2] - (E[X_1])^2
= \tfrac52 - \left(\tfrac12\right)^2
= \frac{9}{4}.
\]

\emph{(b)}  
Since the $X_i$ are i.i.d.,
\[
E[\bar X_n] = E[X_1] = \tfrac12,
\qquad
\operatorname{Var}(\bar X_n) = \frac{\operatorname{Var}(X_1)}{n}
= \frac{9}{4n}.
\]
By Chebyshev’s inequality,
\[
\P\!\left(|\bar X_n - \tfrac12| > \varepsilon\right)
\le \frac{\operatorname{Var}(\bar X_n)}{\varepsilon^2}
= \frac{9}{4n\varepsilon^2}.
\]
As $n\to\infty$, this bound converges to $0$, so
\[
\bar X_n \xrightarrow{P} \tfrac12.
\]

---

\paragraph{2. WLLN for Random Variables with Increasing Variance.}

First compute the mean:
\[
E[\bar X_n]
= \frac1n \sum_{k=1}^n E[X_k]
= 1.
\]
Using independence,
\[
\operatorname{Var}(\bar X_n)
= \frac{1}{n^2} \sum_{k=1}^n \operatorname{Var}(X_k)
= \frac{1}{n^2} \sum_{k=1}^n \frac{1}{k}.
\]
Since $\sum_{k=1}^n \frac1k \sim \log n$, we obtain
\[
\operatorname{Var}(\bar X_n)
\sim \frac{\log n}{n^2} \to 0.
\]
Chebyshev’s inequality then implies
\[
\bar X_n \xrightarrow{P} 1.
\]
Thus the Weak Law holds even though the individual variances do not vanish.

---

\paragraph{3. Sample Proportion Convergence.}

For Bernoulli$(p)$ variables,
\[
E[X_i]=p, \qquad \operatorname{Var}(X_i)=p(1-p).
\]
Hence
\[
E[\hat p_n]=p,
\qquad
\operatorname{Var}(\hat p_n)=\frac{p(1-p)}{n}.
\]
By the WLLN,
\[
\hat p_n \xrightarrow{P} p.
\]

For the probability bound, Chebyshev’s inequality gives
\[
\P(|\hat p_n - p|>0.05)
\le \frac{p(1-p)}{n(0.05)^2}.
\]
To make this less than $0.01$, it suffices that
\[
\frac{p(1-p)}{n(0.05)^2} < 0.01
\quad\Longrightarrow\quad
n > \frac{p(1-p)}{0.000025}.
\]
This bound is conservative but does not rely on any normal approximation.

---

\paragraph{4. WLLN When Moments Do Not Exist.}

We compute
\[
E[X_1]
= \int_1^\infty x \frac{1}{x^2}\,dx
= \int_1^\infty \frac{1}{x}\,dx
= \infty.
\]
Since the expectation does not exist, the classical WLLN does not apply.
In fact, this heavy-tailed distribution allows rare but extremely large observations that dominate the sample average, preventing stabilization of $\bar X_n$.

---

\paragraph{5. WLLN for Non-Identically Distributed Variables.}

We have
\[
E[Y_n] = 0,
\qquad
\operatorname{Var}(Y_n)
= \frac{1}{n^2}\sum_{k=1}^n \frac{1}{k}
\sim \frac{\log n}{n^2} \to 0.
\]
Applying Chebyshev’s inequality,
\[
\P(|Y_n|>\varepsilon)
\le \frac{\operatorname{Var}(Y_n)}{\varepsilon^2} \to 0.
\]
Thus
\[
Y_n \xrightarrow{P} 0.
\]

---

\paragraph{6. CLT for Bernoulli Variables.}

By the CLT,
\[
\frac{\hat p_n - p}{\sqrt{p(1-p)/n}}
\Rightarrow N(0,1).
\]
For $p=0.3$ and $n=400$,
\[
\sqrt{\frac{p(1-p)}{n}}
= \sqrt{\frac{0.21}{400}}
\approx 0.0229.
\]
Hence
\[
\P(|\hat p_n - p|<0.02)
\approx \P\!\left(|Z|<\frac{0.02}{0.0229}\right)
= \P(|Z|<1.746)
\approx 0.919.
\]

---

\paragraph{7. CLT for Poisson Variables.}

If $X_i\sim\text{Poisson}(1)$, then
\[
S_{200} \sim \text{Poisson}(200),
\qquad
E[S_{200}]=200,
\quad
\operatorname{Var}(S_{200})=200.
\]
Using the normal approximation with continuity correction,
\[
\P(S_{200}\le 230)
\approx \P\!\left(
Z \le \frac{230.5 - 200}{\sqrt{200}}
\right)
= \P(Z \le 2.162)
\approx 0.985.
\]

---

\paragraph{8. CLT Approximation Error.}

If $X_i \sim N(\mu,\sigma^2)$, then the sum
\[
S_n = \sum_{i=1}^n X_i
\sim N(n\mu, n\sigma^2)
\]
\emph{exactly}, for every $n$.
Thus the limiting distribution obtained from the CLT coincides with the exact finite-sample distribution.
In this case the CLT provides no approximation—it is already exact.

---

\paragraph{9. CLT for Uniform Distribution.}

For $X_i \sim U(0,1)$,
\[
E[X_i]=\tfrac12,
\qquad
\operatorname{Var}(X_i)=\tfrac{1}{12}.
\]
Therefore
\[
E[S_{50}]=25,
\qquad
\operatorname{Var}(S_{50})=\frac{50}{12}.
\]
By the CLT,
\[
\P(S_{50}>30)
\approx \P\!\left(
Z > \frac{30-25}{\sqrt{50/12}}
\right)
= \P(Z>2.45)
\approx 0.0071.
\]

---

\paragraph{10. Lindeberg Condition Check.}

\emph{(a)}  
By symmetry,
\[
E[X_k]=0.
\]
The variance is
\[
\operatorname{Var}(X_k)
= E[X_k^2]
= k^2\!\left(\frac{1}{2k} + \frac{1}{2k}\right)
= k.
\]

\emph{(b)}  
Let $S_n=\sum_{k=1}^n X_k$. Then
\[
\operatorname{Var}(S_n)
= \sum_{k=1}^n \operatorname{Var}(X_k)
= \sum_{k=1}^n k
\sim \frac{n^2}{2}.
\]
Thus the natural scaling for a CLT would be $S_n / n$.

However, with probability $1/k$, the variable $X_k$ takes values of size $k$.
These jumps are \emph{not negligible} relative to the standard deviation of $S_n$, which is of order $n$.
Consequently, for any $\varepsilon>0$,
\[
\E\!\left[
X_k^2 \mathbf{1}_{\{|X_k|>\varepsilon n\}}
\right]
\approx k^2 \cdot \frac{1}{k}
= k,
\]
so the Lindeberg condition fails.
Hence the CLT does \emph{not} apply.

---


\paragraph{11. CLT + Delta Method.}

For $X_i\sim\text{Exponential}(1)$,
\[
E[X_i]=1,
\qquad
\operatorname{Var}(X_i)=1.
\]
By the CLT,
\[
\sqrt{n}(\bar X_n - 1) \Rightarrow N(0,1).
\]
Let $g(x)=\log x$, which is differentiable at $x=1$ with $g'(1)=1$.
By the delta method,
\[
\sqrt{n}(\log\bar X_n - \log 1)
\Rightarrow N\!\left(0,(g'(1))^2\right)
= N(0,1).
\]
Thus the asymptotic distribution is standard normal.

---

\paragraph{12. CLT for Non-Rectangular Domains.}

\emph{(a)}  
Let $R=\sqrt{X^2+Y^2}$. For a point uniformly distributed in the unit disk,
the radial density is
\[
f_R(r)=2r, \qquad 0\le r\le 1.
\]
Hence
\[
E[R]
= \int_0^1 r(2r)\,dr
= \frac{2}{3}.
\]

\emph{(b)}  
We compute
\[
E[R^2]
= \int_0^1 r^2(2r)\,dr
= \frac12,
\]
so
\[
\operatorname{Var}(R)
= \frac12 - \left(\frac23\right)^2
= \frac{1}{18}.
\]
Since the $R_i$ are i.i.d., the CLT yields
\[
\sqrt{n}(R_n - \tfrac23)
\Rightarrow N\!\left(0,\frac{1}{18}\right).
\]


\end{document}

%%%%%%%%%%%%%%%%%%%%%%%%%%%%%%%%%%%%%%%%%%%%%%%%%%%%%%%%%%%%%%%%%%%%%%%%%%%%%%%%%%%%%%%%%%%%%%%%%%%%%%%
%%%%%%%%%%%%%%%%%%%%%%%%%%%%%%%%%%%%%%%%%%%%%%%%%%%%%%%%%%%%%%%%%%%%%%%%%%%%%%%%%%%%%%%%%%%%%%%%%%%%%%%
%%%%%%%%%%%%%%%%%%%%%%%%%%%%%%%%%%%%%%%%%%%%%%%%%%%%%%%%%%%%%%%%%%%%%%%%%%%%%%%%%%%%%%%%%%%%%%%%%%%%%%%
%%%%%%%%%%%%%%%%%%%%%%%%%%%%%%%%%%%%%%%%%%%%%%%%%%%%%%%%%%%%%%%%%%%%%%%%%%%%%%%%%%%%%%%%%%%%%%%%%%%%%%%


\paragraph*{Detailed Explanation of the Lindeberg Condition.}

Recall that for a triangular array of independent random variables
$\{X_k\}$ with partial sums
\[
S_n = \sum_{k=1}^n X_k,
\qquad
s_n^2 = \operatorname{Var}(S_n),
\]
the \emph{Lindeberg condition} states that for every $\varepsilon>0$,
\[
\frac{1}{s_n^2}
\sum_{k=1}^n
\E\!\left[
X_k^2 \, \mathbf{1}_{\{|X_k|>\varepsilon s_n\}}
\right]
\;\longrightarrow\; 0.
\]
This condition formalizes the requirement that no single summand
contributes a non-negligible fraction of the total variance through
rare but large deviations.

\vspace{0.5em}

In the present example,
\[
\P(X_k = k)=\frac{1}{2k},
\qquad
\P(X_k = -k)=\frac{1}{2k},
\qquad
\P(X_k=0)=1-\frac{1}{k}.
\]
By symmetry, $E[X_k]=0$, and the variance is
\[
\operatorname{Var}(X_k)
= E[X_k^2]
= k^2\!\left(\frac{1}{2k}+\frac{1}{2k}\right)
= k.
\]
Therefore, the variance of the sum is
\[
s_n^2
= \operatorname{Var}(S_n)
= \sum_{k=1}^n k
= \frac{n(n+1)}{2}
\sim \frac{n^2}{2},
\]
so the natural scale of fluctuations is
\[
s_n \sim \frac{n}{\sqrt{2}}.
\]

\vspace{0.5em}

We now examine the Lindeberg condition.
The indicator $\mathbf{1}_{\{|X_k|>\varepsilon s_n\}}$ is nonzero only
when $|X_k|=k$ and
\[
k > \varepsilon s_n \sim \varepsilon \frac{n}{\sqrt{2}}.
\]
Thus only indices $k$ of order $n$ contribute to the sum.

For such $k$,
\[
\E\!\left[
X_k^2 \, \mathbf{1}_{\{|X_k|>\varepsilon s_n\}}
\right]
= k^2 \cdot \P(|X_k|=k)
= k^2 \cdot \frac{1}{k}
= k.
\]
Summing over all indices with $k>\varepsilon n/\sqrt{2}$ gives
\[
\sum_{k>\varepsilon n/\sqrt{2}} k
\;\approx\;
\int_{\varepsilon n/\sqrt{2}}^n x\,dx
\;\sim\; C n^2
\quad\text{for some } C>0.
\]

Dividing by $s_n^2 \sim n^2/2$, we obtain
\[
\frac{1}{s_n^2}
\sum_{k=1}^n
\E\!\left[
X_k^2 \, \mathbf{1}_{\{|X_k|>\varepsilon s_n\}}
\right]
\;\longrightarrow\;
\text{a positive constant}.
\]
Since this limit is not zero, the Lindeberg condition fails.

\vspace{0.5em}

\emph{Interpretation.}
Although each $X_k$ is usually zero, it occasionally takes values of
size $k$ with probability $1/k$.
For large $k$, these jumps are of the same order as the standard
deviation of the entire sum $S_n$.
Because such jumps occur often enough to contribute a fixed proportion
of the total variance, the averaging mechanism behind the Central Limit
Theorem breaks down.
Consequently, the CLT does not apply to this sequence.

\end{document}