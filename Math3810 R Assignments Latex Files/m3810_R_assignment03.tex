\documentclass[11pt]{article}
\usepackage[suffix=Solutions]{teaching-header}
\usepackage{listings,tikz,enumitem,booktabs}

\lstdefinelanguage{R}{
      keywords={if,else,while,for,in,next,break,function,TRUE,FALSE,NULL,Inf,NA,NaN,switch,repeat,return,require,library},
      keywordstyle=\color{blue}\bfseries,
      identifierstyle=\color{black},
      comment=[l]{\#},
      commentstyle=\color{gray}\ttfamily,
      string=[b]{"},
      stringstyle=\color{red}\ttfamily,
      morecomment=[l]{//},
      morestring=[b]{'},
      sensitive=true,
      morekeywords={print,summary,plot,lm,glm,data,frame,read.csv,write.csv,factor,levels,names,colnames,rownames,
      head,tail,str,dim,length,class,typeof,mode,is.na,is.null,is.finite,is.infinite,is.nan,as.numeric,as.character,
      as.factor,as.Date,as.POSIXct,as.matrix,as.data.frame,rbind,cbind,merge,subset,aggregate,tapply,apply,lapply,sapply,
      mapply,vapply,replicate,seq,rep,c,list,matrix,array,data.frame,table,hist,boxplot,barplot,pie,curve,lines,points,text,
      abline,legend,par,mtext,title,xlab,ylab,xlim,ylim,main,sub,col,pch,cex,lty,lwd,type,bg,fg,args,options,warnings,errors,
      message,stop,warning,error,try,tryCatch,withCallingHandlers,on.exit,debug,browser,trace,recover,options,getOption,setOption},
    }

\def\assessmenttype{introductory homework}
\def\assessmenttypecap{Introductory Homework}
\def\assessmentname{\#3}  
\def\assessmentnamesol{\#3 Solutions}
\def\duedate{Wednesday, October 2}
\def\duetime{3:30pm}
\def\printfancyheader{1} 

\def\classnum{3810}
\def\classtitle{Probability}
\def\classtitleshort{Probability}
\def\classsec{001}
\def\classterm{Fall 2025}
\def\instructor{Robert Rostermundt}
%\def\hmwknum{\#2}

\ifnum\printfancyheader=1
\pagestyle{myheadings}
\else
    \ifnum\printsol=0
        \pagestyle{empty}
    \else
        \pagestyle{plain}
    \fi
\fi

\ifnum\printsol=0
\title{\vspace{-1in}Math\classnum\;-\;\classtitle\\
    Section\;\classsec\;-\;\classterm\\
    \assessmenttypecap\;\assessmentname}
\author{University of Colorado Denver / College of Liberal Arts and Sciences}
\date{Department of Mathematics - \instructor}
\markright{Math\classnum\;-\;\classtitle,\;\assessmenttypecap\;\assessmentname, UCD, \classterm, \instructor}
\else
\title{\vspace{-1in}Math\classnum\;-\;\classtitle\\
    Section\;\classsec\;-\;\classterm\\
    \assessmenttypecap\;\assessmentnamesol}
\author{University of Colorado Denver / College of Liberal Arts and Sciences}
\date{Department of Mathematics - \instructor}
\markright{Math\classnum\;-\;\classtitle,\;\assessmenttypecap\;\assessmentname, UCD, \classterm, \instructor}
\fi

\begin{document}
\ifnum\printfancyheader=1
\maketitle\thispagestyle{empty}
\else
    \ifnum\printsol=0
        \begin{center}{\large Math \classnum--\classsec, UCD, \classterm, Homework \assessmentname}\\
        \smallskip
        Due by \duedate\, at \duetime.\\
        \smallskip
        {\em Late homeworks will not be accepted without prior approval. Please no email submissions!}
        \end{center}
    \else
        \begin{center}{Math \classnum--\classsec, UCD, \classterm \hfill Homework \assessmentnamesol}
        \end{center}
        \vskip 2mm
    \fi
\hrule
\vskip 5mm
\fi

\normalsize

\ifnum\printsol=0
\hrule\bigskip\noindent\textbf{Name:}\hspace*{3in}
\textbf{Student Number:}\bigskip\hrule
\vspace*{5mm}
\fi

\renewcommand{\S}{\mathbb{S}}
\renewcommand{\P}{\mathbb{P}}
\newcommand{\prob}[1]{\mathbb{P}\left(#1\right)}
\newcommand{\Prob}[1]{\mathbb{P}\Big(#1\Big)}

\section*{Instructions}
Show all reasoning clearly. All simulation results should be reproducible and clearly labeled.
You may use R for all computations.

\section*{Problems}
\begin{enumerate}[leftmargin=*]

\item \textbf{Binomial Simulation}
\begin{enumerate}
\item Simulate 100 Bernoulli trials with $p=0.3$.
\item Compute the empirical mean and variance.
\item Compare to theoretical values.
\end{enumerate}

\item \textbf{Multiple Experiments}
\begin{enumerate}
\item Repeat 3 times with 100 trials each.
\item Record empirical proportions each time.
\item Comment on variability.
\end{enumerate}

\item \textbf{Poisson Approximation}
\begin{enumerate}
\item Simulate 1000 Poisson($\lambda=2$) observations.
\item Plot histogram and overlay theoretical PMF.
\item Compute sample mean and variance.
\end{enumerate}

\item \textbf{Law of Large Numbers}
\begin{enumerate}
\item Simulate sums of $n$ i.i.d. Bernoulli($p=0.5$) for $n=10,100,1000,10000$.
\item Plot the running proportion of successes.
\item Comment on convergence to true probability.
\end{enumerate}

\item \textbf{Discussion}
\begin{itemize}
\item Explain why empirical frequencies converge as $n$ increases.
\item How does randomness affect small samples?
\end{itemize}

\end{enumerate}

\ifnum\printsol=1
\section*{Solutions}
\begin{enumerate}[leftmargin=*]

\item
\begin{lstlisting}[language=R]
x <- rbinom(100, size=1, prob=0.3)
mean(x)
var(x)
\end{lstlisting}

\item
\begin{lstlisting}[language=R]
for(i in 1:3){
  x <- rbinom(100,1,0.3)
  print(mean(x))
}
\end{lstlisting}

\item
\begin{lstlisting}[language=R]
y <- rpois(1000, lambda=2)
hist(y, prob=TRUE)
points(0:10, dpois(0:10, 2), col="red", pch=19)
mean(y)
var(y)
\end{lstlisting}

\item
\begin{lstlisting}[language=R]
nvec <- c(10,100,1000,10000)
for(n in nvec){
  x <- rbinom(n,1,0.5)
  plot(cumsum(x)/seq(1,n), type="l", main=paste("n =", n))
}
\end{lstlisting}

\item
Empirical frequencies converge due to the Law of Large Numbers.  
Small samples show more variability because randomness has larger influence.

\end{enumerate}

\vskip 5mm
\hrule
\vskip 5mm
\begin{center}
\bf Please let me know if you have any questions, comments, or corrections!
\end{center}
\fi

\end{document}
