\documentclass[11pt]{article}
\usepackage[suffix=Solutions]{teaching-header}
\usepackage{listings,tikz,enumitem,booktabs}

\lstdefinelanguage{R}{
      keywords={if,else,while,for,in,next,break,function,TRUE,FALSE,NULL,Inf,NA,NaN,switch,repeat,return,require,library},
      keywordstyle=\color{blue}\bfseries,
      identifierstyle=\color{black},
      comment=[l]{\#},
      commentstyle=\color{gray}\ttfamily,
      string=[b]{"},
      stringstyle=\color{red}\ttfamily,
      morecomment=[l]{//},
      morestring=[b]{'},
      sensitive=true,
      morekeywords={print,summary,plot,lm,glm,data,frame,read.csv,write.csv,factor,levels,names,colnames,rownames,
      head,tail,str,dim,length,class,typeof,mode,is.na,is.null,is.finite,is.infinite,is.nan,as.numeric,as.character,
      as.factor,as.Date,as.POSIXct,as.matrix,as.data.frame,rbind,cbind,merge,subset,aggregate,tapply,apply,lapply,sapply,
      mapply,vapply,replicate,seq,rep,c,list,matrix,array,data.frame,table,hist,boxplot,barplot,pie,curve,lines,points,text,
      abline,legend,par,mtext,title,xlab,ylab,xlim,ylim,main,sub,col,pch,cex,lty,lwd,type,bg,fg,args,options,warnings,errors,
      message,stop,warning,error,try,tryCatch,withCallingHandlers,on.exit,debug,browser,trace,recover,options,getOption,setOption},
    }

%%%%%%%%%%%%%%%%%%%%%%%%%%%%%%%%%%%%%%%%%%%%%%%%%%%%%%%%%%%%%%%%%%%%%%%%%%%%
%%%%%%%%%%%%%%%%%%%%%%%%%%%%%%%%%%%%%%%%%%%%%%%%%%%%%%%%%%%%%%%%%%%%%%%%%%%%


%\ifnum\printsol=0 (when no solutions printed)
%Do something
%	\else  (when solutions are printed)
%Do something else
%\fi


% Assessment specific definitions to fill in title and page headers. Other course specific definitions are located in classinfo document which is accessed by \input command below. 
\def\assessmenttype{introductory homework}
\def\assessmenttypecap{Introductory Homework}
\def\assessmentname{\#1}  % Such as #1 or a word title
\def\assessmentnamesol{\#1 Solutions}
\def\duedate{Wednesday, September 18}
\def\duetime{3:30pm}
\def\printfancyheader{1} % Beware that changing the header will change custom pagebreaks for boxed solutions

\def\classnum{3810}
\def\classtitle{Probability}
\def\classtitleshort{Probability}
\def\classsec{001}
\def\classterm{Fall 2025}
\def\instructor{Robert Rostermundt}
%\def\hmwknum{\#2}  % Class Specific Information for title header

%If use fancy title header then include custom header on all pages. If don't use fancy title header then include pagenumbers only when solutions printed
\ifnum\printfancyheader=1
\pagestyle{myheadings}
	\else
		\ifnum\printsol=0
				\pagestyle{empty}
			\else
				\pagestyle{plain}
		\fi		
\fi


% Creates a fancy title header for front page. This is turned on by boolean \printfancyheader which later invokes \maketitle command. Otherwise a simpler header is displayed.
\ifnum\printsol=0 % Determine if non-solution fancyheader or solution fancyheader is displayed
\title{\vspace{-1in}Math\classnum\;-\;\classtitle\\
	Section\;\classsec\;-\;\classterm\\
	\assessmenttypecap\;\assessmentname}
	\author{University of Colorado Denver / College of Liberal Arts 	and Sciences}
	\date{Department of Mathematics - \instructor}

	\markright{Math\classnum\;-\;\classtitleshort,\;\assessmenttypecap\;\assessmentname, UCD, \classterm, \instructor}
	
		\else

\title{\vspace{-1in}Math\classnum\;-\;\classtitle\\
	Section\;\classsec\;-\;\classterm\\
	\assessmenttypecap\;\assessmentnamesol}
	\author{University of Colorado Denver / College of Liberal Arts 	and Sciences}
	\date{Department of Mathematics - \instructor}

	\markright{Math\classnum\;-\;\classtitleshort,\;\assessmenttypecap\;\assessmentname, UCD, \classterm, \instructor}

\fi


%%%%%%%%%%%%%%%%%%%%%%%%%%%%%%%%%%%%%%%%%%%%%%%%%%%%%%%%%%%%%%%%%%%%%%%%%%%%
%%%%%%%%%%%%%%%%%%%%%%%%%%%%%%%%%%%%%%%%%%%%%%%%%%%%%%%%%%%%%%%%%%%%%%%%%%%%
\begin{document}
%%%%%%%%%%%%%%%%%%%%%%%%%%%%%%%%%%%%%%%%%%%%%%%%%%%%%%%%%%%%%%%%%%%%%%%%%%%%
%%%%%%%%%%%%%%%%%%%%%%%%%%%%%%%%%%%%%%%%%%%%%%%%%%%%%%%%%%%%%%%%%%%%%%%%%%%%
\ifnum\printfancyheader=1
\maketitle\thispagestyle{empty}
	\else	
		\ifnum\printsol=0

				\begin{center}{\large Math \classnum--\classsec, UCD, \classterm, Homework \assessmentname}\\
\smallskip
Due by \duedate\, at \duetime.\\
\smallskip
{\em Late homeworks will not be accepted without prior approval. Please no email submissions!}
				\end{center}
			\else

				\begin{center}{Math \classnum--\classsec, UCD, \classterm \hfill Homework \assessmentnamesol}
				\end{center}
\vskip 2mm

		\fi
		
\hrule
\vskip 5mm

\fi



%%%%%%%%%%%%%%%%%%%%%%%%%%%%%%%%%%%%%%%%%%%%%%%%%%%%%%%%%%%%%%%%%
%%%%%%%%%%%%%%%%%%%%%%%%%%%%%%%%%%%%%%%%%%%%%%%%%%%%%%%%%%%%%%%%%


\normalsize

\ifnum\printsol=0

%%%%%%%%%%%%%%%%%%%%%%%%%%%%%%%%%%%%%%%%%%%%%%%%%%%%%%
\hrule\bigskip\noindent\textbf{Name:}\hspace*{3in}
\textbf{Student Number:}\bigskip\hrule
%%%%%%%%%%%%%%%%%%%%%%%%%%%%%%%%%%%%%%%%%%%%%%%%%%%%%%

%%%%%%%%%%%%%%%%%%%%%%%%%%%%%%%%%%%%%%%%%%%%%%%%%%%%%%
\vspace*{5mm}
\begin{itemize}\itemsep=0in
	\item This is an open note \assessmenttype. You may use your notes and your book on this \assessmenttype.
%	\item Write all answers in the spaces provided on the \assessmenttype.
	\item Throughout the \assessmenttype, show your work so that your reasoning is clear. Otherwise no credit will be given.
	\item If you are asked to \emph{find} or \emph{write} a formula or an expression without proof, you do not need to show any work but must present your solution using correct mathematical notation.
	\item If you are asked to \emph{give} or \emph{cite} a definition or a result that we discussed in class, you must use complete sentences, correct mathematical notation, and state all conditions or assumptions.
	\item If you are asked to \emph{prove} or \emph{show} a formula or a result, you must give a proof using complete sentences and correct mathematical notation. If you can correctly cite a result that we have proved in class or that is proven in the book, you can use it and do not have to prove it again. 
	\item The following table gives the total points for each problem on this \assessmenttype.

\begin{center}
{\renewcommand{\arraystretch}{1.5}
\renewcommand{\tabcolsep}{0.2cm}
\begin{tabular}{||c||c||c||c||c||c||}
\hline
\rule{0pt}{3ex}Problem & Points & Score/10 & Problem & Points & Score/10 \\ \hline \hline
1  & 12 & \hspace{1.0in}  & 6 & 8  & \hspace{1.0in}\\ \hline
2  & 12 & \hspace{1.0in}  & 7 & 8  & \hspace{1.0in}\\ \hline
3  & 8  & \hspace{1.0in}  & 8 & 6  & \hspace{1.0in}\\ \hline
4  & 6  & \hspace{1.0in}  & 9 & 6  & \hspace{1.0in}\\ \hline
5  & 17 & \hspace{1.0in}  & * & *  & \hspace{1.0in}\\ \hline\hline
\rule{0pt}{3ex}Total & 55 & \hspace{1.0in} & & 28 & \\ \hline
\end{tabular}}
\end{center}

\vspace*{5mm}

\end{itemize}
\vspace*{5mm}

\begin{flushright}
Total Score:\;\underline{\hspace*{2cm}}\;/\;10

\end{flushright}

\vfill\eject

\fi


%%%%%%%%%%%%%%%%%%%%%%%%%%%%%%%%%%%%%%%%%%%%%%%%%%%%%%%%%%%%%%%%%
%%%%%%%%%%%%%%%%%%%%%%%%%%%%%%%%%%%%%%%%%%%%%%%%%%%%%%%%%%%%%%%%%


\renewcommand{\S}{\mathbb{S}} % For defining a set S 
\renewcommand{\P}{\mathbb{P}} % For defining a probabilty measure
\newcommand{\prob}[1]{\mathbb{P}\left(#1\right)} % For defining a probability measure with argument and normal parentheses
\newcommand{\Prob}[1]{\mathbb{P}\Big(#1\Big)} % For defining a probability measure with argument and large parenthesis


%%%%%%%%%%%%%%%%%%%%%%%%%%%%%%%%%%%%%%%%%%%%%%%%%%%%%%%%%%%%%%%%%
%%%%%%%%%%%%%%%%%%%%%%%%%%%%%%%%%%%%%%%%%%%%%%%%%%%%%%%%%%%%%%%%%
\ifnum\printsol=1
\vspace*{2mm}
\hrule
\vskip 8mm

\fi


%%%%%%%%%%%%%%%%%%%%%%%%%%%%%%%%%%%%%%%%%%%%%%%%%%%%%%%%%%%%%%%%%%%%
%%%%%%%%%%%%%%%%%%%%%%%%%%%%%%%%%%%%%%%%%%%%%%%%%%%%%%%%%%%%%%%%%%%%
%%%%%%%%%%%%%%%%%%%%%%%%%%%%%%%%%%%%%%%%%%%%%%%%%%%%%%
\section*{Instructions}
%%%%%%%%%%%%%%%%%%%%%%%%%%%%%%%%%%%%%%%%%%%%%%%%%%%%%%
Show all reasoning clearly. All simulation results should be reproducible and clearly labeled.
You may use R for all computations.

%%%%%%%%%%%%%%%%%%%%%%%%%%%%%%%%%%%%%%%%%%%%%%%%%%%%%%
\section*{Problems}
%%%%%%%%%%%%%%%%%%%%%%%%%%%%%%%%%%%%%%%%%%%%%%%%%%%%%%

\begin{enumerate}[leftmargin=*]

%----------------------------------------------------
\item \textbf{Basic R Warm-Up}

\begin{enumerate}
\item Use R to compute:
\[
2^5, \qquad \frac{17}{3}, \qquad \sqrt{2}.
\]

\item Store the value of $\sqrt{2}$ in a variable \texttt{x} and compute $x^2 - 2$.
\end{enumerate}

%----------------------------------------------------
\item \textbf{Simulating a Coin Toss}

\begin{enumerate}
\item Use \texttt{sample()} to simulate one fair coin toss with outcomes H and T.
\item Simulate 10 independent coin tosses and store the outcomes.
\item Count the number of heads.
\end{enumerate}

%----------------------------------------------------
\item \textbf{Empirical Probability}

\begin{enumerate}
\item Simulate 100 independent coin tosses.
\item Compute the empirical proportion of heads.
\item Repeat the experiment three times.
\item Comment on whether the empirical proportions are identical or different.
\end{enumerate}

%----------------------------------------------------
\item \textbf{Dice Experiment}

\begin{enumerate}
\item Simulate 60 rolls of a fair six-sided die.
\item Estimate the probability that the outcome is greater than 4.
\item Compare your estimate to the theoretical probability.
\end{enumerate}

%----------------------------------------------------
\item \textbf{Conceptual Question}

In your own words, explain:
\begin{itemize}
\item What an empirical probability is,
\item Why empirical probabilities vary from run to run,
\item How this connects to the idea of long-run frequency.
\end{itemize}

\end{enumerate}

%%%%%%%%%%%%%%%%%%%%%%%%%%%%%%%%%%%%%%%%%%%%%%%%%%%%%%
\ifnum\printsol=1
%%%%%%%%%%%%%%%%%%%%%%%%%%%%%%%%%%%%%%%%%%%%%%%%%%%%%%
\vskip 5mm
\hrule
\vskip 5mm
\section*{Solutions}
%%%%%%%%%%%%%%%%%%%%%%%%%%%%%%%%%%%%%%%%%%%%%%%%%%%%%%

\begin{enumerate}[leftmargin=*]

\item
\begin{lstlisting}[language=R]
2^5
17 / 3
sqrt(2)

x <- sqrt(2)
x^2 - 2
\end{lstlisting}

\item
\begin{lstlisting}[language=R]
sample(c("H","T"), 1)

tosses <- sample(c("H","T"), 10, replace = TRUE)
sum(tosses == "H")
\end{lstlisting}

\item
\begin{lstlisting}[language=R]
t1 <- sample(c("H","T"), 100, replace = TRUE)
mean(t1 == "H")

t2 <- sample(c("H","T"), 100, replace = TRUE)
mean(t2 == "H")

t3 <- sample(c("H","T"), 100, replace = TRUE)
mean(t3 == "H")
\end{lstlisting}

The proportions differ slightly due to randomness but cluster near $0.5$.

\item
\begin{lstlisting}[language=R]
rolls <- sample(1:6, 60, replace = TRUE)
mean(rolls > 4)
\end{lstlisting}

Theoretical probability: $P(X>4)=2/6=1/3$.

\item
Empirical probabilities are computed from observed data rather than exact formulas.
They vary due to random sampling, but as the number of trials increases,
they stabilize near the true probability, illustrating the Law of Large Numbers.

\end{enumerate}

\vskip 5mm
\hrule
\vskip 5mm
\begin{center}
\bf Please let me know if you have any questions, comments, or corrections!
\end{center}

%%%%%%%%%%%%%%%%%%%%%%%%%%%%%%%%%%%%%%%%%%%%%%%%%%%%%%
\fi
%%%%%%%%%%%%%%%%%%%%%%%%%%%%%%%%%%%%%%%%%%%%%%%%%%%%%%

\end{document}
%%%%%%%%%%%%%%%%%%%%%%%%%%%%%%%%%%%%%%%%%%%%%%%%%%%%%%
