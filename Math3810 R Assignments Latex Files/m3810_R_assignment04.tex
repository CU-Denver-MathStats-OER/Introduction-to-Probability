\documentclass[11pt]{article}
\usepackage[suffix=Solutions]{teaching-header}
\usepackage{listings,tikz,enumitem,booktabs}

\lstdefinelanguage{R}{
      keywords={if,else,while,for,in,next,break,function,TRUE,FALSE,NULL,Inf,NA,NaN,switch,repeat,return,require,library},
      keywordstyle=\color{blue}\bfseries,
      identifierstyle=\color{black},
      comment=[l]{\#},
      commentstyle=\color{gray}\ttfamily,
      string=[b]{"},
      stringstyle=\color{red}\ttfamily,
      morecomment=[l]{//},
      morestring=[b]{'},
      sensitive=true,
      morekeywords={print,summary,plot,lm,glm,data,frame,read.csv,write.csv,factor,levels,names,colnames,rownames,
      head,tail,str,dim,length,class,typeof,mode,is.na,is.null,is.finite,is.infinite,is.nan,as.numeric,as.character,
      as.factor,as.Date,as.POSIXct,as.matrix,as.data.frame,rbind,cbind,merge,subset,aggregate,tapply,apply,lapply,sapply,
      mapply,vapply,replicate,seq,rep,c,list,matrix,array,data.frame,table,hist,boxplot,barplot,pie,curve,lines,points,text,
      abline,legend,par,mtext,title,xlab,ylab,xlim,ylim,main,sub,col,pch,cex,lty,lwd,type,bg,fg,args,options,warnings,errors,
      message,stop,warning,error,try,tryCatch,withCallingHandlers,on.exit,debug,browser,trace,recover,options,getOption,setOption},
    }

\def\assessmenttype{introductory homework}
\def\assessmenttypecap{Introductory Homework}
\def\assessmentname{\#4}  
\def\assessmentnamesol{\#4 Solutions}
\def\duedate{Wednesday, October 9}
\def\duetime{3:30pm}
\def\printfancyheader{1} 

\def\classnum{3810}
\def\classtitle{Probability}
\def\classtitleshort{Probability}
\def\classsec{001}
\def\classterm{Fall 2025}
\def\instructor{Robert Rostermundt}
%\def\hmwknum{\#2}

\ifnum\printfancyheader=1
\pagestyle{myheadings}
\else
    \ifnum\printsol=0
        \pagestyle{empty}
    \else
        \pagestyle{plain}
    \fi
\fi

\ifnum\printsol=0
\title{\vspace{-1in}Math\classnum\;-\;\classtitle\\
    Section\;\classsec\;-\;\classterm\\
    \assessmenttypecap\;\assessmentname}
\author{University of Colorado Denver / College of Liberal Arts and Sciences}
\date{Department of Mathematics - \instructor}
\markright{Math\classnum\;-\;\classtitle,\;\assessmenttypecap\;\assessmentname, UCD, \classterm, \instructor}
\else
\title{\vspace{-1in}Math\classnum\;-\;\classtitle\\
    Section\;\classsec\;-\;\classterm\\
    \assessmenttypecap\;\assessmentnamesol}
\author{University of Colorado Denver / College of Liberal Arts and Sciences}
\date{Department of Mathematics - \instructor}
\markright{Math\classnum\;-\;\classtitle,\;\assessmenttypecap\;\assessmentname, UCD, \classterm, \instructor}
\fi

\begin{document}
\ifnum\printfancyheader=1
\maketitle\thispagestyle{empty}
\else
    \ifnum\printsol=0
        \begin{center}{\large Math \classnum--\classsec, UCD, \classterm, Homework \assessmentname}\\
        \smallskip
        Due by \duedate\, at \duetime.\\
        \smallskip
        {\em Late homeworks will not be accepted without prior approval. Please no email submissions!}
        \end{center}
    \else
        \begin{center}{Math \classnum--\classsec, UCD, \classterm \hfill Homework \assessmentnamesol}
        \end{center}
        \vskip 2mm
    \fi
\hrule
\vskip 5mm
\fi

\normalsize

\ifnum\printsol=0
\hrule\bigskip\noindent\textbf{Name:}\hspace*{3in}
\textbf{Student Number:}\bigskip\hrule
\vspace*{5mm}
\fi

\renewcommand{\S}{\mathbb{S}}
\renewcommand{\P}{\mathbb{P}}
\newcommand{\prob}[1]{\mathbb{P}\left(#1\right)}
\newcommand{\Prob}[1]{\mathbb{P}\Big(#1\Big)}

\section*{Instructions}
Show all reasoning clearly. All simulation results should be reproducible and clearly labeled.
You may use R for all computations.

\section*{Problems}
\begin{enumerate}[leftmargin=*]

\item \textbf{Continuous Random Variable Simulation}
\begin{enumerate}
\item Simulate 500 samples from $X \sim N(5,4)$.
\item Compute sample mean and variance.
\item Plot histogram and overlay the theoretical density curve.
\end{enumerate}

\item \textbf{Linear Transformation of RV}
\begin{enumerate}
\item Define $Y = 3X - 2$.
\item Compute mean and variance of $Y$ empirically.
\item Compare to theoretical $E[Y]$ and $Var(Y)$.
\end{enumerate}

\item \textbf{CDF Comparison}
\begin{enumerate}
\item Compute empirical CDF of $X$ and $Y$.
\item Plot both on the same graph with theoretical CDF curves.
\end{enumerate}

\item \textbf{Standardization}
\begin{enumerate}
\item Standardize $X$ to $Z = (X - \bar{X}) / s_X$.
\item Verify mean and variance of $Z$.
\item Plot histogram of $Z$ and overlay standard normal density.
\end{enumerate}

\item \textbf{Discussion}
\begin{itemize}
\item Explain the effect of linear transformations on mean and variance.
\item Explain why standardization produces mean 0 and variance 1.
\end{itemize}

\end{enumerate}

\ifnum\printsol=1
\section*{Solutions}
\begin{enumerate}[leftmargin=*]

\item
\begin{lstlisting}[language=R]
X <- rnorm(500, mean=5, sd=2)
mean(X)
var(X)
hist(X, prob=TRUE)
curve(dnorm(x,5,2), add=TRUE, col="red")
\end{lstlisting}

\item
\begin{lstlisting}[language=R]
Y <- 3*X - 2
mean(Y)
var(Y)
3*mean(X) - 2   # Theoretical
3^2 * var(X)    # Theoretical
\end{lstlisting}

\item
\begin{lstlisting}[language=R]
ecdfX <- ecdf(X)
ecdfY <- ecdf(Y)
plot(ecdfX, col="blue")
lines(ecdfY, col="red")
curve(pnorm(x,5,2), add=TRUE, lty=2)
curve(pnorm(x,3*5-2,3*2), add=TRUE, lty=2)
\end{lstlisting}

\item
\begin{lstlisting}[language=R]
Z <- (X - mean(X)) / sd(X)
mean(Z)
var(Z)
hist(Z, prob=TRUE)
curve(dnorm(x,0,1), add=TRUE, col="green")
\end{lstlisting}

\item
Linear transformations scale and shift mean and variance.  
Standardization shifts mean to 0 and scales variance to 1.

\end{enumerate}

\vskip 5mm
\hrule
\vskip 5mm
\begin{center}
\bf Please let me know if you have any questions, comments, or corrections!
\end{center}
\fi

\end{document}
