\documentclass[11pt]{article}
\usepackage[suffix=Solutions]{teaching-header}
\usepackage{listings,tikz,enumitem,booktabs}

\lstdefinelanguage{R}{
      keywords={if,else,while,for,in,next,break,function,TRUE,FALSE,NULL,Inf,NA,NaN,switch,repeat,return,require,library},
      keywordstyle=\color{blue}\bfseries,
      identifierstyle=\color{black},
      comment=[l]{\#},
      commentstyle=\color{gray}\ttfamily,
      string=[b]{"},
      stringstyle=\color{red}\ttfamily,
      morecomment=[l]{//},
      morestring=[b]{'},
      sensitive=true,
      morekeywords={print,summary,plot,lm,glm,data,frame,read.csv,write.csv,factor,levels,names,colnames,rownames,
      head,tail,str,dim,length,class,typeof,mode,is.na,is.null,is.finite,is.infinite,is.nan,as.numeric,as.character,
      as.factor,as.Date,as.POSIXct,as.matrix,as.data.frame,rbind,cbind,merge,subset,aggregate,tapply,apply,lapply,sapply,
      mapply,vapply,replicate,seq,rep,c,list,matrix,array,data.frame,table,hist,boxplot,barplot,pie,curve,lines,points,text,
      abline,legend,par,mtext,title,xlab,ylab,xlim,ylim,main,sub,col,pch,cex,lty,lwd,type,bg,fg,args,options,warnings,errors,
      message,stop,warning,error,try,tryCatch,withCallingHandlers,on.exit,debug,browser,trace,recover,options,getOption,setOption},
    }

\def\assessmenttype{introductory homework}
\def\assessmenttypecap{Introductory Homework}
\def\assessmentname{\#8}  
\def\assessmentnamesol{\#8 Solutions}
\def\duedate{Wednesday, November 6}
\def\duetime{3:30pm}
\def\printfancyheader{1} 

\def\classnum{3810}
\def\classtitle{Probability}
\def\classtitleshort{Probability}
\def\classsec{001}
\def\classterm{Fall 2025}
\def\instructor{Robert Rostermundt}
%\def\hmwknum{\#2}

\ifnum\printfancyheader=1
\pagestyle{myheadings}
\else
    \ifnum\printsol=0
        \pagestyle{empty}
    \else
        \pagestyle{plain}
    \fi
\fi

\ifnum\printsol=0
\title{\vspace{-1in}Math\classnum\;-\;\classtitle\\
    Section\;\classsec\;-\;\classterm\\
    \assessmenttypecap\;\assessmentname}
\author{University of Colorado Denver / College of Liberal Arts and Sciences}
\date{Department of Mathematics - \instructor}
\markright{Math\classnum\;-\;\classtitle,\;\assessmenttypecap\;\assessmentname, UCD, \classterm, \instructor}
\else
\title{\vspace{-1in}Math\classnum\;-\;\classtitle\\
    Section\;\classsec\;-\;\classterm\\
    \assessmenttypecap\;\assessmentnamesol}
\author{University of Colorado Denver / College of Liberal Arts and Sciences}
\date{Department of Mathematics - \instructor}
\markright{Math\classnum\;-\;\classtitle,\;\assessmenttypecap\;\assessmentname, UCD, \classterm, \instructor}
\fi

\begin{document}
\ifnum\printfancyheader=1
\maketitle\thispagestyle{empty}
\else
    \ifnum\printsol=0
        \begin{center}{\large Math \classnum--\classsec, UCD, \classterm, Homework \assessmentname}\\
        \smallskip
        Due by \duedate\, at \duetime.\\
        \smallskip
        {\em Late homeworks will not be accepted without prior approval. Please no email submissions!}
        \end{center}
    \else
        \begin{center}{Math \classnum--\classsec, UCD, \classterm \hfill Homework \assessmentnamesol}
        \end{center}
        \vskip 2mm
    \fi
\hrule
\vskip 5mm
\fi

\normalsize

\ifnum\printsol=0
\hrule\bigskip\noindent\textbf{Name:}\hspace*{3in}
\textbf{Student Number:}\bigskip\hrule
\vspace*{5mm}
\fi

\renewcommand{\S}{\mathbb{S}}
\renewcommand{\P}{\mathbb{P}}
\newcommand{\prob}[1]{\mathbb{P}\left(#1\right)}
\newcommand{\Prob}[1]{\mathbb{P}\Big(#1\Big)}

\section*{Instructions}
Show all reasoning clearly. All simulation results should be reproducible and clearly labeled.
You may use R for all computations.

\section*{Problems}
\begin{enumerate}[leftmargin=*]

\item \textbf{Transformations of Random Variables}
\begin{enumerate}
\item Let $X \sim N(10,4)$. Define $Y = 3X-5$. Compute the mean and variance of $Y$.
\item Let $X \sim N(5,9)$. Define $Y = -2X+7$. Compute the mean and variance of $Y$.
\item Standardize $X \sim N(\mu, \sigma^2)$ to $Z=(X-\mu)/\sigma$. Compute $P(X<15)$ for $X\sim N(12,16)$ using standardization.
\end{enumerate}

\item \textbf{Simulation of Linear Transformations}
\begin{enumerate}
\item Simulate 50, 100, 1000, and 50000 draws from $X\sim N(5,2^2)$ and $Y=-3X+2$. Plot histograms overlaying $X$ and $Y$.
\item Plot the empirical CDFs of $X$ and $Y$ and compare with theoretical CDFs.
\end{enumerate}

\end{enumerate}

\ifnum\printsol=1
\section*{Solutions}
\begin{enumerate}[leftmargin=*]

\item
\begin{lstlisting}[language=R]
# Example 1
X1 <- 10
VarX1 <- 4
Y1_mean <- 3*X1 - 5
Y1_var <- 3^2 * VarX1

# Example 2
X2 <- 5
VarX2 <- 9
Y2_mean <- -2*X2 + 7
Y2_var <- (-2)^2 * VarX2

# Example 3
mu <- 12
sigma <- 4
Z <- (15 - mu)/sigma
pnorm(Z)
\end{lstlisting}

\item
\begin{lstlisting}[language=R]
set.seed(123)
mu <- 5; sigma <- 2; a <- -3; b <- 2
for(n in c(50,100,1000,50000)){
  X <- rnorm(n, mu, sigma)
  Y <- a*X + b
  hist(X, breaks=50, prob=TRUE, col=rgb(0,0,1,0.5), main=paste("n=",n))
  hist(Y, breaks=50, prob=TRUE, col=rgb(1,0,0,0.5), add=TRUE)
  # Empirical CDF
  ecdf_X <- ecdf(X)
  ecdf_Y <- ecdf(Y)
  plot(ecdf_X, verticals=TRUE, col="blue", main=paste("CDF, n=",n))
  lines(ecdf_Y, col="red", verticals=TRUE)
}
\end{lstlisting}

\end{enumerate}

\vskip 5mm
\hrule
\vskip 5mm
\begin{center}
\bf Please let me know if you have any questions, comments, or corrections!
\end{center}
\fi

\end{document}
